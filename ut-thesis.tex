%%%%%%%%%%%%%%%%%%%%%%%%%%%%%%%%%%%%%%%%%%%%%%%%%%%%%%%%%%%%%%%%%%%%%%
%%
%%  UT-THESIS.TEX
%%
%% This program can be redistributed and/or modified under the terms
%% of the LaTeX Project Public License Distributed from CTAN archives
%% in directory CTAN:/macros/latex/base/lppl.txt.
%%
%%  Copyright (c) 1999 by Francois Pitt
%%  Last Update: 1999 May 13
%%
%%%%%%%%%%%%%%%%%%%%%%%%%%%%%%%%%%%%%%%%%%%%%%%%%%%%%%%%%%%%%%%%%%%%%%
%%
%%  This file is distributed in the hope that it will be useful but
%%  without any warranty (without even the implied warranty of
%%  fitness for a particular purpose).  For a description of this
%%  file's purpose, and instructions on its use, see below.
%%
%%  Feel free to copy and redistribute this file, as long as this
%%  copyright notice remains intact and this file is distributed
%%  along with the companion file `ut-thesis.cls'.
%%
%%  (Thanks to Robert Bernecky for his suggestions on improving the
%%  usefulness and readability of this file.)
%%
%%  Send all bugs, questions, comments, suggestions, etc. to the
%%  author, at <fpitt@cs.utoronto.ca>.
%%
%%%%%%%%%%%%%%%%%%%%%%%%%%%%%%%%%%%%%%%%%%%%%%%%%%%%%%%%%%%%%%%%%%%%%%
%%
%%  Skeleton LaTeX2e file for the preparation of theses at UofT;
%%  conforms to the School of Graduate Studies' guidelines of 07/97.
%%  To be used in conjunction with class file `ut-thesis.cls', whose
%%  features it illustrates.
%%
%%  To comment out parts of a file, use the macro \ignore{...}
%%  around the entire block of text you want to ignore.
%%
%%  To explicitly set the pagestyle of any inserted blank page when
%%  \cleardoublepage occurs, use one of \clearemptydoublepage or
%%  \clearplaindoublepage instead.
%%
%%  For single-spaced quotes or quotations, use the `longquote' and
%%  `longquotation' environments.  For single-spaced, 1 1/2-spaced,
%%  or double-spaced paragraphs, use one of the environments
%%  `singlespaced', `oneandahalfspaced', or `doublespaced'.  More
%%  generally, for paragraphs with a line spacing of `n', use
%%  `\begin{newspacing}{n}...\end{newspacing}'.
%%
%%  All other environments, commands, and options provided by the
%%  `ut-thesis' class will be described below, at the point where
%%  they should appear in the document.
%%
%%  See the companion file `ut-thesis.cls' for more details.
%%
%%%%%%%%%%%%%%%%%%%%%%%%%%%%%%%%%%%%%%%%%%%%%%%%%%%%%%%%%%%%%%%%%%%%%%


%%%%%%%%%%%%         PREAMBLE         %%%%%%%%%%%%

%% Default settings format a final copy (12pt font, single-sided,
%% double-spaced, normal margins, single-spaced notes).  For a rough
%% copy (10pt font, double-sided, double-spaced, normal margins, with
%% the word "DRAFT" printed at each corner of every page), use the
%% `draft' option.  The default line spacing can be changed with one
%% of the following options: `singlespaced', `oneandahalfspaced', or
%% `doublespaced'.  The notes are always single-spaced by default, but
%% can be made to have the same spacing as the rest of the document by
%% using the option `spacednotes'.  The size of the margins can be
%% changed with one of the following options: `narrowmargins' (1 1/4"
%% left, 3/4" others), `normalmargins' (1 1/4" left, 1" others),
%% `widemargins' (1 1/4" all), `extrawidemargins' (1 1/2" all).  Any
%% other standard option for the `report' document class can be used
%% to override the default or draft settings.

%% ***   Add any desired options.   ***
%\documentclass[draft]{ut-thesis}
\documentclass{ut-thesis}


%% ***   Add \usepackage declarations here.   ***
\usepackage{setspace}
%\singlespacing
\onehalfspacing
%\doublespacing



%\RequirePackage[l2tabu, orthodox]{nag}

\usepackage{amsthm,amsmath,amssymb,graphicx,color}
%% load other packages you need
\usepackage[english]{babel}
\usepackage{enumerate}
\usepackage{caption}
\usepackage{bm} % For bold fonts
\usepackage{multirow} % For element-wise matrix sparsification table


%\usepackage[color,draft]{showkeys} % To show labels of keys used


%\usepackage{pdfsync} % REMOVE ME AT THE FINAL VERSION

\usepackage{palatino} % Palatino fonts.

%\usepackage[T1]{fontenc}               % Ensure correct font encoding

\usepackage{microtype} % This should be after the font package (IMPORTANT DEPENDENT ON FONTS)

%%%%%%%%%%%%%%%%%%%%%%%%%%%%%%%%%%%%%%%%%%%%%%%%%%%%%%%%%%%%%%%%%%%
% The order between float hyperref and algorithm REALLY Matters
% Update: It works ! A comment from Iain Murray points out that in the hyperref README, one is told to include the float package BEFORE hyperref , and only then include algorithm (after hyperref).
%%%%%%%%%%%%%%%%%%%%%%%%%%%%%%%%%%%%%%%%%%%%%%%%%%%%%%%%%%%%%%%%%%%
%%%%%%%%%%%%%%%%%%%%%%%%%%%%%%%%%%%%%%%%%%%%%%%%%%%%%%%%%%%%%%%%%%%

%\usepackage{float}

\usepackage[backref=page]{hyperref}
\hypersetup{
    unicode=false,          % non-Latin characters in Acrobat’s bookmarks
    pdftoolbar=true,        % show Acrobat’s toolbar?
    pdfmenubar=true,        % show Acrobat’s menu?
    pdffitwindow=true,      % page fit to window when opened
    pdftitle={Randomized Primitives for Linear Algebra and Applications},    % title
    pdfauthor={Anastasios Zouzias},     % author
    pdfsubject={Randomized Linear Algebra},   % subject of the document
    pdfcreator={Anastasios Zouzias},   % creator of the document
    pdfproducer={Anastasios Zouzias}, % producer of the document
    pdfkeywords={derandomization,probability,conditional expectation method, pessimistic estimators}, % list of keywords
    pdfnewwindow=true,      % links in new window
    colorlinks=false,       % false: boxed links; true: colored links
    linkcolor=red,          % color of internal links
    citecolor=green,        % color of links to bibliography
    filecolor=magenta,      % color of file links
    urlcolor=cyan,          % color of external links
}

%%%%%%%%%%%%%%%%%%%%%%%%%%%%%%%%%%%%%%%%%%%%%%%%%%%%%%%%%%%%%%%%%%%
% For Reverse Citations of the References.
%%%%%%%%%%%%%%%%%%%%%%%%%%%%%%%%%%%%%%%%%%%%%%%%%%%%%%%%%%%%%%%%%%%
%%%%%%%%%%%%%%%%%%%%%%%%%%%%%%%%%%%%%%%%%%%%%%%%%%%%%%%%%%%%%%%%%%%
%%%%%%%%%%%%%%%%%%%%%%%%%%%%%%%%%%%%%%%%%%%%%%%%%%%%%%%%%%%%%%%%%%%
\renewcommand*{\backref}[1]{}
\renewcommand*{\backrefalt}[4]{%
	\ifcase #1 %
		(Not cited) %
	\or
		(Cited on page~#2)%
	\else
		(Cited on pages~#2)%
	\fi
}
\renewcommand*{\backrefsep}{, }
\renewcommand*{\backreftwosep}{ and~}
\renewcommand*{\backreflastsep}{ and~}
%%%%%%%%%%%%%%%%%%%%%%%%%%%%%%%%%%%%%%%%%%%%%%%%%%%%%%%%%%%%%%%%%%%
%%%%%%%%%%%%%%%%%%%%%%%%%%%%%%%%%%%%%%%%%%%%%%%%%%%%%%%%%%%%%%%%%%%
%%%%%%%%%%%%%%%%%%%%%%%%%%%%%%%%%%%%%%%%%%%%%%%%%%%%%%%%%%%%%%%%%%%

\usepackage{algorithm}% http://ctan.org/pkg/algorithm
\usepackage{algpseudocode}% http://ctan.org/pkg/algorithmicx


% For fancy Chapter Headers.
\usepackage{fancyhdr}
\usepackage[Lenny]{fncychap}   % *****

\ChNameVar{\fontsize{16}{18}\usefont{OT1}{phv}{m}{n}\selectfont} % Set the style for name
\ChNumVar{\fontsize{60}{62}\usefont{OT1}{ptm}{m}{n}\selectfont} % sets the style for digit

\ChTitleVar{\Huge\bfseries\rm} % sets the style for Title.
\ChRuleWidth{2.5pt}

% For figure caption.
%\usepackage[small,normal,bf,up]{caption}
%\renewcommand{\captionfont}{\small\itshape}


%% We will completely define our own header strings, so switch the fancy
%% headers on, but nuke all the default values. (Note that this package
%% *has* to load after the geometry package!)

%\usepackage{fancyhdr}
%\pagestyle{fancy}
%\fancyhf{}

%% Now begin customising things. See the fancyhdr docs for more info.

%\renewcommand{\chaptermark}[1]{\markboth{\MakeUppercase{#1}}{}}
%\renewcommand{\sectionmark}[1]{\markright{\MakeUppercase{#1}}{}}
%\renewcommand{\headrulewidth}{0pt}




%Results
%Shortcuts
\newcommand{\R}{\mathbb{R}}
\newcommand{\E}{\mathbb{E}}
\newcommand{\N}{\mathbb{N}}

\newcommand{\x}{\mathbf{x}}
\newcommand{\y}{\mathbf{y}}
\newcommand{\z}{\mathbf{z}}
%\newcommand{\A}{\mathbf{A}}
\newcommand{\bi}{\mathbf{b}}
\newcommand{\ro}{\mathbf{r}}
\newcommand{\w}{\mathbf{w}}
\newcommand{\p}{\mathbf{p}}
\newcommand{\zero}{\mathbf{0}}
\newcommand{\ep}{\epsilon}
\newcommand{\de}{\delta}

%%%%%%%%%%%%%%%%%%%%%%%%%%%%%%%%%%%%%%%%%%%
%   General notation
%%%%%%%%%%%%%%%%%%%%%%%%%%%%%%%%%%%%%%%%%%%
%%%%%%%%%%%%%%%%%%%%%%%%%%%%%%%%%%%%%%%%%%%
% Epsilon
\newcommand{\eps}{\varepsilon}

% Extended condition number of A (J. Demmel)
\newcommand{\kappaFS}{\kappa^2_{\textrm{\tiny F}}}
\newcommand{\kappaF}{\kappa_{\textrm{\tiny F}}}



% Absolute Value of a number
\newcommand{\abs}[1]{\left|#1\right|}

% Optimum
\def\opt{{\textrm{opt}}}

% Polynomial factors
\newcommand{\poly}[1]{\mathrm{poly}\left(#1\right) }

% Polylog(n)
\newcommand{\polylog}[1]{\ensuremath{\operatorname{polylog}\left(#1\right)}}

\newcommand{\sign}[1]{\ensuremath{\operatorname{\textbf{sign}}\left(#1\right)}}
% Big-Oh Notation
\newcommand{\OO}{\mathcal{O}}

% Volume of a set
\newcommand{\vol}[1]{\operatorname{vol}(#1)}

% Real numbers
\newcommand{\RR}{\mathbb{R}}
\newcommand{\reals}{\mathbb{R}}

% Integers numbers
\newcommand{\ZZ}{\mathbb{Z}}

% Natural numbers
\newcommand{\NN}{\mathbb{N}}

% Euler's constant
\newcommand{\e}{\ensuremath{{\textrm e}}}

% Sparsity of matrix
\newcommand{\ravg}{\mathrm{R}_{\mathrm{avg}}}
\newcommand{\cavg}{\mathrm{C}_{\mathrm{avg}}}

% Expectation
\DeclareMathOperator{\EE}{\mathbb{E}}

% Probability
\newcommand{\Prob}[1]{\ensuremath{\mathbb{P}\left(#1\right)}}
% Variance
\newcommand{\var}[1]{\ensuremath{\mathrm{Var}(#1)}}


%%%%%%%%%%%%%%%%%%%%%%%%%%%%%%%%%%%%%%%%%%%
%%%%%%%%%%%%%%%%%%%%%%%%%%%%%%%%%%%%%%%%%%%
% Several norms
%%%%%%%%%%%%%%%%%%%%%%%%%%%%%%%%%%%%%%%%%%%
%%%%%%%%%%%%%%%%%%%%%%%%%%%%%%%%%%%%%%%%%%%
\newcommand{\norm}[1]{\ensuremath{\left\|#1\right\|_2}}
\newcommand{\pnorm}[1]{\ensuremath{\left\|#1\right\|_p}}
\newcommand{\qnorm}[1]{\ensuremath{\left\|#1\right\|_q}}
\newcommand{\infnorm}[1]{\ensuremath{\left\|#1\right\|_\infty}}
\newcommand{\onenorm}[1]{\ensuremath{\left\|#1\right\|_1}}
\newcommand{\frobnorm}[1]{\ensuremath{\left\|#1\right\|_{\textrm{F}}}}
%%%%%%%%%%%%%%%%%%%%%%%%%%%%%%%%%%%%%%%%%%%
%%%%%%%%%%%%%%%%%%%%%%%%%%%%%%%%%%%%%%%%%%%


%=======================
%% Matrix quantities
%=======================
%=======================
% Stable rank of a matrix
\newcommand{\sr}[1]{\ensuremath{\operatorname{\textbf{\footnotesize sr}}\left(#1\right)}}
% Trace of a matrix.
\newcommand{\trace}[1]{\ensuremath{\operatorname{\textbf{tr}}\left(#1\right)}}
%\DeclareMathOperator{\trace}{trace}
% Rank of a matrix
\newcommand{\rank}[1]{\ensuremath{\operatorname{\textbf{{\footnotesize rank}}}\left(#1\right)}}
% Kernel of a matrix
%\newcommand{\ker}[1]{\ensuremath{\mathrm{\textbf{ker}}\left(#1\right)}}
% Image of a matrix
\newcommand{\im}[1]{\ensuremath{\operatorname{\textbf{Im}}\left(#1\right)}}
% Condition number of a matrix
\newcommand{\cond}[1]{\ensuremath{\operatorname{\mathrm{cond}}\left(#1\right)}}
% Matrix exponential
\newcommand{\expm}[1]{\ensuremath{\operatorname{\textbf{\footnotesize exp}}\left[#1\right]}}
% Matrix Hyperbolic Cosine
\newcommand{\coshm}[1]{\ensuremath{\operatorname{\textbf{\footnotesize cosh}}\left[#1\right]}}
% Determinant
\newcommand{\detm}[1]{\ensuremath{\operatorname{\textbf{det}}\left(#1\right)}}
% Pseudo-inverse of a matrix
\newcommand{\pinv}[1]{ {#1}^\dagger}

% Pseudo-inverse of a matrix
\newcommand{\tpinv}[1]{ {#1}^{\dagger\top}}
% # of non-zero entries of a matrix
\newcommand{\nnz}[1]{\ensuremath{\operatorname{\textbf{\footnotesize nnz}}\left(#1\right)}}
% Diagonal Matrix
\newcommand{\diag}[1]{\ensuremath{\operatorname{\textbf{diag}}\left(#1\right)}}
% Column span of a matrix
\newcommand{\colspan}[1]{\mathcal{R}(#1)}
% The dilation of a matrix
\newcommand{\dil}[1]{\ensuremath{\mathcal{D}\left(#1\right)}}
%=======================
%=======================


%%%%%%%%%%%%%%%%%%%%%%%%%%%%%%%%%%%%%%%%%%%
%%%%%%%%%%%%%%%%%%%%%%%%%%%%%%%%%%%%%%%%%%%
%%% Vector and matrix operators
%%%%%%%%%%%%%%%%%%%%%%%%%%%%%%%%%%%%%%%%%%%
%%%%%%%%%%%%%%%%%%%%%%%%%%%%%%%%%%%%%%%%%%%


\newcommand{\vct}[1]{\bm{#1}}
\newcommand{\mtx}[1]{\bm{#1}}

% Inner product
\newcommand{\ip}[2]{\left\langle {#1},\ {#2} \right\rangle}
% Matrix inner product ( A \bullet B := tr(AB)
\newcommand{\mip}[2]{ {#1}\bullet {#2}}



% Special matrices (CONSTANTS)
\newcommand{\Id}{\mathbf{I}}
\newcommand{\J}{\mathbf{J}}
\newcommand{\onemtx}{\bm{1}}
\newcommand{\zeromtx}{\mathbf{0}}




%%%%%%%%%%%%%%%%%%%%%%%%%%%%%%%%%%%%%
\newcommand{\mat}[1]{ {\ensuremath{\mathsf{#1} }}}
%%%%%%%%%%%%%%%%%%%%%%%%%%%%%%%%%%%%%

\def\gammab{{\bm{\gamma}}}
\def\kappab{{\bm{\kappa}}}
\def\sig{{\bm{\Sigma}}}
\def\sigplus{{\bm{\Sigma}^{+}}}
\def\siginv{{\bm{\Sigma}^{-1}}}
\def\bet{{\bm{\beta}}}
\def\one{{\bm{1}}}
\def\exp{\hbox{\textrm exp}}
\def\col{\hbox{\textrm col}}
\def\ker{\hbox{\textrm ker}}
\def\ahat{{\hat\a}}
\def\p{{\mathbf p}}
\def\e{{\mathbf e}}
\def\q{{\mathbf q}}
\def\rb{{\mathbf r}}
\def\s{{\mathbf s}}
\def\u{{\mathbf u}}
\def\v{{\mathbf v}}
\def\d{{\mathbf \delta}}
\def\xhat{{\hat\x}}
\def\yhat{{\hat\y}}
\def\A{\matA}
\def\B{\matB}
\def\C{\matC}
\def\Ahat{\hat\matA}
\def\Atilde{\tilde\matA}
\def\Btilde{\tilde\matB}
\def\Stilde{\tilde\matS}
\def\Utilde{\tilde\matU}
\def\Vtilde{\tilde\matV}
\def\G{{\cl G}}
\def\hset{{\cl H}}
\def\Q{{\bm{Q}}}
\def\U{{\bm{U}}}
\def\V{{\bm{V}}}
\def\win{\hat{\w}}
\def\wopt{\w^*}
\def\matAhat{\hat\mat{A}}
\def\matA{\mat{A}}
\def\matB{\mat{B}}
\def\matC{\mat{C}}
\def\matD{\mat{D}}
\def\matE{\mat{E}}
\def\matG{\mat{G}}
\def\matH{\mat{H}}
\newcommand{\matI}{\Id} % This matrix is reserved for the Identity matrix
\def\matM{\mat{M}}
\def\matP{\mat{P}}
\def\matQ{\mat{Q}}
\def\matR{\mat{R}}
\def\matL{\mat{L}}

\def\matS{\mat{S}}
\def\matT{\mat{T}}
\def\matU{\mat{U}}
\def\matV{\mat{V}}
\def\matW{\mat{W}}
\def\matX{\mat{X}}
\def\matY{\mat{Y}}
\def\matZ{\mat{Z}}
\def\matSig{\mat{\Sigma}}
\def\matOmega{\mat{\Omega}}
\def\matGam{\mat{\Gamma}}
\def\matTh{\mat{\Theta}}

\newcommand{\matPi}{\mat{\Pi}}
\newcommand{\matPsi}{\mat{\Psi}}
\newcommand{\matPhi}{\mat{\Phi}}
\def\w{{\mathbf{w}}}
\def\ein{{\cl E_{\mathrm{in}}}}
\def\eout{{\cl E}}



\def\vecx{\vct{x}}
\def\vecb{\vct{b}}
\def\vecu{\vct{u}}
\def\vecy{\vct{y}}
\def\vecz{\vct{z}}
\def\veco{\vct{o}}
\def\vecw{\vct{w}}
\def\vecv{\vct{v}}
\def\vecp{\vct{p}}



%% CAUTION OVERRIDE NOTATION
% Over-ride notation
\def\b{{\mathbf b}}


\newcommand{\bc}{{\b_{\mathcal{R}(\matA)^\bot } }}
\newcommand{\br}{{\b_{\mathcal{R}(\matA) } }}

% Least squares solution of Ax = b
\newcommand{\xls}{\x_{\textrm{\tiny LS}}}
\newcommand{\xopt}{\x_{\textrm{\tiny opt}}}

\newcommand{\yls}{\y_{\textrm{\tiny LS}}}
\newcommand{\yopt}{\y_{\textrm{\tiny opt}}}

% For rows, columns and entries of a matrix A
\newcommand{\ar}[1]{ \matA^{(#1)}}
\newcommand{\ac}[1]{ \matA_{(#1)}}
\newcommand{\Ae}[1]{ \mat{a}_{#1}}

% For rows, columns and entries of a matrix B
\newcommand{\Br}[1]{ \matB^{(#1)}}
\newcommand{\Bc}[1]{ \matB_{(#1)}}
\newcommand{\Be}[1]{ \matB_{#1}}


% For rows and columns and entries of a matrix L
\newcommand{\Lr}[1]{ \matL^{(#1)}}
\newcommand{\Lc}[1]{ \matL_{(#1)}}
\newcommand{\Le}[1]{ \matL_{#1}}

% Normalized Laplacian
\newcommand{\Ln}{\matL_{\textrm{\tiny norm}}}
\newcommand{\Lnr}[1]{ \matL_{\textrm{\tiny norm}}^{(#1)}}
\newcommand{\Lnc}[1]{ \matL_{\textrm{\tiny norm}_{(#1)}}}
\newcommand{\Lne}[1]{ \matL_{\textrm{\tiny norm}_{#1}} }



% For rows, columns and entries of a matrix Q
\newcommand{\qr}[1]{ \matQ^{(#1)}}
\newcommand{\qc}[1]{ \matQ_{(#1)}}

% For rows, columns and entries of a matrix tilde Q
\newcommand{\tqr}[1]{ \widetilde{\matQ}^{(#1)}}
\newcommand{\tqc}[1]{ \widetilde{\matQ}_{(#1)}}





\newcommand{\argmin}{\operatorname*{arg\; min}}
\newcommand{\argmax}{\operatorname*{arg\; max}}

% Cayley Graph of a group G and subset S.
\newcommand{\Cay}[2]{\ensuremath{\operatorname{Cay}\left(#1; #2\right)}}


\newcommand{\Sym}{\mathcal{S}}


\newcommand{\myMat}[4]{\left[\begin{array}[c]{ll}
#1 & #2 \\
#3 & #4
\end{array}
\right]}



%% The line spacing of the document should be specified using one of
%% the document options given above, but if you need a line spacing
%% that is not provided by the options, you can override the default
%% line spacing for the entire document with the command
%%   `\linespacing{...}'.
%% Note that in order to get the correct appearance, the argument to
%% `\linespacing' must be equal to 1/3 + 2/3 times the desired line
%% spacing (for example, single-spaced = \linespacing{1},
%%                        1 1/2-spaced = \linespacing{1.33}, and
%%                       double-spaced = \linespacing{1.66}).

%% ***   Uncomment and fill in a value, if needed.    ***
%% ***   REMEMBER: You should NOT need to use this.  Use one of   ***
%% ***   the document class options mentionned above instead.     ***
%\linespacing{}

%%%%%%%%%%%%%%%%%%%%%%%%%%%%%%%%%%%%%%%%%%%%%%%%%%%%%%%%%%%%%%%%%%%%%%
%%                                                                  %%
%%                  ***   I M P O R T A N T   ***                   %%
%%                                                                  %%
%%  Fill in the following fields with the required information:     %%
%%   - \degree{...}       name of the degree obtained               %%
%%   - \department{...}   name of the graduate department           %%
%%   - \gradyear{...}     year of graduation                        %%
%%   - \author{...}       name of the author                        %%
%%   - \title{...}        title of the thesis                       %%
%%%%%%%%%%%%%%%%%%%%%%%%%%%%%%%%%%%%%%%%%%%%%%%%%%%%%%%%%%%%%%%%%%%%%%

%% ***   Change this example to appropriate values.   ***
\degree{Doctor of Philosophy}
\department{Computer Science}
\gradyear{2012}
\author{Anastasios Zouzias}
\title{RANDOMIZED PRIMITIVES FOR LINEAR ALGEBRA AND APPLICATIONS}

%% ***   NOTE   ***
%% Put here all other formatting commands that belong in the preamble.



%% For example, to list only down to subsections in table of contents
%% (-1=part, 0=chapter, 1=section, 2=subsection, 3=subsubsection,
%%  4=paragraph, 5=subparagraph, 6=subsubparagraph).
%
\setcounter{tocdepth}{1}


%%%%%%%%%%%%      MAIN  DOCUMENT      %%%%%%%%%%%%

\begin{document}

%% ***   NOTE   ***
%% You should put all of your `\newcommand', `\newenvironment', and
%% `\newtheorem's (in other words, all the global definitions that
%% you will need throughout your thesis) in a separate file and use
%% "\input{filename}" to input it here.

\newtheorem{definition}{Definition}
\newtheorem{theorem}{Theorem}[chapter]
\newtheorem{proposition}[theorem]{Proposition}
\newtheorem{lemma}[theorem]{Lemma}
\newtheorem{corollary}[theorem]{Corollary}
\newtheorem{question}{Question}
\newtheorem{claim}[theorem]{Claim}
\newtheorem{conjecture}[theorem]{Conjecture}
\newtheorem{observation}[theorem]{Observation}
\newtheorem{fact}[theorem]{Fact}
\newtheorem{example}{Example}
\newtheorem{assumption}[theorem]{Assumption}
\newtheorem{remark}{Remark}
\newtheorem{problem}{Problem}



%% This sets the page style and numbering for preliminary sections.
\begin{preliminary}

%% This generates the title page from the information given above.
\maketitle

%% There should be NOTHING between the title page and abstract.

%% This generates the abstract page, with the line spacing adjusted
%% according to SGS guidelines.
\begin{abstract}
The present thesis focuses on the design and analysis of randomized algorithms for accelerating several linear algebraic tasks. In particular, we develop simple, efficient, randomized algorithms for a plethora of fundamental linear algebraic tasks and we also demonstrate their usefulness and applicability to matrix computations and graph theoretic problems. The thesis can be divided into three parts. The first part concentrates on the development of randomized linear algebraic primitives, the second part demonstrates the application of such primitives to matrix computations, and the last part discusses the application of such primitives to graph problems.
%

%
First, we present randomized approximation algorithms for the problems of matrix multiplication, orthogonal projection, vector orthonormalization and principal angles computation (a.k.a. canonical correlation analysis).
%

%
Second, utilizing the tools developed in the first part, we present randomized and provable accurate approximation algorithms for the problems of linear regression and element-wise matrix sparsification. Moreover, we present an efficient deterministic algorithm for selecting a small subset of vectors that are in isotropic position.
%

%
Finally, we exploit well-known interactions between linear algebra and spectral graph theory to develop and analyze graph algorithms. In particular, we present a near-optimal time deterministic construction of expanding Cayley graphs, an efficient deterministic algorithm for graph sparsification and a randomized distributed Laplacian solver that operates under the gossip model of computation.
%
\end{abstract}

%% Anything placed between the abstract and table of contents will
%% appear on a separate page since the abstract ends with \newpage
%% and the table of contents starts with \clearpage.

%% This generates a "dedication" section, if needed.
%% (uncomment to have it appear in the document)
\begin{dedication}
	\vspace{12ex}
	\begin{quotation}
		\begin{center}
			\begin{em}
				\section*{Dedication}
				Dedicated to the memory of Avner Magen
    		\end{em}
		\end{center}
	\end{quotation}
\end{dedication}

%% The `dedication' and `acknowledgements' sections do not create new
%% pages so if you want the two sections to appear on separate pages,
%% you should put an explicit \newpage between them.

\newpage
%% This generates an "acknowledgements" section, if needed.
%% (uncomment to have it appear in the document)
\begin{acknowledgements}
The completion of this manuscript was made possible through the invaluable contributions of a number of people. First, I would like to express my gratitude to my supervisors Mark Braverman and Avner Magen. They both provided me with a deep perspective of theoretical computer science and guidance on many technical aspects of this thesis. Their inspiration has been truly invaluable to me. Moreover, I would like to thank Allan Borodin, Toni Pitassi and Mike Molloy for agreeing to serve as members of my departmental examination committee. Moreover, I would like to sincerely thank Joel A. Tropp for his careful assessment of our work and for providing useful suggestions for improvements.
%

%
During my time in graduate school, I had the great pleasure to collaborate with many researchers. I would like to thank all my co-authors: Haim Avron, Christos Boutsidis, Petros Drineas, Jeff Edmonds, Nick Freris, Piotr Indyk, Pascal Koiran, Avner Magen, Mike Mahoney, Tassos Sidiropoulos, Sivan Toledo and Michail Vlachos. I truly enjoyed collaborating with them and I have learned a lot from each one of them.
%

%
The department of computer science at the university of Toronto (UofT) is a great academic environment with many great scholars. In particular, I would like to thank all the members of the theory group. I have also learned a lot from fellow graduate students and postdocs especially Aki, Arkadev, Brendan, Costis, Dai, Eden, George, Jeremy, Joel, Kaveh, Lila,  Natan, Per, Periklis, Siavosh, Tassos, Wesley, Yevgeniy, Yu and Yuval.
%

%
Many thanks go to IBM Research Zurich for hosting me as a research intern during the summer and fall of 2011. The initial discussions about the randomized gossip algorithms of Chapter~\ref{chap:graph} were initiated there. In addition, I would like to thank the department of computer science at Princeton University for hosting me as a visiting student during the winter of 2012 (especially Mark for setting up everything).
%

%
I would like to thank all Greek graduate students at UofT (a.k.a. Greek mafia or ``grspamites'') for all the moments of joy and fun that we shared together. Last but not least, I would like to thank Maria for her support, encouragement, and for making my journey through graduate school, pleasant and full of beautiful moments.
%

%
Finally, my heartfelt thanks go out to my family and friends in Greece for their constant support and encouragement.
\end{acknowledgements}

%% This generates the Table of Contents (on a separate page).
\tableofcontents

%% This generates the List of Tables (on a separate page), if needed.
%% (uncomment to have it appear in the document)
%\listoftables

%% This generates the List of Figures (on a separate page), if needed.
%% (uncomment to have it appear in the document)
%\listoffigures

%% This generates the List of Algorithms (on a separate page), if needed.
%% (uncomment to have it appear in the document)
%\listofalgorithms


%% End of the preliminary sections: reset page style and numbering.
\end{preliminary}

%%%%%%%%%%%%%%%%%%%%%%%%%%%%%%%%%%%%%%%%%%%%%%%%%%%%%%%%%%%%%%%%%%%%%%
%%  Put your Chapters here; the easiest way to do this is to keep   %%
%%  each chapter in a separate file and `\include' all the files    %%
%%  right here.  Note that each chapter file should start with the  %%
%%  line "\chapter{ChapterName}".  Note that using `\include'       %%
%%  instead of `\input' makes each chapter start on a new page.     %%
%%%%%%%%%%%%%%%%%%%%%%%%%%%%%%%%%%%%%%%%%%%%%%%%%%%%%%%%%%%%%%%%%%%%%%

%% ***   Include chapter files here.   ***
%%%%%%%%%%%%%%%%%%%%%%%%%%%%%%%%%%%%%%%%%%%%%%%%%%%
%%%%%%%%%%%%%%%%%%%%%%%%%%%%%%%%%%%%%%%%%%%%%%%%%%%
\chapter{Introduction}\label{chap:intro}
%%%%%%%%%%%%%%%%%%%%%%%%%%%%%%%%%%%%%%%%%%%%%%%%%%%
%%%%%%%%%%%%%%%%%%%%%%%%%%%%%%%%%%%%%%%%%%%%%%%%%%%
%
Randomness has served as an important resource and indispensable idea in the theory of computation. There is a plethora of successful paradigms of the use of randomness in theoretical computer science including complexity theory (interactive proofs, PCP), distributed computation, and randomized approximation algorithms in combinatorial optimization (randomized rounding), computational geometry (coresets) and machine learning theory (VC-dimension), to name a few. Randomness has also been the driving force in discrete mathematics towards a better understanding of combinatorial structures via the probabilistic method.
%

%
The present thesis focuses on the following fundamental question:
\begin{center}
	How can we utilize randomness to accelerate linear algebraic computations?
\end{center}
%

%
The design and analysis of deterministic ``exact'' algorithms for linear algebraic tasks including multiplying matrices, solving a system of linear equations, computing the rank, the singular values or any other interesting quantities associated with matrices has a very rich literature both in the pure and the applied mathematics literature~\cite{book:Demmel,book:GVL}.
%

%
On the other hand, the first appearance of a randomized algorithm for approximating matrix computations via dimensionality reduction appeared in the papers of Papadimitriou et al.~\cite{LSI} and Frieze et al.~\cite{FKV98}. The authors of~\cite{LSI}, motivated by an application to term-document indexing (latent semantic indexing), proposed a randomized algorithm for efficient low rank matrix approximation using random projections. The paper of~\cite{FKV98} analyzed a randomized dimensionality reduction approach utilizing non-uniform row/column sampling for the low rank matrix approximation problem. In the sequel, the idea of utilizing randomness to approximate matrices inspired researchers to design and analyze randomized algorithms for approximating matrix multiplication~\cite{matrixmult:drineas}, low rank matrix approximation~\cite{lowrank:FKV,lowrank:drineas,lra:PNAS2007,HMT} and matrix decomposition~\cite{matrixdecomp:drineas} to name a few. Most of this work was motivated by the need of processing very large data-sets which are usually modeled by a matrix representation. In particular, a large body of work on the the design of randomized algorithms for large matrix problems has been recently developed. The current state of the rapidly growing literature in this research area has been partially summarized in the following surveys~\cite{book:Mahoney,book:spectral,HMT}.
%

%
The scope of this thesis is to contribute to the aforementioned line of research by designing and analyzing simple, efficient, randomized approximation algorithms for several fundamental linear algebraic tasks and, in addition, demonstrate their usefulness and applicability to matrix computations and graph theoretic problems.

%
%%%%%%%%%%%%%%%%%%%%%%%%%%%%%%%%%%%%%%%%%%%%%%%%%%%
%%%%%%%%%%%%%%%%%%%%%%%%%%%%%%%%%%%%%%%%%%%%%%%%%%%
\section{Organization of the Thesis}
%%%%%%%%%%%%%%%%%%%%%%%%%%%%%%%%%%%%%%%%%%%%%%%%%%%
%%%%%%%%%%%%%%%%%%%%%%%%%%%%%%%%%%%%%%%%%%%%%%%%%%%
Below, we outline the structure of this thesis and highlight the contributions of the individual chapters. The thesis can be divided into three parts. The first part, consisting mainly of Chapter~\ref{chap:rnla}, concentrates on the design and analysis of several randomized linear algebraic primitives. The second part (Chapter~\ref{chap:ma}) demonstrates the application of such linear algebraic tools to several matrix computations. Finally, the third part (Chapter~\ref{chap:graph}) discusses the application of such primitives to graph theoretic problems.
%%%%%%%%%%%%%%%%%%%%%%%%%%%%%%%%%%%%%%%%%%%%%%%%%%%
%%%%%%%%%%%%%%%%%%%%%%%%%%%%%%%%%%%%%%%%%%%%%%%%%%%
\subsection*{Chapter~\ref{chap:rnla}: Randomized Approximate Linear Algebraic Primitives}
%%%%%%%%%%%%%%%%%%%%%%%%%%%%%%%%%%%%%%%%%%%%%%%%%%%
%%%%%%%%%%%%%%%%%%%%%%%%%%%%%%%%%%%%%%%%%%%%%%%%%%%
%
Chapter~\ref{chap:rnla} discusses randomized linear algebraic primitives such as approximate matrix multiplication, approximate orthogonal projection, approximate vector orthonormalization and approximate computation of a particular notion of distance between two linear subspaces. Below we briefly outline the main contributions of this chapter.
%

%
The research of~\cite{lowrank:FKV} focuses on using non-uniform row sampling to speed-up the running time of several matrix computations. The subsequent developments of~\cite{matrixmult:drineas,lowrank:drineas, matrixdecomp:drineas} also study the performance of Monte-Carlo algorithms on primitive matrix algorithms including the matrix multiplication problem with respect to the Frobenius norm, see also~\cite{lowrank:rankone:VR}. Sarlos~\cite{sarlos} extended (and improved) this line of research using random projections. Here, following the above line of research, we improve the analysis of the above randomized algorithms for approximating matrix multiplication with respect to the operator norm.
%

%
In addition, a randomized iterative algorithm for approximately computing orthogonal projections is presented. That is, given any vector and linear subspace represented as the column span of a matrix $\matA$, the proposed algorithm converges exponentially to the orthogonal projection of the given vector onto the column span of $\matA$. The convergence rate of the algorithm depends, roughly speaking, on the condition number of $\matA$.
%

%
Based on the aforementioned approximate orthogonal projection algorithm, we present a randomized iterative, amenable to parallel implementation, algorithm for orthonormalizing a set of high dimensional vectors. The algorithm might be effective compared to the classical Gram Schmidt orthonormalization for the case of sparse and sufficiently well-conditioned set of vectors.
%

%
Finally, an efficient randomized algorithm for approximating the principal angles and principal vectors between two subspaces is presented. To the best of our knowledge, the proposed algorithm is the first provable accurate approximation algorithm that is more efficient than the state-of-the-art exact algorithms~\cite{GZ95} for the case of very high-dimensional subspaces.
\begin{center}
This chapter is based on joint work with Freris~\cite{REK} and on joint work with Avron, Boutsidis and Toledo~\cite{approxCCA}
\end{center}
%
%
%%%%%%%%%%%%%%%%%%%%%%%%%%%%%%%%%%%%%%%%%%%%%%%%%%%
%%%%%%%%%%%%%%%%%%%%%%%%%%%%%%%%%%%%%%%%%%%%%%%%%%%
\subsection*{Chapter~\ref{chap:ma}: Matrix Algorithms}
%%%%%%%%%%%%%%%%%%%%%%%%%%%%%%%%%%%%%%%%%%%%%%%%%%%
%%%%%%%%%%%%%%%%%%%%%%%%%%%%%%%%%%%%%%%%%%%%%%%%%%%
%
In Chapter~\ref{chap:ma}, we present randomized and provable accurate approximate algorithms for the problems of linear regression and element-wise matrix sparsification. Moreover, we analyze an efficient deterministic algorithm for selecting a small subset of vectors that are in isotropic position. Below, we briefly outline the main contributions of this chapter.
%

%
We present a randomized iterative algorithm that, given any system of linear equations, exponentially converges in expectation to the minimum Euclidean norm least squares solution. The expected number of arithmetic operations required to obtain an estimate of given accuracy is proportional to the square condition number of the system multiplied by the number of non-zeros entries of the input matrix. The proposed algorithm which we call \emph{randomized extended Kaczmarz} is an extension of the randomized Kaczmarz method that was analyzed by Strohmer and Vershynin and resolves a question left open in~\cite{SV06,RK}.
%

%
Given a set of vectors in isotropic position, we present an efficient deterministic algorithm for selecting a subset of thes vectors that are approximately close to isotropic position. The proposed algorithm builds on important and strong results from numerical linear algebra including the Fast Multipole Method~\cite{FMM:CGR} (FMM) and eigenvalue solvers of matrices after rank-one updates~\cite{Gu:update}.
%

%
Element-wise matrix sparsification was pioneered by Achlioptas and McSherry~\cite{matrix:sparsification:AM01,matrix:sparsification:optas}. Achlioptas and McSherry described sampling-based algorithms that select a small number of entries from an input matrix $\matA $ to construct a sparse sketch $\widetilde{\matA} $, which is close to $\matA$ in the operator norm. We present a simple matrix sparsification algorithm that achieves the best known upper bounds for element-wise matrix sparsification and its analysis is based on matrix concentration inequalities. Moreover, using the matrix hyperbolic cosine algorithm (Section~\ref{sec:balancing}), we present the first deterministic algorithm and strong sparsification bounds for symmetric matrices that have an approximate diagonally dominant property.
\begin{center}
This chapter is based on joint work with Magen~\cite{chernoff:matrix_valued:MZ11}, on joint work with Drineas~\cite{matrix:sparsification:IPL2011} and on joint work with Freris~\cite{REK}.
\end{center}
%
%%%%%%%%%%%%%%%%%%%%%%%%%%%%%%%%%%%%%%%%%%%%%%%%%%%
%%%%%%%%%%%%%%%%%%%%%%%%%%%%%%%%%%%%%%%%%%%%%%%%%%%
\subsection*{Chapter~\ref{chap:graph}: Graph Algorithms}
%%%%%%%%%%%%%%%%%%%%%%%%%%%%%%%%%%%%%%%%%%%%%%%%%%%
%%%%%%%%%%%%%%%%%%%%%%%%%%%%%%%%%%%%%%%%%%%%%%%%%%%
It is well-known that linear algebra serves as an extremely useful tool for analyzing and understanding several properties of graphs, most notably graph expansion. In this chapter, we exploit such connections to design and analyze graph algorithms such as near-optimal deterministic constructions of expanding Cayley graphs, efficient deterministic algorithms for graph sparsification and Laplacian solvers under the gossip (a.k.a. epidemic) model of distributed computation.

%
First, the Alon-Roichman theorem asserts that Cayley graphs obtained by choosing a logarithmic number of group elements independently and uniformly at random are expanders~\cite{expander:AlonRoichman:orig}. Wigderson and Xiao's derandomization of the matrix Chernoff bound implies a deterministic $\OO(n^4 \log n )$ time algorithm for constructing Alon-Roichman graphs~\cite{chernoff:matrix_valued:derand:WX08}. Independently, Arora and Kale generalized the multiplicative weights update (MWU) method to the matrix-valued setting and, among other interesting implications, they improved the running time to $\OO(n^3\polylog{n})$~\cite{phdthesis:Kale:2008}. Here we further improve the running time to $\OO(n^2 \log^3 n)$ utilizing the matrix hyperbolic cosine algorithm and exploiting the group structure of the problem.
%

%
Second, the spectral graph sparsification problem poses the question whether any dense graph can be approximated by a sparse graph while preserving all eigenvalues of the difference of their Laplacian matrices to an arbitrarily small relative error~\cite{graph:sparsifier:ICM2010}; the resulting graphs are usually called \emph{spectral sparsifiers}. An efficient randomized algorithm to construct an $(1+\eps)$-spectral sparsifier with $\OO(n\log n /\eps^2)$ edges was given in~\cite{graph:sparsifiers:eff_resistance}. Furthermore, an $(1+\eps)$-spectral sparsifier with $\OO(n/\eps^2)$ edges can be computed in $\OO(mn^3/\eps^2)$ deterministic time~\cite{graph:sparsifiers:twice_ram}. Here we present an efficient deterministic algorithm for spectrally sparsifying dense graphs. The main contribution here is the following: given a weighted dense graph $H=(V,E)$ on $n$ vertices with positive weights and $0< \eps <1$, there is a deterministic algorithm that returns an $(1+\eps)$-spectral sparsifier with $\OO(n/ \eps^2)$ edges in $\widetilde{\OO}(n^4 \log n /\eps^2$ $ \max\{ \log^2 n, 1/\eps^2 \})$ time.
%

%
Third, we present a randomized distributed algorithm for solving Laplacian linear systems with exponential convergence in expectation under the gossip model of computation. Gossip algorithms for distributed computation are based on a gossip or rumor-spreading type of asynchronous message exchange protocol. The analysis of the proposed algorithm relies on the advances in randomized iterative least squares solvers including the randomized extended Kaczmarz method discussed in Chapter~\ref{chap:ma}.
%
%
\begin{center}
This chapter is based on joint work with Freris~\cite{CDC12} and on the work of~\cite{ICALP12}.
\end{center}
%%%%%%%%%%%%%%%%%%%%%%%%%%%%%%%%%%%%%%%%%%%%%%%%%%%
%\clearpage
%%%%%%%%%%%%%%%%%%%%%%%%%%%%%%%%%%%%%%%%%%%%%%%%%%%
\section{Preliminaries}
%%%%%%%%%%%%%%%%%%%%%%%%%%%%%%%%%%%%%%%%%%%%%%%%%%%
We now introduce the mathematical notation that will be used throughout the thesis and we also present several basic results from linear algebra and probability theory. In addition, we state known facts about uniform sampling rows from matrices with orthonormal columns and we present the matrix Bernstein inequality and Minsker's version of the matrix Bernstein inequality~\cite{minsker}. Finally, we present a matrix generalization of Spencer's hyperbolic cosine algorithm~\cite{hyperbolic_cosine:Spencer}
%
%
%%%%%%%%%%%%%%%%%%%%%%%%%%%%%%%%%%%%%%%%%%%%%%%%%%%
\paragraph{Basic Notation}
%%%%%%%%%%%%%%%%%%%%%%%%%%%%%%%%%%%%%%%%%%%%%%%%%%%
%
For an integer $m$, let $[m]:=\{1,\ldots,m\}$. We denote by $\RR, \ZZ$ and $\NN$ the reals, integers and natural numbers, respectively. For any positive number $x$, the base-$2$ logarithm and natural logarithm of $x$ is denoted by $\log(x)$ and $\ln (x)$, respectively. Occasionally, we might prefer to hide $\log\log(\cdot)$ factors under the big-Oh notation, we make this explicit by using $\widetilde{\OO}(\cdot)$. All matrices contain real entries. We use capital letters $\matA,\matB,\ldots$ to denote matrices and bold lower-case letters $\x,\y,\ldots$ to denote column vectors. We denote by $\zeromtx$, the all-zeroes matrix, $\J$ for the all-ones matrix and $\Id$ for the identity matrix and by $\e_1,\e_2,\ldots ,\e_n$ the standard basis vectors of $\reals^n$. Occasionally, we explicitly specify the dimensions of these matrices by adding a subscript, e.g., $\Id_n$ is the $n\times n $ identity matrix. $\Sym^{n\times n}$ denotes the set of symmetric matrices of size $n$. We denote the rows and columns of any $m\times n$ matrix $\matA$ by $\ar{1}, \ldots , \ar{m}$ and $\ac{1},\ldots , \ac{n}$, respectively. $\colspan{\matA}$ denotes the range (or column span) of $\matA$, i.e., $\colspan{\matA}:=\{\matA \x\ | \ \x\in\RR^n\}$ and $\colspan{\matA}^{\bot}$ denotes the orthogonal complement of $\colspan{\matA}$. Let $\x\in\reals^n$, we denote by $\diag{\x}$ the diagonal matrix containing $x_1,x_2,\ldots ,x_n$. For a square matrix $\matM$, we also write $\diag{\matM}$ to denote the diagonal matrix that contains the diagonal entries of $\matM$. Let $\x\in\reals^n$ and $\y\in\reals^n$ viewed as row or column vectors, then $\x\otimes \y$ is the $n\times n$ matrix such that $ (\x \otimes \y)_{i,j} =x_i y_j$. We denote the inner product between two vectors $\x$ and $\y$ of the same dimensions by $\ip{\x}{\y}:=\sum_{i} x_i y_i$. We denote by $[\matA ; \matB]$ the matrix obtained by concatenating the columns of $\matB$ next to the columns of $\matA$. Moreover, we denote by $\nnz{\cdot}$ the number of non-zero entries of its argument matrix.

%
%%%%%%%%%%%%%%%%%%%%%%%%%%%%%%%%%%%%%%%%%%%%%%%%%%%
\subsection{Linear Algebra}
%%%%%%%%%%%%%%%%%%%%%%%%%%%%%%%%%%%%%%%%%%%%%%%%%%%
%
The following discussion reviews several definitions and facts from linear algebra; for more details, see~\cite{book:matrix:Bhatia,book:GVL,book:matrix_analysis:HornJohnson}. Let $\matA$ be an $m\times n$ matrix of rank $r$. We denote $\norm{\matA}=\max \{ \norm{\matA \x}~|~\norm{\x} =1 \}$, $\infnorm{\matA} = \max_{i\in{[m]}} \sum_{j\in{[n]}} |\Ae{ij}|$ and by $\frobnorm{\matA}=\sqrt{\sum_{i,j}{\Ae{ij}^2}}$ the Frobenius norm of $\matA$. Also $\rank{\matA}$ and $\sr{\matA}:=\frobnorm{\matA}^2/\norm{\matA}^2$ is the rank and \emph{stable rank} of $\matA$. Observe that $\sr{\matA}\leq \rank{\matA}$. The trace of a square matrix $\matB$, i.e., the sum of its diagonal elements, is denoted as $\trace{\matB}$. A matrix $\matP$ of size $n$ is called projector if it is idempotent, i.e., $\matP^2 =\matP$.
%

% SVD, condition numbers etc...
A symmetric matrix $\matA$ is positive semi-definite (PSD), denoted by $\zeromtx \preceq \matA$, if $\x^\top \matA \x \geq 0$ for every vector $\x$. For two symmetric matrices $\matX,\matY$, we say that $\matY\succeq \matX$ if and only if $\matY-\matX$ is a positive semi-definite (PSD) matrix. Moreover, a symmetric matrix $\matA$ of size $n$ is called \emph{diagonally dominant} if $|\matA_{ii}| \geq \sum_{j\neq i} |\matA_{ij}|$ for every $i\in{[n]}$. Given any matrix $\matA$, its \emph{dilation} is defined as $\dil{\matA} = \left[\begin{matrix}
        \zeromtx      & \matA \\
	\matA^\top & \zeromtx
       \end{matrix}\right].
$
It is easy to verify that $\lambda_{\max}(\dil{\matA}) = \norm{\matA}$.
%

%
Let $\matA = \matU\matSig \matV^\top$ be the truncated singular value decomposition (SVD) of $\matA$ where $\matU\in\RR^{m\times r}$ with $\matU^\top \matU = \Id_r$, $\Sigma$ is the diagonal matrix of size $r$ containing the non-zero singular values $\sigma_1(\matA),\sigma_2(\matA),\ldots , \sigma_r(\matA)$ of $\matA$ in non-increasing order, and $\matV\in\RR^{n\times r}$ with $\matV^\top \matV = \Id_r$. The singular value decomposition can be computed using $\OO(mn\min (m,n))$ arithmetic operations. Whenever the matrix $\matA$ is clear from the context, we will refer to $\sigma_1(\matA)$ and $\sigma_{\rank{\matA}}(\matA)$ as $\sigma_{\max}$ and $\sigma_{\min}$, respectively. The Moore-Pensore pseudo-inverse of $\matA$ is denoted by $\pinv{\matA}:= \matV \matSig^{-1}\matU^\top$. Recall that $\norm{\pinv{\matA}} = 1/\sigma_{\min}$.
%We denote the inner product between two row or column vectors $\x$ and $\y$ of the same dimensions by $\ip{\x}{\y}:=\sum_{i} x_i y_i$.
For any non-zero real matrix $\matA$, we define
%%%%%%%%%%%%%%%%%%%%%%%%%%%%%%%%%%%%%%%%%%%%%%%%%%%%%%%
\begin{equation}\label{eq:kappa}
\kappaFS(\matA) := \frobnorm{\matA}^2 \norm{\pinv{\matA}}^2.
\end{equation}
%%%%%%%%%%%%%%%%%%%%%%%%%%%%%%%%%%%%%%%%%%%%%%%%%%%%%%%
Related to this is the scaled square condition number introduced by Demmel in~\cite{cond:Demmel}, see also~\cite{RK}. It is easy to check that the above parameter $\kappaFS(\matA)$ is related with the condition number of $\matA$, $\cond{\matA}:= \sigma_{\max} / \sigma_{\min}$, via the inequalities: $\cond{\matA}^2 \leq \kappaFS(\matA) \leq \rank{\matA} \cdot \cond{\matA^2}^2$. We denote by $\nnz{\cdot}$ the number of non-zero entries of its argument matrix. We define the \emph{average row sparsity} and \emph{average column sparsity} of $\matA$ by $\ravg$ and $\cavg$, respectively, as follows:
\[	\ravg := \sum_{i=1}^{m} q_i \nnz{\ar{i}}\quad\text{and}\quad \cavg := \sum_{j=1}^{n} p_{j} \nnz{\ac{j}}\]
where $p_j := \norm{\ac{j}}^2 / \frobnorm{\matA}^2$ for every $i\in{[n]}$ and $q_i := \norm{\ar{i}}^2 / \frobnorm{\matA}^2$ for every $i\in{[m]}$. The following fact will be used in Chapter~\ref{chap:ma}.
%
%%%%%%%%%%%%%%%%%%%%%%%%%%%%%%%%%%%%%%%%%%%%%%%%%%%%%%%
\begin{fact}\label{fact:xls}
Let $\matA$ be any non-zero real $m\times n$ matrix and $\b\in\RR^m$. Denote by $\xls:= \pinv{\matA}\b$. Then $\xls = \pinv{\matA}\br$.
\end{fact}
%%%%%%%%%%%%%%%%%%%%%%%%%%%%%%%%%%%%%%%%%%%%%%%%%%%%%%%
%
Finally, we frequently use the inequality $1-t\leq \exp(-t)$ for every $t\leq 1$.
%
%%%%%%%%%%%%%%%%%%%%%%%%%%%%%%%%%%%%%%%%%%%%%%%%%%%
%\clearpage
%%%%%%%%%%%%%%%%%%%%%%%%%%%%%%%%%%%%%%%%%%%%%%%%%%%
%\subsection*{Functions of Matrices}
\paragraph{Functions of Matrices.}
%%%%%%%%%%%%%%%%%%%%%%%%%%%%%%%%%%%%%%%%%%%%%%%%%%%
Here we review some basic facts about matrix functions including the matrix exponential and the matrix hyperbolic cosine function, for more details see~\cite{book:Higham:Matrix_fcn}. The matrix exponential of a symmetric matrix $\matA$ is defined as $\expm{\matA} = \Id + \sum_{k=1}^{\infty} \frac{\matA^k}{k!}$. Let $\matA=\matQ\Lambda \matQ^\top$ be the eigendecomposition of $\matA$. It is easy to see that $\expm{\matA} = \matQ \expm{\Lambda} \matQ^\top$. For any square matrices $\matA$ and $\matB$ of the same size that commute, i.e., $\matA\matB=\matB\matA$, we have that $ \expm{\matA+\matB} = \expm{\matA}\expm{\matB}$. In general, when $\matA$ and $\matB$ do not commute, the following estimate is known for symmetric matrices.
\begin{lemma}\cite{ineq:trace_exp:Golden,ineq:trace_exp:Thompson}\label{lem:ineq:golden_thompson}
For any symmetric matrices $\matA$ and $\matB$, $\trace{\expm{ \matA + \matB}} \leq \trace{\expm{\matA}\expm{\matB}}$.
\end{lemma}
The following fact about matrix exponential for rank one matrices will be also useful.
\begin{lemma}\label{lem:expm:outerprod}
	Let $\x$ be a non-zero vector in $\reals^n$. Then $ \expm{\x \otimes \x} = \Id_n + \frac{\e^{\norm{\x}^2} - 1}{\norm{\x}^2} \x\otimes \x$.
Similarly, $\expm{-\x \otimes \x} = \Id_n - \frac{1 - \e^{-\norm{\x}^2}}{\norm{\x}^2} \x \otimes \x$.
\end{lemma}
%%%%%%%%%%%%%%%%%%%%%%%%%%%%%%%%%%%%%%%%%%%%%%%%%%%
\begin{proof}
%(of Lemma~\ref{lem:expm:outerprod})
%%%%%%%%%%%%%%%%%%%%%%%%%%%%%%%%%%%%%%%%%%%%%%%%%%%
The proof is immediate by the definition of the matrix exponential. Notice that $(\x \otimes \x)^k = \norm{\x}^{2(k-1)} \x\otimes \x$ for $k\geq 1$, hence
\begin{eqnarray*}
		\expm{\x \otimes \x}  & = & \Id + \sum_{k=1}^{\infty}{ \frac{(\x \otimes \x)^k}{k!}} \  =\  \Id + \sum_{k=1}^{\infty}{ \frac{\norm{\x}^{2(k-1)} \x\otimes \x}{k!}}
				\ = \ \Id + \frac{\e^{\norm{\x}^2} - 1}{\norm{\x}^2} \x\otimes \x.
\end{eqnarray*}
Similar considerations give that $\expm{-\x\otimes \x } = \Id - \frac{1 - \e^{-\norm{\x}^2}}{\norm{\x}^2} \x \otimes \x$.
%%%%%%%%%%%%%%%%%%%%%%%%%%%%%%%%%%%%%%%%%%%%%%%%%%%
\end{proof}
%%%%%%%%%%%%%%%%%%%%%%%%%%%%%%%%%%%%%%%%%%%%%%%%%%%
%

%
%
Let us define the \emph{matrix hyperbolic cosine} function of a symmetric matrix $\matA$ as $ \coshm{\matA} := (\expm{\matA} + \expm{-\matA}) /2$. Next, we state a few properties of the matrix hyperbolic cosine.
\begin{lemma}\label{lem:dil_vs_expm}
Let $\matA$ be a symmetric matrix of size $n$. Then $\trace{\expm{\dil{\matA}}} = 2 \trace{ \coshm{\matA} }$.
\end{lemma}
%%%%%%%%%%%%%%%%%%%%%%%%%%%%%%%%%%%%%%%%%%%%%%%%%%%
\begin{proof}
%(of Lemma~\ref{lem:dil_vs_expm})
%%%%%%%%%%%%%%%%%%%%%%%%%%%%%%%%%%%%%%%%%%%%%%%%%%%
Set $\matB :=\dil{\matA} = \left[ \begin{matrix} \zeromtx & \matA \\
 \matA^\top & \zeromtx
\end{matrix}\right]$. Notice that for any integer $k\geq 1$, $\matB^{2k} = \left[ \begin{matrix}
 \matA^{2k} & \zeromtx \\
 \zeromtx & \matA^{2k}
\end{matrix}\right]$ and $
\matB^{2k+1} = \left[ \begin{matrix}
 \zeromtx & \matA^{2k+1} \\
 \matA^{2k+1} & \zeromtx
\end{matrix}\right]$. Since the odd powers of $\matB$ are trace-less, it follows that
\begin{eqnarray*}
      \trace{ \expm{ \matB }} &  =  &  \trace{ \Id_{2n}  + \sum_{k=1}^{\infty}  \frac{\matB^{2k}}{(2k ) !}    + \sum_{k=0}^{\infty}  \frac{\matB^{2k+1}}{(2k + 1 ) !} }
			  \  =  \  \trace{ \Id_{2n}  + \sum_{k=1}^{\infty}  \frac{\matB^{2k}}{(2k ) !} } \\
			  &  =  &  2 \trace{ \Id_{n}  + \sum_{k=1}^{\infty}  \frac{\matA^{2k}}{(2k ) !} }
			  \  =  \  \trace{ \expm{\matA} + \expm{- \matA}  }
			  \  =  \  2\trace{ \coshm{\matA}}.
\end{eqnarray*}
%%%%%%%%%%%%%%%%%%%%%%%%%%%%%%%%%%%%%%%%%%%%%%%%%%%
\end{proof}
%%%%%%%%%%%%%%%%%%%%%%%%%%%%%%%%%%%%%%%%%%%%%%%%%%%
%
\begin{lemma}\label{lem:coshm_with_proj}
Let $\matA$ be a symmetric matrix and $\matP$ be a projector matrix that commutes with $\matA$, i.e., $\matP\matA=\matA\matP$. Then $\coshm{\matP\matA} = \matP\coshm{\matA} + \Id - \matP$.
\end{lemma}
%%%%%%%%%%%%%%%%%%%%%%%%%%%%%%%%%%%%%%%%%%%%%%%%%%%
\begin{proof}
%(of Lemma~\ref{lem:coshm_with_proj})
%%%%%%%%%%%%%%%%%%%%%%%%%%%%%%%%%%%%%%%%%%%%%%%%%%%
By the definition of $\coshm{\cdot}$, it suffices to show that $\expm{\matP\matA}=\matP\expm{\matA}+ \Id - \matP$,
\begin{eqnarray*}
	\expm{\matP\matA}= \Id + \sum_{k=1}^{\infty} \frac{(\matP\matA)^k}{k!} = \Id + \matP\sum_{k=1}^{\infty} \frac{\matA^k}{k!} = \matP\expm{\matA}+ \Id - \matP.
\end{eqnarray*}
%%%%%%%%%%%%%%%%%%%%%%%%%%%%%%%%%%%%%%%%%%%%%%%%%%%
\end{proof}
%%%%%%%%%%%%%%%%%%%%%%%%%%%%%%%%%%%%%%%%%%%%%%%%%%%
%%%%%%%%%%%%%%%%%%%%%%%%%%%%%%%%%%%%%%%%%%%%%%%%%%%
\begin{lemma}\label{lem:trace:incr_psd}
For any positive semi-definite symmetric matrix $\matA$ of size $n$ and any two symmetric matrices $\matB,\matC$ of size $n$, $\matB\preceq \matC$ implies $\trace{\matA\matB} \leq \trace{\matA\matC}$.
\end{lemma}
\begin{proof}
Conjugating by $\matA^{1/2}$ the PSD inequality $\matB \preceq \matC$ (Lemma~\ref{lem:pert3}), it follows that $\matA^{1/2} \matB \matA^{1/2} \preceq \matA^{1/2} \matC \matA^{1/2}$. Taking the trace operator over both sides implies that $\trace{\matA^{1/2} \matB \matA^{1/2}}  \leq \trace{\matA^{1/2} \matC \matA^{1/2}}$. To conclude use the cyclic property of the trace on both sides, i.e., $\trace{\matA^{1/2} \matB \matA^{1/2}} = \trace{\matA^{1/2} \matA^{1/2} \matB } = \trace{\matA \matB}$.
\end{proof}
%
%%%%%%%%%%%%%%%%%%%%%%%%%%%%%%%%%%%%%%%%%%%%%%%%%%%
%\clearpage
%%%%%%%%%%%%%%%%%%%%%%%%%%%%%%%%%%%%%%%%%%%%%%%%%%%
%%%%%%%%%%%%%%%%%%%%%%%%%%%%%%%%%%%%%%%%%%%%%%%%%%%

%%%%%%%%%%%%%%%%%%%%%%%%%%%%%%%%%%%%%%%%%%%%%%%%%%%
%\clearpage
%%%%%%%%%%%%%%%%%%%%%%%%%%%%%%%%%%%%%%%%%%%%%%%%%%%
%\subsection*{Functions of Matrices}
\paragraph{Matrix Perturbation.}
%%%%%%%%%%%%%%%%%%%%%%%%%%%%%%%%%%%%%%%%%%%%%%%%%%%
%
The next discussion reviews a few results from matrix perturbation theory; for more details, see~\cite{book:perturbation:stewart,book:GVL,book:matrix:Bhatia}. The next lemma states that if a symmetric positive semi-definite matrix $\widetilde{\matC}$ approximates the Rayleigh quotient of a symmetric positive semi-definite matrix $\matC$, then the eigenvalues of $\widetilde{\matC}$ also approximate the eigenvalues of $\matC$.
\begin{lemma}\label{lem:rayleight_to_eig}
Let $0<\eps<1$. Assume $\matC$, $\widetilde{\matC}$ are $n\times n$ symmetric positive semi-definite matrices, such that the following inequality holds $ (1-\eps)\x^\top \matC \x \leq \x^\top \widetilde{\matC} \x \leq (1+\eps) \x^\top \matC \x$ for every $\x\in{\RR^n}$. Then, for $i=1,\dots, n$ the eigenvalues of $\matC$ and $\widetilde{\matC}$ are the same up-to an error factor $\eps$, i.e.,
%
\[(1-\eps)\lambda_i(\matC) \leq \lambda_i(\widetilde{\matC}) \leq (1+\eps) \lambda_i(\matC).\]
%
\end{lemma}
\begin{proof}
The proof is an immediate consequence of the Courant-Fischer's characterization of the eigenvalues. First notice that by hypothesis, $\matC$ and $\widetilde{\matC}$ have
the same null space. Hence we can assume without loss of generality, that $\lambda_i(\matC), \lambda_i(\widetilde{\matC}) > 0$ for all $i=1,\dots, n$. Let $\lambda_i(\matC)$ and $\lambda_i(\widetilde{\matC})$ be the eigenvalues (in non-decreasing order) of $\matC$ and $\widetilde{\matC}$, respectively. The Courant-Fischer min-max theorem~\cite[p.~394]{book:GVL} expresses the eigenvalues as
%%%%%%%%%%%%%%%%
\begin{equation}\label{eqn:Courant_Fischer}
\lambda_i(\matC) = \min_{\mathcal{S}^i}\max_{\x\in{S^i} } \frac{\x^\top \matC \x}{\x^\top \x},
\end{equation}
%%%%%%%%%%%%%%%%
where the minimum is over all $i$-dimensional subspaces $\mathcal{S}^i$. Let the subspaces $\mathcal{S}^{i}_0$ and $\mathcal{S}^i_1$ where the minimum is achieved for the eigenvalues of $\matC$ and $\widetilde{\matC}$, respectively. Then, it follows that
%
\[
\lambda_i (\widetilde{\matC}) = \min_{\mathcal{S}^i}\max_{\x\in{\mathcal{S}^i} } \frac{\x^\top \widetilde{\matC} \x}{\x^\top \x}\leq \max_{\x\in{S^i_0} } \frac{\x^\top \widetilde{\matC} \x}{\x^\top \matC \x} \frac{\x^\top \matC \x}{\x^\top \x} \leq (1+\eps)\lambda_i (\matC).
\]
and similarly,
\[
\lambda_i (\matC) = \min_{\mathcal{S}^i}\max_{\x\in{\mathcal{S}^i} } \frac{\x^\top \matC \x}{\x^\top \x}\leq \max_{\x\in{S^i_1} } \frac{\x^\top \matC \x}{\x^\top \widetilde{\matC} \x} \frac{\x^\top \widetilde{\matC} \x}{\x^\top \x} \leq \frac{\lambda_i (\widetilde{\matC})}{1-\eps}.
\]
Therefore, it follows that for $i=1,\dots , n$: $(1- \eps) \lambda_i(\matC) \leq \lambda_i(\widetilde{\matC}) \leq (1+\eps) \lambda_i(\matC)$.
\end{proof}
%
We now state two known matrix perturbation results and a simple but useful property of the psd ordering.
\begin{lemma}[Theorem 3.3 in~\cite{EI95}]\label{lem:pert1}
Let $\matPsi \in \RR^{p \times q}$ and $ \matPhi =  \matD_L \matPsi \matD_R $ with $\matD_L \in \RR^{p \times p}$ and $\matD_R \in \RR^{q \times q}$ being non-singular matrices.
Let $\gamma = \max\{  \norm{ \matD_L \matD_L^\top - \matI_p },  \norm{ \matD_R^\top \matD_R - \matI_q } \} $. Then, for all $i=1,\ldots,\rank{\matPsi}$:
$  |  \sigma_i\left( \matPhi \right) -  \sigma_i\left( \matPsi \right)   |  \le \gamma \cdot  \sigma_i\left( \matPsi \right). $
\end{lemma}

\begin{lemma}[Weyl's inequality for singular values; Corollary 7.3.8 in~\cite{book:matrix_analysis:HornJohnson}]\label{lem:pert2}
Let $\matPhi, \matPsi \in \RR^{m \times n}$. Then, for all $i=1,\ldots, \min( m,n):$
$| \sigma_i\left(\matPhi\right)- \sigma_i\left(\matPsi\right)  |  \le \norm{\matPhi - \matPsi} $.
\end{lemma}
%

%
\begin{lemma}[Conjugating the PSD ordering; Observation 7.7.2 in~\cite{book:matrix_analysis:HornJohnson}]\label{lem:pert3}
Let $\matPhi, \matPsi \in \RR^{n \times n}$ are symmetric matrices with $\matPhi \preceq \matPsi$. Then, for every $n \times m$ matrix $\matZ:$
$\matZ^\top \matPhi \matZ \preceq \matZ^\top \matPsi \matZ$.
\end{lemma}
%
%%%%%%%%%%%%%%%%%%%%%%%%%%%%%%%%%%%%%%%%%%%%%%%%%%%
\subsection{Probabilistic Tools}
%%%%%%%%%%%%%%%%%%%%%%%%%%%%%%%%%%%%%%%%%%%%%%%%%%%
%
We abbreviate the terms ``independently and identically distributed'' and ``almost surely'' with i.i.d. and a.s., respectively.

%
The first tool is the so-called subspace Johnson-Lindenstrauss lemma. Such a result was obtained in \cite{sarlos} (see also~\cite[Theorem~1.3]{jl:manifold}) although it appears implicitly in results extending the original Johnson Lindenstrauss lemma~\cite{JL84} (see~\cite{magen07}). The techniques for proving such a result with possible worse bound are not new and can be traced back even to Milman's proof of Dvoretsky theorem~\cite{Dvoretsky:Milman}.
\begin{lemma}\label{lem:jl_subspace} (Subspace JL lemma \cite{sarlos})
Let $\mathcal{W} \subseteq \RR^d$ be a linear subspace of dimension $k$ and
$\eps\in{(0, 1/3)}$. Let $\matR$ be a $t\times d$ random sign matrix rescaled by $1/\sqrt{t}$, namely $R_{ij} = \pm 1/\sqrt{t}$ with equal probability.
Then
\begin{align}\label{eq:jl_subspace}
\Prob{ (1-\eps) \norm{\w}^2 \leq \norm{\matR\w}^2 \leq (1+\eps)\norm{\w}^2,\ \forall\ \w\in\mathcal{W} } \geq 1 - c_2^k \cdot \exp (- c_1 \eps^2 t),
\end{align}
where $c_1 = \frac1{16 \cdot 36},c_2 = 18$.
\end{lemma}
\begin{proof}
The  statement in our lemma has been proved in Corollary $11$ of~\cite{sarlos}, see also~\cite[Theorem~$1.3$]{jl:manifold} for a restatement. More precisely, repeat the proof of Corollary $11$ of~\cite{sarlos} paying attention to the constants. That is, set $\matC=\matW^\top \matR^\top \matR \matW -\matI_k$ where the column span of $\matW$ equals to $\mathcal{W}$ and $\eps_0=1/2$ in Lemma~$10$ of~\cite{sarlos}. Then, apply the JL transform~\cite{Ach03} with (rescaled) accuracy $\eps/4$ on each vector of the set $T':=\{\matW^\top \vct{x}\ |\ \vct{x} \in T \}$ where $T$ is from Lemma $10$ of~\cite{sarlos}, hence $|T'| \leq e^{k\ln (18)}$. So,
	%
\begin{equation}\label{ineq:sigma_preserved}
	\Prob{\left(  \forall i = 1,\dots ,k :\  1-\eps \leq \sigma_i(\matW^\top \matR^\top \matR \matW) \leq 1+\eps \right) } \geq 1 - e^{k\ln (18) } e^{-\eps^2 r/(36 \cdot 16) } .
\end{equation}
%
Therefore, whenever the event of Ineq.~\eqref{ineq:sigma_preserved} holds, it implies that $\norm{\matC} = \norm{\matW^\top \matR^\top \matR\matW - \Id_k} \leq \eps$, which is equivalent to the statement of Ineq.~\eqref{eq:jl_subspace}.
\end{proof}
%
%
%
%
Next we present a standard lemma that bounds the spectral norm of any matrix $\matA$ when it's multiplied by a random sign matrix that is rescaled by $1/\sqrt{t}$ presented to us by Mark Rudelson (personal communication, 2010). If random Gaussian matrices are used in the following lemma, then it is a direct consequence of Gaussian measure concentration for Lipschitz functions. The use of random sign matrices makes the argument a bit more involved, but such arguments are standard in the literature.
\begin{lemma}\label{lem:Rudelson}
Let $\matA$ be an $m\times n$ real matrix and fix $t\geq 1$. Let $\matR$ be a $t\times n$ random sign matrix rescaled by $1/\sqrt{t}$. For every $\tau>0$
\begin{equation}
 \Prob{ \norm{ \matA\matR^\top } \geq 2\frobnorm{\matA} / \sqrt{t} + 2 \norm{\matA} + \norm{\matA} \tau / \sqrt{t} }\ \leq\ e^{-\tau^2/8}.
\end{equation}
\end{lemma}
\begin{proof}
Let $\matG$ be a $t\times n$ matrix whose entries are independent Gaussian random variables. Then by the Gordon-Chev\`{e}t inequality\footnote{For example, set $\matS=\Id_t, \matT=\matA$ in \cite[Proposition~$10.1$,~p.~$54$]{HMT}.}
\begin{align*}
 	\EE\norm{\matA\matG^\top} 	& \leq  \norm{\Id_t}\frobnorm{\matA} + \frobnorm{\Id_t}\norm{\matA} =   \frobnorm{\matA} + \norm{\matA}\sqrt{t}.
\end{align*}

Let $\odot$ denote the entrywise product\footnote{For any two matrices $\matB$ and $\matC$ of the same size $(\matB \odot \matC)_{ij} = \matB_{ij}\matC_{ij} $} (also known as Hadamard product) between two matrices of the same size. Write $\matG$ as $\matE \odot |\matG|$ where $\matE_{ij} = \sign{\matG_{ij}}$ and the $(i,j)$ entry of $|\matG|$ equals $ |\matG_{ij}|$. Note that $\matE$ and $|\matG|$ are independent and $\matE $ is a random sign matrix. It follows that
\[\EE[ \matE\odot |\matG| \ |\ \matE] = \sqrt{\frac{2}{\pi}} \matE\]
since $\EE |g| = \sqrt{2/\pi}$ for a Gaussian random variable $g$. Multiply the above from the right with $\matA^\top$ and take norms on both sides
\[\norm{ \EE\left[ (\matE\odot |\matG|) \matA^\top \ |\ \matE\right]  }= \norm{\sqrt{\frac{2}{\pi}} \matE \matA^\top}.\]
By Jensen's inequality, it follows that $\norm{\sqrt{\frac{2}{\pi}} \matE \matA^\top} \leq \EE \left[ \norm{(\matE\odot |\matG|) \matA^\top } \ |\ \matE \right] $. Taking expectation with respect to $\matE$, it follows that
\[ \sqrt{\frac{2}{\pi}} \EE \norm{ \matE \matA^\top } \leq \EE \norm{\matG \matA^\top }.\]
Since $\matE$ is a random sign matrix, it follows that
\[\EE \norm{\matA \matR^\top} \leq  \sqrt{\frac{\pi}{2t}} \EE \norm{\matG \matA^\top } \leq 2(\frobnorm{\matA} /\sqrt{t} + \norm{\matA}).\]
%
%
%
Define the function $f:[-1,1]^{t\times n} \to \RR$ by $f(\matS) = \norm{\frac1{\sqrt{t} } \matA\matS^\top}$. The calculation above shows that $\EE f(\matS)\leq 2(\frobnorm{\matA}/\sqrt{t} + \norm{\matA})$ where the expectation is over a uniformly at random element of $\{\pm 1\}^{t\times n}$. We bound the Lipschitz constant of $f$. Let $S_1, S_2 \in [-1,1]^{t\times n}$,
\[ |f(S_1) - f(S_2)| \leq \norm{ \frac1{\sqrt{t}} \matA (\matS_1^\top - \matS_2^\top )} \leq \frac1{\sqrt{t}} \norm{\matA} \frobnorm{\matS_1 - \matS_2}\]
where we used the triangle inequality and standard properties of matrix norms.

Since $f$ is convex and $(\norm{\matA}/\sqrt{t})$-Lipschitz as a function of the entries of $\matS$, Talagrand's measure concentration inequality for product measures~\cite[Equation~1.8]{talagrand} or~\cite[Theorem~5.9, p. 100]{book:Ledoux} yields that for every $\tau>0$
\begin{equation*}
\Prob{ \norm{ \matA\matR^\top } \geq \EE \norm{\matA \matR^\top} + \frac{\norm{\matA}\tau}{\sqrt{t}} } \leq \exp (-\tau^2 /8).
\end{equation*}
It follows that for every $\tau>0$
\[\Prob{\norm{ \matA\matR^\top } \geq  2(\frobnorm{\matA}/\sqrt{t} + \norm{\matA}) + \norm{\matA}\tau / \sqrt{t}} \leq \exp (-\tau^2 /8).\]
\end{proof}
%
%%%%%%%%%%%%%%%%%%%%%%%%%%%%%%%%%%%%%%%%%%%%%%%%%%%
\subsection{Matrix Coherence and Sampling from an Orthonormal Matrix}
%%%%%%%%%%%%%%%%%%%%%%%%%%%%%%%%%%%%%%%%%%%%%%%%%%%
%
%
Given a matrix $\matA$ with $m$ rows, the \emph{coherence} of $\matA$ is defined as $\mu(\matA)=\max_{i\in{[m]}}\norm{ \e_i^\top \matU_{\matA}  }$, where $\e_i$ is the $i$-th standard basis (column) vector of $\RR^m$. Coherence is an important quantity in the analysis of randomized matrix algorithms. Note that the coherence of $\matA$ is a property of the column space of $\matA$, and does not depend on the actual choice of $\matA$. Therefore, if $\colspan{\matA} = \colspan{\matB}$ then $\mu(\matA) = \mu(\matB)$. Furthermore, it is easy to verify that if $\colspan{\matA} \subseteq \colspan{\matB}$ then $\mu(\matA) \leq \mu(\matB)$. Finally, we mention that for every matrix $\matA$ with $m$ rows: $\rank{\matA}/m \leq \mu(\matA) \leq 1$.
%

%
In this thesis, we quite often focus on tall-and-thin matrices, i.e. matrices with (much) more rows than columns. The following lemma formalizes that coherence affects the amount of rows needed to be uniformly sampled from a matrix with orthonormal columns so that the resulting sampled matrix remains close to orthonormal (i.e., its singular values are close to one). We need to setup the following notation before stating the next lemma. Given a subset of indices $T \subseteq [m]$, the corresponding sampling matrix $\matS$ is the $|T|\times m$ matrix obtained by discarding from $\matI_{m}$ the rows whose index is not in $T$. Note that $\matS \matA$ is the matrix obtained by keeping only the rows in $\matA$ whose index {\em appears} in $T$.
%
%%%%%%%%%%%%%%%%%%%%%%%%%%%%%%%%%%%%%%%%%%%%%%%%%%%
\begin{lemma}[Sampling from Orthonormal Matrix, Corollary to Lemma~3.4 in~\cite{Tro11}]
\label{lemma:sampling-ortho}
Let $\matQ \in \RR^{m \times d}$ have orthonormal columns. Let $0 < \eps < 1$ and $0 < \delta < 1$. Let $r$ be an integer such that
\[
6 \eps^{-2} m  \mu(\matQ) \log (3d/\delta) \leq r \leq m \,.
\]
Let $T$ be a random subset of $[m]$ of cardinality $r$, drawn from a uniform distribution over such subsets, and let $\matS$ be the $|T|\times m$ sampling matrix corresponding to $T$ rescaled by $\sqrt{m/r}$. Then, with probability of at least $1-\delta$, for $i\in[d]$: $  \sqrt{1-\eps} \le \sigma_i(\matS \matQ) \le  \sqrt{1+\eps}$.
\end{lemma}
%%%%%%%%%%%%%%%%%%%%%%%%%%%%%%%%%%%%%%%%%%%%%%%%%%%
%%%%%%%%%%%%%%%%%%%%%%%%%%%%%%%%%%%%%%%%%%%%%%%%%%%
\begin{proof}
Apply Lemma 3.4 in~\cite{Tro11} with the following choice of parameters:
$\ell = \alpha M \log(k/\delta),$
$\alpha = 6/\eps^2,$ and
$\delta_{tropp} = \eta = \eps$.
Here, $\ell$, $\alpha$, $M$, $k$, $\eta$ are the parameters of  Lemma 3.4 in~\cite{Tro11};
also $\delta_{tropp}$ plays the role of $\delta$, an error parameter, of  Lemma 3.4 in~\cite{Tro11}.
$\eps$ and $\delta$ are from our Lemma.
\end{proof}
%%%%%%%%%%%%%%%%%%%%%%%%%%%%%%%%%%%%%%%%%%%%%%%%%%%
In the above lemma, $T$ is obtained by sampling coordinates from $[m]$ \emph{without} replacement. Similar results can be shown for sampling with replacement, or using Bernoulli variables~\cite{LS:Ipsen12}.
%%%%%%%%%%%%%%%%%%%%%%%%%%%%%%%%%%%%%%%%%%%%%%%%%%%
%%%%%%%%%%%%%%%%%%%%%%%%%%%%%%%%%%%%%%%%%%%%%%%%%%%
\subsection{Randomized Walsh-Hadamard Transform}\label{sec:wht}
%%%%%%%%%%%%%%%%%%%%%%%%%%%%%%%%%%%%%%%%%%%%%%%%%%%
%%%%%%%%%%%%%%%%%%%%%%%%%%%%%%%%%%%%%%%%%%%%%%%%%%%
Matrices with high coherence pose a problem for algorithms based on uniform row sampling. One way to circumvent this problem is to use a coherence-reducing transformation.
%The crucial observation is that most problems can be safely transformed using unitary matrix. This is also true for our problem: $\sigma_i(\matQ \matA, \matQ \matB) = \sigma_i(\matA, \matB)$ if $\matQ$ is unitary. If the unitary matrix is chosen correctly, it can reduce coherence.
One popular coherence-reducing transformation is the Randomized Walsh-Hadamard Transform (RHT) matrix introduced in the paper of Ailon and Chazelle~\cite{FJlT}. We start with the definition of the deterministic Walsh-Hadamard Transform matrix. Fix an integer $m = 2^h$, for $h = 1,2,3, \ldots$. The (non-normalized) $m \times m$ matrix of the Walsh-Hadamard Transform (WHT) is defined recursively as,
%
\[\matH_m = \left[
\begin{array}{cc}
  \matH_{m/2} &  \matH_{m/2} \\
  \matH_{m/2} & -\matH_{m/2}
\end{array}\right],
%
\qquad \mbox{with} \qquad
%
\matH_2 = \left[
\begin{array}{cc}
  +1 & +1 \\
  +1 & -1
\end{array}\right].
\]
%
The $m \times m$ normalized matrix of the Walsh-Hadamard transform is $\matH = m^{-\frac{1}{2}} \matH_m$. The recursive nature of the WHT allows us to compute $\matH \matX$ for an $m \times n$ matrix $\matX$ in time $O(m n \log(m))$.
However, in our case we are interested in $\matS \matH \matX$ where $\matS$ is a $r$-row sampling matrix. To compute $\matS \matH \matX$ only $\OO(m n \log(r))$ operations
suffice (Theorem 2.1 in~\cite{AL09}).
\begin{definition}[Randomized Walsh-Hadamard Transform (RHT)]
\label{def:rht}
Let $m = 2^h$ for some positive integer $h$. A \emph{Randomized Walsh-Hadamard Transform (RHT)} is an $m \times m$ matrix of the form $\matTh =  \matH  \matD $
where $\matD$ is a random diagonal matrix of size $m$ whose entries are independent random signs, and $\matH$ is a normalized Walsh-Hadamard matrix of size $m$.
\end{definition}
The following lemma demonstrate that the application of $\matTh$ reduces the coherence of any fixed matrix.
\begin{lemma} [RHT bounds Coherence, Lemma 3.3 in~\cite{Tro11}]
\label{lem:rht-reduce}
Let $\matA$ be an $m\times n$  ($m \ge n$, $m=2^h$ for some positive integer $h$) matrix, and let $\matTh$ be an RHT. Then, with probability of at least $1-\delta$,
\[\mu(\matTh \matA) \leq \frac1{m}\left( \sqrt{n} + \sqrt{8 \log(m / \delta)} \right)^2\,.\]
\end{lemma}
%} % END OF IGNORE
%%%%%%%%%%%%%%%%%%%%%%%%%%%%%%%%%%%%%%%%%%%%%%%%%%%
%\clearpage
%%%%%%%%%%%%%%%%%%%%%%%%%%%%%%%%%%%%%%%%%%%%%%%%%%%
%%%%%%%%%%%%%%%%%%%%%%%%%%%%%%%%%%%%%%%%%%%%%%%%%%%
%\clearpage
%%%%%%%%%%%%%%%%%%%%%%%%%%%%%%%%%%%%%%%%%%%%%%%%%%%
%\section{Pessimistic Estimators for the Matrix Bernstein}\label{sec:derand_Bernstein}
\subsection{Matrix Concentration Inequalities}
%\label{sec:balancing}
%%%%%%%%%%%%%%%%%%%%%%%%%%%%%%%%%%%%%%%%%%%%%%%%%%%
%%%%%%%%%%%%%%%%%%%%%%%%%%%%%%%%%%%%%%%%%%%%%%%%%%%
%
%
In the present thesis, the analysis of several algorithms will rely on matrix probability inequalities for the sum of independent and identically distributed random matrices. Many classical probability inequalities (such as Chernoff-Hoeffding, Bernstein, Azuma, etc.) have been extended to the matrix setting~\cite{chernoff:matrix_valued:AW,chernoff:matrix_valued:Tropp}, see also~\cite{Tropp:nips} for a detailed overview. Here, we will use the matrix Bernstein inequality and Minsker's extension of the matrix Bernstein inequality~\cite{minsker}. The following version of the matrix Bernstein inequality (~\cite[Theorem~$3.2$]{recht:simple_completion}) slightly rephrased to better suit our notation will suffice for the purposes of this thesis, see also~\cite{chernoff:matrix_valued:Tropp} for improved bounds.
%
\begin{theorem}\label{thm:matrix_valued:Bernstein}
Let $\matM_1,\ldots ,\matM_t$ be i.i.d. copies of a random symmetric matrix $\matM$ of size $n$ such that $\EE{ \matM}=\zeromtx_n $, $\norm{\matM}\leq \gamma$ a.s. and $\norm{\EE \matM^2 } \leq \rho^2$. Then, for any $\eps > 0$,
%
\begin{equation}\label{ineq:matrix_Bernstein}
\Prob{ \norm{\frac{1}{t} \sum_{k=1}^{t} \matM_k } > \eps } \ \leq\ 2n \exp\left(-\frac{t\eps^2}{2\rho^2 + 2\gamma \eps/3}\right).
\end{equation}
%
\end{theorem}
%
%
Minsker proved an extension of the matrix Bernstein inequality that depends on a dimension parameter that may be smaller than the dimensions of the matrix samples~\cite{minsker} improving upon the work of~\cite{HKZ12}, see also~\cite{Oliv10} for a related bound. Here, we state the following version of Minsker's inequality which can be found in~\cite{Tropp:nips}.
\begin{theorem}\cite[Theorem~7.3.1 combined with Equation 7.3.2]{Tropp:nips}\label{thm:Minsker}
Let $\matX_1,\matX_2, \ldots ,\matX_t$ be a sequence of random symmetric matrices that satisfy
\[ \EE \matX_k = 0 \quad \text{and}\quad \lambda_{\max}(\matX_k) \leq \gamma\]
for every $k\in{[t]}$. Define $\matY = \sum_{k=1}^{t} \matX_k$. Define $d = d(\matY) = \trace{\EE( \matY^2) }/ \norm{\EE(\matY^2)}$ and $\sigma^2 = \sigma^2(\matY) = \norm{\EE(\matY^2)}$. Then, for $\tau > \sigma + \gamma /3$,
\begin{equation}\label{ineq:Minsker}
	\Prob{ \lambda_{\max}(\matY) \geq \tau } \leq 4 d \cdot\exp\left(-\frac{\tau^2/2}{\sigma^2 + \gamma \tau/3}\right).
\end{equation}
Moreover, suppose that $\EE(\matY^2) \preceq \matV$ for some positive semi-definite matrix $\matV$. Then for all $\tau> \norm{\matV}^{1/2} + \gamma /3$,
\begin{equation}\label{ineq:Minsker2}
	\Prob{ \lambda_{\max}(\matY) \geq \tau } \leq 4 \frac{\trace{\matV}}{\norm{\matV}} \cdot\exp\left(-\frac{\tau^2/2}{\norm{\matV} + \gamma \tau/3}\right).
\end{equation}
\end{theorem}
As was noticed in~\cite[p.78]{Tropp:nips}, the bound displayed in Inequality~\eqref{ineq:Minsker} may be not easy to apply because an estimate of the parameter $d$ may not be available or possible to obtain. However, the moreover part of the above theorem (Inequality~\eqref{ineq:Minsker2}) allows more flexibility whenever Theorem~\ref{thm:Minsker} is applied.
%%%%%%%%%%%%%%%%%%%%%%%%%%%%%%%%%%%%%%%%%%%%%%%%%%%
%\clearpage
%%%%%%%%%%%%%%%%%%%%%%%%%%%%%%%%%%%%%%%%%%%%%%%%%%%
%\section{Pessimistic Estimators for the Matrix Bernstein}\label{sec:derand_Bernstein}
\subsection{Balancing Matrices: a matrix hyperbolic cosine algorithm}\label{sec:balancing}
%%%%%%%%%%%%%%%%%%%%%%%%%%%%%%%%%%%%%%%%%%%%%%%%%%%
%%%%%%%%%%%%%%%%%%%%%%%%%%%%%%%%%%%%%%%%%%%%%%%%%%%
%
In many settings, it is desirable to convert the above matrix concentration inequalities into \emph{efficient} deterministic procedures; namely, to derandomize the proofs. Wigderson and Xiao presented an efficient derandomization of the matrix Chernoff bound by generalizing Raghavan's method of pessimistic estimators to the matrix-valued setting~\cite{chernoff:matrix_valued:derand:WX08}.
%

%
In this section, we present a generalization of Spencer's hyperbolic cosine algorithm to the matrix-valued setting~\cite{hyperbolic_cosine:Spencer} which corresponds to a derandomization of the matrix Bernstein inequality. In an earlier preliminary manuscript~\cite{matrix:hypercosine_zouzias}, the generalization of Spencer's hyperbolic cosine algorithm was also based on the method of pessimistic estimators as in~\cite{chernoff:matrix_valued:derand:WX08}. However, here we present a proof which is based on a simple averaging argument. We should highlight a few advantages of our result compared to a recent derandomization of the matrix Chernoff inequality~\cite{chernoff:matrix_valued:derand:WX08}. First, our construction does not rely on composing two separate estimators (or potential functions) to achieve operator norm bounds and second it does not require knowledge of the sampling probabilities of the matrix samples as in~\cite{chernoff:matrix_valued:derand:WX08}. In addition, the algorithm of~\cite{chernoff:matrix_valued:derand:WX08} requires computations of matrix expectations with matrix exponentials which are in many cases computationally expensive, see~\cite[Footnote~$6$, p. $63$]{chernoff:matrix_valued:derand:WX08}. Later in this thesis (Chapters~\ref{chap:ma} and~\ref{chap:graph}), we demonstrate that overcoming these limitations leads to faster and in some cases simpler algorithms.
%

%
We briefly describe Spencer's balancing vectors game and then generalize it to the matrix-valued setting~\cite[Lecture~$4$]{book:probmeth:Spencer}. Let a two-player perfect information game between Alice and Bob. The game consists of $n$ rounds. On the $i$-th round, Alice sends a vector $\v_i$ with $\infnorm{\v_i}\leq 1$ to Bob, and Bob has to decide on a sign $s_i\in{\{\pm 1\}}$ knowing only his previous choices of signs and $\{\v_{k}\}_{k < i}$.
At the end of the game, Bob pays Alice $\infnorm{\sum_{i=1}^{n} s_i \v_i}$. We call the latter quantity, the \emph{value} of the game.
%
%

%
%
It has been shown in~\cite{Spencer:balanc_vct} that, in the above limited online variant, Spencer's six standard deviations bound~\cite{sixDeviation:Spencer} does not hold and the best value that we can hope for is $\Omega(\sqrt{n \ln n})$. Such a bound is easy to obtain by picking the signs $\{s_i\}$ uniformly at random. Indeed, a direct application of Azuma's inequality to each coordinate of the random vector $\sum_{i=1}^{n} s_i \v_i$ together with a union bound over all the coordinates gives a bound of $\OO(\sqrt{n\ln n})$.
%

%
Now, we generalize the balancing vectors game to the matrix-valued setting. That is, Alice now sends to Bob a sequence $\{\matM_i\}$ of symmetric matrices of size $n$ with\footnote{A curious reader may ask him/her-self why the operator norm is the right choice. It turns out the the operator norm is the correct matrix-norm analog of the $\ell_\infty$ vector-norm, viewed as the \emph{infinity} Schatten norm on the space of matrices.} $\norm{\matM_i}\leq 1$, and Bob has to pick a sequence of signs $\{s_i\}$ so that, at the end of the game, the quantity $\norm{\sum_{i=1}^{n} s_i \matM_i} $ is as small as possible. Notice that the balancing vectors game is a restriction of the balancing matrices game in which Alice is allowed to send only diagonal matrices with entries bounded in absolute value by one. Similarly to the balancing vectors game, using matrix-valued concentration inequalities, one can prove that Bob has a randomized strategy that achieves at most $\OO(\sqrt{n\ln n})$ w.p. at least $1/n$. Indeed,
%
%
\begin{lemma}\label{lem:balanc_mtx}
Let $\matM_i \in \Sym^{n\times n}$, $\norm{\matM_i} \leq 1$, $1 \leq i \leq n$. Pick $s_i^*\in{\{\pm 1\} }$ uniformly at random for every $i\in{[n]}$. Then $\norm{\sum_{i=1}^{n} s_i^* \matM_i} = \OO(\sqrt{  n \ln n})$ w.p. at least $1/n$.
\end{lemma}
%%%%%%%%%%%%%%%%%%%%%%%%%%%%%%%%%%%%%%%%%%%%%%%%%%%
\begin{proof}
%(of Lemma~\ref{lem:balanc_mtx})
%%%%%%%%%%%%%%%%%%%%%%%%%%%%%%%%%%%%%%%%%%%%%%%%%%%
We wish to apply matrix Azuma's inequality, see~\cite[Theorem~$7.1$]{chernoff:matrix_valued:Tropp}. For every $j\in{[n]}$, define the matrix-valued difference sequence $f_j: [2] \to \Sym^{n\times n}$ as $f_j(k) = (2(k -1) -1 )M_j $ with $\norm{f_j(\cdot)} \leq 1$. Let $X$ be a uniform random variable over the set $\{1,2\}$. Then $\EE_X f_j(X)= \zeromtx_n$. Set $\eps = \sqrt{10\ln (2 n^2) / n}$. Matrix-valued Azuma's inequality tells us that w.p. at least $1/n$, a random set of signs $\{s_j\}_{j\in{[n]}}$ satisfies $\norm{\frac1{n} \sum_{j=1}^{n} s_j \matM_j } \leq \eps$. Rescale the last inequality to conclude.
\end{proof}
%%%%%%%%%%%%%%%%%%%%%%%%%%%%%%%%%%%%%%%%%%%%%%%%%%%
%
%%%%%%%%%%%%%%%%%%%%%%%%%%%%%%%%%%%%%%%%%%%%%%%%%%%
%
Now, let's assume that Bob wants to achieve the above probabilistic guarantees using a \emph{deterministic} strategy. Is it possible? We answer this question in the affirmative by generalizing Spencer's hyperbolic cosine algorithm (and its proof) to the matrix-valued setting. We call the resulting algorithm \emph{matrix hyperbolic cosine} (Algorithm~\ref{alg:matrix:hyperbolic}). It is clear that this simple greedy algorithm implies a deterministic strategy for Bob that achieves the probabilistic guarantees of Lemma~\ref{lem:balanc_mtx} (set $f_j\sim s_j \matM_j$, $t=n$ and $\eps = \OO(\sqrt{ \ln n / n})$ and notice that $\gamma,\rho^2$ are at most one).
%

%
Algorithm~\ref{alg:matrix:hyperbolic} requires an extra assumption on its random matrices compared to Spencer's original algorithm. That is, we assume that our random matrices have uniformly bounded their ``matrix variance'', denoted by $\rho^2$. This requirement is motivated by the fact that in the applications that are studied in this paper such an assumption translates bounds that depend quadratically on the matrix dimensions to bounds that depend linearly on the dimensions.
%

%
We will need the following technical lemma for proving the main result of this section, which is a Bernstein type argument generalized to the matrix-valued setting~ \cite{chernoff:matrix_valued:Tropp}.
%%%%%%%%%%%%%%%%%%%%%%%%%%%%%%%%%%%%%%%%%%%%%%%%%%%
\begin{lemma}\label{lem:bounding_w}
%%%%%%%%%%%%%%%%%%%%%%%%%%%%%%%%%%%%%%%%%%%%%%%%%%%
Let $f:[m] \to \Sym^{n\times n}$ with $\norm{f(i)} \leq \gamma$ for all $i\in{[m]}$. Let $X$ be a random variable over $[m]$ such that $\EE{f(X)}=\zeromtx$ and $\norm{\EE f(X)^2 } \leq \rho^2$. Then, for any $\theta >0$, $\norm{\EE [ \expm{ \dil{ \theta f(X)} }]}\ \leq\ \exp\left( \rho^2( \e^{\theta \gamma } -1 - \theta \gamma )/ \gamma^2\right).$ In particular, for any $0 < \eps < 1$, setting $\theta = \eps /\gamma$ implies that $\EE [ \expm{ \dil{ \eps f(X) / \gamma} }] \preceq \e^{\eps^2 \rho^2 / \gamma^2} \Id_{2n}$.
\end{lemma}
%%%%%%%%%%%%%%%%%%%%%%%%%%%%%%%%%%%%%%%%%%%%%%%%%%%
%
Now we are ready to prove the correctness of the matrix hyperbolic cosine algorithm.
%
%\vspace*{-4.0ex}
\begin{algorithm}{}
	\caption{Matrix Hyperbolic Cosine}\label{alg:matrix:hyperbolic}
\begin{algorithmic}[1]
\Procedure{Matrix-Hyperbolic}{$\{f_j\}$, $\eps$, $t$}\Comment{$f_j:[m] \to \Sym^{n\times n}$ as in Theorem~\ref{thm:hypercosine:main}, $0 < \eps < 1$.}
\State Set $\theta = \eps /\gamma $
\For {$i=1$ to $t$}
	\State Compute $x_i^*\in{[m]}$: $ x_i^* = \argmin_{k\in{[m]}}\trace{\coshm{ \theta \sum_{j=1}^{i-1} f_j(x_j^*) + \theta f_i(k) }} $
\EndFor
\State \textbf{Output:} $t$ indices $x_1^*, x_2^*, \ldots ,x_t^*$ such that $\norm{ \frac1{t} \sum_{j=1}^{t} f_j(x_j^*) } \leq \frac{\gamma \ln( 2n)}{t\eps } + \frac{\eps\rho^2}{\gamma} $
\EndProcedure
\end{algorithmic}
\end{algorithm}
%\vspace*{-4.0ex}
%%%%%%%%%%%%%%%%%%%%%%%%%%%%%%%%%%%%%%%%%%%%%%%%%%%
\begin{theorem}\label{thm:hypercosine:main}
%%%%%%%%%%%%%%%%%%%%%%%%%%%%%%%%%%%%%%%%%%%%%%%%%%%
Let $f_j:[m] \to \Sym^{n\times n}$ with $\norm{f_j(i)} \leq \gamma$ for all $i\in{[m]}$ and $j=1,2,\ldots$. Suppose that there exists independent random variables $X_1,X_2,\ldots $ over $[m]$ such that $\EE{f_j(X_j)}=\zeromtx$ and $\norm{\EE f_j(X_j)^2 } \leq \rho^2$. Algorithm~\ref{alg:matrix:hyperbolic} with input $\{f_j\},\eps, t$ outputs a set of indices $\{x_j^*\}_{j\in{[t]}}$ over $[m]$ such that $ \norm{ \frac1{t}\sum_{j=1}^{t} f_j(x_j^*)} \leq \frac{ \gamma \ln (2n)}{t\eps} +  \frac{\eps \rho^2}{\gamma}.$
%%%%%%%%%%%%%%%%%%%%%%%%%%%%%%%%%%%%%%%%%%%%%%%%%%%
\end{theorem}
%%%%%%%%%%%%%%%%%%%%%%%%%%%%%%%%%%%%%%%%%%%%%%%%%%%
%%%%%%%%%%%%%%%%%%%%%%%%%%%%%%%%%%%%%%%%%%%%%%%%%%%
\begin{proof}
%(of Theorem~\ref{thm:hypercosine:main})
%%%%%%%%%%%%%%%%%%%%%%%%%%%%%%%%%%%%%%%%%%%%%%%%%%%
%%%%%%%%%%%%%%%%%%%%%%%%%%%%%%%%%%%%%%%%%%%%%%%%%%%
Using the notation of Algorithm~\ref{alg:matrix:hyperbolic}, for every $i=1,2,\ldots , t$, define recursively $\matW(i) := \theta \sum_{j=1}^{i} f_j(x_j^*)$ and the potential function $\Phi^{(i)} := 2\trace{\coshm{\matW(i)}}$. For all steps $i=1,2,\ldots , t$, we will prove that
\begin{align}\label{ineq:barrier_incr}
  \Phi^{(i)}   & \leq  \Phi^{(i-1)} \exp\left( \eps^2 \rho^2/\gamma^2  \right).
\end{align}
%
Assume that the algorithm has fixed the first $(i-1)$ indices $x_1^*,\ldots ,x_{(i-1)}^* $. An averaging argument applied on the expression of the argmin of Step $4$ gives that
%
%
\begin{align*}
\EE_{X_i}  2\trace{\coshm{ \theta  \matW(i - 1) + \theta  f_i(X_i)}} &   =  \EE_{X_i}  \trace{\expm{ \theta \dil{ \matW(i - 1)} + \theta \dil{ f_i(X_i)} }} \\
                                                                   & \leq    \trace{\expm{ \dil{\theta \matW(i - 1)}} \EE_{X_i} \expm{\dil{ \theta f_i(X_i)} }} \\
                                                                   & \leq   \trace{\expm{ \dil{\theta \matW(i - 1)}}  \Id_{2n}} \exp\left( \eps^2 \rho^2 / \gamma^2 \right) \\
                                                                   &   =    \Phi^{(i-1)} \exp\left( \eps^2 \rho^2 / \gamma^2 \right)
\end{align*}
%
where in the first inequality we used Lemma~\ref{lem:dil_vs_expm} and linearity of dilation, in the second inequality we used the Golden-Thompson inequality (Lemma~\ref{lem:ineq:golden_thompson}) and linearity of trace, in the third inequality we used Lemma~\ref{lem:trace:incr_psd} together with Lemma~\ref{lem:bounding_w} and in the last equality we used again Lemma~\ref{lem:dil_vs_expm}. Since the algorithm seeks the minimum of the expression in Step $4$, it follows that
\[\Phi^{(i)} \leq \EE_{X_i}  2\trace{\expm{ \theta \dil{ \matW(i-1)} + \theta \dil{ f_i(X_i)} }}\]
which proves Ineq.~\eqref{ineq:barrier_incr}. Apply $t$ times Ineq.~\eqref{ineq:barrier_incr} to conclude that $\Phi^{(t)} \leq \Phi^{(0)} \exp\left( t\frac{\eps^2 \rho^2}{ \gamma^2} \right).$
%
Recall that $\Phi^{(0)} = 2\trace{\coshm{\zeromtx_n}} = 2\trace{\Id_n }  = 2n$. On the other hand, we can lower bound $\Phi^{(t)}$
%
\[\Phi^{(t)} = 2\trace{\coshm{\theta \sum_{j=1}^{t}  f_j(x_j^*)} }  \geq \exp\left(\norm{ \theta \sum_{j=1}^{t}  f_j(x_j^*)  }\right). \]
%
The last inequality follows since $2\trace{\coshm{ \matC}} = 2\sum_{i=1}^{n} \cosh ( \lambda_i ( \matC )) \geq 2\cosh\left(\lambda_{\max}( \matC )\right) + 2\cosh\left( \lambda_{\min}(\matC)\right)  \geq \exp (\norm{\matC})$ for any symmetric matrix $\matC$ . Take logarithms on both sides and divide by $\theta$, we conclude that $\norm{ \sum_{j=1}^{t}  f_j(x_j^*) } \leq \frac{\ln (2n)}{\theta} + t\frac{\eps^2 \rho^2}{\theta \gamma^2}$.
%
Rescale by $t$ the last inequality to conclude the proof.
%
%
%%%%%%%%%%%%%%%%%%%%%%%%%%%%%%%%%%%%%%%%%%%%%%%%%%%
\end{proof}
%%%%%%%%%%%%%%%%%%%%%%%%%%%%%%%%%%%%%%%%%%%%%%%%%%%
%
We conclude with an open question\footnote{The author would like to thank Toni Pitassi for posing this question.} related to Spencer's six standard deviation bound~\cite{sixDeviation:Spencer}. Does Spencer's six standard deviation bound holds under the matrix setting? More formally, given any sequence of $n$ symmetric matrices $\{\matM_i\}$ with $\norm{ \matM_i}\leq 1$, does there exist a set of signs $\{s_i\}$ so that $\norm{ \sum_{i=1}^{n} s_i \matM_i} = \OO(\sqrt{n})$?
%
%%%%%%%%%%%%%%%%%%%%%%%%%%%%%%%%%%%%%%%%%%%%%%%%%%%
%%%%%%%%%%%%%%%%%%%%%%%%%%%%%%%%%%%%%%%%%%%%%%%%%%%
%%%%%%%%%%%%%%%%%%%%%%%%%%%%%%%%%%%%%%%%%%%%%%%%%%%
%%%%%%%%%%%%%%%%%%%%%%%%%%%%%%%%%%%%%%%%%%%%%%%%%%%
%
%
%
%
%
%
%
 % Introduction
\chapter{Randomized Approximate Linear Algebraic Primitives}\label{chap:rnla}
%
%
In the present chapter\footnote{A preliminary version of Section~\ref{sec:apps:matrix_mult} appeared in~\cite{chernoff:matrix_valued:MZ11} (joint work with Avner Magen). The approximate orthogonal projection algorithm appeared in~\cite{REK} (joint work with Nick Freris), whereas the section on approximately vector orthonormalization is new. Section~\ref{sec:approxCCA} appeared online in~\cite{approxCCA} (joint work with Haim Avron, Christos Boutsidis and Sivan Toledo).} we design and analyze randomized approximation algorithms for the tasks of approximately computing the product of two matrices, approximately computing orthogonal projections, approximately orthonormalizing a set of vectors and approximately computing the principal angles between two linear subspaces.
%
%
%%%%%%%%%%%%%%%%%%%%%%%%%%%%%%%%%%%%%%%%%%%%%%%%%%%%%%%%%%%%%%%%%%%
%%%%%%%%%%%%%%%%%%%%%%%%%%%%%%%%%%%%%%%%%%%%%%%%%%%%%%%%%%%%%%%%%%%
\section{Approximate Matrix Multiplication}\label{sec:apps:matrix_mult}
%%%%%%%%%%%%%%%%%%%%%%%%%%%%%%%%%%%%%%%%%%%%%%%%%%%%%%%%%%%%%%%%%%%
%%%%%%%%%%%%%%%%%%%%%%%%%%%%%%%%%%%%%%%%%%%%%%%%%%%%%%%%%%%%%%%%%%%
%
Computing the product of two square matrices is one of the most basic operation in computational mathematics. Until the 1970's it was believed that matrix multiplication requires a cubic number of operations using the naive algorithm. In his paper, Strassen presented the first sub-cubic algorithm~\cite{matrixmult:strassen}. After Strassen's surprising result, researchers believed that it might be possible to multiply two square matrices in near-linear time and hence they worked towards this direction~\cite{matrixmult:CW,matrixmult:group}, see also~\cite{matrixmult:virginia} for recent developments. Here we focus on approximately computing the matrix product of two matrices under a particular matrix norm. The algorithms that will be analyzed here originate from~\cite{mm:Cohen,MM:focs} and~\cite{sarlos}.
%

%
The research of~\cite{lowrank:FKV} focuses on using non-uniform row sampling to speed-up the running time of several matrix computations. The subsequent developments of~\cite{matrixmult:drineas,lowrank:drineas, matrixdecomp:drineas} also study the performance of Monte-Carlo algorithms on primitive matrix algorithms including the matrix multiplication problem with respect to the Frobenius norm. Sarlos~\cite{sarlos} extended (and improved) this line of research using random projections. Most of the bounds for approximating matrix multiplication in the literature are mostly with respect to the Frobenius norm~\cite{matrixmult:drineas, sarlos, CW_stoc09}. In some cases, the techniques that are utilized for bounding the Frobenius norm also imply \emph{weak} bounds for the spectral norm, see~\cite[Theorem~4]{matrixmult:drineas} or~\cite[Corollary~11]{sarlos}. Here we prove the first non-trivial bounds on matrix multiplication under the spectral norm.
%

%
In particular, we analyze approximation algorithms for matrix multiplication with respect to the spectral norm. Let $\matA\in{\RR^{m\times n}}$ and $\matB\in{\RR^{n \times p}}$ be two matrices and $\eps>0$ an accuracy parameter. We approximate the product $\matA \matB$ using sketches $\widetilde{\matA}\in{\RR^{m\times t}}$ and $\widetilde{\matB}\in{\RR^{t\times p}}$, where $t\ll n$, such that
\begin{equation*}
 \norm{\widetilde{\matA} \widetilde{\matB} - \matA \matB} \leq \eps \norm{\matA}\norm{\matB}
\end{equation*}
holds with sufficiently high probability. We analyze two different sampling procedures for constructing $\widetilde{\matA}$ and $\widetilde{\matB}$; one of them is done by i.i.d. non-uniform sampling rows from $\matA^\top$ and $\matB$ and the other by taking random linear combinations of their rows. We prove bounds on $t$ that depend only on the intrinsic dimensionality of $\matA$ and $\matB$, that is their rank and their stable rank. We should note that the algorithms that will be analyzed here are not new. Namely, the non-uniform sampling row/column approach traces back to the papers of~\cite{mm:Cohen,MM:focs,lowrank:rankone:VR}, and the random sign matrix approach originates from~\cite{sarlos}. The approach of approximating matrix multiplication using element-wise matrix sparsification will not be discussed here, see~\cite[Section 5]{matrixmult:drineas}.
%

%
For achieving bounds that depend on rank when taking random linear combinations we employ standard tools from high-dimensional geometry such as the subspace Johnson-Lindenstrauss lemma (Lemma~\ref{lem:jl_subspace}). For bounds that depend on the smaller parameter of stable rank this approach itself seems weak. However, we show that\footnote{This argument was pointed out to us by Mark Rudelson.} in combination with a simple truncation argument it is amenable to provide such bounds. To handle similar bounds for row sampling, we utilize\footnote{We thank Joel Tropp for his suggestion of using Minsker's version of the matrix Bernstein inequality.} matrix concentration inequalities; more precisely we use Minsker's version of the matrix Bernstein inequality, see Theorem~\ref{thm:Minsker}. Thanks to this inequality, we are able to give bounds that depend only on the stable rank of the input matrices.
%

%
We highlight the usefulness of our approximate matrix multiplication bounds by supplying an application in Chapter~\ref{chap:ma}. In particular, we give an approximation algorithm for the $\ell_2$-regression problem that returns an approximate solution by randomly projecting the initial problem to dimensions linear on the rank of the constraint matrix (Sections~\ref{sec:LS}).
%

%
We now state a theorem that gives bounds on the required number of samples for approximate matrix multiplication using non-uniform row/column samples and random projections.
%\clearpage
%%%%%%%%%%%%%%%%%%%%%%%%%%%%%%%%%%%%%%%%%%
%		Matrix Multiplication
%%%%%%%%%%%%%%%%%%%%%%%%%%%%%%%%%%%%%%%%%%
%%%%%%%%%%%%%%%%%%%%%%%%%%%%%%%%%%%%%%%%%%
\begin{theorem}\label{thm:matrixmult}
Let $0< \eps < 1/2$, $0<\delta<1$, and let $\matA\in{\RR^{m\times n}}$, $\matB\in{\RR^{ n\times p}}$ both having rank and stable rank at most $r$ and $\widetilde{r}$, respectively. The
following hold:
\begin{enumerate}[(i)]
 \item
Let $\matR$ be a $t\times n$ random sign matrix rescaled by $1/\sqrt{t}$. Denote by $\widetilde{\matA}= \matA\matR^\top$ and $\widetilde{\matB}=\matR\matB$.
\begin{enumerate}[(a)]
 \item
 If $t=\Omega( \frac{r}{\eps^{2}} \log (1/\delta) )$ then
%
\[ \mathbb{P}( \forall \x\in\RR^m, \y\in\RR^p, \  |\x^\top (\widetilde{\matA} \widetilde{\matB} - \matA \matB)\y| \leq \eps \norm{\x^\top \matA} \norm{\matB\y}) \geq 1-\delta.\]
%
\item
If $t=\Omega(\frac{\widetilde{r}}{\eps^4} \log (1/\delta) )$ then
%
\[ \Prob{\norm{\widetilde{\matA} \widetilde{\matB} - \matA \matB} \leq \eps \norm{\matA} \norm{\matB}
} \geq 1- \delta. \]
%
\end{enumerate}
\item
Let $p_i =\frac{\norm{\ac{i}}^2 + \norm{\Br{i}}^2 }{S} $ be a probability distribution over $[n]$, where $S=\frobnorm{\matA}^2+\frobnorm{\matB}^2$. Draw $t$ i.i.d. samples from $\{p_i\}$ and define the $n\times t$ sampling matrix $\matS$ by:
\[
\matS_{ij} = \begin{cases} 1/\sqrt{tp_i}  &\mbox{if j-th trial equals to i} \\
0 & \mbox{otherwise }.
\end{cases}
\]
Set $\widetilde{\matA} = \matA \matS \in\RR^{m\times t}$ and $\widetilde{\matB} = \matS^\top \matB\in\RR^{t\times p}$. If $t\geq 20\widetilde{r} \ln ( 16\widetilde{r}/\delta) /\eps^2 $, then
%
\[ \Prob{\norm{\widetilde{\matA} \widetilde{\matB} - \matA \matB} \leq \eps \norm{\matA} \norm{\matB} } \geq 1- \delta. \]
%
%
\end{enumerate}
\end{theorem}
Part (\textit{i.b}) follows from (\textit{i.a}) via a truncation argument. This was pointed out to us by Mark Rudelson~(personal communication). To understand the significance and the differences between the different components of this theorem, we first note that the probabilistic event of part (\textit{i.a}) is superior to the probabilistic event of (\textit{i.b}) and (\textit{ii}). Indeed, when $\matA=\matB^\top$ the former implies that $|\x^\top (\widetilde{\matA}^\top \widetilde{\matA} - \matA^\top \matA) \x| < \eps \cdot \x^\top \matA^\top \matA \x$ for every $\x$, which is stronger than $\norm{\widetilde{\matA}^\top \widetilde{\matA} - \matA^\top \matA} \leq \eps \norm{\matA}^2$. Also notice that part (\textit{i}) is essential computationally inferior to (\textit{ii}) as it gives the same bound while it is more expensive computationally to multiply the matrices by random sign matrices than just sampling their rows. However, the advantage of part (\textit{i}) is that the sampling process is \emph{oblivious}, i.e., does not depend on the input matrices. We also note that the special case of part (\textit{ii}) where $\matA=\matB^\top$ is precisely ~\cite[Theorem~3.1]{lowrank:rankone:VR}. In its present generality Theorem~\ref{thm:matrixmult} (i) is tight as can be seen by the reduction of~\cite[Theorem~2.8]{CW_stoc09}
\footnote{Although the reduction of \cite{CW_stoc09} deals with the Frobenius norm and it is also applicable here since $\norm{\cdot} \leq  \frobnorm{\cdot}$.}. A stronger bound of $t=\Omega( \sqrt{\sr{\matA}\sr{\matB}} \log(\sqrt{\sr{\matA}\sr{\matB}}/\delta) /\eps^2 )$ compared to the bound in Theorem~\ref{thm:matrixmult} (ii) has been obtained in~\cite{HKZ12}. The approach of~\cite{HKZ12} uses a similar extension of the matrix Bernstein inequality as Minsker's extension.
%

%
In a nutshell, the importance of deriving tights bounds for approximate matrix multiplication lies on the fact that several linear algebraic problems can be reduced to primitive problems including matrix multiplication.
%

%

Before proving Theorem~\ref{thm:matrixmult}, we give a sufficient property that a linear map must satisfy in order to guarantee such spectral matrix multiplication bounds as in Theorem~\ref{thm:matrixmult}.
\begin{definition}
Given a fixed subspace $\mathcal{W}$ of $\RR^n$ and any $0<\eps <1$, a linear transformation $\Pi$ from $\RR^n$ to $\RR^t$, $t<n$, is called an $\eps$-subspace embedding (with repsect to $\mathcal{W}$) if
\[
(1-\eps) \norm{\w}^2 \leq \norm{\Pi \w}^2 \leq (1+\eps) \norm{\w}^2,\quad \text{for all }\w\in \mathcal{W}.
\]
\end{definition}
For example, Lemma~\ref{lem:jl_subspace} tells us that given any $k$-dimensional subspace of $\RR^n$, an $t\times n$ random sign matrix rescaled by $1/\sqrt{t}$ where $t=\Omega(\frac{k}{\eps^2} \log(1/\delta))$ is an $\eps$-subspace embedding with probability at least $1-\delta$. Moreover, assuming the notation of Lemma~\ref{lemma:sampling-ortho} and Lemma~\ref{lem:rht-reduce}, it follows that the randomized subsampled Hadamard transform $\matS\matTh$ (assuming the notation of the two latter lemmas) is also an $\eps$-subspace embedding with probability at least $1-\delta$ provided that $t= 12\eps^{-2}(k + \log(n/\delta))\log(k/\delta) $. The following lemma states the connection between subspace embeddings and approximate matrix multiplication.
\begin{lemma}
Let $\matA$ be any $m\times n$ matrix, $\matB$ be any $n \times p$ matrix, and $0<\eps <1$. If $\Pi$ is an $\eps$-subspace embedding of $\colspan{[\matA^\top\ \matB]}$, then
\[
\norm{\matA\Pi^\top \Pi \matB - \matA \matB} \leq \eps \norm{\matA}\norm{\matB}.
\]
\end{lemma}
\begin{proof}
Let $\matU_{\matA} \matSig_{\matA} \matV_{\matA}^\top$ and $\matU_{\matB} \matSig_{\matB} \matV_{\matB}^\top$ be the singular value decompositions of $\matA$ and $\matB$, respectively. Moreover, let $\matU$ be an $n\times (r_{\matA}+r_{\matB})$ matrix whose columns form an orthonormal basis for $\colspan{[\matA^\top\ \matB]}$. By the assumption of $\eps$-subspace embedability, it follows that
\[
(1-\eps) \matI \preceq \matU^\top \Pi^\top \Pi \matU \preceq (1+\eps) \matI,
\]
or equivalently, $\norm{\matU^\top \Pi^\top \Pi \matU - \matI } \leq \eps$. Now, it follows that
\begin{align*}
	\norm{\matA\Pi^\top \Pi \matB - \matA \matB} & = \norm{\matU_{\matA} \matSig_{\matA} (\matV_{\matA}^\top\Pi^\top \Pi \matU_{\matB} - \matV_{\matA}^\top \matU_{\matB} )\matSig_{\matB} \matV_{\matB}^\top } \ \leq \ \norm{\matU_{\matA} \matSig_{\matA}} \norm{\matV_{\matA}^\top\Pi^\top \Pi \matU_{\matB} - \matV_{\matA}^\top \matU_{\matB} } \norm{\matSig_{\matB} \matV_{\matB}^\top} \\
						& = \norm{\matA} \norm{\matV_{\matA}^\top\Pi^\top \Pi \matU_{\matB} - \matV_{\matA}^\top \matU_{\matB} } \norm{\matB}
\end{align*}
using the SVD of $\matA$ and $\matB$, the sub-multiplicity and the unitarity invariance property of the spectral norm. Now, since the columns of $\matV_{\matA}$ are spanned by the columns of $\matU$, it follows that there exists a unitary matrix $\matW_{\matA}$, so that $\matV_{\matA} = \matU \matW_{\matA}$ with $\norm{\matW_{\matA}}^2 = \norm{\matW_{\matA}^\top \matW_{\matA}} = \norm{\matW_{\matA}^\top \matU^\top \matU\matW_{\matA}}= \norm{\matV_{\matA}}^2 =1$. Similarly, there exists $\matW_{\matB}$ so that $\matU_{\matB} = \matU \matW_{\matB}$ with $\norm{\matW_{\matB}}=1$. Using the same reasoning as above,
\begin{align*}
	\norm{\matV_{\matA}^\top\Pi^\top \Pi \matU_{\matB} - \matV_{\matA}^\top \matU_{\matB} } = \norm{\matW_{\matA}^\top (\matU^\top\Pi^\top \Pi \matU - \matU^\top \matU) \matW_{\matB}}  \leq  \norm{\matW_{\matA}^\top} \norm{\matU^\top\Pi^\top \Pi \matU - \matI} \norm{\matW_{\matB}} \leq \eps.
\end{align*}
\end{proof}
%

%
Now, we briefly discuss the translation of the above theoretical bounds to fully specified algorithmic solutions for approximating matrix products. Recall that the input of an approximate randomized matrix multiplication algorithm is $\matA$, $\matB$, $\eps$ and $\delta$. The output of the algorithm is $\widetilde{\matA}$ and $\widetilde{\matB}$ that must satisfy $\norm{\widetilde{\matA}\widetilde{\matB} -\matA \matB } \leq \eps \norm{\matA} \norm{\matB}$ with probability at least $1-\delta$. To translate any of the above bounds to an actual algorithm, someone has to specify the sampling procedure (non-uniform row/column sampling, random sign matrices) and the parameter $t$ (number of samples/number of dimensions to project). Since computing the rank of a matrix is not an easy task (compared to matrix multiplication), the bounds that depend on the rank of the input matrices can be useful only under very restricted cases. For example, whenever a priori bounds on the rank of the input matrices are known. On the other hand, the bounds that depend on the stable rank can be of practical value since approximating the stable rank of a given matrix corresponds to approximating the ratio between its Frobenius norm and its spectral norm. Efficient (randomized) algorithms for the relative approximation of the spectral norm of a given matrix have been obtained in~\cite{KW92}. Using the methods of~\cite{KW92}, we can relatively overestimate the stable rank of the input matrices with high probability and set $t$ to this overestimate value. This approach will incur an extra constant multiplicative factor on the bounds described in Theorem~\ref{thm:matrixmult}. The following generic algorithm (Algorithm~\ref{alg:MM}) outlines the approach.
%
\begin{algorithm}{}
	\caption{Generic Framework for approximate matrix multiplication}\label{alg:MM}
\begin{algorithmic}[1]
\Procedure{}{$\matA$, $\matB$, $\eps$, $\delta$}\Comment{$\matA\in\RR^{m\times n}, \matB\in\RR^{n\times p} $, $0<\eps <1/2$, $0<\delta < 1$}
\State Fix sampling procedure: non-uniform row/column sampling or using random sign matrices.
\State Overestimate/approximate the corresponding parameter $t$
\State Compute $\widetilde{\matA}$ and $\widetilde{\matB}$ using Theorem~\ref{thm:matrixmult}
\State Output: $\widetilde{\matA}$ and $\widetilde{\matB}$ that satisfy with probability at least $1-\delta$: $\norm{\widetilde{\matA}\widetilde{\matB} - \matA \matB} \leq \eps \norm{\matA} \norm{\matB}$
\EndProcedure
\end{algorithmic}
\end{algorithm}
%

We devote the rest of the present section to prove Theorem~\ref{thm:matrixmult}.
%
%
\begin{proof}(of Theorem~\ref{thm:matrixmult})
%
%%%%%%%%%%%%%%%%%%%%%%%%%%%%%%%%%%%%%%%%%%%%%%%%%%%%%%%%%%%%%%%%%%%
%%%%%%%%%%%%%%%%%%%%%%%%%%%%%%%%%%%%%%%%%%%%%%%%%%%%%%%%%%%%%%%%%%%
\paragraph{Part (\textit{i.a}):}
%%%%%%%%%%%%%%%%%%%%%%%%%%%%%%%%%%%%%%%%%%%%%%%%%%%%%%%%%%%%%%%%%%%
%%%%%%%%%%%%%%%%%%%%%%%%%%%%%%%%%%%%%%%%%%%%%%%%%%%%%%%%%%%%%%%%%%%
We prove the following more general theorem from which Theorem~\ref{thm:matrixmult} (\textit{i.a}) follows by plugging in $t\geq \frac{2r}{c_1\eps^2} \ln (c_2) \ln(1/\delta)$ where $c_1,c_2$ is as in Theorem~\ref{thm:matrixmult:restated}.
%%%%%%%%%%%%%%%
\begin{theorem}\label{thm:matrixmult:restated}
%%%%%%%%%%%%%%%
Let $\matA\in{\RR^{m\times n}}$ and $\matB\in{\RR^{ n\times p}}$. Assume that the ranks of $\matA$ and $\matB$ are at most $r$. Let $\matR$ be a $t\times n$ random sign matrix rescaled by $1/\sqrt{t}$. Denote by $\widetilde{\matA}= \matA\matR^\top$ and $\widetilde{\matB}= \matR\matB$. The following inequality
holds
\[ \Prob{ \forall \x\in\RR^m, \y\in\RR^p, \quad  |\x^\top (\widetilde{\matA} \widetilde{\matB} - \matA \matB)\y| \leq \eps \norm{\x^\top \matA} \norm{\matB\y} }  \geq 1- c_2^{r} \exp (-c_1 \eps^2 t), \]
where $c_1 = \frac1{16 \cdot 36},c_2 = 18$.
%%%%%%%%%%%%%%%
\end{theorem}
%%%%%%%%%%%%%%%
%%%%%%%%%%%%%%%
\begin{proof}
%(of Theorem~\ref{thm:matrixmult:restated})
%%%%%%%%%%%%%%%
Let $\matA=\matU_{\matA}\Sigma_{\matA} \matV_{\matA}^\top$, $\matB=\matU_{\matB} \Sigma_{\matB} \matV^\top_{\matB}$ be the singular value decomposition of $\matA$ and $\matB$ respectively. Notice that $\matU_{\matA}\in{\RR^{n\times r_{\matA} } }, \matU_{\matB}\in{\RR^{n\times r_{\matB} }}$, where $r_{\matA}$ and $r_{\matB}$ is the rank of $\matA$ and $\matB$, respectively.
%

%
Let $\x_1\in{\RR^m},\x_2\in{\RR^{p}}$ two arbitrary unit vectors. Let $\w_1= \x_1^\top \matA$ and $\w_2=\matB \x_2$. Recall that
%
\[\norm{\matA \matR^\top \matR\matB - \matA \matB} = \sup_{\x_1\in{\mathbb{S}^{m-1}}, \x_2\in{\mathbb{S}^{p-1}} } | \x_1^\top(\matA \matR^\top \matR\matB - \matA \matB)\x_2|.\]
%
We will bound the last term for any arbitrary vector. Denote with $\mathcal{V}$ the subspace\footnote{We denote by $\text{colspan}(\matA)$ the subspace generated by the columns of $\matA$.} $\text{colspan}(\matU_{\matA})\cup \text{colspan}(\matU_{\matB})$ of $\RR^n$. Notice that the size of $\text{dim}(\mathcal{V}) \leq r_{\matA} + r_{\matB} \leq 2r$. Applying Lemma~\ref{lem:jl_subspace} to $\mathcal{V}$, we get that with probability at least $1-c_2^{r}\exp(-c_1\eps^2 t)$ that
\begin{equation}\label{eq:matrixmult}
\forall\ \v \in{\mathcal{V}}: \  \ |\norm{\matR \v}^2- \norm{\v}^2 | \leq \eps \norm{\v}^2.
\end{equation}
Therefore we get that for any unit vectors $\v_1,\v_2\in{\mathcal{V}}$:
\begin{align*}
 \ip{\matR\v_1}{\matR\v_2}	&   =   \dfrac{\norm{ \matR\v_1 + \matR\v_2}^2-\norm{\matR\v_1 - \matR\v_2}^2}{4} \ \leq \ \dfrac{(1+\eps)\norm{\v_1+\v_2}^2-(1-\eps)\norm{\v_1-\v_2}^2}{4}\\
  		    	&   =   \dfrac{\norm{\v_1+\v_2}^2-\norm{\v_1-\v_2}^2}{4} \ + \  \eps \dfrac{\norm{\v_1+\v_2}^2+\norm{\v_1-\v_2}^2}{4}\\
			&   =   \ip{\v_1}{\v_2} + \eps \frac{\norm{\v_1}^2+\norm{\v_2}^2}{2}\ =\ \ip{\v_1}{\v_2} + \eps,
\end{align*}
where the first equality follows from the Parallelogram law, the first inequality follows from Equation~\eqref{eq:matrixmult}, and the last inequality since $\v_1,\v_2$ are unit vectors. By similar considerations we get that $\ip{\matR\v_1}{\matR\v_2}  \geq \ip{\v_1}{\v_2} - \eps$. By linearity of $\matR$, we get that
%
\[\forall \v_1,\v_2 \in{\mathcal{V} }: \  \ |(\matR\v_1)^\top \matR\v_2 - \v_1^\top \v_2 | \leq \eps \norm{\v_1}\norm{\v_2} .  \]
%
Notice that $\w_1,\w_2\in{\mathcal{V} }$, hence $ |\w_1 \matR^\top \matR \w_2 - \ip{\w_1}{\w_2}| \leq \eps \norm{\w_1}\norm{\w_2} = \eps \norm{\x_1^\top \matA}\norm{\matB\x_2}$.
\end{proof}
%%%%%%%%%%%%%%%%%%%%%%%%%%%%%%%%%%%%%%%%%%%%%%%%%%%%%%%%%%%%%%%%%%%
%%%%%%%%%%%%%%%%%%%%%%%%%%%%%%%%%%%%%%%%%%%%%%%%%%%%%%%%%%%%%%%%%%%
\paragraph{Part (\textit{i.b}):}
%%%%%%%%%%%%%%%%%%%%%%%%%%%%%%%%%%%%%%%%%%%%%%%%%%%%%%%%%%%%%%%%%%%
%%%%%%%%%%%%%%%%%%%%%%%%%%%%%%%%%%%%%%%%%%%%%%%%%%%%%%%%%%%%%%%%%%%
Recall Lemma~\ref{lem:Rudelson}. Using this lemma together with Theorem~\ref{thm:matrixmult} (\textit{i.a}) and a truncation argument we can prove part (\textit{i.b}).
%%%%%%%%%%%%%%%%%%%%%%%%%%%%%%%%%%%%%%%%%%%%%%%%%%%%%%%%%%%%%%%%%%%%%%%%%
%%%%%%%%%%%%%%%%%%%%%%%%%%%%%%%%%%%%%%%%%%%%%%%%%%%%%%%%%%%%%%%%%%%%%%%%%
\begin{proof}(of Theorem~\ref{thm:matrixmult} (\textit{i.b}))
It suffices to prove that if $t = \Omega(\frac{\widetilde{r}}{\eps^4}\log(1/\delta))$ then $\norm{\frac{\matA}{\norm{\matA}} \matR^\top \matR \frac{\matB}{\norm{\matB}} - \frac{\matA}{\norm{\matA}} \frac{\matB}{\norm{\matB}}} \leq \eps$ with probability at least $1-\delta$. Therefore, by homogeneity assume that $\norm{\matA}=\norm{\matB}=1$. Let $\matA_k$ denote the best rank $k$ approximation of $\matA$ for any $1\leq k \leq \rank{\matA}$. Set $\theta =\lfloor  \frac{1600 \max\{ \sr{\matA}, \sr{\matB} \}}{\eps^2}\rfloor$ and fix $t= \frac{2\theta \ln(c_2)}{c_1 \eps^2} \ln(2/\delta) +  8 \ln(8/\delta)$  where $c_1,c_2$ are the constants in Theorem~\ref{thm:matrixmult:restated}. Define $\widehat{\matA} = \matA - \matA_\theta$, $\widehat{\matB} = \matB- \matB_\theta$. Since $\frobnorm{\matA}^2 = \sum_{j=1}^{\rank{\matA}} \sigma_j(\matA)^2$,
\begin{align*}
 	\norm{\widehat{\matA}} \ \leq\ \dfrac{\frobnorm{\matA} }{\sqrt{\theta}} \leq \dfrac{\eps}{40}, \mbox{ and } \norm{\widehat{\matB}}	\ \leq \ \dfrac{\frobnorm{\matB} }{\sqrt{\theta}} \leq \dfrac{\eps}{40}.
\end{align*}
By triangle inequality, it follows that
\begin{align}
 	\norm{ \widetilde{\matA} \widetilde{\matB} - \matA \matB}  &\leq  \norm{ \matA_\theta \matR^\top  \matR \matB_\theta - \matA_\theta \matB_\theta}\label{ineq:rud1} \\
							& +  \norm{ \widehat{\matA} \matR^\top \matR \matB_\theta}   +   \norm{ \matA_\theta \matR^\top  \matR \widehat{\matB} } + \norm{ \widehat{\matA} \matR^\top \matR \widehat{\matB}} \label{ineq:rud2}\\
						    &   +   \norm{ \widehat{\matA} \matB_\theta} + \norm{ \matA_\theta \widehat{\matB} } + \norm{ \widehat{\matA} \widehat{\matB} }\label{ineq:rud3}.
\end{align}
The quantities displayed in Equation~\eqref{ineq:rud3} are bounded as follows
\begin{equation}\label{mm:rud0}
 \norm{ \widehat{\matA} \matB_\theta} + \norm{ \matA_\theta \widehat{\matB} } + \norm{ \widehat{\matA} \widehat{\matB} } \leq \eps^2/1600 + \eps /40 +\eps /40 \leq \eps
\end{equation}
using standard properties of matrix norms.
%

%
The terms displayed in Equation~\eqref{ineq:rud2} can be bounded using Lemma~\ref{lem:Rudelson}. Indeed, apply Lemma~\ref{lem:Rudelson} with $\tau = \sqrt{t}$ for each matrix $\hat{\matA}, \hat{\matB}, \matA_{\theta}, \matB_{\theta}$. Notice that $2\frobnorm{\hat{\matA}}/\sqrt{t} + 3\norm{\hat{\matA}} \leq 5\eps /40$ since $t\geq \theta$. Similarly, $2\frobnorm{\hat{\matB}}/\sqrt{t} + 3\norm{\hat{\matB}} < 5\eps / 40$. Also $2\frobnorm{\matA_\theta} /\sqrt{t} +3 \norm{\matA_\theta} \leq 4$ and similarly $2\frobnorm{\matB_\theta} /\sqrt{t} +3 \norm{\matB_\theta} < 4$. A union bound on the application of Lemma~\ref{lem:Rudelson} to $\hat{\matA}, \hat{\matB}, \matA_{\theta}$ and $\matB_{\theta}$ implies that the following event
\begin{equation}\label{mm:rud1}
 \left\{ \norm{\hat{\matA} \matR^\top } \geq 5\eps / 40 \right\} \cup \left\{ \norm{\matA_{\theta} \matR^\top } \geq 4 \right\} \cup \left\{ \norm{\hat{\matB} \matR } \geq 5\eps / 40  \right\} \cup \left\{ \norm{\matB_{\theta} \matR } \geq 4 \right\}
\end{equation}
holds with probability at most $4\exp(-t /8)$. The later probability is at most $\delta /2$ ($t\geq 8 \ln(8/\delta)$). Whenever the event~\eqref{mm:rud1} does not hold, it follows
\[ \norm{ \widehat{\matA} \matR^\top \matR \matB_\theta}   +   \norm{ \matA_\theta \matR^\top  \matR \widehat{\matB} } + \norm{ \widehat{\matA} \matR^\top \matR \widehat{\matB}} \leq 4 \cdot 5\eps /40 + 4\cdot 5\eps /40 + 25\eps^2 / 40^2 \leq 2\eps . \]
Finally the term on the right hand side of~\eqref{ineq:rud1} can be bounded using Theorem~\ref{thm:matrixmult} (i.a). Since $t\geq \frac{2\theta \ln(c_2)}{c_1 (\eps/10 )^2} \ln(2/\delta)$,
\begin{equation}\label{mm:rud2}
\Prob{ \norm{ \matA_\theta \matR^\top  \matR \matB_\theta - \matA_\theta \matB_\theta}  \geq \eps } \leq \delta /2.
\end{equation}
A union bound on~\eqref{mm:rud1} and~\eqref{mm:rud2} together with Inequality~\eqref{mm:rud0} implies that
\[ \norm{ \widetilde{\matA} \widetilde{\matB} - \matA \matB} \leq \eps +  2 \eps + \eps  = 4\eps\]
with probability at least $1-\delta$. Rescale $\eps$ to conclude.
\end{proof}
%%%%%%%%%%%%%%%%%%%%%%%%%%%%%%%%%%%%%%%%%%%%%%%%%%%%%%%%%%%%%%%%%%%
%%%%%%%%%%%%%%%%%%%%%%%%%%%%%%%%%%%%%%%%%%%%%%%%%%%%%%%%%%%%%%%%%%%
\paragraph{Part (\textit{ii}):}
%%%%%%%%%%%%%%%%%%%%%%%%%%%%%%%%%%%%%%%%%%%%%%%%%%%%%%%%%%%%%%%%%%%
%%%%%%%%%%%%%%%%%%%%%%%%%%%%%%%%%%%%%%%%%%%%%%%%%%%%%%%%%%%%%%%%%%%
The proof is an application of Minsker's extension the matrix Bernstein inequality~\ref{thm:Minsker}. It suffices to prove that if $t\geq 20\widetilde{r} \ln ( 16\widetilde{r}/\delta) /\eps^2 $ then $\norm{\frac{\matA}{\norm{\matA}} \matS \matS^\top \frac{\matB}{\norm{\matB}} - \frac{\matA}{\norm{\matA}} \frac{\matB}{\norm{\matB}}} \leq \eps$ with probability at least $1-\delta$. Therefore, by homogeneity assume that $\norm{\matA}=\norm{\matB}=1$. Now, $S = \frobnorm{\matA}^2 + \frobnorm{\matB}^2 \leq 2\widetilde{r}$.

Define\footnote{Recall that for any $n$ dimensional (row or column) vector $\x$ and $m$ dimensional (row or column) vector $\y$, $\x\otimes \y $ is the $n\times m$ whose $(i,j)$ entry equals to $x_i y_j$.} $\matW_i:= \frac1{p_i} \dil{\ac{i} \otimes \Br{i}} - \dil{\matA \matB}$ for every $i\in{[n]}$. Define the random matrix $\matM$ of size $m+p$ to be equal to $\matW_i$ with probability $p_i$. Clearly, $\EE \matM = \zeromtx_{(m+p)}$.

Let $\matM_1,\matM_2, \ldots , \matM_t$ be i.i.d. copies of $\matM$, then the random matrix $\frac1{t} \sum_{i=1}^{t} \matM_i $ can be alternatively described using the sampling matrix $\matS$ as $\dil{\matA \matS \matS^\top \matB} - \dil{\matA \matB}$. Indeed, fix any realization of $\matS$; namely, assume that $\matS_{l_j j} = 1\sqrt{tp_j}$ for some $l_j\in{[n]}$. Then, it follows that
%
\[
\dil{\matA\matS \matS^\top \matB} -\dil{\matA\matB} = \sum_{j=1}^{t} \dil{(\matA \matS_{(j))} \otimes (\matS_{(j)}^\top \matB)} - \dil{\matA\matB} = \frac1{t} \sum_{j=1}^{t} \matW_{l_j}.
\]
by the linearity of $\dil{\cdot}$ and the definition of $\matS$.
%
It follows that
\[\lambda_{\max}(\matM) \leq \norm{\matM} = \max_{i\in{[n]}} \norm{\matW_i} \leq 1 + S \max_{i\in{[n]}} \frac{\norm{\ac{i}}\norm{\Br{i}}}{\norm{\ac{i}}^2 +\norm{\Br{i}}^2} \leq 1 + S/2  \leq = 2 \widetilde{r}\]
where we used the arithmetic/geometric mean inequality in the numerator and the inequality $S\geq 1$. Now, we bound the second moment $\rho^2 = \norm{\EE \matM^2}$. First, notice that for any $i\in{[n]}$,
\[\matW_i^2 = \frac1{p_i^2} \dil{\ac{i} \otimes \Br{i}}^2 - \frac1{p_i}\dil{\ac{i} \otimes \Br{i}} \dil{\matA\matB} - \dil{\matA\matB}\frac1{p_i}\dil{\ac{i} \otimes \Br{i}} +\dil{\matA\matB}^2.\]
Therefore, $\EE\matM^2 = \sum_{i=1}^{n} p_i \matW_i^2 = \sum_{i=1}^{n}\frac1{p_i} \dil{\ac{i} \otimes \Br{i}}^2 - \dil{\matA\matB}^2 $ by linearity. Next, we upper bound $\EE\matM^2$ in the psd ordering
\begin{align*}
	\EE \matM^2 & \preceq \sum_{i=1}^{n}\frac1{p_i} \dil{\ac{i} \otimes \Br{i}}^2
				\ \preceq  S \sum_{i=1}^{n} \frac1{\norm{\ac{i}}^2 + \norm{\Br{i}}^2 } \myMat{\norm{\Br{i}}^2 \ac{i}\otimes \ac{i} }{\zeromtx}{\zeromtx}{\norm{\ac{i}}^2 \Br{i} \otimes \Br{i}}\\
				& =  S \sum_{i=1}^{n} \myMat{ \ac{i}\otimes \ac{i} }{\zeromtx}{\zeromtx}{\Br{i} \otimes \Br{i}} = S \myMat{ \matA \matA^\top}{\zeromtx}{\zeromtx}{\matB^\top \matB}
\end{align*}
where the first psd inequality follows by adding the psd matrix $\dil{\matA\matB}^2$, and the second psd inequality by adding the psd matrix
\[\sum_{i=1}^{n} \frac1{p_i} \myMat{\norm{\ac{i}}^2 \ac{i}\otimes \ac{i} }{\zeromtx}{\zeromtx}{\norm{\Br{i}}^2 \Br{i} \otimes \Br{i}}.\]
Hence $\EE \matY^2 \preceq \matV$, where
%
\[
\matV := \frac{S}{t} \myMat{\matA\matA^\top}{\zeromtx}{\zeromtx}{\matB^\top \matB}.
\]
%
It follows that $\norm{\matV} = S\max(\norm{\matA}^2, \norm{\matB}^2)/t  = S/t \leq 2\widetilde{r}/t$. Also, $\frac{\trace{\matV}}{\norm{\matV}} = \trace{\matA\matA^\top} + \trace{\matB^\top \matB} = \frobnorm{\matA}^2 + \frobnorm{\matB}^2 \leq 2\widetilde{r}$.

Set $\matX_i = \frac1{t} \matM_i$ for every $i\in{[t]}$ in Theorem~\ref{thm:Minsker} and notice that $\matY = \sum_{i=1}^{t}\matX_i$. Moreover, $\EE \matX_i =\zeromtx$, $\norm{\matX_i}\leq 2\widetilde{r} / t$ and $\EE \matY^2 \leq \matV$. Given any $0<\eps < 1$ and $0<\delta <1$, set $t := 20 \widetilde{r} /\eps^2 \ln( 16\widetilde{r}/\delta)$. It holds that
\[
\norm{\matV}^{1/2} + \gamma /3 \leq \sqrt{2\widetilde{r} /t} + 2\widetilde{r}/t  = \eps \sqrt{\frac{2\widetilde{r}}{20\widetilde{r}\ln(16\widetilde{r}/\delta)}} + \frac{2\eps^2 \widetilde{r}}{60\widetilde{r}\ln(16\widetilde{r}/\delta)} \leq \eps
\]
using that $\ln(16\widetilde{r}/\delta) \geq 1$ for every $0<\delta < 1$. Now, we are in position to apply Theorem~\ref{thm:Minsker} (Inequality~\ref{ineq:Minsker2}) with $\tau=\eps$ and $t$ implies that (since $\eps > \norm{\matV}^{1/2} + \gamma /3 \leq \sqrt{2\widetilde{r} /t} + 2\widetilde{r}/t$)
\begin{align}
\Prob{ \lambda_{\max}(\matY) \geq \eps } &\leq 4 \frac{\trace{\matV}}{\norm{\matV}} \cdot\exp\left(-\frac{\eps^2/2}{\norm{\matV} + \gamma \eps/3}\right)
										 \ \leq 8 \widetilde{r} \cdot\exp\left(-\frac{t\eps^2}{ 4\widetilde{r}  + 4\widetilde{r}/3}\right)
										 \ \leq \delta /2 \label{mm:minsker1}
\end{align}
where the second inequality follows by the upper bounds on $\norm{\matV}$ and $\gamma$ and the third inequality follows by the range of values on $t$. Apply the same argument as above to the random matrix $-\matY$ to bound $\lambda_{\min}(\matY)$ as follows
\begin{equation}\label{mm:minsker2}
\Prob{ \lambda_{\min}(\matY) \leq  - \eps } \leq \delta / 2.
\end{equation}
Union bound both Inequalities~\eqref{mm:minsker1} and~\eqref{mm:minsker2} to conclude that for every $0<\eps <1$ and every $0<\delta<1$, if $t\geq 20 \widetilde{r}\ln(16\widetilde{r}/\delta)/\eps^2 $, then
\[\Prob{ \norm{\matY}\geq \eps} \leq \delta.\]
To conclude recall that
\[\Prob{\norm{\matY} \geq \eps } = \Prob{ \norm{\dil{\matA\matS\matS^\top \matB- \matA\matB }} \geq \eps } = \Prob{\norm{ \matA\matS\matS^\top \matB - \matA \matB}\geq \eps }.\]
\end{proof}
%
%
%
%%%%%%%%%%%%%%%%%%%%%%%%%%%%%%%%%%%%%%%%%%%%%%%%%%%%%%%
%%%%%%%%%%%%%%%%%%%%%%%%%%%%%%%%%%%%%%%%%%%%%%%%%%%%%%%
\section{Approximate Orthogonal Projection}
%%%%%%%%%%%%%%%%%%%%%%%%%%%%%%%%%%%%%%%%%%%%%%%%%%%%%%%
%%%%%%%%%%%%%%%%%%%%%%%%%%%%%%%%%%%%%%%%%%%%%%%%%%%%%%%
%
In the present section, we present a randomized iterative algorithm (Algorithm~\ref{alg:randOP}) that, given any vector $\b$ and a linear subspace represented as the column span of a matrix $\matA$, computes an approximation to the orthogonal projection of $\b$ onto the column span of $\matA$ (denoted by $\br$, $\br=\matA\pinv{\matA}\b$). The exact version of this problem is a fundamental geometric primitive.
%

%
The work of~\cite{ROP:CRT11} presents an efficient approximation algorithm for this problem. The main idea behind this paper was to approximately solve the overdetermined linear system $\matA \x = \b$ as an intermediate step, i.e., compute an approximate least squares solution $\tilde{\x}_{\textrm{\tiny LS}}$. Then, return $\matA \tilde{\x}_{\textrm{\tiny LS}}$ as the approximate solution since $\br=\matA\pinv{\matA}\b$. The motivation behind their work was to accelerate interior point methods for convex optimization, see~\cite{book:Wright}, since the core of interior point methods is based on a particular orthogonal projection.
%
\begin{algorithm}{}
	\caption{Randomized Orthogonal Projection}\label{alg:randOP}
\begin{algorithmic}[1]
\Procedure{}{$\matA$, $\b$, $T$}\Comment{$\matA\in\RR^{m\times n}, \b\in\RR^m$, $T\in \NN$}
\State Initialize $\z^{(0)} =\b$
\For {$k=0,1,2,\ldots, T - 1 $ }
	\State Pick $j_k\in[n]$ with probability $p_j:=\norm{\ac{j}}^2/\frobnorm{\matA}^2,\ j\in [n]$
	\State Set $ \z^{(k+1)} = \left(\Id_m - \frac{\ac{j_k} \ac{j_k}^\top }{\norm{\ac{j_k}}^2}\right) \z^{(k)}$
\EndFor
\State Output $\z^{(T)}$
\EndProcedure
\end{algorithmic}
\end{algorithm}
%
Algorithm~\ref{alg:randOP} is iterative. Initially, it starts with $\z^{(0)}=\b$. At the $k$-th iteration, the algorithm randomly selects a column $\ac{j}$ of $\matA$ for some $j$, and updates $\z^{(k)}$ by projecting it onto the orthogonal complement of the space of $\ac{j}$. The claim is that randomly selecting the columns of $\matA$ with probability proportional to their square norms implies that the algorithm converges to $\bc$ in expectation. After $T$ iterations, the algorithm outputs $\z^{(T)}$ and by orthogonality $\b-\z^{(T)}$ serves as an approximation for $\br$. The next theorem bounds the expected rate of convergence for Algorithm~\ref{alg:randOP}.
%
%
\begin{theorem}\label{thm:randOP}
Let $\matA\in\RR^{m\times n}$, $\b\in\RR^m$ and $T>1$ be the input to Algorithm~\ref{alg:randOP}. Fix any integer $T\geq k>0$. In exact arithmetic, after $k$ iterations of Algorithm~\ref{alg:randOP} it holds that
\[\EE \norm{\z^{(k)} - \bc }^2 \leq \left(1 -\frac1{\kappaFS(\matA)}\right)^k  \norm{\br}^2.\]
Moreover, each iteration of Algorithm~\ref{alg:randOP} requires in expectation (over the random choices of the algorithm) at most $5\cavg(\matA)$ arithmetic operations.
\end{theorem}
\begin{remark}\label{rem:randOP}
A suggestion for a stopping criterion for Algorithm~\ref{alg:randOP} is to regularly check: $ \frac{\norm{\matA^\top \z^{(k)}}}{\frobnorm{\matA} \norm{\z^{(k)}}} \leq \eps$ for some given accuracy $\eps>0$. It is easy to see that whenever this criterion is satisfied, it holds that $\norm{\bc-\z^{(k)} } / \norm{\z^{(k)}} \leq \eps \kappaF(\matA)$, i.e., $\b-\z^{(k)}\approx \br$.
\end{remark}
%

%
We devote the rest of this subsection to prove Theorem~\ref{thm:randOP}. Define $\matP (j):= \Id_m - \frac{\ac{j} \ac{j}^\top}{\norm{\ac{j}}^2}$ for every $j\in [n]$. Observe that $\matP (j) \matP (j) = \matP (j)$, i.e., $\matP (j)$ is a projector matrix. Let $X$ be a random variable over $\{1,2,\ldots, n\}$ that picks index $j$ with probability $\norm{\ac{j}}^2/\frobnorm{\matA}^2$. It is clear that $\EE [\matP(X)] = \Id_m - \matA\matA^\top /\frobnorm{\matA}^2$. Later we will make use of the following fact.
%
%
\begin{fact}\label{lem:technical}
For every vector $\vecu$ in the column space of $\matA$, it holds $\norm{\left(\Id_m - \frac{\matA\matA^\top }{\frobnorm{\matA}^2}\right) \vecu} \leq \left(1 - \frac{\sigma^2_{\min}}{\frobnorm{\matA}^2} \right) \norm{\vecu}$.
\end{fact}
%
%

%
Define $\e^{(k)}:= \z^{(k)} - \bc$ for every $k\geq 0$. A direct calculation implies that
\[\e^{(k)} = \matP (j_k) \e^{(k-1)}.\]
Indeed, $\e^{(k)} = \z^{(k)} - \bc = \matP(j_k) \z^{(k-1)} - \bc = \matP(j_k) (\e^{(k-1)} + \bc ) - \bc = \matP(j_k) \e^{(k-1)}$ using the definitions of $\e^{(k)}$, $\z^{(k)}$, $\e^{(k-1)}$ and the fact that $\matP (j_k) \bc = \bc$ for any $j_k\in{[n]}$. Moreover, it is easy to see that for every $k\geq0$ $\e^{(k)}$ is in the column space of $\matA$, since $\e^{(0)} = \b - \bc = \br\in \colspan{\matA} $, $\e^{(k)}= \matP (j_k) \e^{(k-1)}$ and in addition $\matP(j_k)$ is a projector matrix for every $j_k\in [n]$.

Let $X_1,X_2,\ldots $ be a sequence of independent and identically distributed random variables distributed as $X$. For ease of notation, we denote by $\EE_{k-1}[\cdot] = \EE_{X_k} [\cdot\ |\ X_1, X_2, \ldots, X_{k-1}]$, i.e., the conditional expectation conditioned on the first $(k-1)$ iteration of the algorithm. It follows that
\begin{align*}
	\EE_{k-1} \norm{ \e^{(k)}}^2 &  =   \EE_{k-1} \norm{ \matP (X_k) \e^{(k-1)} }^2 \ = \ \EE_{k-1} \ip{ \matP (X_k) \e^{(k-1)} }{ \matP (X_k) \e^{(k-1)} } \\
							 &  =   \EE_{k-1} \ip{\e^{(k-1)} }{ \matP (X_k)\matP (X_k) \e^{(k-1)} } \ = \  \ip{\e^{(k-1)} }{ \EE_{k-1} [\matP (X_k)] \e^{(k-1)} } \\
							 &\leq  \norm{\e^{(k-1)}} \norm{ \left(\Id_m - \frac{\matA\matA^\top }{\frobnorm{\matA}^2}\right) \e^{(k-1)} }
							 \ \leq \ \left(1 - \frac{\sigma^2_{\min}}{\frobnorm{\matA}^2}\right) \norm{\e^{(k-1)}}^2
\end{align*}
where we used linearity of expectation, the fact that $\matP (\cdot)$ is a projector matrix, Cauchy-Schwarz inequality and Fact~\ref{lem:technical}. Repeating the same argument $k-1$ times we get that
\[\EE\norm{ \e^{(k)}}^2 \leq \left(1 -  \frac1{\kappaFS(\matA)}\right)^k \norm{\e^{(0)}}^2.\]
%
Note that $\e^{(0)} = \b - \bc = \br$ to conclude.
%Set $k\geq \kappaFS(\matA)\ln(\norm{\br}^2 /\eps^2)$ and recall the inequality $1-t \leq \exp(-t)$ for $t \leq 1$ to conclude.
%

%
Step $5$ can be rewritten as $\z^{(k+1)} = \z^{(k)} - \left(\ip{\ac{j_k}}{\z^{(k)}} / \norm{\ac{j_k}}^2\right) \ac{j_k}$. At every iteration, the inner product and the update from $\z^{(k)}$ to $\z^{(k+1)}$ require at most $5\nnz{\ac{j_k}}$ operations for some $j_k\in{[n]}$; hence in expectation each iteration requires at most $\sum_{j=1}^{n}  p_j 5\nnz{\ac{j}} = 5\cavg(\matA)$ operations.
%



%%%%%%%%%%%%%%%%%%%%%%%%%%%%%%%%%%%%%%%%%%%%%%%%%%%%%%%
%%%%%%%%%%%%%%%%%%%%%%%%%%%%%%%%%%%%%%%%%%%%%%%%%%%%%%%
\section{Approximate Orthonormalization}
%%%%%%%%%%%%%%%%%%%%%%%%%%%%%%%%%%%%%%%%%%%%%%%%%%%%%%%
%%%%%%%%%%%%%%%%%%%%%%%%%%%%%%%%%%%%%%%%%%%%%%%%%%%%%%%
Given a set of column vectors $\{\ac{1},\ac{2},\ldots , \ac{n}\}\subset \RR^m$ forming an $m\times n$ real matrix $\matA$, the problem of computing an orthonormal basis for their span is a fundamental computational primitive in numerical linear algebra. It is the main ingredient, in direct algorithms for solving least squares~\cite{book:Bjork}, in iterative linear system solver such as GMRES~\cite{book:Saad}, and in eigenvalue algorithms such as the Arnoldi process to name a few. Due to numerical issues and finite precision representation of real numbers, exact orthonormalization is not feasible in general\footnote{A modified version of the classical Gram-Schmidt is known to be numerically stable~\cite{QR:numerics}.}. A natural relaxation of the exact orthonormalization problem is to require approximate orthogonality. In this section, we study the problem of computing an approximate orthonormal basis of a set of vectors, i.e., a set of basis vectors whose pair-wise inner products is close to zero. We mainly focus on iterative algorithms.
%

%
There is a rich body of work on iterative algorithms for approximate orthonormalization. In 1970, Kovarik proposed two iterative algorithms for approximate orthonormalization that have quadratic convergence~\cite{QR:Kovarik}. The main drawback of Kovarik's algorithms is that each iteration is computational expensive, i.e., Algorithm B of~\cite{QR:Kovarik} requires matrix inversion, see~\cite{QR:Popa} and references therein for several improvements. Although, the aforementioned algorithms are interesting from a theoretical point of view, they are inferior compared to the classical solutions in terms of computational efficiency.
%

%
It has been observed that Gram-Schmidt may produce a set of vectors which is far from being orthogonal under finite precision computations~\cite{QR:Bj67}. Such issues motivated researchers to study iterative versions of the Gram-Schmidt process where each step of the process is iteratively applied until a desired accuracy has been achieved. Iterative Gram-Schmidt algorithms with improved orthogonality have been proposed in~\cite{QR:DGKS} and~\cite{QR:Ruhe}, see also~\cite{QR:Hoffman,QR:numericsII} and~\cite{QR:USSR}.
%

%
Finally, to the best of our knowledge, Rokhlin and Tygert implicitly presented the first randomized algorithm for approximately orthonormalizing a set of vectors~\cite{RT08}. Assume that $m\gg n$, the algorithm of~\cite{RT08} proceeds as follows: First, it randomly projects the columns of $\matA$ using the subsampled randomized Fourier transform\footnote{The subsampled randomized Fourier transform shares similar properties with the subsampled randomized Walsh-Hadamard transform, see Definition~\ref{def:rht}. A similar bound holds by using the subsampled randomized Walsh-Hadamard transform.} to $\OO(n^2)$ dimensions. Then, the algorithm applies a QR decomposition on the projected column vectors denoted by $\widetilde{\matQ}\widetilde{\matR}$. The main argument of~\cite{RT08} is that the columns of the product $\matA \widetilde{\matR}^{-1}$ are approximately orthonormal with constant probability.
%

%
In the following section we present a randomized, amenable to parallelization, iteratively-based algorithm for the case of vectors whose corresponding $m\times n$ matrix $\matA$ is sparse and sufficiently well-conditioned.


%$\matQ \matR$ be the QR decomposition of $\matA$, i.e, $\matQ$ is an $m\times n $ matrix with orthonormal columns that spans $\colspan{\matA}$ and $\matR$ is an $n\times n$ upper triangular non-singular matrix. Recall that the classical Gram-Schmidt process for computing the QR decomposition requires $\OO(mn^2)$ operations.
%

%%%%%%%%%%%%%%%%%%%%%%%%%%%%%%%%%%%%%%%%%%%%%%%%%%%
%%%%%%%%%%%%%%%%%%%%%%%%%%%%%%%%%%%%%%%%%%%%%%%%%%%
\subsection{A Randomized Parallel Orthonormalization Algorithm}\label{sec:result}
%%%%%%%%%%%%%%%%%%%%%%%%%%%%%%%%%%%%%%%%%%%%%%%%%%%
%%%%%%%%%%%%%%%%%%%%%%%%%%%%%%%%%%%%%%%%%%%%%%%%%%%
In this section, we analyze a randomized algorithm for approximate vector orthonormalization. The main feature of the algorithm is that it is amenable to a parallel implementation; a feature that is not present on the classical Gram-Schmidt process. Due to the approximate nature of the algorithm, the algorithm will not be able to distinguish between a set of vectors which is linearly dependent and a set which is close to being linearly dependent. We define the above notion of ``closeness'' by saying that a matrix $\matA$ containing as columns a set of $n$ vectors is \emph{$\gamma$-orthogonalizable}, if the norm of the projection of $\ac{i+1}$ onto the complement of the span of $\{\ac{1},\ldots ,\ac{i}\}$ is at least $\gamma\norm{\ac{i+1}}$ for every $i\in{[n-1]}$. More concisely, if $\norm{(\Id - \matA_{[i]}\pinv{(\matA_{[i]})})\ac{i+1}} \geq \gamma \norm{\ac{i+1}}$ for every $i\in{[n-1]}$, where $\matA_{[i]}$ is the $m\times i$ matrix containing the first $i$ columns of $\matA$. To justify the definition above, we note that any matrix $\matA$ with linearly independent columns is a $\gamma$-orthogonalizable matrix for some $\gamma>0$. However, the discussion here involves approximate algorithms for distinguishing between linear dependence and independence, therefore we need a more robust notion than the notion of linear independence. The above definition captures this requirement by enforcing that the projection of any column vector into the span of all prior column vectors is not being negligible.
%
%
%
%%%%%%%%%%%%%%%%%%%%%%%%%%%%%%%%%%%%%%%%%%%%%%%%%%%
\begin{algorithm}{}
	\caption{Randomized Sparse GS (RSGS)}\label{alg:RSGS}
\begin{algorithmic}[1]
	\Procedure{RSGS}{$\matA$, $T$}\Comment{$\matA\in\RR^{m\times n}$, $\eps>0$}
	\State Sort the columns of $\matA$ with respect to their sparsity, i.e., $\nnz{\ac{i}} \leq \nnz{\ac{j}}$ if $i<j$.
	\State Let $\widetilde{\matQ}$ be the $m\times n$ zeroes matrix; initialize $\tqc{1} = \ac{1} / \norm{\ac{1}}$
	\For{$i=2,\ldots, n$}
	\State 	Apply Algorithm~\ref{alg:randOP} with input $(\matA (:, 1 : (i-1) ))$, $\ac{i}$ and $T$. Output $\z(i)$
	\State Set $\tqc{i} = \z (i) / \norm{\z (i)}$.
	\EndFor
	\State \textbf{Output:} the $m\times n$ matrix $\widetilde{\matQ}$
\EndProcedure
\end{algorithmic}
\end{algorithm}
%%%%%%%%%%%%%%%%%%%%%%%%%%%%%%%%%%%%%%%%%%%%%%%%%%%
%
%
%
%%%%%%%%%%%%%%%%%%%%%%%%%%%%%%%%%%%%%%%%%%%%%%%%%%%
\begin{theorem}
%
Let $0<\eps< 1/2$, $0<\delta <1$ and let $\matA$ be an $m\times n$ matrix which is $\eps$-orthogonalizable after the reordering of Step $2$. Algorithm~\ref{alg:RSGS} with input $T\geq \kappaFS(\matA) \ln( \frac{n}{\delta\eps^4})$ outputs, with probability at least $1-\delta$, an $m\times n$ matrix $\widetilde{\matQ}$ with the following properties:
\begin{enumerate}[(a)]
	\item
	The columns of $\widetilde{\matQ}$ span $\colspan{\matA}$.
	\item
	The condition number of $\widetilde{\matQ}$ is bounded by $1+2\eps$. In other words, the columns of $\widetilde{\matQ}$ are nearly orthonormal, i.e., $\norm{\widetilde{\matQ}^\top \widetilde{\matQ} - \matI} \leq 2\eps$.
\end{enumerate}
The expected running time of Algorithm~\ref{alg:RSGS} is $\OO(\cavg(\matA) n \kappaFS(\matA) \ln(\frac{n}{\delta \eps^4}))$, which is at most $\OO(\nnz{\matA} n \cond{\matA}^2\ln(\frac{n}{\delta \eps^4}))$.
\end{theorem}
%%%%%%%%%%%%%%%%%%%%%%%%%%%%%%%%%%%%%%%%%%%%%%%%%%%
%%%%%%%%%%%%%%%%%%%%%%%%%%%%%%%%%%%%%%%%%%%%%%%%%%%
\begin{remark}
	In Step $5$ of Algorithm~\ref{alg:RSGS}, it is not clear how to specify the parameter $T$ to be greater than $\kappaFS(\matA) \ln( \frac{n}{\delta\eps^4})$ without a priori knowledge of $\sigma_{\min}(\matA)$. The stopping criterion discussed in Remark~\ref{rem:randOP} can be used as an alternative termination criterion in Step 5.
\end{remark}
%%%%%%%%%%%%%%%%%%%%%%%%%%%%%%%%%%%%%%%%%%%%%%%%%%%
We devote the rest of this section to prove the above theorem. First, we claim that the average column sparsity of $\matA_{[i]}$ for every $i\in{[n]}$ is upper bounded by the average column sparsity of $\matA$ after Step~2.
%
\begin{lemma}\label{lem:sparsity}
After Step $2$ of Algorithm~\ref{alg:RSGS}, it holds that $\text{C}_{avg}(\matA_{[i-1]}) \leq \text{C}_{\text{avg}}(\matA_{[i]})$ for every $1 < i \leq n $.
\end{lemma}
%
\begin{proof}
It suffices to prove that $\frac1{\frobnorm{\matA_{[i - 1]}}^2} \sum_{j=1}^{i - 1} \norm{\ac{j} }^2 \nnz{\ac{j}}  \leq  \frac1{\frobnorm{\matA_{[i]}}^2} \sum_{j=1}^{i} \norm{\ac{j} }^2 \nnz{\ac{j}}$. Re-arranging terms, it is equivalent to
\begin{align*}
	\sum_{j=1}^{i - 1} \norm{\ac{j} }^2 \nnz{\ac{j}} & \leq  \left( 1 - \frac{  \norm{\ac{i}}^2 }{\frobnorm{\matA_{[i]} }^2} \right) \sum_{j=1}^{i} \norm{\ac{j} }^2 \nnz{\ac{j}} \quad \text{or}\\
	 \frac1{\frobnorm{\matA_{[i]} }^2}  \sum_{j=1}^{i} \norm{\ac{j} }^2 \nnz{\ac{j}} & \leq  \nnz{\ac{i}}.
\end{align*}
%
Recall that the columns of $\matA$ are sorted in ascending order in terms of their sparsity. Hence, the result follows since the left hand side is the expected column sparsity of $\matA_{[i]}$, which is at most $\nnz{\ac{i}}$ by Step $2$.
\end{proof}
Moreover, we claim that the ratio of the Frobenius norm over the smallest non-zero singular value of all the sub-matrices of $\matA$ is upper bound by the corresponding ratio of $\matA$.
%
%
\begin{lemma}\label{lem:submatrixcond}
%
Fix $1<i\leq n$. Let $\matA$ be an $m\times n$ matrix. Then $\kappaFS(\matA_{[i]}) \leq \kappaFS(\matA)$.
\end{lemma}
%
%
\begin{proof}
First, it is obvious that $\frobnorm{\matA_{[i]}}^2 \leq \frobnorm{\matA}^2$. It suffices to lower bound $\sigma_{\min}(\matA_{[i]})$.
%
\begin{align*}
	\sigma_{\min}(\matA) = \min_{\x\neq \zero,\ \matA\x\neq\zero } \frac{\norm{\matA \x}}{\norm{\x}} \ \leq \ \min_{x_{i+1}=x_{i+2}=\ldots = x_n = 0,\ \x\neq \zero,\ \matA\x\neq\zero } \frac{\norm{\matA \x}}{\norm{\x}}\ = \ \min_{\y\neq \zero,\ \matA_{[i]}\y\neq\zero } \frac{\norm{\matA_{[i]} \y}}{\norm{\y}} = \sigma_{\min}(\matA_{[i]}).
\end{align*}
%
The first inequality holds since
\[\left\{\x\in\RR^n\ |\ \x\neq \zero,\ \matA\x\neq\zero,\ x_{i+1}=x_{i+2}=\ldots =x_n=0 \right\} \subseteq \left\{\x\in\RR^n\ |\ \x\neq \zero,\ \matA\x\neq\zero  \right\}.\]
\end{proof}
%
%

%
By construction it holds that $\norm{\tqc{i}}^2 = 1$. It suffices to show that $\left|\ip{\tqc{i}}{\tqc{j}}\right| \leq 2\eps$ for any $j<i$. First notice that, for every $1< i\leq n$, $\kappaFS(\matA_{[i]}) \leq \kappaFS(\matA)$ (Lemma~\ref{lem:submatrixcond}). By the choice of $T$, apply Theorem~\ref{thm:randOP} for every $i=2,\ldots ,n$ with $\delta' = \delta/n$ and take a union bound over all $i$ to conclude that with probability at least $1-\delta$:
\begin{align}\label{ineq:main}
	\norm{\z (i) - (\matI_m- \matA_{[i-1]} \pinv{\matA_{[i-1]}}) \ac{i} } \leq \eps^2 \norm{ \ac{i}}\quad \text{for all }1<i\leq n.
\end{align}
Condition on the event that Equation~\eqref{ineq:main} holds from now on. Fix any $1<i\leq n $ and $j<i$. For notation convenience, set $\matP = (\matI_m- \matA_{[i-1]} \pinv{\matA_{[i-1]}})$ and let $\matW$ be an $m\times (i-1)$ matrix whose columns form a basis for $\text{colspan}(\matA_{[i-1]})$. Condition (b) is satisfied since:
\begin{align*}
\left|\ip{\tqc{i}}{ \tqc{j}}\right| &   =   \left|\ip{\tqc{i}}{ \matW \u}\right| \quad\text{where }\tqc{j} = \matW \u \text{ for some }\u\in \RR^{i-1}\\
 						   &   =   \left|\ip{\z(i)}{ \matW \u}\right| / \norm{\z(i)} \\
 						   &   =   \left|\ip{\z(i) - \matP \ac{i} + \matP \ac{i} }{ \matW \u}\right| / \norm{\z(i)} \\
 						   &   =   \left|\ip{\z(i) - \matP \ac{i}}{ \tqc{j}}\right| / \norm{\z(i)} \quad\text{since }\matP \matW = \zero\\
 						   & \leq  \norm{\z(i) - \matP \ac{i} } \norm{ \tqc{j}} / \norm{\z(i)} \quad \text{Cauchy-Schwarz Ineq.}\\
 						   & \leq  \eps^2 \frac{\norm{\ac{i} }}{ \norm{\z(i)}} \quad\text{Ineq.}~\eqref{ineq:main}.
\end{align*}
It follows that $|\ip{\tqc{i}}{ \tqc{j}} |\leq 2\eps $, since $\norm{\z(i)} \geq \norm{\matP \ac{i}} - \norm{\z(i) - \matP\ac{i}}\geq \eps \norm{\ac{i}} - \eps^2 \norm{\ac{i}} \geq \eps /2 \norm{\ac{i}}$
using the triangle inequality, the assumption that $\matA$ is $\eps$-orthogonalizable and Ineq.~\eqref{ineq:main}, and the fact that $\eps <1/2$.
%

%
Now, we analyze the running time of the algorithm. Lemma~\ref{lem:sparsity} tells us that, for every $1< i\leq n$, the average column sparsity of the matrices $\matA_{[i]}$ are upper bounded by $\cavg(\matA)$. Therefore, Step $5$ of Algorithm~\ref{alg:RSGS} requires in expectation $\OO(\cavg(\matA) T)$ operations.
%

%
Observe that $\cavg(\matA) \kappaFS(\matA) = \sum_{j=1}^{n}\norm{\ac{j}}^2 \nnz{\ac{j}} / \sigma^2_{\min}(\matA) \leq \nnz{\matA} \cond{\matA}^2$, where the inequality follows since $\max_{j\in{[n]}} \norm{\ac{j}}^2 \leq \sigma_{\max}^2(\matA)$.
%
%
%
%
%
%
%%%%%%%%%%%%%%%%%%%%%%%%%%%%%%%%%%%%%%%%%%%%%%%%%%%%%%%
%%%%%%%%%%%%%%%%%%%%%%%%%%%%%%%%%%%%%%%%%%%%%%%%%%%%%%%
\section{Approximate Principal Angles}\label{sec:approxCCA}
%%%%%%%%%%%%%%%%%%%%%%%%%%%%%%%%%%%%%%%%%%%%%%%%%%%%%%%
%%%%%%%%%%%%%%%%%%%%%%%%%%%%%%%%%%%%%%%%%%%%%%%%%%%%%%%
%
Canonical Correlation Analysis (CCA), introduced by H. Hotelling in 1936~\cite{Hot36}, is an important technique in statistics, data analysis, and data mining. CCA  has been successfully applied in many machine learning applications, e.g. clustering~\cite{CKLS09}, learning of word embeddings~\cite{DFU11}, sentiment classification~\cite{DRFU12}, discriminant learning~\cite{SFGT12}, object recognition~\cite{KKC07} and activity recognition from video~\cite{LAMCS11}.
In many ways CCA is analogous to Principal Component Analysis (PCA), but instead of analyzing a single data-set (in matrix form), the goal of CCA is to analyze the relation between a pair of data-sets (each in matrix form). From a statistical point of view, PCA extracts the maximum covariance directions between elements in a single matrix, whereas CCA finds the direction of maximal correlation between a pair of matrices. From a linear algebraic point of view, CCA measures the similarities between two subspaces (those spanned by the columns of each of the two matrices analyzed). From a geometric point of view, CCA computes the cosine of the \emph{principal angles} between the two subspaces.
% I moved the Geo-PoV here since we talk about PoV here and not later.
%

%
There are different ways to define the canonical correlations (a.k.a. principal angles) of a pair of matrices, and all these methods are equivalent~\cite{GZ95}.
%From an application point of view, statistically-oriented definitions are often the most appropriate, but
The following linear algebraic formulation of Golub and Zha~\cite{GZ95} serves our algorithmic point of view best.
\begin{definition}\label{def}
Let $\matA \in \RR^{m \times n}$ and $\matB \in \RR^{m \times \ell}$ , and assume that $p = \rank{\matA} \geq \rank{\matB} = q$.
The {\em canonical correlations}  $\sigma_1\left( \matA, \matB \right) \ge \sigma_2\left( \matA, \matB \right) \ge \cdots \ge \sigma_q\left( \matA, \matB \right)$
of the matrix pair $(\matA, \matB)$ are defined recursively by the following formula:
\[ \sigma_i\left(\matA, \matB \right) = \max_{ \x \in {\cal A}_i, \y \in {\cal B}_i }  \sigma \left( \matA \x, \matB \y \right) = : \sigma\left( \matA \x_i, \matB \y_i \right)  ,\quad i=1,\ldots ,q\]
where
\begin{itemize}

\item $ \sigma\left(\u, \v \right)  = | \u^\top \v | / \left( \norm{\u} \norm{\v} \right)$,

\item $ {\cal A}_i = \{ \x : \matA \x \neq \bf{0}, \matA \x \perp \{ \matA \x_1,\ldots,\matA \x_{i-1} \} \} $,

\item $ {\cal B}_i = \{ \y : \matB \y \neq \bf{0}, \matB \y \perp \{ \matB \y_1,\ldots,\matB \y_{i-1} \} \} $.
\end{itemize}
The unit vectors $\matA \x_1 / \norm{\matA \x_1}, \dots, \matA \x_q / \norm{\matA \x_q}, \matB \y_1 / \norm{\matB \y_1}, \dots, \matB \y_q / \norm{\matB \y_q}$ are called the {\em canonical} or {\em principal vectors}\footnote{Note that the canonical vectors are \emph{not} uniquely defined.}.
\end{definition}
In this section, we present a randomized algorithm that computes an additive-error approximation to all the canonical correlations of a matrix pair asymptotically faster compared to the standard method of Bj{\"o}rck and Golub~\cite{BG73}. To the best of our knowledge this is the first sub-cubic time approximation algorithm for CCA.

Our algorithm is based on \emph{dimensionality reduction}: given a pair of matrices $(\matA, \matB)$, we transform the pair to a new pair
$(\hat{\matA}, \hat{\matB})$ that has much fewer rows, and then compute the canonical correlations of the new pair exactly, e.g. using the Bj{\"o}rck and Golub algorithm. We prove that with high probability the canonical correlations of $(\hat{\matA}, \hat{\matB})$ are close to the canonical correlations of $(\matA, \matB)$.
%Now, any CCA algorithm can be applied on $(\hat{\matA}, \hat{\matB})$; in our analysis we assume that the Bj{\"o}rck and Golub algorithm is used.
The transformation of $(\matA, \matB)$ into $(\hat{\matA}, \hat{\matB})$ is done in two steps. First, we apply the \emph{Randomized Walsh-Hadamard Transform (RHT)} to both $\matA$ and $\matB$. This is a unitary transformation, so the canonical correlations are preserved exactly. On the other hand, we show that with high probability, the transformed matrices have their ``information'' equally spread among all the input rows, so now the transformed matrices are amenable to uniform sampling. In the second step, we uniformly sample (without replacement) a sufficiently large set of rows and rescale them to form $(\hat{\matA}, \hat{\matB})$. The combination of RHT and uniform sampling is often called \emph{Subsampled Randomized Walsh-Hadamard Transform (SRHT)} in the literature~\cite{Tro11}. Note that
other variants of dimensionality reduction~\cite{sarlos} might be appropriate as well, but for concreteness we focus on the SRHT.
% witout "as well" appropriate will imply "better", which we do not want to say (and don't think so)

Our dimensionality reduction scheme is particularly effective when the matrices are tall-and-thin, that is they have much more rows than columns. Targeting such matrices is natural: in typical CCA applications, columns typically correspond to features or labels and rows correspond to samples or training data. By computing the CCA on as many instances as possible (as much training data as possible), we get the most reliable estimates of application-relevant quantities.
However in current algorithms adding instances (rows) is expensive, e.g. in Bj{\"o}rck and Golub algorithm we pay $\OO(n^2+\ell^2)$ for each row. Our algorithm allows practitioners to run CCA on huge data sets because we reduce the cost of an extra row, making it not much more expensive than $\OO(n+\ell)$.
%
%
%%%%%%%%%%%%%%%%%%%%%%%%%%%%%%%%%%%%%%%%%%%%%%%%%%%%%%%
%%%%%%%%%%%%%%%%%%%%%%%%%%%%%%%%%%%%%%%%%%%%%%%%%%%%%%%
\paragraph{The Bj{\"o}rck and Golub Algorithm}
%
%%%%%%%%%%%%%%%%%%%%%%%%%%%%%%%%%%%%%%%%%%%%%%%%%%%%%%%
%%%%%%%%%%%%%%%%%%%%%%%%%%%%%%%%%%%%%%%%%%%%%%%%%%%%%%%
There are quite a few algorithms to compute the canonical correlations ~\cite{GZ95}. One of the most popular methods is due to Bj{\"o}rck and Golub~\cite{BG73}. It is based on the following observation.
\begin{theorem}\label{thm:bjork-golub}
Suppose $\matQ \in \RR^{m \times p}$ ($m \geq p$) and $\matW \in \RR^{m \times q}$ ($m \geq q$), both having orthonormal columns. The canonical correlations of $(\matQ, \matW)$ are the top $\min\{p,q\}$ singular values of $\matQ^\top \matW$.
\end{theorem}
The canonical correlations of the pair $(\matA,\matB)$ is a property of the subspace spanned by $\matA$ and $\matB$. So, Theorem~\ref{thm:bjork-golub} implies that once we have a pair of matrices $\matQ$ and $\matW$ with orthonormal columns whose column space spans the same column space of $\matA$ and $\matB$, respectively, then all we need is to compute the singular values of $\matQ^\top \matW$.  Bj{\"o}rck and Golub suggest the use of QR decompositions, but  $\matU_\matA$ and $\matU_\matB$ will serve as well. Both options require $ \OO \left(m\left(n^2 + \ell^2\right) \right)$ time; we use the latter approach here.
%
\begin{corollary}\label{cor:bjork-golub}
Frame Definition~\ref{def}. Then, for $i\in[q]$:
$
\sigma_i(\matA, \matB) = \sigma_i(\matU^\top_\matA \matU_\matB)\,.
$
\end{corollary}
%
%
%
%%%%%%%%%%%%%%%%%%%%%%%%%%%%%%%%%%%%%%%%%%%%%%%%%%%%%%%
%%%%%%%%%%%%%%%%%%%%%%%%%%%%%%%%%%%%%%%%%%%%%%%%%%%%%%%
\subsection{Perturbation Bounds for Matrix Products}\label{sec:pert}
%%%%%%%%%%%%%%%%%%%%%%%%%%%%%%%%%%%%%%%%%%%%%%%%%%%%%%%
%%%%%%%%%%%%%%%%%%%%%%%%%%%%%%%%%%%%%%%%%%%%%%%%%%%%%%%
%
This section states three technical lemmas which analyze the perturbation of the singular values of the product of a pair of matrices after dimensionality reduction.
These lemmas are essential for our analysis in subsequent sections, but they might be of independent interest as well.
%
\begin{lemma}\label{lem:pert4}
Let $\matA \in \RR^{m \times n}$ ($m \geq n$)  and $\matB \in \RR^{m \times \ell}$ ($m \geq \ell$). Define $\matC := [\matA ; \matB] \in \R^{m\times(n+\ell)}$, and suppose $\matC$ has rank $\omega$, so $\matU_{\matC}\in\R^{m\times\omega}$.  Let $\matS \in \R^{r \times m}$ be any matrix such that %$\rank( \matS \matU_{\matC})=\omega$
$ \sqrt{1-\eps} \leq \sigma_{\omega}\left(\matS \matU_{\matC} \right) \leq \sigma_1\left(\matS \matU_{\matC} \right)  \le \sqrt{1+\eps},$ for some $0 < \eps < 1$ . Then, for $i=1,\dots,\min(n,\ell)$,
\[|\sigma_i \left( \matA^\top\matB \right)  - \sigma_i\left( \matA^\top \matS ^\top \matS \matB \right)| \le \eps\cdot\norm{\matA}\cdot\norm{\matB}\,.\]
\end{lemma}
%
\begin{proof}
%\paragraph{Bounding $|\sigma_i \left( \matU_{\matA}^\top\matU_{\matB} \right)  - \sigma_i\left( \matU_{\matA}^\top \matS ^\top \matS \matU_{\matB} \right)|$.}
Using Weyl's inequality for the singular values of arbitrary matrices (Lemma~\ref{lem:pert2}) we obtain,
%
\begin{align*}
	| \sigma_i \left( \matA^\top\matB  \right)   - \sigma_i\left( \matA^\top \matS ^\top \matS \matB \right)|%\\
	 & \leq  \norm{  \matA^\top \matS ^\top \matS \matB - \matA^\top\matB }\\
     &  =    \norm{  \matV_{\matA}\matSig_{\matA}\left(\matU_{\matA}^\top \matS ^\top \matS \matU_{\matB}  - \matU_{\matA}^\top\matU_{\matB} \right)\matSig_{\matB} \matV^\top_{\matB} } \\
     & \leq  \norm{  \matU_{\matA}^\top \matS^\top \matS \matU_{\matB} - \matU_{\matA}^\top \matU_{\matB} } \cdot\norm{\matA}\cdot\norm{\matB}\,.
\end{align*}
%
Next, we argue that $ \norm{  \matU_{\matA}^\top \matS^\top \matS \matU_{\matB} - \matU_{\matA}^\top \matU_{\matB} } \le \norm{\matU_{\matC}^\top \matS^\top \matS \matU_{\matC}   - \matI_{\omega}}$.
Indeed, we now have
\begin{align*}
	\norm{  \matU_{\matA}^\top \matS^\top \matS \matU_{\matB} - \matU_{\matA}^\top \matU_{\matB} } %\\
	& =  \sup_{\norm{\w}=1,\ \norm{\z}=1} | \w^\top \matU_{\matA}^\top \matS^\top \matS \matU_{\matB} \z - \w^\top \matU_{\matA}^\top \matU_{\matB}\z | \\
	& =  \sup_{\norm{\x}=\norm{\y} = 1,\ \x\in{\mathcal{R}(\matU_{\matA})},\ \y\in{\mathcal{R} (\matU_{\matB})} } | \x^\top \matS^\top \matS \y - \x^\top \y | \\
	& \leq  \sup_{\norm{\x}=\norm{\y} = 1,\ \x\in{\mathcal{R}(\matU_{\matC})},\ \y\in{\mathcal{R}(\matU_{\matB}) }} | \x^\top \matS^\top \matS \y - \x^\top \y | \\
	& \leq  \sup_{\norm{\x}=\norm{y} = 1,\ \x\in{\mathcal{R}(\matU_{\matC})},\ \y\in{\mathcal{R}(\matU_{\matC})}} | \x^\top \matS^\top \matS \y - \x^\top \y | \\
	&   =   \sup_{\norm{\w}=1,\ \norm{\z}=1} | \w^\top \matU_{\matC}^\top \matS^\top \matS \matU_{\matC} \z  - \w^\top\matU_{\matC}^\top \matU_{\matC} \z | \\
	&   =   \norm{\matU_{\matC}^\top \matS^\top \matS \matU_{\matC}   - \matI_{\omega}}.
\end{align*}
In the above, all the equalities follow by the definition of the spectral norm of a matrix while the two inequalities follow
because $\mathcal{R}(\matU_{\matA}) \subseteq \mathcal{R}(\matU_{\matC})$ and $\mathcal{R}(\matU_{\matB}) \subseteq \mathcal{R}(\matU_{\matC})$, respectively.

To conclude the proof, recall that we assumed that for $i \in [\omega]$:
$1-\eps \le \lambda_i \left( \matU_{\matC}^\top \matS^\top \matS \matU_{\matC} \right) \le 1+\eps$.
\end{proof}

\begin{lemma}\label{lem:pert5}
Let $\matA \in \R^{m \times n}$ ($m \geq n$) and $\matB \in \R^{m \times \ell}$ ($m \geq \ell$).  Let $\matS \in \R^{r \times m}$ be any matrix such that $\rank{\matS \matA} = \rank{\matA}$ and $\rank{\matS \matB}=\rank{\matB}$, and all singular values of $\matS \matU_{\matA}$ and $\matS \matU_{\matB}$ are inside $[\sqrt{1-\eps},\sqrt{1+\eps}]$ for some
$0 < \eps < 1/2$.
Then, for $i=1,\dots,\min(n,\ell)$,
\[|\sigma_i\left( \matU_{\matA}^\top \matS^\top \matS \matU_{\matB} \right) -   \sigma_i \left( \matU_{\matS\matA}^\top\matU_{\matS \matB} \right) |
\le 2 \eps \left( 1 + \eps \right)\,.\]
\end{lemma}
\begin{proof}
For every $i=1,\ldots,q$ we have,
\begin{align*}
|\sigma_i\left( \matU_{\matA}^\top \matS^\top \matS \matU_{\matB} \right) - \sigma_i \left( \matU_{\matS\matA}^\top\matU_{\matS\matB} \right) |
 & = |\sigma_i\left(  \matSig_{\matA}^{-1} \matV_{\matA}^\top \matA^\top \matS^\top \matS \matB \matV_{\matB} \matSig_{\matB}^{-1}   \right)
 -  \sigma_i \left( \matSig_{\matS\matA}^{-1} \matV_{\matS\matA}^\top \matA^\top \matS^\top \matS \matB \matV_{\matS \matB} \matSig_{\matS \matB}^{-1} \right) | \\
& \leq  \gamma \cdot \sigma_i\left( \matSig_{\matA}^{-1} \matV_{\matA}^\top\matA^\top \matS^\top \matS \matB \matV_{\matB} \matSig_{\matB}^{-1} \right)
\ =\ \gamma \cdot \sigma_i\left( \matU_{\matA}^\top \matS^\top \matS \matU_{\matB} \right) \\
& \leq \gamma \cdot \norm{ \matU_{\matA}^\top \matS^\top } \cdot  \sigma_i\left( \matS \matU_{\matB} \right)
\leq \gamma \cdot  \left( 1 + \eps\right)
\end{align*}
with $ \gamma = \max(
\norm{\matSig_{\matS\matA}^{-1} \matV_{\matS\matA}^\top \matV_{\matA} \matSig_{\matA}^2 \matV_{\matA}^\top \matV_{\matS\matA} \matSig_{\matS\matA}^{-1} - \matI_p}  ,
\norm{\matSig_{\matS\matB}^{-1} \matV_{\matS\matB}^\top \matV_{\matB} \matSig_{\matB}^2 \matV_{\matB}^\top \matV_{\matS\matB} \matSig_{\matS\matB}^{-1} - \matI_q}
)\,.$
%

%
In the above, the first inequality follows using\footnote{Set $\matPsi = \matSig_{\matA}^{-1} \matV_{\matA}^\top \matA^\top \matS^\top \matS \matB \matV_{\matB} \matSig_{\matB}^{-1}$, $\matD_L := \matSig^{-1}_{\matS \matA}\matV_{\matS\matA}^\top \matV_{\matA} \matSig_{\matA}$ and $\matD_R:=\matSig_{\matB}\matV_{\matB}^\top \matV_{\matS\matB} \matSig_{\matS\matB}^{-1}$. $\matD_L$ and $\matD_R$ are non-singular, as a product of non-singular matrices. Moreover, $\matD_L\Psi \matD_R = \matSig_{\matS\matA}^{-1} \matV_{\matS\matA}^\top \matA^\top \matS^\top \matS \matB \matV_{\matS \matB} \matSig_{\matS \matB}^{-1}$, since $\matA = \matA \matV_{\matA}\matV_{\matA}^\top$, $\matB = \matB \matV_{\matB}\matV_{\matB}^\top$. }  Lemma~\ref{lem:pert1}, while the second follows
because for any two matrices $\matX, \matY:$ $\sigma_i(\matX \matY) \le \norm{\matX} \sigma_i(\matY)$.
Finally, in the third inequality we used the fact that $\norm{ \matU_{\matA}^\top \matS^\top } \le \sqrt{1+\eps}$ and $\sigma_i\left( \matS \matU_{\matB} \right) \le \sqrt{1+\eps}$.

We now bound
$\norm{\matSig_{\matS\matA}^{-1} \matV_{\matS\matA}^\top \matV_{\matA} \matSig_{\matA}^2 \matV_{\matA}^\top \matV_{\matS\matA} \matSig_{\matS\matA}^{-1} - \matI_p}$.
The second term in the max expression of $\gamma$ can be bounded in a similar fashion, so we omit the proof.
\begin{align*}
	 \norm{\matSig_{\matS\matA}^{-1} \matV_{\matS\matA}^\top \matV_{\matA} \matSig_{\matA}^2 \matV_{\matA}^\top \matV_{\matS\matA} \matSig_{\matS\matA}^{-1} - \matI_p}
	& =  \norm{\matSig_{\matS\matA}^{-1} \matV_{\matS\matA}^\top \matA^\top\matA  \matV_{\matS\matA} \matSig_{\matS\matA}^{-1} - \matI_p}\\
	& =  \norm{ \matU_{\matS\matA}^\top \tpinv{(\matS\matA)} \matA^\top \matA  \pinv{(\matS\matA )} \matU_{\matS\matA}
     -   \matU_{\matS \matA}^\top \matU_{\matS \matA}\matU_{\matS \matA}^\top \matU_{\matS \matA}}\\
	& =  \norm{ \matU_{\matS\matA}^\top \left( \tpinv{(\matS\matA)} \matA^\top \matA  \pinv{(\matS\matA )} - \matU_{\matS \matA} \matU_{\matS \matA}^\top\right) \matU_{\matS\matA}}\\
	& =  \norm{\tpinv{(\matS\matA)} \matA^\top \matA  \pinv{(\matS\matA )} - \matU_{\matS \matA} \matU_{\matS \matA}^\top}
\end{align*}
where we used $\matA^\top \matA = \matV_{\matA} \matSig_{\matA}^2 \matV_{\matA}^\top$, $\pinv{(\matS\matA )} \matU_{\matS\matA} = \matV_{\matS\matA} \matSig_{\matS\matA}^{-1}$.
Recall that, all the singular values of $\matS \matU_{\matA}$ are between $\sqrt{1 - \eps}$ and $\sqrt{1 + \eps}$ : $ (1-\eps) \matI_p \preceq \matU_{\matA}^\top \matS^\top \matS \matU_{\matA} \preceq (1+\eps ) \matI_p$.
%Conjugating the PSD ordering with $\matSig_{\matA} \matV_{\matA}^\top$ (see Lemma~\ref{lem:pert3}), it follows that
%\begin{equation}
%(1-\eps/2) \matA^\top \matA \preceq \matA^\top \matS^\top \matS \matA \preceq (1+\eps/2) \matA^\top \matA.
%\end{equation}
%Conjugating the PSD ordering with $\pinv{(\matS \matA)}$ (see Lemma~\ref{lem:pert3}), it follows that
%$$(1-\eps/2) \tpinv{(\matS\matA)} \matA^\top \matA  \pinv{(\matS\matA )}
%\preceq \matU_{\matS \matA} \matU_{\matS \matA}^\top \preceq (1+\eps/2) \tpinv{(\matS\matA)} \matA^\top \matA  \pinv{(\matS\matA )}$$
%since $\matS\matA \pinv{(\matS\matA)} = \matU_{\matS\matA} \matU_{\matS\matA}^\top$.
Conjugating the PSD ordering with $\matSig_{\matA} \matV_{\matA}^\top\pinv{(\matS \matA)}$ (see Lemma~\ref{lem:pert3}), it follows that
\begin{align*}
 (1  - \eps) \tpinv{(\matS\matA)} \matA^\top \matA  \pinv{(\matS\matA )}
  \preceq  \matU_{\matS \matA} \matU_{\matS \matA}^\top  \preceq  (1+\eps) \tpinv{(\matS\matA)} \matA^\top \matA  \pinv{(\matS\matA )}
\end{align*}
since $\matA^\top \matA = \matV_{\matA} \matSig^2_{\matA} \matV_{\matA}^\top $ and $\matS\matA \pinv{(\matS\matA)} = \matU_{\matS\matA} \matU_{\matS\matA}^\top$.
Rearranging terms, it follows that
\begin{align*}
	\frac1{1+\eps} \matU_{\matS \matA} \matU_{\matS \matA}^\top  & \preceq  \tpinv{(\matS\matA)} \matA^\top \matA  \pinv{(\matS\matA )}
	 \preceq  \frac1{1-\eps} \matU_{\matS \matA} \matU_{\matS \matA}^\top
\end{align*}
Since $0 < \eps < 1/2$, it holds that $\frac1{1-\eps/3}  \leq 1 + 2\eps$ and $\frac1{1+\eps} \geq 1 - \eps$, hence
\begin{align*}
-2\eps \matU_{\matS \matA} \matU_{\matS \matA}^\top
\preceq  \tpinv{(\matS\matA)} \matA^\top \matA  \pinv{(\matS\matA )} - \matU_{\matS \matA} \matU_{\matS \matA}^\top
 \preceq  2\eps \matU_{\matS \matA} \matU_{\matS \matA}^\top\,.
\end{align*}
This implies that $\norm{\tpinv{(\matS\matA)} \matA^\top \matA  \pinv{(\matS\matA )}  -  \matU_{\matS \matA} \matU_{\matS \matA}^\top}
  \leq  2\eps \norm{ \matU_{\matS \matA} \matU_{\matS \matA}^\top } = 2\eps$. Indeed, let $\x_+$ be the unit eigenvector of the symmetric matrix  $\tpinv{(\matS\matA)} \matA^\top \matA  \pinv{(\matS\matA )} - \matU_{\matS \matA} \matU_{\matS \matA}^\top$ corresponding to its maximum eigenvalue. The PSD ordering implies that
\begin{align*}
\lambda_{\max}\left( \tpinv{(\matS\matA)} \matA^\top \matA  \pinv{(\matS\matA )} - \matU_{\matS \matA} \matU_{\matS \matA}^\top \right)
\leq  2 \eps \x_+^\top \matU_{\matS \matA} \matU_{\matS \matA}^\top \x_+ \leq 2\eps \norm{\matU_{\matS \matA} \matU_{\matS \matA}^\top}= 2\eps.
\end{align*}
Similarly,
$\lambda_{\min}\left( \tpinv{(\matS\matA)} \matA^\top \matA  \pinv{(\matS\matA )} - \matU_{\matS \matA} \matU_{\matS \matA}^\top \right)> - 2\eps$, which shows the claim.
\end{proof}

\begin{lemma}\label{lem:pert6}
Repeat the conditions of Lemma~\ref{lem:pert4}.
Then, for all $\w \in \R^n$ and $\y \in \R^{\ell}$, we have
\[
\abs{ \w^\top \matA^\top \matB \y - \w^\top \matA^\top \matS^\top \matS \matB \y  } \leq \eps \cdot \norm{\matA \w} \cdot \norm{\matB \y}.
\]
\end{lemma}
\begin{proof}
Let $\matE =  \matU_{\matA}^\top \matS^\top \matS \matU_{\matB} - \matU_{\matA}^\top \matU_{\matB}$. Now,
%$ \abs{ \w^\top \matA^\top \matB \y - \w^\top \matA^\top \matS^\top \matS \matB \y  } =$
\begin{align*}
\abs{ \w^\top \matA^\top \matB \y - \w^\top \matA^\top \matS^\top \matS \matB \y  } & =   \abs{ \w^\top \matV_{\matA}\matSig_{\matA} \matE \matSig_{\matB} \matV_{\matB}^\top \y  }
\leq  \norm{\w^\top \matV_{\matA}\matSig_{\matA}} \norm{\matE} \norm{\matSig_{\matB} \matV_{\matB}^\top \y}\\
&=    \norm{\w^\top \matV_{\matA}\matSig_{\matA}\matU_{\matA}^\top} \norm{\matE} \norm{\matU_{\matB}\matSig_{\matB} \matV_{\matB}^\top \y}
=    \norm{\w^\top \matA^\top} \norm{\matE} \norm{\matB \y}\\
&=    \norm{\matE}  \norm{\matA \w} \norm{\matB \y}
\end{align*}
Now, Lemma 7
%Lemma~\ref{lem:pert4}
ensures that $\norm{\matE} \le \eps$.
\end{proof}

%
%%%%%%%%%%%%%%%%%%%%%%%%%%%%%%%%%%%%%%%%%%%%%%%%%%%%%%%
%%%%%%%%%%%%%%%%%%%%%%%%%%%%%%%%%%%%%%%%%%%%%%%%%%%%%%%
\subsection{CCA of Row Sampled Pairs}
%%%%%%%%%%%%%%%%%%%%%%%%%%%%%%%%%%%%%%%%%%%%%%%%%%%%%%%
%%%%%%%%%%%%%%%%%%%%%%%%%%%%%%%%%%%%%%%%%%%%%%%%%%%%%%%
%

Given $\matA$ and $\matB$, one straightforward way to accelerate CCA is to sample rows uniformly from both matrices, and to compute the CCA of the smaller matrices. %Most methods for computing the canonical correlations benefit from smaller matrices.
In this section we show that if we sample enough rows, then the canonical correlations of the sampled pair are close to the canonical correlations of the original pair. Furthermore, the canonical weights of the sampled pair can be used to find approximate canonical vectors. Not surprisingly, the sample size depends on the coherence. More specifically, it depends on the coherence of $[\matA ; \matB]$.

\begin{theorem}\label{thm1}
Suppose $\matA \in \R^{m \times n}$ ($m \geq n$) has rank $p$ and $\matB \in \R^{m \times \ell}$ ($m \geq \ell$) has rank $q \le p$. Let $0 < \eps < 1/2$ be an accuracy parameter and $0 < \delta < 1$ be a failure probability parameter. Let $\omega = \rank{[\matA ; \matB]} \leq p+q$. Let $r$ be an integer such that
\[
54 \eps^{-2} m  \mu([\matA ; \matB]) \log ( 12 \omega /\delta) \leq r \leq m \,.
\]
Let $T$ be a random subset of $[m]$ of cardinality $r$, drawn from a uniform distribution over such subsets,
and let $\matS \in \R^{r \times m}$ be the sampling matrix corresponding to $T$ rescaled by $\sqrt{m/r}$.
Denote $\hat{\matA} = \matS \matA$ and $\hat{\matB} = \matS  \matB$.

Let $\hat{\sigma}_1,\dots,\hat{\sigma}_q$  be the exact canonical correlations of $(\hat{\matA}, \hat{\matB})$,
and let
\[
\w_1=\hat{\x}_1 / \norm{\hat{\matA} \hat{\x}_1}, \dots, \w_p=\hat{\x}_q / \norm{\hat{\matA} \hat{\x}_q}\,,
\quad \text{and}\quad
\p_1=\hat{\y}_1 / \norm{\hat{\matB} \hat{\y}_1}, \dots, \p_q = \hat{\y}_q / \norm{\hat{\matB} \hat{\y}_q}
\]
be the exact canonical weights of $(\hat{\matA}, \hat{\matB})$. With probability of at least $1-\delta$ all the following hold simultaneously:
\begin{enumerate}[(a)]
    \item
    (Approximation of Canonical Correlations) For every $i=1,2,\ldots ,q$: $ | \sigma_i\left(\matA, \matB \right) - \sigma_i\left( \matS \matA, \matS  \matB \right)| \le  \eps + 2\eps^2 / 9 = O(\eps)\,.$
	\item
	(Approximate Orthonormal Bases) The vectors $\{\matA \w_i\}_{i\in{[q]}}$ form an approximately orthonormal basis. That is,
    for any $c \in [q]$,
%
\[\frac{1}{1+\eps/3} \leq \norm{\matA \w_c}^2 \leq \frac{1}{1-\eps/3}\,,\]
%
and for any $i\neq j$,
\[|
\ip{\matA \w_i}{ \matA \w_j}| \leq \frac{\eps}{3 - \eps}.
\]
Similarly, for the set of $\{\matB \p_i\}_{i\in{[q]}}$.
	\item
(Approximate Correlation) For every $i=1,2,\ldots ,q$:
\[
\frac{\sigma_i(\matA,\matB)}{1+\eps/3} - \frac{\eps/3}{1-\eps/9} \leq \sigma(\matA \w_i, \matB \p_i) \leq \frac{\sigma_i(\matA,\matB)}{1-\eps/3} + \frac{\eps/3}{(1-\eps/3)^2}\,.
\]
\end{enumerate}
\end{theorem}
%
%
\begin{proof}
Let $\matC :=  [\matU_{\matA} ; \matU_{\matB}]$. Lemma~\ref{lemma:sampling-ortho} implies that each of the following three assertions hold with probability of at least $1-\delta/3$, hence all three hold simultaneously with probability of at least $1-\delta$:
\begin{itemize}
\item For every $r\in[p]$: $1-\eps/3 \le \sigma_r^2(\matS \matU_{\matA}) \le  1+\eps/3\,.$
\item For every $k\in[q]$: $1-\eps/3 \le \sigma_k^2(\matS \matU_{\matB}) \le  1+\eps/3\,.$
\item For every $h\in[\omega]$: $1-\eps/3 \le \sigma_h^2(\matS \matU_{\matC}) \le  1+\eps/3\,.$
\end{itemize}
We now show that if indeed all three hold, then (a)-(c) hold as well.

{\bf Proof of (a).} Corollary~\ref{cor:bjork-golub} implies that
$\sigma_i(\matA, \matB) = \sigma_i(\matU^\top_\matA \matU_\matB)$ and
$\sigma_i(\matS \matA, \matS \matB) = \sigma_i(\matU^\top_{\matS \matA} \matU_{\matS \matB})$.
We now use the triangle inequality to get,
\begin{align*}
| \sigma_i \left( \matA, \matB \right) -  \sigma_i \left(\matS \matA, \matS\matB \right) |
    & =   | \sigma_i \left( \matU_{\matA}^\top\matU_{\matB} \right) -   \sigma_i \left( \matU_{\matS\matA}^\top\matU_{\matS\matB} \right) |  \\
%    & =  | \sigma_i \left( \matU_{\matA}^\top\matU_{\matB} \right)  - \sigma_i\left( \matU_{\matA}^\top \matS^\top \matS \matU_{\matB} \right) +  \sigma_i\left( \matU_{\matA}^\top \matS^\top \matS \matU_{\matB} \right) -   \sigma_i \left( \matU_{\matS\matA}^\top\matU_{\matS\matB} \right) | \\
    & \leq   | \sigma_i \left( \matU_{\matA}^\top\matU_{\matB} \right)  - \sigma_i\left( \matU_{\matA}^\top \matS^\top \matS \matU_{\matB} \right)|
 	   +    |\sigma_i\left( \matU_{\matA}^\top \matS^\top \matS \matU_{\matB} \right) -   \sigma_i \left( \matU_{\matS\matA}^\top\matU_{\matS\matB} \right) |.
\end{align*}
To conclude the proof, use Lemma~\ref{lem:pert4} and Lemma~\ref{lem:pert5} to bound these two terms, respectively.

{\bf Proof of (b).} For any $c \in [q]$, $\norm{\matA \w_c} = \norm{\matA \w_c} / \norm{\hat{\matA} \w_c}$ since $\norm{\hat{\matA} \w_c} = 1$. Now use Lemma~\ref{lem:pert6}. For any $i\neq j$
\begin{align*}
	|\ip{\matA \w_i}{ \matA \w_j }| & \leq  |\w_i^\top \hat{\matA}^\top \hat{\matA}  \w_j|
	   +   |\w_i^\top (\hat{\matA}^\top \hat{\matA} - \matA^\top \matA)\w_j|
       =   |\w_i^\top (\hat{\matA}^\top \hat{\matA} - \matA^\top \matA)\w_j| \\
	& \leq  \frac{\eps}{3} \norm{\matA \w_i} \norm{\matA \w_j}
	 \leq  \frac{\eps/3}{1-\eps/3} \norm{\hat{\matA}\w_i}\norm{\hat{\matA}\w_j}
	   =   \frac{\eps}{3-\eps}.
\end{align*}
In the above, we used the triangle inequality, the fact that the $\w_i$'s are the canonical weights of $\hat{\matA}$, and Lemma~\ref{lem:pert6}.

{\bf Proof of (c).} We only prove the upper bound. The lower bound is similar, and we omit it.\\
\small
%$\sigma\left(\matA \w_i, \matB \p_i\right) = $
\begin{align*}
\sigma\left(\matA \w_i, \matB \p_i\right)	 & =  \frac{\ip{\matA\w_i}{\matB\p_i}}{\norm{\matA\w_i}\norm{\matB\p_i}}
					 						 \leq \frac{1}{1-\eps/3}\cdot\ip{\matA\w_i}{\matB\p_i}
                                               =   \frac{1}{1-\eps/3}\cdot \left( \ip{\hat{\matA}\w_i}{\hat{\matB}\p_i}
                                                 +  \w^\top_i\left(\matA^\top\matB - \hat{\matA}^\top\hat{\matB} \right)\p_i \right) \\
                                              &   \leq   \frac{\sigma\left(\hat{\matA} \x_i, \hat{\matB}\y_i\right)}{1-\eps/3}
                                                 +   \frac{\eps/3}{1-\eps/3}\cdot\norm{\matA\w_i}\cdot\norm{\matB\p_i}
											     \leq   \frac{\sigma\left(\hat{\matA} \w_i, \hat{\matB}\p_i\right)}{1-\eps/3} + \frac{\eps/3}{(1-\eps/3)^2}
\end{align*}
In the above,
the first equality follows by the definition of $\sigma(\cdot,\cdot)$,
the first inequality by using $1=\norm{\hat{\matA}\w_i}^2 \leq (1+\eps) \norm{\matA\w_i}^2$ (same holds for $\matB\p_i$),
the second inequality from Lemma~\ref{lem:pert6},
the third inequality  by using $(1-\eps)\norm{\matA\w_i}^2\leq \norm{\hat{\matA} \w_i}^2 =1$ (same holds for $\matB\p_i$),
and the last inequality by (a).
\end{proof}





\subsection{Fast Approximate CCA}\label{sec:alg}

First, we define what we mean by approximate CCA.
\begin{definition}[Approximate CCA] \label{def:approxCCA}
For $0 \leq \eta \leq 1$, an {\em $\eta$-approximate CCA of $(\matA, \matB)$}, is a set of positive numbers $\hat{\sigma}_1,\dots,\hat{\sigma}_q$ together with a set of vectors $\w_1,\dots,\w_q$ for $\matA$ and a set of vectors $\p_1,\dots,\p_q$ for $\matB$, such that
\begin{enumerate}[(a)]
\item For every $i\in[q]$, $|\sigma_i(\matA, \matB) - \hat{\sigma}_i | \leq \eta\,.$
\item For every $i\in[q]$, \[|\norm{\matA \w_i}^2 - 1 | \leq \eta\,,\] and for $i\neq j$, \[|\ip{\matA \w_i}{ \matA \w_j}| \leq \eta\,.\] Similarly, for the set of $\{\matB \p_i\}_{i\in{[q]}}$.
\item For every $i\in[q]$, $|\sigma_i(\matA, \matB) - \sigma(\matA \w_i, \matB \p_i) | \leq \eta\,.$
\end{enumerate}
\end{definition}
We are now ready to present our fast algorithm for approximate CCA of a pair of tall-and-thin matrices. Algorithm~\ref{alg:approx} gives the pseudo-code description of our algorithm.

The analysis in the previous section (Theorem~\ref{thm1}) shows that if we sample enough rows, the canonical correlations and weights of the sampled matrices are an $O(\eps)$-approximate CCA of $(\matA, \matB)$. However, to turn this observation into a concrete algorithm we need an upper bound on the coherence of $[\matA ; \matB]$. It is conceivable that in certain scenarios such an upper bound might be known in advance, or that it can be computed quickly~\cite{DMMW12}. However, even if we know the coherence, it might be as large as one, which will imply that sampling the entire matrix is needed.

To circumvent this problem, our algorithm uses the RHT to reduce the coherence of the matrix pair before sampling rows from it. In particular, instead of sampling rows from $(\matA, \matB)$  we sample rows from $(\matTh \matA, \matTh\matB)$, where $\matTh$ is a RHT matrix (Definition~\ref{def:rht}). This unitary transformation bounds the coherence with high probability, so we can use Theorem~\ref{thm1} to compute the number of rows required for an $O(\eps)$-approximate CCA.
We now sample the transformed pair $(\matTh \matA, \matTh\matB)$ to obtain $(\hat{\matA}, \hat{\matB})$. Now the canonical correlations and weights of $(\hat{\matA}, \hat{\matB})$ are computed and returned.
\begin{algorithm}[t]
\caption{Fast Approximate CCA}
\label{alg:approx}
\begin{algorithmic}[1]
\State {\bf Input:} $\matA \in \RR^{m \times n}$ of rank $p$, $\matB \in \RR^{m \times \ell}$ of rank $q$, $0< \eps < 1/2$, and $\delta$ ($n\geq l$, $p \ge q$).
\medskip
\State $r \longleftarrow \min(54\eps^{-2}\left[\sqrt{n+\ell} + \sqrt{8\log(12m/\delta)} \right]^2 \log (3(n+\ell)/\delta), m)$
\State Let $\matS$ be the sampling matrix of a random subset of $[m]$ of cardinality $r$ (uniform distribution).
\State Draw a random diagonal matrix $\matD$ of size $m$ with $\pm 1$ on its diagonal with equal probability.
\State $\hat{\matA} \longleftarrow \matS \matH \cdot (\matD \matA)$ using fast subsampled WHT (see Section~\ref{sec:wht}).
\State $\hat{\matB} \longleftarrow \matS \matH \cdot (\matD \matB)$ using fast subsampled WHT (see Section~\ref{sec:wht}).
\State Compute and return the canonical correlations and the canonical weights of $( \hat{\matA},\hat{\matB} )$
(e.g. using Bj{\"o}rck and Golub's algorithm).
\end{algorithmic}
\end{algorithm}

\begin{theorem}\label{thm:alg}
With probability of at least $1-\delta$, Algorithm~\ref{alg:approx} returns an $O(\eps)$-approximate CCA of $(\matA, \matB)$. Assuming Bj{\"o}rck and Golub's algorithm is used in line 7, Algorithm~\ref{alg:approx} runs in time
$$O\left(  m n \log{m} +  \eps^{-2}\left[\sqrt{n} + \sqrt{\log(m/\delta)}\right]^2 \log(n/\delta) n^2\right)\,.$$
\end{theorem}

\begin{proof}
%First we argue about the quality of approximation of Algorithm~\ref{alg:approx} and then about its time complexity.
Lemma~\ref{lem:rht-reduce} ensures that with probability of at least $1-\delta/2$,
$$\mu([\matTh \matA; \matTh \matB]) \leq \frac{1}{m} \left( \sqrt{n+\ell} + \sqrt{8\log(3m/\delta)} \right)^2\,.$$
Assuming that the last inequality holds, Theorem~\ref{thm1} ensures that with probability of at least $1-\delta/2$,
the canonical correlations and weights of $(\hat{\matA},  \hat{\matB})$ form an $O(\eps)$-approximate CCA of
$(\matTh \matA, \matTh \matB)$. By the union bound, both events hold together with probability of at least $1-\delta$.
The RHT transforms applied to $\matA$ and $\matB$ are unitary, so for every $\eta$, an $\eta$-approximate CCA of $(\matTh \matA, \matTh \matB)$ is also an $\eta$-approximate CCA of $(\matA, \matB)$ (and vice versa).

{\bf Running time analysis.}
Step 2 takes $O(1)$ operations. Step 3 requires $O(r)$ operations.
Step 4 requires $O(m)$ operations.
Step 5 involves the multiplication of $\matA$ with $\matS \matH \matD$ from the left.
Computing $\matD \matA$ requires $O(mn)$ time. Multiplying $\matS \matH$ by $\matD \matA$ using fast subsampled WHT requires
$O(m n \log r )$ time, as explained in Section~\ref{sec:wht}.
Similarly, step 6 requires $O( m \ell \log r )$ operations.
Finally, step 7 takes $O( r n \ell + r (n^2 + \ell^2))$ time.  Assuming that $n \geq \ell$, the total running time is
$O(r n^2 + m n \log(r))$. Plugging the value for $r$, and using the fact that $r \leq m$, established our running time bound.

%Computing $\hat{\matA}$ involves flipping the signs of $\matA$ ($O(mn)$ operations), computing a WHT transform of an $m \times n$ matrix ($O(mn \log(m))$ operations), and sampling $r$ rows ($O(rn)$ operations). However, we are interested only in the sample rows, so by coupling the WHT and the sampling, we can compute $\hat{\matA}$ in
%$O(m n \log(r))$ time\footnote{In practice it might be better to compute all entries since it allows the use of optimized libraries.}. Similarly, $O(m \ell \log(r))$ time for $\hat{\matB}$.

%There is more than one way to compute the correlations of $(\hat{\matA}, \hat{\matB})$. If we use Bj{\"o}rck and Golub's algorithm, the running time is $O(r (n^2 + \ell^2))$. Since we assume that $n \geq \ell$, the total running time is
%$O(r n^2 + m n \log(r))$, where $r = \min(54\eps^{-2}\left[\sqrt{n+\ell} + \sqrt{8\log(12m/\delta)} \right]^2 \log (3(n+\ell)/\delta), m)$. Plugging in the value of $r$, and using the fact that $r \leq m$, we find that the total running time is $O\left( \eps^{-2}\left[\sqrt{n} + \sqrt{\log{m/\delta}}\right]^2 \log(n/\delta) n^2 + m n \log{m}\right)$.

%\vspace*{-0.05in}
\end{proof}

%\vspace*{-0.15in}

% Removal candidate:
From a practical point of view, our algorithm is useful
for measuring the size of the correlated subspace, and obtaining
the principal vectors of it. $\eps$ is $0.1$, or perhaps $0.01$. So for reasonably high correlations,
say above $0.2$, we get some useful information. However, for lower correlations we
get no information at all. Furthermore, it is too expensive to compute
all the principal vectors, but once we know the size of the correlated subspace
we can use the approximate weights to compute the vectors for that subspace.

%%%%%%%%%%%%%%%%%%%%%%%%%%%%%%%%%%%%%%%%%%%%%%%%%%%%%%%%%%%%%%%%%%
% Optimality of bounds
%%%%%%%%%%%%%%%%%%%%%%%%%%%%%%%%%%%%%%%%%%%%%%%%%%%%%%%%%%%%%%%%%%
\subsection{Relative vs. Additive Error}\label{sec:error:lowerbound}
%%%%%%%%%%%%%%%%%%%%%%%%%%%%%%%%%%%%%%%%%%%%%%%%%%%%%%%%%%%%%%%%%%
Now, we demonstrate that, unless $r \approx m$, it is not possible to replace the additive error guarantees of Theorem~\ref{thm:alg} with relative error guarantees.
%
\begin{lemma}
Assume that given any matrix pair ($\matA$, $\matB$) and any constant $0<\eps <1$, Algorithm~\ref{alg:approx} computes a pair $(\hat{\matA}, \hat{\matB})$ by setting a sufficient large value for $r$ in Step $2$ so that the canonical correlations are relatively preserved with constant probability, i.e., with constant probability:
\begin{align*}
	(1-\eps) \sigma_i(\matA,\matB) \leq \sigma_i(\hat{\matA},\hat{\matB}) \leq (1+\eps) \sigma_i(\matA,\matB), \quad i=1,\ldots ,q.
\end{align*}
Then, it follows that $r = \Omega( m/\log(m))$.
\end{lemma}
%
%
\begin{proof}
The proof follows by a reduction to the set disjointness communication complexity problem. In particular, assume that Alice gets an $\x\in\{0,1\}^m$ as input and Bob gets $\y\in\{0,1\}^m$. Their goal is to decide if there exists $i\in{[m]}$ so that $x_i = y_i = 1$ by exchanging as less information as possible. It is known that the randomized communication complexity of this problem is $\Omega(m)$, see~\cite{DISJ} for a modern proof.

Set $\eps = 1/2$ and let $\delta$ be a constant in Algorithm~\ref{alg:approx}. Now, Alice and Bob can compute $\widetilde{\x} = \sqrt{m}\matS \matH\matD \x$ and $\widetilde{\y} = \sqrt{m}\matS \matH\matD \y$, respectively (using shared randomness). Then, Alice sends to Bob $\widetilde{\x}$. With constant probabilty, it holds
%
	\[ \frac1{2}\frac{\ip{\x}{\y}}{\norm{\x}\norm{\y}} \leq \frac1{r} \frac{\ip{\widetilde{\x}}{\widetilde{\y}}}{\norm{\widetilde{\x}}\norm{\widetilde{\y}}} \leq \frac{3}{2} \frac{\ip{\x}{\y}}{\norm{\x}\norm{\y}}. \]
%
Now, Bob can decide if there exists $i$, so that $x_i=y_i =1$ by checking if $\ip{\widetilde{\x}}{\widetilde{\y}}$ is zero or not. Hence, this protocol decides the set disjointness problem. Now, since $\sqrt{m}\matS\matH \matD$ is an $r\times m$ matrix with entries from $\{-1,+1\}$, it follows that $\infnorm{\widetilde{\x}} \leq m$. Therefore, we can encode $\widetilde{\x}$ using at most $r \log(2m)$ bits. It follows by the linear lower bound for set disjointness that $r\log(2m) \geq C m$ for some constant $C>0$.
\end{proof}
%
%
%
%
 % Randomized Linear Algebraic Primitives
%%%%%%%%%%%%%%%%%%%%%%%%%%%%%%%%%%%%%%%%%%%%%%%%%%%%%%%
%%%%%%%%%%%%%%%%%%%%%%%%%%%%%%%%%%%%%%%%%%%%%%%%%%%%%%%
\chapter{Matrix Algorithms}\label{chap:ma}
%%%%%%%%%%%%%%%%%%%%%%%%%%%%%%%%%%%%%%%%%%%%%%%%%%%%%%%
%%%%%%%%%%%%%%%%%%%%%%%%%%%%%%%%%%%%%%%%%%%%%%%%%%%%%%%
In this chapter\footnote{Section~\ref{sec:LS} appeared in~\cite{chernoff:matrix_valued:MZ11} (joint work with Avner Magen) and in~\cite{REK} (joint work with Nick Freris). The section on the element-wise matrix sparsification problem appeared in~\cite{matrix:sparsification:IPL2011} (joint work with Petros Drineas). The fast isotropic vector sparsification algorithm appeared in~\cite{ICALP12}.}, we develop and analyze randomized approximation algorithms for two matrix computational problems; the least squares problem (also known as linear regression) and the element-wise matrix sparsification problem. Moreover, we present a deterministic algorithm for isotropic vector sparsification and, as a consequence, spectral sparsification.
%
%
%%%%%%%%%%%%%%%%%%%%%%%%%%%%%%%%%%%%%%%%%%
%%%%%%%%%%%%%%%%%%%%%%%%%%%%%%%%%%%%%%%%%%
%		l_2 regression
%%%%%%%%%%%%%%%%%%%%%%%%%%%%%%%%%%%%%%%%%%
%%%%%%%%%%%%%%%%%%%%%%%%%%%%%%%%%%%%%%%%%%
\section{Randomized Approximate Least Squares}\label{sec:LS}
%%%%%%%%%%%%%%%%%%%%%%%%%%%%%%%%%%%%%%%%%%
%%%%%%%%%%%%%%%%%%%%%%%%%%%%%%%%%%%%%%%%%%
%
Let $\matA$ be an $m\times n$ non-zero real matrix and $\b$ be a real vector of size $m$. In the present section we analyze randomized algorithms for the least squares problem. The least squares problem is formally defined as follows:
\begin{equation}\label{eq:ls}
\text{Compute }\x\in \RR^n \text{ that is a minimizer of } \min_{\x\in\RR^n} \norm{\matA \x - \b}^2.
\end{equation}
To ensure uniqueness on the above minimization problem, it suffices to impose the requirement of returning a minimizing vector $\x$ of Eqn.~\eqref{eq:ls} that additionally has the minimum Euclidean norm. In this case, the minimum Euclidean norm vector that minimizes~Eqn.~\eqref{eq:ls} equals to $\xls=\pinv{\matA}\b$. Recall that the standard direct methods for computing $\xls$ require $\OO(mn^2)$ arithmetic operations. The main objective here is to design faster algorithms that compute an approximation to $\xls$.
%

%
Here, two randomized algorithms are presented; each of which exploits randomness in a different manner. The first algorithm is effective in the case of \emph{overdetermined} linear systems\footnote{By overdetermined linear systems, we called linear systems that have much more constraints than variables.} and it is based on the dimensionality reduction paradigm~\cite{l2_regression:drineas06,sarlos,low_rank:STOC09,CW_stoc09,drineas:tensor_sparsification,chernoff:matrix_valued:MZ11,fasterLS}. The first algorithm is due to~\cite{sarlos}; here the main contribution is that we obtain tighter analysis than the analysis of~\cite{sarlos}. It is worth-mentioning that several surprising dimensionality reduction techniques have been quite recently obtained in which the projection step can be performed in input sparsity time, see~\cite{ls:nnzA,OSNAP,MP12}. The second algorithm which we call \emph{randomized extended Kaczmarz} (REK) is a randomized iterative algorithm that exponentially converges to $\xls$ in expectation.
%

%
%
%%%%%%%%%%%%%%%%%%%%%%%%%%%%%%%%%%%%%%%%%%%%%%%%%%%%%%%
%%%%%%%%%%%%%%%%%%%%%%%%%%%%%%%%%%%%%%%%%%%%%%%%%%%%%%%
\subsubsection*{Least squares solvers}\label{sec:related}
%%%%%%%%%%%%%%%%%%%%%%%%%%%%%%%%%%%%%%%%%%%%%%%%%%%%%%%
%%%%%%%%%%%%%%%%%%%%%%%%%%%%%%%%%%%%%%%%%%%%%%%%%%%%%%%
%
We give a brief discussion on least squares solvers including deterministic direct and iterative algorithms together with recently proposed randomized algorithms. For a detailed discussion on deterministic methods, the reader is referred to~\cite{book:Bjork}. In addition, we place our contributions in context with prior work.
\paragraph{Deterministic algorithms}
In the literature, several methods have been proposed for solving least squares problems of the form~\eqref{eq:ls}. Here we briefly describe a representative sample of such methods including the use of QR factorization with pivoting, the use of the singular value decomposition (SVD) and iterative methods such as Krylov subspace methods applied on the normal equations~\cite{book:Saad}. LAPACK provides robust implementations of the first two methods; DGELSY uses QR factorization with pivoting and DGELSD uses the singular value decomposition~\cite{LAPACK}. For the iterative methods, LSQR is equivalent to applying the conjugate gradient method on the normal equations~\cite{PS82} and it is a robust and numerically stable method.
\paragraph{Randomized algorithms}
To the best of our knowledge, most randomized algorithms proposed in the theoretical computer science literature for approximately solving least squares are mainly based on the following generic two step procedure: first randomly (and efficiently) project the linear system into sufficiently many dimensions, and second return the solution of the down-sampled linear system as an approximation to the original optimal solution~\cite{petrosSODA06,sarlos,CW09,fasterLS}, see also~\cite{ls:nnzA}. Concentration of measure arguments imply that the optimal solution of the down-sampled system is close to the optimal solution of the original system. The accuracy of the approximate solution using this approach depends on the sample size and to achieve relative accuracy $\eps$, the sample size should depend inverse polynomially on $\eps$. This fact implies that these approaches are unsuitable for the high-precision regime of error.

A different approach is the so called randomized preconditioning method, see~\cite{RT08,AMT10}. The authors of~\cite{AMT10} implemented Blendenpik, a high-precision least squares solver. Blendenpik consists of two steps. In the first step, the input matrix is randomly projected and an effective preconditioning matrix is extracted from the projected matrix. In the second step, an iterative least squares solver such as the LSQR algorithm of Paige and Saunders~\cite{PS82} is applied on the preconditioned system. Blendenpik is effective for overdetermined and underdetermined problems.

A parallel iterative least squares solver based on normal random projections called LSRN was recently implemented by Meng, Saunders and Mahoney~\cite{lsrn}. LSRN consists of two phases. In the first preconditioning phase, the original system is projected using random normal projection from which a preconditioner is extracted. In the second step, an iterative method such as LSQR or the Chebyshev semi-iterative method~\cite{Chebyshev} is applied on the preconditioned system. This approach is also effective for over-determined and under-determined least squares problems assuming the existence of a parallel computational environment.
%

%
A detailed numerical evaluation of the randomized extended Kaczmarz method has been obtained in~\cite{REK}. Here we highlight only its main points. In~\cite{REK}, the randomized extended Kaczmarz algorithm was compared against DGELSY, DGELSD, Blendenpik. LSRN~\cite{lsrn} did not perform well under a setup in which no parallelization is allowed. The numerical evaluation of Section~\cite[Section 5]{REK} indicates that the randomized extended Kaczmarz
is effective on the case of sparse, well-conditioned and strongly rectangular (both overdetermined and underdetermined) least squares problems. On the other hand, a preconditioned version of the randomized extended Kaczmarz did not perform well under the case of ill-conditioned matrices.
%
%%%%%%%%%%%%%%%%%%%%%%%%%%%%%%%%%%%%%%%%%%
%%%%%%%%%%%%%%%%%%%%%%%%%%%%%%%%%%%%%%%%%%
\subsection{Dimensionality Reduction for Least Squares}
%%%%%%%%%%%%%%%%%%%%%%%%%%%%%%%%%%%%%%%%%%
%%%%%%%%%%%%%%%%%%%%%%%%%%%%%%%%%%%%%%%%%%
We present an approximation algorithm for the least-squares regression problem; given an $m\times n$, $m\gg n$, matrix $\matA$ of rank $r$ and a vector $\b\in\RR^m$ we want to compute $\xls=\pinv{\matA}\b$ that minimizes $\norm{\matA\x-\b}$ over all $\x\in\RR^n$ and has the minimum Euclidean norm. In the paper~\cite{l2_regression:drineas06}, Drineas et al. show that if we non-uniformly sample $t=\Omega(n^2/\eps^2)$ rows from $\matA$ and $\b$, then with high probability the optimum solution of the $t\times n$ sampled problem will be within $(1+\eps)$ close to the original problem. The main drawback of their approach is that finding or even approximating the sampling probabilities is computationally intractable, i.e., requires $\OO(mn^2)$ operations. Sarlos~\cite{sarlos} improved the above bound to $t=\Omega( n\log n/\eps^2)$ and gave the first $o(mn^2)$ relative error approximation algorithm for this problem.
%

%
In the next theorem we eliminate the extra logarithmic multiplicative factor from Sarlos bounds and replace the dimension (number of variables) $n$ with the rank $r$ of the constraints matrix $\matA$. We should point out that independently, the same bound as our Theorem~\ref{thm:ell2_regression} was obtained by Clarkson and Woodruff~\cite{CW_stoc09} (see also~\cite{fasterLS} and~\cite{ls:nnzA} for more recent improvements). The proof of Clarkson and Woodruff uses heavy machinery and a completely different approach. In a nutshell they manage to improve the matrix multiplication bound with respect to the Frobenius norm. They achieve this by bounding higher moments of the Frobenius norm of the approximation  viewed as a random variable instead of bounding the \emph{local} differences for each coordinate of the product. To do so, they rely on intricate moment calculations spanning over four pages, see~\cite{CW_stoc09} for more. On the other hand, the proof of the present $\ell_2$-regression bound uses only basic matrix analysis, elementary deviation bounds and $\eps$-net arguments. More precisely, we argue that Theorem~\ref{thm:matrixmult} (\textit{i.a}) immediately implies that by randomly-projecting to dimensions linear in the intrinsic dimensionality of the constraints, i.e., the rank of $\matA$, is sufficient as the following theorem indicates.
%%%%%%%%%%%%%%%%%%%%%%%%%%%%%%%%%%%%%%%%%%
%		l_2 regression
%%%%%%%%%%%%%%%%%%%%%%%%%%%%%%%%%%%%%%%%%%
%%%%%%%%%%%%%%%%%%%%%%%%%%%%%%%%%%%%%%%%%%
\begin{theorem}\label{thm:ell2_regression}
Let $\matA\in{\RR^{m\times n}}$ be a matrix of rank $r$ and $\b\in\RR^m$. Let $0<\eps<1/3$, $0<\delta <1$, $\matR$ be a $t\times m$ random sign matrix rescaled by $1/\sqrt{t}$ and $\widetilde{\x}_{\text{opt}}=\pinv{(\matR\matA)} \matR\b$.
\begin{itemize}
 \item
If $t=\Omega(\frac{r}{\eps}\log (1/\delta))$, then with probability at least $1-\delta$,
\begin{equation}\label{ineq:regression:approx}
 \norm{\b-\matA\widetilde{\x}_{\text{opt}}} \leq (1+\eps) \norm{\b-\matA \xls }.
\end{equation}
\item
If $t=\Omega(\frac{r}{\eps^2}\log(1/\delta))$, then with probability at least $1-\delta$,
\begin{equation}\label{ineq:regression:x_opt}
 \norm{\x_{\text{opt}} - \widetilde{\x}_{\text{opt}}} \leq \dfrac{\eps}{\sigma_{\min}(\matA)}\norm{\b-\matA\xls}.
\end{equation}
\end{itemize}
\end{theorem}
\begin{remark}
The above result can be easily generalized to the case where $\b$ is an $m\times p$ matrix $\matB$ of rank at most $r$ (see proof). This is known as the generalized $\ell_2$-regression problem in the literature, i.e., $\arg\min_{\matX\in{n\times p}}\norm{\matA\matX-\matB}$ where $\matB$ is an $m\times p$ rank $r$ matrix.
\end{remark}
%%%%%%%%%%%%%%%%%%%%%%%%%%%%%%%%%%%%%%%%%%%%%%%%%%%%%%%%%%%%%%%%%%%
%%%%%%%%%%%%%%%%%%%%%%%%%%%%%%%%%%%%%%%%%%%%%%%%%%%%%%%%%%%%%%%%%%%
\begin{proof}
%(of Theorem~\ref{thm:ell2_regression})
%%%%%%%%%%%%%%%%%%%%%%%%%%%%%%%%%%%%%%%%%%%%%%%%%%%%%%%%%%%%%%%%%%%
%%%%%%%%%%%%%%%%%%%%%%%%%%%%%%%%%%%%%%%%%%%%%%%%%%%%%%%%%%%%%%%%%%%
Similarly as the proof in~\cite{sarlos}. Let $\matA=\matU\matSig \matV^\top$ be the SVD of $\matA$. Let  $\b=\matA \xopt + \w$, where $\w\in\RR^m$ and $\w\bot$\text{colspan(\matA)}. Also let $\matA(\widetilde{\x}_{\text{opt}} - \xopt)=\matU\y$, where $\y \in \RR^{\rank{\matA}}$. Our goal is to bound this quantity
\begin{align}
 \norm{\b-\matA\widetilde{\x}_{\text{opt}}}^2 &= \norm{\b-\matA(\widetilde{\x}_{\text{opt}} - \xls) - \matA\xls }^2\nonumber
                            \ =\  \norm{\w - \matU\y}^2 \nonumber
			    \ = \ \norm{\w}^2 +\norm{\matU\y}^2,  \quad \text{since }\w\bot \text{colspan}(U)\nonumber  \\
				& =  \norm{\w}^2 + \norm{\y}^2, \quad \text{since} \matU^\top \matU = \Id. \label{eq:l2_basic}
\end{align}
It suffices to bound the norm of $\y$, i.e., $\norm{\y} \leq 3\eps \norm{\w}$. Recall that given $\matA,\b$ the vector $\w$ is uniquely defined. On the other hand, vector $\y$ depends on the random projection $\matR$. Next we show the connection between $\y$ and $\w$ through the ``normal equations''.
\begin{align}
	\matR\matA\widetilde{\x}_{\text{opt}}  &= \matR\b +\w_2 \implies \nonumber
	\matR\matA\widetilde{\x}_{\text{opt}}   = \matR(\matA\xls + \w) + \w_2 \implies \nonumber \\
	\matR\matA(\widetilde{\x}_{\text{opt}} - \x_{\text{opt}})  &= \matR\w + \w_2 \implies \nonumber
	\matU^\top \matR^\top \matR \matU \y  = \matU^\top \matR^\top \matR\w  + \matU^\top  \matR^\top  \w_2 \implies
\nonumber \\
	\matU^\top \matR^\top \matR \matU \y  &= \matU^\top \matR^\top \matR\w \label{eq:random_l2},
\end{align}
where $\w_2\bot\text{colspan}(\matR)$, and used this fact to derive Ineq.~\eqref{eq:random_l2}. A crucial observation is that the  $\text{colspan}(\matU)$ is perpendicular to $\w$. Set $\matA=\matB=\matU$ in Theorem~\ref{thm:matrixmult} (i.a), and set $\eps' = \sqrt{\eps}$, and $t=\Omega( \frac{r}{\eps'^2}\log(1/\delta))$. Notice that $\rank{\matA}+\rank{\matB} \leq 2r$, hence with probability at least $1-\delta/2$ we know that $1-\eps' \leq \sigma_i(\matR\matU) \leq 1 + \eps'$. It follows that
\begin{equation}\label{eq:ls:1}
\norm{\matU^\top \matR^\top \matR \matU \y } \geq (1-\eps')^2 \norm{\y}.
\end{equation}
A similar argument (set $\matA=\matU$ and $\matB=\w$ in Theorem~\ref{thm:matrixmult} (i.a)) guarantees that (since $t\geq \Omega( \frac{r}{\eps'^2}\log(1/\delta))$)
\begin{equation}\label{eq:ls:2}
	\norm{\matU^\top \matR^\top \matR\w } = \norm{\matU^\top \matR^\top \matR\w - \matU^\top \w} \leq \eps' \norm{\matU} \norm{\w}= \eps' \norm{\w}
\end{equation}
with probability at least $1-\delta /2$. Recall that $\norm{\matU}  = 1$, since $\matU^\top \matU = \Id_n$. Therefore, condition on both the events~\eqref{eq:ls:1} and~\eqref{eq:ls:2} (which occur w.p. at least $1-\delta$) and take Euclidean norms on both sides of Equation~\eqref{eq:random_l2} to conclude that
\[ \norm{ \y} \leq \dfrac{\eps'}{(1-\eps')^2} \norm{\w} \leq 4\eps' \norm{\w}. \]
Summing up, it follows from Equation~\eqref{eq:l2_basic} that, with probability at least $1-\delta$, $\norm{\b-\matA\widetilde{\x}_{\text{opt}}}^2 \leq (1+16\eps'^2) \norm{\w} = (1 + 16\eps) \norm{\b-\matA\xopt}^2.$ This proves Ineq.~\eqref{ineq:regression:approx}.
%

%
Ineq.~\eqref{ineq:regression:x_opt} follows directly from the bound on the norm of $\y$ repeating the above proof for $\eps' \leftarrow \eps $. First recall that $\xopt$ is in the row span of $\matA$, since $\xopt = \matV\matSig^{-1} \matU^\top \b$ and the columns of $\matV$ span the row space of $\matA$. Similarly for $\widetilde{\x}_{\text{opt}}$ since the row span of $\matR \matA$ is contained in the row-span of $\matA$. Indeed, $\eps \norm{\w} \geq \norm{\y} =\norm{\matU\y} = \norm{\matA(\xopt - \widetilde{\x}_{\text{opt}}) } \geq \sigma_{\min(\matA)} \norm{ \xopt - \widetilde{\x}_{\text{opt}} }$.
\end{proof}
%
%%%%%%%%%%%%%%%%%%%%%%%%%%%%%%%%%%%%%%%%%%
%%%%%%%%%%%%%%%%%%%%%%%%%%%%%%%%%%%%%%%%%%
\subsection{Randomized Extended Kaczmarz}\label{sec:REK}
%%%%%%%%%%%%%%%%%%%%%%%%%%%%%%%%%%%%%%%%%%
%%%%%%%%%%%%%%%%%%%%%%%%%%%%%%%%%%%%%%%%%%
%
%
The Kaczmarz method is an iterative projection algorithm for solving linear systems of equations~\cite{K}. Due to its simplicity, the Kaczmarz method has found numerous applications including image reconstruction, distributed computation and signal processing to name a few~\cite{K:apps:CFM92,book:K:apps:H80,book:K:apps:Nat01,CDC12}, see~\cite{K:apps} for more applications. The Kaczmarz method has also been rediscovered in the field of image reconstruction and called ART (Algebraic Reconstruction Technique)~\cite{ART}, see also~\cite{book:Zenios,book:K:apps:H80} for additional references. It has been also applied to more general settings, see~\cite[Table~1]{K:apps} and \cite{K:tompk,K:rate:MC77} for non-linear versions of the Kaczmarz method.

Throughout Section~\ref{sec:REK}, all vectors are assumed to be column vectors. The Kaczmarz method operates as follows: Initially, it starts with an arbitrary vector $\x^{(0)}\in\RR^n$. In each iteration, the Kaczmarz method goes through the rows of $\matA$ in a cyclic manner\footnote{That is, selecting the indices of the rows from the sequence $1,2,\ldots , m , 1,2 , \ldots$.} and for each selected row, say $i$-th row $\ar{i}$, it orthogonally projects the current estimate vector onto the affine hyperplane defined by the $i$-th constraint of $\matA\x = \b$, i.e., $\{\x\ |\ \ip{\ar{i}}{\x} = b_i\}$ where $\ip{\cdot}{\cdot}$  is the Euclidean inner product. More precisely, assuming that the $i_k$-th row has been selected at $k$-th iteration, then the $(k+1)$-th estimate vector $\x^{(k+1)}$ is inductively defined by
\[\x^{(k+1)} := \x^{(k)} + \lambda_k\frac{b_{i_k} - \ip{\ar{i_k}}{ \x^{(k)}}}{\norm{\ar{i_k}}^2} \ar{i_k}\]
where $\lambda_k \in \RR$ are the so-called relaxation parameters and $\norm{\cdot}$ denotes the Euclidean norm. The original Kaczmarz method corresponds to $\lambda_k = 1$ for all $k\geq 0$ and all other setting of $\lambda_k$'s are usually referred as the \emph{relaxed Kaczmarz method} in the literature~\cite{K:apps,book:Galantai}.

Kaczmarz proved that this process converges to the unique solution for square non-singular matrices~\cite{K}, but without any attempt to bound the rate of convergence. Bounds on the rate of convergence of the Kaczmarz method are given in~\cite{K:rate:MC77}, \cite{K:rate:Ansorge} and \cite[Theorem~4.4, p.120]{book:Galantai}. In addition, an error analysis of the Kaczmarz method under the finite precision model of computation is given in~\cite{phdthesis:K:error,K:error}.
%

%
%More precisely, let $\widehat{\matA}\in\RR^{m\times n}$ whose row set is the rows of $\matA$ normalized to unit length. Then it follows that after a sweep over all rows of $\matA$, Kaczmarz's method improves its estimate by $1- \det (\widehat{\matA}^\top \widehat{\matA})$.

Nevertheless, the Kaczmarz method converges even if the linear system $\matA\x = \b$ is overdetermined ($m>n$) and has no solution. In this case and provided that $\matA$ has full column rank, the Kaczmarz method converges to the least squares estimate. This was first observed by Whitney and Meany~\cite{K:rate:WM67} who proved that the relaxed Kaczmarz method converges provided that the relaxation parameters are within $[0,2]$ and $\lambda_k\to 0$, see also~\cite[Theorem~1]{K:relax:CEG83}, \cite{K:Tanabe} and~\cite{K:relax:Hanke90} for additional references.
%STATE the result here. (THEOREM 4.32 from BOOK, see also Herman, Lent, Lutz and CEG82). where and later Tanabe proved it~\cite{K:Tanabe}.

In the literature there was empirical evidence that selecting the rows non-uniformly at random may be more effective than selecting the rows via Kaczmarz's cyclic manner~\cite{RK:HM93,K:apps:CFM92}. Towards explaining such an empirical evidence, Strohmer and Vershynin proposed a simple randomized variant of the Kaczmarz method that has exponential convergence \emph{in expectation}~\cite{RK} assuming that the linear system is solvable; see also~\cite{LS:RCD} for extensions to linear constraints. A randomized iterative algorithm that computes a sequence of random vectors $\x^{(0)}, \x^{(1)}, \ldots$ is said to \emph{converge in expectation} to a vector $\x^*$ if and only if $\EE \norm{\x^{(k)} - \x^*}^2\to 0$ as $k\to\infty$, where the expectation is taken over the random choices of the algorithm. Soon after~\cite{RK}, Needell analyzed the behavior of the randomized Kaczmarz method for the case of full column rank linear systems that do not have any solution~\cite{Needell09}. Namely, Needell proved that the randomized Kaczmarz estimate vector is (in the limit) within a fixed distance from the least squares solution and also that this distance is proportional to the distance of $\b$ from the column space of $\matA$. In other words, Needell proved that the randomized Kaczmarz method is effective for least squares problems whose least squares error is negligible.

We present a randomized iterative least squares solver (REK, Algorithm~\ref{alg:REK}) that converges in expectation to $\xls$. REK is based on~\cite{RK,Needell09} and inspired by~\cite{popa}. More precisely the proposed algorithm can be thought of as a randomized variant of Popa's extended Kaczmarz method~\cite{popa}, therefore we named it as \emph{randomized extended Kaczmarz}.

%In this setting, it turns out that the randomized Kaczmarz estimate vector is within a fixed distance from the least squares solution. In addition, the distance is proportional, roughly speaking,  to the distance of $\b$ from the column space of $\matA$ which in general can be arbitrarily large.
%
%
% We mostly follow this book~\cite{book:Galantai}.
%Books : \cite{book:Bodewig}, \cite{book:Gastinel}


%
\paragraph{Roadmap}
%
%
%In Section~\ref{sec:related}, we briefly discuss related work on the design of deterministic and randomized algorithms for solving least squares problems.
First, we discuss the convergence properties of the randomized Kaczmarz algorithm for solvable systems as in~\cite{RK} (Theorem~\ref{thm:RK:consistent}) and also recall its analysis for non-solvable systems (Theorem~\ref{thm:RK:inconsistent}). Second, we present and analyze the randomized extended Kaczmarz algorithm.
%Finally, in Section~\ref{sec:impl} we provide a numerical evaluation of the proposed algorithm.
%
%
%
%%%%%%%%%%%%%%%%%%%%%%%%%%%%%%%%%%%%%%%%%%%%%%%%%%%%%%%
%%%%%%%%%%%%%%%%%%%%%%%%%%%%%%%%%%%%%%%%%%%%%%%%%%%%%%%
\subsubsection*{Randomized Kaczmarz}
%\label{sec:RK}
%%%%%%%%%%%%%%%%%%%%%%%%%%%%%%%%%%%%%%%%%%%%%%%%%%%%%%%
%%%%%%%%%%%%%%%%%%%%%%%%%%%%%%%%%%%%%%%%%%%%%%%%%%%%%%%
%
%
%
Strohmer and Vershynin proposed the following randomized variant of Kaczmarz algorithm~(Algorithm~\ref{alg:RK}), see~\cite{RK} for more details. The following theorem is a restatement of the main result of~\cite{RK} without imposing the full column rank assumption.
%%%%%%%%%%%%%%%%%%%%%%%%%%%%%%%%%%%%%%%%%%%%%%%%%%%%%%%
\begin{theorem}\label{thm:RK:consistent}
Let $\matA\in\RR^{m\times n}$, $\b\in\RR^m$ and $T>1$ be the input to Algorithm~\ref{alg:RK}. Assume that $\matA \x = \b$ has a solution and denote $\xls:=\pinv{\matA}\b$. In exact arithmetic, Algorithm~\ref{alg:RK} converges to $\xls$ in expectation:
	\begin{equation}
		\EE \norm{\x^{(k)} - \xls}^2 \leq \left(1 - \frac1{\kappaFS(\matA)}\right)^k \norm{\x^{(0)} - \xls}^2\quad \forall\ k>0.
	\end{equation}
\end{theorem}
%%%%%%%%%%%%%%%%%%%%%%%%%%%%%%%%%%%%%%%%%%%%%%%%%%%%%%%
%%%%%%%%%%%%%%%%%%%%%%%%%%%%%%%%%%%%%%%%%%%%%%%%%%%%%%%
\begin{remark}
The above theorem has been proved in~\cite{RK} for the case of full column rank. Also, the rate of expected convergence in~\cite{RK} is $1-1/\widetilde{\kappa}^2 (\matA)$ where $\widetilde{\kappa}^2(\matA) := \frobnorm{\matA}^2 / \sigma_{\min{(m,n)}}(\matA^\top \matA)$. Notice that if $\rank{\matA}< n$, then $\widetilde{\kappa}^2(\matA)$ is infinite whereas $\kappaFS(\matA)$ is bounded.
\end{remark}
%%%%%%%%%%%%%%%%%%%%%%%%%%%%%%%%%%%%%%%%%%%%%%%%%%%%%%%
%
%
%%%%%%%%%%%%%%%%%%%%%%%%%%%%%%%%%%%%%%%%%%%%%%%%%%%%%%%
\begin{algorithm}{}
	\caption{Randomized Kaczmarz~\cite{RK}}\label{alg:RK}
\begin{algorithmic}[1]
\Procedure{}{$\matA$, $\b$, $T$}\Comment{$\matA\in\RR^{m\times n}, \b\in\RR^m$}
\State Set $\x^{(0)}$ to be any vector in the row space of $\matA$
\For {$k=0,1,2,\ldots , T-1$ }
	\State Pick $i_k\in[m]$ with probability $q_i:=\norm{\ar{i}}^2/\frobnorm{\matA}^2, i\in [m]$
	\State Set $ \x^{(k+1)} = \x^{(k)}  + \frac{b_{i_k} - \ip{\x^{(k)}}{\ar{i_k} }}{\norm{\ar{i_k}}^2} \ar{i_k}$
\EndFor
\State Output $\x^{(T)}$
\EndProcedure
\end{algorithmic}
\end{algorithm}
%%%%%%%%%%%%%%%%%%%%%%%%%%%%%%%%%%%%%%%%%%%%%%%%%%%%%%%
%
%
%
We devote the rest of this subsection to prove Theorem~\ref{thm:RK:consistent} following~\cite{RK}. The proof is based on the following two elementary lemmas which both appeared in~\cite{RK}. However, in our setting, the second lemma is not identical to that in~\cite{RK}.
%%%%%%%%%%%%%%%%%%%%%%%%%%%%%%%%%%%%%%%%%%%%%%%%%%%
\begin{lemma}[Orthogonality]\label{lem:ortho}
Assume that $\matA\x= \b$ has a solution and use the notation of Algorithm~\ref{alg:RK}, then $\x^{(k+1)} -\xls$ is perpendicular to $\x^{(k+1)} - \x^{(k)}$ for any $k\geq 0$. In particular, in exact arithmetic it holds that $\norm{\x^{(k+1)} - \xls}^2 = \norm{\x^{(k)} - \xls}^2 - \norm{\x^{(k+1)} - \x^{(k)}}^2$.
\end{lemma}
%%%%%%%%%%%%%%%%%%%%%%%%%%%%%%%%%%%%%%%%%%%%%%%%%%%
%
%
%%%%%%%%%%%%%%%%%%%%%%%%%%%%%%%%%%%%%%%%%%%%%%%%%%%
\begin{proof}
%(of Lemma~\ref{lem:ortho})
It suffices to show that $\ip{\x^{(k+1)} - \xls}{\x^{(k+1)} - \x^{(k)}} = 0$. For notational convenience, let $\alpha_i := \frac{b_i - \ip{\x^{(k)}}{\ar{i}}}{\norm{\ar{i}}^2}$ for every $i\in{[m]}$. Assume that $\x^{(k+1)} = \x^{(k)} + \alpha_{i_k} \ar{i_k}$ for some arbitrary $i_k\in [m]$. Then,
\begin{align*}
	\ip{\x^{(k+1)} - \xls}{\x^{(k+1)} - \x^{(k)} }  & =   \ip{\x^{(k+1)} - \xls}{ \alpha_{i_k} \ar{i_k}} \ = \ \alpha_{i_k}\left(\ip{\x^{(k+1)} }{ \ar{i_k}} -  b_{i_k}\right)
\end{align*}
using the definition of $\x^{(k+1)}$, and the fact that $\ip{\xls}{\ar{i_k}} = b_{i_k}$ since $\xls$ is a solution to $\matA\x=\b$. Now, by the definition of $\alpha_{i_k} $, $\ip{\x^{(k+1)}}{\ar{i_k}} = \ip{\x^{(k)}}{\ar{i_k}} + \alpha_{i_k} \norm{\ar{i_k}}^2 = \ip{\x^{(k)}}{\ar{i_k}} + b_{i_k} - \ip{\x^{(k)}}{\ar{i_k}} = b_{i_k}$.
\end{proof}
%%%%%%%%%%%%%%%%%%%%%%%%%%%%%%%%%%%%%%%%%%%%%%%%%%%
%
%
The above lemma provides a formula for the error at each iteration. Ideally, we seek to minimize the error at each iteration which is equivalent to maximizing $\norm{\x^{(k+1)} - \x^{(k)}}$ over the choice of the row projections of the algorithm. The next lemma suggests that by randomly picking the rows of $\matA$ reduces the error in expectation.
%%
%%
%%%%%%%%%%%%%%%%%%%%%%%%%%%%%%%%%%%%%%%%%%%%%%%%%%%%%%%
\begin{lemma}[Expected Error Reduction]\label{lem:avg}
Assume that $\matA\x=\b$ has a solution. Let $Z$ be a random variable over $[m]$ with distribution $\Prob{Z=i} = \frac{\norm{\ar{i}}^2}{\frobnorm{\matA}^2}$ and assume that $\x^{(k)}$ is a vector in the row space of $\matA$. If $\x^{(k+1)} := \x^{(k)} + \frac{b_Z - \ip{\x^{(k)}}{\ar{Z}}}{\norm{\ar{Z}}^2} \ar{Z}$ (in exact arithmetic), then
\begin{equation}
\EE_{Z}\norm{\x^{(k+1)} - \xls}^2 \leq \left(1 - \frac1{\kappaFS(\matA)}\right) \norm{\x^{(k)} - \xls}^2.
\end{equation}
\end{lemma}
%%%%%%%%%%%%%%%%%%%%%%%%%%%%%%%%%%%%%%%%%%%%%%%%%%%
%
%%%%%%%%%%%%%%%%%%%%%%%%%%%%%%%%%%%%%%%%%%%%%%%%%%%
\begin{proof}
%(of Lemma~\ref{lem:avg})
%
In light of Lemma~\ref{lem:ortho}, it suffices to show that $\EE_{Z}\norm{\x^{(k+1)} - \x^{(k)}}^2 \geq \frac1{\kappaFS(\matA)} \norm{\x^{(k)} - \xls}^2$.
%
By the definition of $\x^{(k+1)}$, it follows
\begin{align*}
\EE_{Z}\norm{\x^{(k+1)} - \x^{(k)}}^2  &= \EE_{Z} \left[\left(\frac{b_Z - \ip{\x^{(k)}}{\ar{Z} }}{\norm{\ar{Z}}^2}\right)^2 \norm{\ar{Z}}^2\right] \ =\  \EE_{Z} \frac{\ip{\xls - \x^{(k)} }{\ar{Z}}^2}{\norm{\ar{Z}}^2} \\
 & =   \sum_{i=1}^{m} \frac{\ip{\xls - \x^{(k)}}{\ar{i}}^2}{\frobnorm{\matA}^2} = \frac{\norm{\matA (\xls - \x^{(k)})}^2}{\frobnorm{\matA}^2}.
\end{align*}
By hypothesis, $\x^{(k)}$ is in the row space of $\matA$ for any $k$ when $\x^{(0)}$ is; in addition,
the same is true for $\xls$ by the definition of pseudo-inverse~\cite{book:GVL}. Therefore,
$\norm{\matA(\xls - \x^{(k)})} \geq \sigma_{\min} \norm{\xls - \x^{(k)} }$.
\end{proof}
%%%%%%%%%%%%%%%%%%%%%%%%%%%%%%%%%%%%%%%%%%%%%%%%%%%
%
%
%%%%%%%%%%%%%%%%%%%%%%%%%%%%%%%%%%%%%%%%%%%%%%%%%%%
%
%
Theorem~\ref{thm:RK:consistent} follows by iterating Lemma \ref{lem:avg}, we get that
\[\EE \norm{\x^{(k+1)} - \xls}^2 \leq \left(1 - \frac1{\kappaFS(\matA)}\right)^k \norm{\x^{(0)} - \xls}^2.\]
%%%%%%%%%%%%%%%%%%%%%%%%%%%%%%%%%%%%%%%%%%%%%%%%%%%%%%%
%%%%%%%%%%%%%%%%%%%%%%%%%%%%%%%%%%%%%%%%%%%%%%%%%%%%%%%
%
%
%%%%%%%%%%%%%%%%%%%%%%%%%%%%%%%%%%%%%%%%%%%%%%%%%%%%%%%
%%%%%%%%%%%%%%%%%%%%%%%%%%%%%%%%%%%%%%%%%%%%%%%%%%%%%%%
\subsubsection{Randomized Kaczmarz Applied to Noisy Linear Systems}
%\label{sec:noisyRK}
%%%%%%%%%%%%%%%%%%%%%%%%%%%%%%%%%%%%%%%%%%%%%%%%%%%%%%%
%%%%%%%%%%%%%%%%%%%%%%%%%%%%%%%%%%%%%%%%%%%%%%%%%%%%%%%
The analysis of Strohmer and Vershynin is based on the restrictive assumption that the linear system has a solution. Needell made a step further and analyzed the more general setting in which the linear system does not have any solution and $\matA$ has full column rank~\cite{Needell09}. In this setting, it turns out that the randomized Kaczmarz algorithm computes an estimate vector that is within a fixed distance from the solution; the distance is proportional to the norm of the ``noise vector'' multiplied by $\kappaFS(\matA)$~\cite{Needell09}. The following theorem is a restatement of the main result in~\cite{Needell09} with two modifications: the full column rank assumption on the input matrix is dropped and the additive term $\gamma$ of Theorem~$2.1$ in~\cite{Needell09} is improved to $\norm{\w}^2/\frobnorm{\matA}^2$. The only technical difference here from~\cite{Needell09} is that the full column rank assumption is not necessary.
%
\begin{theorem}\label{thm:RK:inconsistent}
Assume that the system $\matA \x = \y$ has a solution for some $\y\in\RR^m$. Denote by $\x^{*} := \pinv{\matA}\y$. Let $\hat\x^{(k)}$ denote the $k$-th iterate of the randomized Kaczmarz algorithm applied to the linear system $\matA \x = \b$ with $\b:=\y + \w$ for any fixed $\w\in\RR^m$, i.e., run Algorithm~\ref{alg:RK} with input $(\matA,\b)$. In exact arithmetic, it follows that
%
\begin{equation}\label{ineq:relaxRK}
\EE \norm{\hat\x^{(k)} - \x^{*}}^2 \le \left(1-\frac1{\kappaFS(\matA)}\right)\EE \norm{\hat\x^{(k-1)} - \x^{*}}^2 + \frac{\norm{\w}^2}{\frobnorm{\matA}^2}.
\end{equation}
%
%
In particular,
\[\EE \norm{\hat\x^{(k)} - \x^{*}}^2 \le \left(1-\frac1{\kappaFS(\matA)}\right)^k \norm{\x^{(0)}- \x^{*}}^2 + \frac{\norm{\w}^2}{\sigma^2_{\min} }.\]
%
%
\end{theorem}
%
%
%
\begin{proof}
%(of Theorem~\ref{thm:RK:inconsistent})
As in~\cite{Needell09}, for any $i\in{[m]}$ define the affine hyper-planes:
%
\begin{align*}
\mathcal{H}_i &:= \{\x: \ip{\ar{i}}{\x} = y_i\}\\
\mathcal{H}_i^{w_i} &:= \{\x: \ip{\ar{i} }{\x} = y_i + w_i\}
\end{align*}
Assume for now that at the $k$-th iteration of the randomized Kaczmarz algorithm applied on $(\matA, \b)$, the $i$-th row is selected. Note that $\hat\x^{(k)}$ is the projection of $\hat\x^{(k-1)}$ on $\mathcal{H}_i^{w_i}$ by the definition of the randomized Kaczmarz algorithm on input $(\matA,\b)$. Let us denote the projection of $\hat\x^{(k-1)}$ on $\mathcal{H}_i$ by $\x^{(k)}$. The two affine hyper-planes $\mathcal{H}_i,\mathcal{H}_i^{w_i}$ are \emph{parallel} with common normal $\ar{i}$, so $\x^{(k)}$ is the projection of $\hat\x^{(k)}$ on $\mathcal{H}_i$ and the minimum distance between $\mathcal{H}_i$ and $\mathcal{H}_i^{w_i}$ equals $|w_i| / \norm{\ar{i}}$. In addition, $\x^{*}\in \mathcal{H}_i$ since $\ip{\x^{*}}{\ar{i}} = y_i$, therefore by orthogonality we get that
\begin{equation}\label{eq:1}
\norm{\hat\x^{(k)} - \x^{*}}^2 = \norm{\x^{(k)} - \x^{*}}^2 + \norm{\hat\x^{(k)} - \x^{(k)}}^2.
\end{equation}
Since $\x^{(k)}$ is the projection of $\hat\x^{(k-1)}$ onto $\mathcal{H}_i$ (that is to say, $\x^{(k)}$ is a randomized Kaczmarz step applied on input $(\matA, \y)$ where the $i$-th row is selected on the $k$-th iteration) and $\hat\x^{(k-1)}$ is in the row space of $\matA$, Lemma~\ref{lem:avg} tells us that
\begin{equation}\label{eq:2}
 \EE\norm{\x^{(k)} - \x^{*}}^2 \le \left(1 - \frac1{\kappaFS(\matA)}\right) \norm{\hat\x^{(k-1)} - \x^{*}}^2.
\end{equation}
Note that for given selected row $i$ we have $ \norm{\hat\x^{(k)} - \x^{(k)}}^2= \frac{w_i^2}{\norm{\ar{i}}^2}$;
by the distribution of selecting the rows of $\matA$ we have that
\begin{equation}\label{eq:3}
\E\norm{\hat\x^{(k)} - \x^{(k)}}^2 = \sum_{i=1}^{m} q_i \frac{w_i^2}{\norm{\ar{i}}^2} = \frac{\norm{\w}^2}{\frobnorm{\matA}^2}.
\end{equation}
Inequality~\eqref{ineq:relaxRK} follows by taking expectation on both sides of Equation~\eqref{eq:1} and bounding its resulting right hand side using Equations~\eqref{eq:2} and~\eqref{eq:3}. Applying Inequality~\eqref{ineq:relaxRK} inductively, it follows that
\[\EE \norm{\hat\x^{(k)} - \x^{*}}^2 \le \left(1-\frac1{\kappaFS(\matA)}\right)^k\norm{\x^{(0)}- \x^{*}}^2 + \frac{\norm{\w}^2}{\frobnorm{\matA}^2}\sum_{i=0}^{k} \left(1-\frac1{\kappaFS(\matA)}\right)^i,\]
where we used that $\x^{(0)}$ is in the row space of $\matA$. The latter sum is bounded above by $\sum_{i=0}^{\infty} \left(1-\frac1{\kappaFS(\matA)}\right)^i =\frobnorm{\matA}^2 / \sigma^2_{\min}$.
\end{proof}
%%%%%%%%%%%%%%%%%%%%%%%%%%%%%%%%%%%%%%%%%%%%%%%%%%%%%%%
%%%%%%%%%%%%%%%%%%%%%%%%%%%%%%%%%%%%%%%%%%%%%%%%%%%%%%%
%
%%%%%%%%%%%%%%%%%%%%%%%%%%%%%%%%%%%%%%%%%%%%%%%%%%%%%%%
%%%%%%%%%%%%%%%%%%%%%%%%%%%%%%%%%%%%%%%%%%%%%%%%%%%%%%%
\subsubsection*{Randomized Extended Kaczmarz}
%\label{sec:result}
%%%%%%%%%%%%%%%%%%%%%%%%%%%%%%%%%%%%%%%%%%%%%%%%%%%%%%%
%%%%%%%%%%%%%%%%%%%%%%%%%%%%%%%%%%%%%%%%%%%%%%%%%%%%%%%
%
%
Given any least squares problem, Theorem~\ref{thm:RK:inconsistent} with $\w = \bc$ tells us that the randomized Kaczmarz algorithm works well for least square problems whose least squares error is very close to zero, i.e., $\norm{\w}\approx 0$. Roughly speaking, in this case the randomized Kaczmarz algorithm approaches the minimum $\ell_2$-norm least squares solution up to an additive error that depends on the distance between $\b$ and the column space of $\matA$.

Here, the main observation is that it is possible to efficiently reduce the norm of the ``noisy'' part of $\b$, $\bc$ (using Algorithm~\ref{alg:randOP}) and then apply the randomized Kaczmarz algorithm on a new linear system whose right hand side vector is now arbitrarily close to the column space of $\matA$, i.e., $\matA\x\approx \br$. This idea together with the observation that the least squares solution of the latter linear system is equal (in the limit) to the least squares solution of the original system (see Fact~\ref{fact:xls}) implies a randomized algorithm for solving least squares.

Next we present the randomized extended Kaczmarz algorithm which is a specific combination of the randomized orthogonal projection algorithm together with the randomized Kaczmarz algorithm.
%%
%
%%%%%%%%%%%%%%%%%%%%%%%%%%%%%%%%%%%%%%%%%%%%%%%%%%%%%%%
%%%%%%%%%%%%%%%%%%%%%%%%%%%%%%%%%%%%%%%%%%%%%%%%%%%%%%%
\paragraph{The algorithm}
%%%%%%%%%%%%%%%%%%%%%%%%%%%%%%%%%%%%%%%%%%%%%%%%%%%%%%%
%%%%%%%%%%%%%%%%%%%%%%%%%%%%%%%%%%%%%%%%%%%%%%%%%%%%%%%
%
\begin{algorithm}{}
	\caption{Randomized Extended Kaczmarz (REK)}\label{alg:REK}
\begin{algorithmic}[1]
\Procedure{}{$\matA$, $\b$, $\eps$}\Comment{$\matA\in\RR^{m\times n}, \b\in\RR^m$, $\eps >0$}
\State Initialize $\x^{(0)}=\zeromtx$ and $\z^{(0)} =\b$
\For {$k=0,1,2,\ldots $ }
	\State Pick $i_k\in[m]$ with probability $q_i:=\norm{\ar{i}}^2/\frobnorm{\matA}^2, i\in [m]$
	\State Pick $j_k\in[n]$ with probability $p_j:=\norm{\ac{j}}^2/\frobnorm{\matA}^2,\ j\in [n]$
	\State Set $ \z^{(k+1)} = \z^{(k)} - \frac{\ip{\ac{j_k}}{\z^{(k)}}}{\norm{\ac{j_k}}^2}\ac{j_k}  $
	\State Set $ \x^{(k+1)} = \x^{(k)}  + \frac{b_{i_k} - z^{(k)}_{i_k} - \ip{\x^{(k)}}{\ar{i_k} }}{\norm{\ar{i_k}}^2} \ar{i_k}$
	\State\label{alg:stopping} Check every $8 \min (m,n)$ iterations and terminate if it holds:
	\[  \frac{\norm{\matA \x^{(k)} - (\b - \z^{(k)}) }}{\frobnorm{\matA} \norm{\x^{(k)}}}  \leq \eps \quad\text{and} \quad \frac{\norm{\matA^\top \z^{(k)}}}{\frobnorm{\matA}^2\norm{\x^{(k)}}} \leq \eps.\]

\EndFor
\State Output $\x^{(k)}$
\EndProcedure
\end{algorithmic}
\end{algorithm}
%%%%%%%%%%%%%%%%%%%%%%%%%%%%%%%%%%%%%%%%%%%%%%%%%%%%%%%
%%%%%%%%%%%%%%%%%%%%%%%%%%%%%%%%%%%%%%%%%%%%%%%%%%%%%%%
%
%
The proposed algorithm consists of two components. The first component consisting of Steps $5$ and $6$ is responsible to implicitly maintain an approximation to $\br$ formed by $\b-\z^{(k)}$. The second component, consisting of Steps 4 and 7, applies the randomized Kaczmarz algorithm with input $\matA$ and the current approximation $\b -\z^{(k)}$ of $\br$, i.e., applies the randomized Kaczmarz on the system $\matA \x = \b - \z^{(k)}$. Since $\b - \z^{(k)}$ converges to $\br$, $\x^{(k)}$ will eventually converge to the minimum Euclidean norm solution of $\matA \x = \br$ which equals to $\xls=\pinv{\matA}\b$ (see Fact~\ref{fact:xls}).
%

The stopping criterion of Step~\ref{alg:stopping} was decided based on the following analysis. Assume that the termination criteria are met for some $k>0$. Let $\z^{(k)} = \bc + \w$ for some $\w\in\colspan{\matA}$ (which holds by the definition of $\z^{(k)}$). Then,
\begin{align*}
	\norm{\matA^\top \z^{(k)}} &= \norm{\matA^\top (\bc + \w)} = \norm{\matA^\top \w} \geq \sigma_{\min}(\matA)\norm{\z^{(k)} - \bc}.
\end{align*}
By rearranging terms and using the second part of the termination criterion, it follows that $\norm{\z^{(k)} - \bc} \leq \eps \frac{\frobnorm{\matA}^2}{\sigma_{\min}} \norm{\x^{(k)}}$. Now,
\begin{align*}
	\norm{\matA(\x^{(k)} - \xls)} & \leq \norm{\matA\x^{(k)} - (\b - \z^{(k)}) } + \norm{ \b- \z^{(k)} - \matA\xls} \\
								  & \leq \eps \frobnorm{\matA} \norm{\x^{(k)}} + \norm{\bc - \z^{(k)}} \\
								  & \leq \eps \frobnorm{\matA} \norm{\x^{(k)}} + \eps \frac{\frobnorm{\matA}^2}{\sigma_{\min}} \norm{\x^{(k)}},
\end{align*}
where we used the triangle inequality, the first part of the termination rule together with $\br = \matA\xls$ and the above discussion.
Now, since $\x^{(k)},\xls \in\colspan{\matA^\top}$, it follows that
\begin{equation}\label{REK:forwardErr}
			\frac{\norm{\x^{(k)} - \xls } }{\norm{\x^{(k)}}} \leq \eps \kappaF(\matA) ( 1 + \kappaF(\matA)).
\end{equation}
Equation~\eqref{REK:forwardErr} demonstrates that the forward error of REK after termination is bounded.
%
%%%%%%%%%%%%%%%%%%%%%%%%%%%%%%%%%%%%%%%%%%%%%%%%%%%%%%%
%%%%%%%%%%%%%%%%%%%%%%%%%%%%%%%%%%%%%%%%%%%%%%%%%%%%%%%
\paragraph{Rate of convergence}
%%%%%%%%%%%%%%%%%%%%%%%%%%%%%%%%%%%%%%%%%%%%%%%%%%%%%%%
%%%%%%%%%%%%%%%%%%%%%%%%%%%%%%%%%%%%%%%%%%%%%%%%%%%%%%%
%
The following theorem bounds the expected rate of convergence of Algorithm~\ref{alg:REK}.
%
%
\begin{theorem}\label{thm:REK}
After $T>1$ iterations, in exact arithmetic, Algorithm~\ref{alg:REK} with input $\matA$ (possibly rank-deficient) and $\b$ computes a vector $\x^{(T)}$ such that
%
%
\[ \EE \norm{\x^{(T)} - \xls}^2 \leq \left(1 - \frac1{\kappaFS(\matA)}\right)^{\lfloor T/2\rfloor}\left(1 + 2\cond{\matA}\right) \norm{\xls}^2.\]
%
%
\end{theorem}
%
%
%%%%%%%%%%%%%%%%%%%%%%%%%%%%%%%%%%%%%%%%%%%%%%%%%%%%%%%
\begin{proof}
%%%%%%%%%%%%%%%%%%%%%%%%%%%%%%%%%%%%%%%%%%%%%%%%%%%%%%%
%
%
For the sake of notation, set $\alpha =1- 1/\kappaFS(\matA)$ and denote by $\EE_{k}[\cdot] := \EE[\cdot \ |\ i_0,j_0, i_1,j_1,\ldots , i_k, j_k]$, i.e., the conditional expectation with respect to the first $k$ iterations of Algorithm~\ref{alg:REK}. Observe that Steps $5$ and $6$ are independent from Steps $4$ and $7$ of Algorithm~\ref{alg:REK}, so Theorem~~\ref{thm:randOP} implies that for every $l\geq 0$
\begin{equation}\label{eq:improve}
\EE \norm{\z^{(l)} - \bc }^2 \leq \alpha^l \norm{\br}^2 \leq \norm{\br}^2.
\end{equation}
Fix a parameter $k^* := \lfloor T/2 \rfloor$. After the $k^*$-th iteration of Algorithm~\ref{alg:REK}, it follows from Theorem~\ref{thm:RK:inconsistent} (Inequality~\eqref{ineq:relaxRK}) that
\begin{align*}
	\EE_{(k^*-1)} \norm{\x^{(k^*)} - \xls}^2 & \leq  \alpha  \norm{\x^{(k^* - 1)} - \xls}^2 + \frac{ \norm{\bc - \z^{(k^* - 1)}}^2}{\frobnorm{\matA}^2}.
\end{align*}
Indeed, the randomized Kaczmarz algorithm is executed with input $(\matA, \b - \z^{(k^*-1)})$ and current estimate vector $\x^{(k^* -1)}$. Set $\y=\br$ and $\w=\bc - \z^{(k^*-1)}$ in Theorem~\ref{thm:RK:inconsistent} and recall that $\xls=\pinv{\matA} \b = \pinv{\matA}\br = \pinv{\matA}\y$.
%

%
Now, averaging the above inequality over the random variables $i_1,j_1, i_2, j_2, \ldots, i_{k^* - 1}, j_{k^* - 1}$ and using linearity of expectation, it holds that
\begin{align}
	\EE \norm{\x^{(k^*)} - \xls}^2 & \leq  \alpha  \EE\norm{\x^{(k^* - 1)} - \xls}^2 + \frac{ \EE\norm{\bc - \z^{(k^* - 1)}}^2}{\frobnorm{\matA}^2}\label{eq:star}\\
	& \leq  \alpha \EE \norm{\x^{(k^* - 1)} - \xls}^2 + \frac{\norm{\br}^2}{\frobnorm{\matA}^2} \quad  \text{by Ineq.}~\eqref{eq:improve}\nonumber\\
	& \leq  \ldots \leq \alpha^{k^*}\norm{\x^{(0)} - \xls}^2 + \sum_{l=0}^{k^* - 2} \alpha^l \frac{\norm{\br}^2}{\frobnorm{\matA}^2},\quad (\text{repeat the above } k^* - 1 \text{ times}) \nonumber \\
	& \leq  \norm{\xls}^2 + \sum_{l=0}^{\infty} \alpha^l \frac{\norm{\br}^2}{\frobnorm{\matA}^2}, \quad \text{since } \alpha <1\text{ and }\x^{(0)} = \zero . \nonumber
\end{align}
Simplifying the right hand side using the fact that $\sum_{l=0}^{\infty} \alpha^l = \frac1{1-\alpha} = \kappaFS(\matA)$, it follows
\begin{equation}\label{eq:stopTime}
		\EE \norm{\x^{(k^*)} - \xls}^2 \leq  \norm{\xls}^2 + \norm{\br}^2/\sigma_{\min}^2.
\end{equation}
Moreover, observe that for every $l\geq 0$
\begin{equation}\label{eq:better}
\EE \norm{\bc - \z^{(l+k^*)}}^2 \leq \alpha^{l+k^*} \norm{\br}^2 \leq \alpha^{k^*}\norm{\br}^2.
\end{equation}
Now for any $k>0$, similar considerations as Ineq.~\eqref{eq:star} implies that
\begin{align*}
\EE \norm{\x^{(k + k^*)} - \xls}^2  & \leq   \alpha \EE \norm{\x^{(k + k^* - 1)} - \xls}^2 + \frac{\EE \norm{\bc - \z^{(k - 1 + k^*)}}^2}{\frobnorm{\matA}^2} \\
									& \leq   \ldots \leq \alpha^k \EE \norm{\x^{(k^*)} - \xls}^2 + \sum_{l=0}^{k - 1}\alpha^{(k - 1) - l} \frac{\EE \norm{\bc - \z^{(l + k^*)}}^2}{\frobnorm{\matA}^2} \quad \text{(by induction)}\\
									& \leq   \alpha^k \EE \norm{\x^{(k^*)} - \xls}^2 + \frac{\alpha^{k^*}\norm{\br}^2}{\frobnorm{\matA}^2} \sum_{l=0}^{k - 1}\alpha^{l} \quad  \text{(by Ineq.~\eqref{eq:better})} \\
									& \leq   \alpha^k \left(\norm{\xls}^2 + \norm{\br}^2/\sigma_{\min}^2\right)  + \alpha^{k^*}\norm{\br}^2 / \sigma_{\min}^2 \quad  \left(\text{by Ineq.~\eqref{eq:stopTime}}\right) \\
									&   =   \alpha^k \norm{\xls}^2 + (\alpha^{k}+\alpha^{k^*}) \norm{\br}^2/\sigma_{\min}^2\\
									&   \leq   \alpha^k \norm{\xls}^2 + (\alpha^{k}+\alpha^{k^*}) \cond{\matA} \norm{\xls}^2 \quad \text{since }\norm{\br} \leq \sigma_{\max} \norm{\xls} \\
									&   \leq \alpha^{k^*} (1 + 2\cond{\matA})\norm{\xls}^2.
\end{align*}
To derive the last inequality, consider two cases. If $T$ is even, set $k=k^*$, otherwise set $k=k^*+1$. In both cases, $(\alpha^{k}+\alpha^{k^*}) \leq 2\alpha^{k^*}$.
%
%%%%%%%%%%%%%%%%%%%%%%%%%%%%%%%%%%%%%%%%%%%%%%%%%%%%%%%
\end{proof}
%%%%%%%%%%%%%%%%%%%%%%%%%%%%%%%%%%%%%%%%%%%%%%%%%%%%%%%
%
%%%%%%%%%%%%%%%%%%%%%%%%%%%%%%%%%%%%%%%%%%%%%%%%%%%%%%%
%%%%%%%%%%%%%%%%%%%%%%%%%%%%%%%%%%%%%%%%%%%%%%%%%%%%%%%
\paragraph{Theoretical bounds on time complexity}
%%%%%%%%%%%%%%%%%%%%%%%%%%%%%%%%%%%%%%%%%%%%%%%%%%%%%%%
%%%%%%%%%%%%%%%%%%%%%%%%%%%%%%%%%%%%%%%%%%%%%%%%%%%%%%%
%
In this section, we discuss the running time complexity of the randomized extended Kaczmarz (Algorithm~\ref{alg:REK}). Recall that REK is a Las-Vegas randomized algorithm, i.e., the algorithm always outputs an ``approximately correct'' least squares estimate (satisfying~\eqref{REK:forwardErr}) but its runnning time is a random variable. Given any fixed accuracy parameter $\eps>0$ and any fixed failure probability $0<\delta<1$ we bound the number of iterations required by the algorithm to terminate with probability at least $1-\delta$.
\begin{lemma}\label{lem:runtime}
Fix an accuracy parameter $0<\eps<2$ and failure probability $0<\delta<1$. In exact arithmetic, Algorithm~\ref{alg:REK} terminates after at most
	\[T^*:=2\kappaFS(\matA) \ln \left(\frac{32(1+2\cond{\matA})}{\delta\eps^2}\right)\]
iterations with probability at least $1-\delta$.
\end{lemma}
\begin{proof}
	Denote $\alpha := 1 - 1/ \kappaFS(\matA)$ for notational convenience. It suffices to prove that with probability at least $1-\delta$ the conditions of Step~\ref{alg:stopping} of Algorithm~\ref{alg:REK} are met. Instead of proving this, we will show that:
\begin{enumerate}
	\item With probability at least $1-\delta/2$: $\norm{(\b - \z^{(T^*)}) - \br} \leq \eps\norm{\br} / 4$.
	\item With probability at least $1-\delta/2$: $\norm{\x^{(T^*)}-\xls} \leq \eps \norm{\xls}/4$.
\end{enumerate}
Later we prove that Items (1) and (2) imply the Lemma. First we prove Item (1). By the definition of the algorithm,
\begin{align*}
	\Prob{ \norm{ (\b - \z^{(T^*)}) - \br } \geq \eps\norm{\br} / 4} & =  \Prob{ \norm{\bc - \z^{(T^*)}  }^2 \geq \eps^2\norm{\br}^2/16} \\
																	& \leq  \frac{16 \EE \norm{  \z^{(T^*)} - \bc }^2}{\eps^2\norm{\br}^2} \\
																	& \leq  16\alpha^{T^*}/\eps^2  \leq \delta/2
\end{align*}
the first equality follows since $\b -\br = \bc$, the second inequality is Markov's inequality, the third inequality follows by Theorem~\ref{thm:randOP}, and the last inequality since $T^*\geq \kappaFS(\matA) \ln(\frac{32}{\delta \eps^2})$.

Now, we prove Item (2):
\begin{align*}
		\Prob{ \norm{\x^{(T^*)}-\xls} \leq \eps\norm{\xls}/4} & \leq  \frac{ 16\EE \norm{\x^{(T^*)}-\xls}^2 }{\eps^2\norm{\xls}^2} \\
																					& \leq  16\alpha^{\lfloor T^*/2\rfloor} (1+2\cond{\matA})/ \eps^2  \leq \delta/2.
\end{align*}
the first inequality is Markov's inequality, the second inequality follows by Theorem~\ref{thm:REK}, and the last inequality follows provided that $T^* \geq 2\kappaFS(\matA) \ln \left(\frac{32(1+2\cond{\matA})}{\delta\eps^2}\right)$
%

%
A union bound on the complement of the above two events (Item (1) and (2)) implies that both events happen with probability at least $1-\delta$. Now we show that conditioning on Items (1) and (2), it follows that REK terminates after $T^*$ iterations, i.e.,
\[ \norm{\matA \x^{(T^*)} - (\b - \z^{(T^*)}) } \leq \eps \frobnorm{\matA}\norm{\x^{(T^*)}}\quad \text{and}\quad \frac{\norm{\matA^\top \z^{(k)}}}{\frobnorm{\matA}^2\norm{\x^{(k)}}} \leq \eps.\]
We start with the first condition. First, using triangle inequality and Item 2, it follows that
\begin{equation}\label{ineq:xlsxk}
\norm{\x^{(T^*)}} \geq  \norm{\xls} - \norm{\xls-\x^{(T^*)}}  \geq (1 - \eps/4 ) \norm{\xls}.
\end{equation}
Now,
\begin{align*}
	\norm{\matA \x^{(T^*)} - (\b - \z^{(T^*)}) } & \leq \norm{\matA \x^{(T^*)} - \br} + \norm{ (\b - \z^{(T^*)}) - \br } \\
	 											 & \leq \norm{\matA (\x^{(T^*)} - \xls)} + \eps \norm{\br} / 4 \\
		 										 & \leq \sigma_{\max} \norm{\x^{(T^*)} - \xls} + \eps \norm{\matA\xls} / 4 \\
												 & \leq \eps \sigma_{\max} \norm{\xls} /2 \\
												 & \leq \frac{\eps/2}{1- \eps/4} \norm{\x^{(T^*)}} \leq \eps \norm{\x^{(T^*)}}
\end{align*}
where the first inequality is triangle inequality, the second inequality follows by Item $1$ and $\br =\matA\xls$, the third and forth inequality follows by Item $2$ and the fifth inequality holds by Inequality~\eqref{ineq:xlsxk} and the last inequality follows since $\eps < 2$.
The second condition follows since
\begin{align*}
	\norm{\matA^\top \z^{(T^*)}} & = \norm{\matA^\top (\bc - \z^{(T^*)})}  \leq \sigma_{\max} \norm{\bc - \z^{(T^*)}} \\
								 & \leq \eps \sigma_{\max} \norm{\br}/4  \leq \eps \sigma^2_{\max} \norm{\xls}/4 \\
								 & \leq \frac{\eps/4}{1-\eps /4} \sigma^2_{\max} \norm{\x^{(T^*)} } \leq \eps \frobnorm{\matA}^2 \norm{\x^{(T^*)} }.
\end{align*}
the first equation follows by orthogonality, the second inequality assuming Item (2), the third inequality follows since $\br=\matA\xls$, the forth inequality follows by~\eqref{ineq:xlsxk} and the final inequality since $\eps <2$.
\end{proof}

Lemma~\ref{lem:runtime} bounds the number of iterations with probability at least $1-\delta$, next we bound the total number of arithmetic operations in worst case~(Eqn.~\eqref{eq:runtime:det}) and in expectation~(Eqn.~\eqref{eq:runtime:rand}). Let's calculate the computational cost of REK in terms of floating-point operations (flops) per iteration. For the sake of simplicity, we ignore the additional (negligible) computational overhead required to perform the sampling operations (see Section~\cite{REK} for more details) and checking for convergence.

Each iteration of Algorithm~\ref{alg:REK} requires four level-1 BLAS operations (two \emph{DDOT} operations of size $m$ and $n$, respectively, and two \emph{DAXPY} operations of size $n$ and $m$, respectively) and additional four flops. In total, $4(m+n)+2$ flops per iteration.
%

%
Therefore by Lemma~\ref{lem:runtime}, with probability at least $1-\delta$, REK requires at most
\begin{equation}\label{eq:runtime:det}
5 (m +n ) \cdot T^* \leq  10 (m+n) \rank{\matA}  \cond{\matA} \ln \left(\frac{32(1+2\cond{\matA})}{\delta\eps^2}\right)
\end{equation}
arithmetic operations (using that $\kappaFS(\matA) \leq \rank{\matA} \cond{\matA}$).
%

%
Next, we bound the \emph{expected running time} of REK for achieving the above guarantees for any fixed $\eps$ and $\delta$. Obviously, the expected running time is at most the quantity in~\eqref{eq:runtime:det}. However, as we will see shortly the expected running time is proportional to $\nnz{\matA}$ instead of $(m+n)\rank{\matA}$.

Exploiting the (possible) sparsity of $\matA$, we first show that each iteration of Algorithm~\ref{alg:REK} requires at most $5(\cavg + \ravg)$ operations in expectation. For simplicity of presentation, we assume that we have stored $\matA$ in compressed column sparse format \emph{and} compressed row sparse format~\cite{book:templates}.

Indeed, fix any $i_k\in{[m]}$ and $j_k\in{[n]}$ at some iteration $k$ of Algorithm~\ref{alg:REK}. Since $\matA$ is both stored in compressed column and compressed sparse format, Steps $7$ and Step $8$ can be implemented in $5 \nnz{\ac{j_k}}$ and 5$\nnz{\ar{i_k}}$, respectively.

By the linearity of expectation and the definitions of $\cavg$ and $\ravg$, the expected running time after $T^*$ iterations is at most $5T^* (\cavg + \ravg)$. It holds that (recall that $p_j = \norm{\ac{j}}^2/\frobnorm{\matA}^2$)
\begin{align*}
	\cavg T^* & =  \frac{2}{\frobnorm{\matA}^2}\left(\sum_{j=1}^{n} \norm{\ac{j}}^2 \nnz{\ac{j}}\right)\frac{\frobnorm{\matA}^2}{\sigma_{\min}^2}\ln \left(\frac{32(1+2\cond{\matA})}{\delta\eps^2}\right)\\
			  & =  2\frac{\sum_{j=1}^{n} \norm{\ac{j}}^2 \nnz{\ac{j}}}{\sigma_{\min}^2}\ln \left(\frac{32(1+2\cond{\matA})}{\delta\eps^2}\right)\\
			  & \leq 2\sum_{j=1}^{n}\nnz{\ac{j}}\frac{\max_{j\in{[n]}} \norm{\ac{j}}^2}{\sigma_{\min}^2} \ln \left(\frac{32(1+2\cond{\matA})}{\delta\eps^2}\right)\\
			& \leq 2\nnz{\matA}\cond{\matA} \ln \left(\frac{32(1+2\cond{\matA})}{\delta\eps^2}\right)
\end{align*}
using the definition of $\cavg$ and $T^*$ in the first equality and the fact that $\max_{j\in{[n]}} \norm{\ac{j}}^2 \leq \sigma_{\max}^2$ and $\sum_{j=1}^{n}\nnz{\ac{j}} = \nnz{\matA}$ in the first and second inequality. A similar argument shows that $\ravg T^* \leq 2\nnz{\matA} \cond{\matA} \ln \left(\frac{32(1+2\cond{\matA})}{\delta\eps^2}\right)$ using the inequality $\max_{i\in{[m]}} \norm{\ar{i}}^2 \leq \sigma_{\max}^2$.
%

%
Hence by Lemma~\ref{lem:runtime}, with probability at least $1-\delta$, the expected number of arithmetic operations of REK is at most
\begin{equation}\label{eq:runtime:rand}
 20 \nnz{\matA} \cond{\matA}\ln \left(\frac{32(1+2\cond{\matA})}{\delta\eps^2}\right).
\end{equation}
In other words, the expected running time analysis is much tighter than the worst case displayed in Equation~\eqref{eq:runtime:det} and is proportional to $\nnz{\matA}$ times the square condition number of $\matA$.
%
%
%
%%%%%%%%%%%%%%%%%%%%%%%%%%%%%%%%%%%%%%%%%%%%%%%%%%%
%\clearpage
%%%%%%%%%%%%%%%%%%%%%%%%%%%%%%%%%%%%%%%%%%%%%%%%%%%
%%%%%%%%%%%%%%%%%%%%%%%%%%%%%%%%%%%%%%%%%%%%%%%%%%%
\section{Fast Isotropic Sparsification}\label{sec:fast_isotrop_sparse}
%%%%%%%%%%%%%%%%%%%%%%%%%%%%%%%%%%%%%%%%%%%%%%%%%%%
%%%%%%%%%%%%%%%%%%%%%%%%%%%%%%%%%%%%%%%%%%%%%%%%%%%
A set of $n$-dimensional vectors $\x_1,\x_2, \ldots , \x_m$ is in \emph{isotropic position} if $\sum_{i=1}^{m} \x_i\otimes \x_i $ equals to the identity matrix. Let $\matA$ be then $m\times n$ matrix with $m\gg n$ whose row set consists of $\{\x_1,\x_2,\ldots ,\x_m\}$. Given $0 <\eps < 1$ and $\matA$, the isotropic sparsification problem is the problem of selecting a small subset of rows of $\matA$ that (after rescaling) their sum of outer products spectrally approximates the identity matrix within $\eps$ in the operator norm.
%

%
The matrix Bernstein inequality (see~\cite{chernoff:matrix_valued:Tropp}) tells us that there exists such a set with size $\OO(n\log n /\eps^2)$. Indeed, set $f(i)=\ac{i} \otimes \ac{i} / p_i - \Id_n$ where $p_i = \norm{\ac{i}}^2 / \frobnorm{\matA}^2$. A calculation shows that $\gamma$ and $\rho^2$ are $\OO(n)$. Moreover, Algorithm~\ref{alg:matrix:hyperbolic} implies an $\OO(mn^4 \log n /\eps^2)$ time algorithm for finding such a set. The running time of Algorithm~\ref{alg:matrix:hyperbolic} for rank-one matrix samples can be improved to $\OO(mn^3 \polylog{n} /\eps^2)$ by exploiting their rank-one structure. More precisely, using fast algorithms for computing all the eigenvalues of matrices after rank-one updates~\cite{Gu:update}. Next we show that we can further improve the running time by a more careful analysis.
%
%
%
%%%%%%%%%%%%%%%%%%%%%%%%%%%%%%%%%%%%%%%%%%%%%%%%%%%
%%%%%%%%%%%%%%%%%%%%%%%%%%%%%%%%%%%%%%%%%%%%%%%%%%%
%\vspace*{-3.5ex}
\begin{algorithm}{}
	\caption{Fast Isotropic Sparsification}\label{alg:fast:isotrop}
\begin{algorithmic}[1]
\Procedure{Isotrop}{$\matA$, $\eps$} \Comment{$\matA\in{\reals^{m\times n}}$, $\sum_{k=1}^{m} \ac{k} \otimes \ac{k} = \Id_n$ and $0 < \eps <1$}
\State Set $\theta = \eps / n $, $t=\OO( n \ln n/\eps^2)$, and $\ac{k} \leftarrow \ac{k}/\sqrt{p_k}$ for every $k\in{[m]}$, where $p_k=\norm{\ac{k} }^2/n$
%\State Compute $w_+ = \e^{-\theta} (1+\frac{\e^{\theta n} -1}{n} )$ and $w_- = \e^{\theta} (1-\frac{1 - \e^{-\lambda n}}{n})$
\State Set $\Lambda_{\{0\}} = \zeromtx_n$ and $\matZ = \sqrt{\theta}\ \matA$
\For {$i=1$ to $t$}
%	\State Set $w_i^+ = (w_{+})^{t-i} \e^{-\theta i}$ and $w_i^{-} = (w_{-})^{t-i}\e^{\theta i}$
	\State $x_i^* = \argmin_{k\in{[m]}}{\trace{\expm{ \Lambda_{\{i - 1\}} + \matZ_{(k)} \otimes \matZ_{(k)} }\e^{-\theta i}  + \expm{- \Lambda_{\{i - 1\}} - \matZ_{(k)} \otimes \matZ_{(k)} }\e^{\theta i}  }    } $ \Comment{Apply $m$ times Lemma~\ref{lem:comp_eigs}}
	\State $[\Lambda_{\{i\}}, U_{\{i\}}] = \textbf{eigs} ( \Lambda_{\{i - 1\}} + \matZ_{(x_i^*)} \otimes \matZ_{(x_i^*)} )$ \Comment{\textbf{eigs} computes eigensystem}
	\State $\matZ = \matZ  \matU_{\{i\}} $ \Comment{Apply fast matrix-vector multiplication }
\EndFor
\State \textbf{Output:} $t$ indices $x_1^*, x_2^*, \ldots ,x_t^*,\ x_i^* \in{[m]}$  s.t. $\norm{ \sum_{k=1}^{t} \frac{ \ac{x_k^*} \otimes \ac{x_k^*} }{ tp_{x_k^*}} - \Id_n } \leq \eps $
\EndProcedure
\end{algorithmic}
\end{algorithm}
%\vspace*{-4.0ex}
%%%%%%%%%%%%%%%%%%%%%%%%%%%%%%%%%%%%%%%%%%%%%%%%%%%
%%%%%%%%%%%%%%%%%%%%%%%%%%%%%%%%%%%%%%%%%%%%%%%%%%%
%

%
We show how to improve the running time of Algorithm~\ref{alg:matrix:hyperbolic} to $\OO(\frac{mn^2}{\eps^2} \polylog{n, \frac1{\eps}})$ utilizing results from numerical linear algebra including the Fast Multipole Method~\cite{FMM:CGR} (FMM) and ideas from~\cite{Gu:update}. The main idea behind the improvement is that the trace is invariant under any change of basis. At each iteration, we perform a change of basis so that the matrix corresponding to the previous choices of the algorithm is diagonal. Now, Step $4$ of Algorithm~\ref{alg:matrix:hyperbolic} corresponds to computing all the eigenvalues of $m$ different eigensystems with special structure, i.e., diagonal plus a rank-one matrix. Such eigensystem can be solved in $\OO(n \polylog{n})$ time using the FMM as was observed in~\cite{Gu:update}. However, the problem now, is that at each iteration we have to represent all the vectors $\ac{i}$ in the new basis, which may cost $\OO(mn^2)$. The key observation is that the change of basis matrix at each iteration is a Cauchy matrix (see Appendix). It is known that matrix-vector multiplication with Cauchy matrices can be performed efficiently and numerically stable using FMM. Therefore, at each iteration, we can perform the change of basis in $\OO(mn\polylog{n})$ and $m$ eigenvalue computations in $\OO(mn\polylog{n})$ time. The next theorem states that the resulting algorithm (Algorithm~\ref{alg:fast:isotrop}) runs in $\OO(mn^2 \polylog{n})$ time. We need the following technical lemma, before stating the theorem.
%
%
%
%
%
%%%%%%%%%%%%%%%%%%%%%%%%%%%%%%%%%%%%%%%%%%%%%%%%%%%
\begin{lemma}\label{lem:technical_cosh}
%%%%%%%%%%%%%%%%%%%%%%%%%%%%%%%%%%%%%%%%%%%%%%%%%%%
Assume that the first $(i-1)$ indices, $i< t$ have been fixed by Algorithm~\ref{alg:fast:isotrop}. Let $\Phi_k^{(i)}$ be the value of the potential function when the index $k$ has been selected at the next iteration of the algorithm. Similarly, let $\widetilde{\Phi}_k^{(i)}$ be the (approximate) value of the potential function computed using Lemma~\ref{lem:comp_eigs} within an additive error $\delta>0$ for all eigenvalues. Then,
\begin{align*}
	\e^{-\delta} \Phi_k^{(i)} \leq \widetilde{\Phi}^{(i)}_k \leq  \e^{\delta} \Phi_k^{(i)}
\end{align*}
\end{lemma}
%
%
%%%%%%%%%%%%%%%%%%%%%%%%%%%%%%%%%%%%%%%%%%%%%%%%%%%
\begin{proof}
%%%%%%%%%%%%%%%%%%%%%%%%%%%%%%%%%%%%%%%%%%%%%%%%%%%
Let $\tau_1,\tau_2 , \ldots , \tau_n$ be the eigenvalues of $\Lambda_{\{i -1\}} + Z_{(k)} \otimes Z_{(k)}$. Let $\widetilde{\tau}_1, \widetilde{\tau}_2 ,\dots ,\widetilde{\tau}_n$ be the approximate eigenvalues of the latter matrix when computed via Lemma~\ref{lem:comp_eigs} within an additive error $\delta>0 $, i.e, $|\widetilde{\tau}_j - \tau_j| \leq \delta$ for all $j\in{[n]}$.

First notice that, by Step $5$ of Algorithm~\ref{alg:fast:isotrop}, $\Phi_k^{(i)} = 2 \sum_{j=1}^{n} \cosh (\tau_j - \lambda i)$. Similarly, $\widetilde{\Phi}_k^{(i)}:= 2 \sum_{j=1}^{n} \cosh (\widetilde{\tau}_j - \lambda i)$. By the definition of the hyperbolic cosine, we get that
\begin{align*}
	\sum_{j=1}^{n} \cosh (\widetilde{\tau}_j - \lambda i )   & =  \sum_{j=1}^{n} \cosh (\tau_j - \lambda i  + \widetilde{\tau}_j - \tau_j )  \\
	& = \frac1{2}\sum_{j=1}^{n} \left[\exp (\tau_j - \lambda i)\exp(\widetilde{\tau}_j - \tau_j ) + \exp (-\tau_j + \lambda i)\exp(-\widetilde{\tau}_j + \tau_j )\right].
\end{align*}
To derive the upper bound notice that $\sum_{j=1}^{n} \cosh (\widetilde{\tau}_j - \lambda i )  \leq  \sum_{j=1}^{n} \cosh (\tau_j - \lambda i) \max_{j\in{[n]}}\{ \exp(\widetilde{\tau}_j - \tau_j), \exp ( - \widetilde{\tau}_j + \tau_j) \}$
and the maximum is upper bounded by $\exp(\delta)$. Similarly, for the lower bound.
%we get that
%\begin{align*}
%\sum_{j=1}^{n} \cosh (\widetilde{\tau}_j - \lambda i ) & \geq & \sum_{j=1}^{n} \cosh (\tau_j - \lambda i) \min_{j\in{[n]}}\{ \exp(\widetilde{\tau}_j - \tau_j), \exp ( - %\widetilde{\tau}_j + \tau_j) \}
%\end{align*}
%and the minimum is lower bounded by $\exp(-\delta)$.
%%%%%%%%%%%%%%%%%%%%%%%%%%%%%%%%%%%%%%%%%%%%%%%%%%%
\end{proof}
%%%%%%%%%%%%%%%%%%%%%%%%%%%%%%%%%%%%%%%%%%%%%%%%%%%
%
%
%
%
%
\begin{theorem}\label{thm:derand:isotrop:fast}
Let $\matA$ be an $m\times n$ matrix with $\matA^\top \matA = \Id_n$, $m\geq n$ and $ 0 < \eps <1$. Algorithm~\ref{alg:fast:isotrop} returns at most $t=\OO(n  \ln n/\eps^2)$ indices $x_1^*,x_2^*,\ldots x_t^*$ over $[m]$ with corresponding scalars $s_1,s_2,\ldots ,s_t$ using $\widetilde{\OO}(mn^2 \log^3 n /\eps^2 )$ operations such that
\begin{equation}\label{eq:main_thm:fast:main_eqn}
	\norm{ \sum_{i=1}^{t} s_i \ac{x_i^*} \otimes \ac{x_i^*} - \Id_n} \leq \eps.
\end{equation}
\end{theorem}
%%%%%%%%%%%%%%%%%%%%%%%%%%%%%%%%%%%%%%%%%%%%%%%%%%%
%\clearpage
%%%%%%%%%%%%%%%%%%%%%%%%%%%%%%%%%%%%%%%%%%%%%%%%%%%
%%%%%%%%%%%%%%%%%%%%%%%%%%%%%%%%%%%%%%%%%%%%%%%%%%%
%%%%%%%%%%%%%%%%%%%%%%%%%%%%%%%%%%%%%%%%%%%%%%%%%%%
\begin{proof}
%(of Theorem~\ref{thm:derand:isotrop:fast})
%%%%%%%%%%%%%%%%%%%%%%%%%%%%%%%%%%%%%%%%%%%%%%%%%%%
%%%%%%%%%%%%%%%%%%%%%%%%%%%%%%%%%%%%%%%%%%%%%%%%%%%
%%%%%%%%%%%%%%%%%%%%%%%%%%%%%%%%%%%%%%%%%%%%%%%%%%%
The proof consists of three steps: (\emph{a}) we show that Algorithm~\ref{alg:fast:isotrop} is a reformulation of Algorithm~\ref{alg:matrix:hyperbolic}; (\emph{b}) we prove that in Step $5$ of Algorithm~\ref{alg:fast:isotrop} it is enough to compute the values of the potential function within a sufficiently small multiplicative error using Lemma~\ref{lem:comp_eigs}, and (\emph{c}) we give the advertised bound on the running time of Algorithm~\ref{alg:fast:isotrop}.
%

%
Set $p_i = \norm{\ac{i}}^2/\frobnorm{\matA}^2$, $f(i) = \ac{i}\otimes \ac{i}/p_i - \Id_n$ and $s_i=1/p_i$ for every $i\in{[m]}$. Observe that $\frobnorm{\matA}^2=\trace{\matA^\top \matA} = \trace{\Id_n} = n$. Let $X$ be a random variable distributed over $[m]$ with probability $p_i$. Notice that $\EE{f(X)} = \zeromtx_n$ and $\gamma = n $, since $\norm{f(i)} = \norm{n \ac{i}\otimes \ac{i}/\norm{\ac{i}}^2 - \Id_n } \leq n $ for every $i\in{[m]}$. Moreover, a direct calculation shows that $\EE{f(X)^2} = \EE{ (\ac{X} \otimes \ac{X}/p_X)^2} - \Id_n = n\sum_{i=1}^{m} \ac{i}\otimes \ac{i} - \Id_n = (n-1)\Id_n $, hence $\rho^2 \leq  n$. Algorithm~\ref{alg:matrix:hyperbolic} with $t=\OO(n\ln n /\eps^2)$ returns indices $x_1^*,x_2^*,\dots, x_t^*$ so that $\norm{\frac1{t} \sum_{j=1}^t f_j(x_j^*)} \leq \frac{\gamma \ln ( 2n)}{t \eps} + \eps \rho^2 /\gamma \leq 2\eps$. We next prove by induction that the same set of indices are also returned by Algorithm~\ref{alg:fast:isotrop}.
%

%
For ease of presentation, rescale every row of the input matrix $\matA$, i.e., set $ \widehat{\matA}_{(k)} = \ac{k} \sqrt{ \theta / p_{k}}$ for every $k\in{[m]}$ (see Steps $2$ and $3$ of Algorithm~\ref{alg:fast:isotrop}). For sake of the analysis, let us define the following sequence of symmetric matrices of size $n$
\begin{align*}
\matT_{\{0\}} &:= \zeromtx_n,\\
\matT_{\{i\}} &:=  \matT_{\{i - 1\}} + \widehat{\matA}_{(x_i^*)} \otimes \widehat{\matA}_{(x_i^*)}  \text{ for } i\in{[t]}
\end{align*}
with eigenvalue decompositions $\matT_{\{i\}} = \matQ_{\{i\}} \Lambda_{\{i\}} \matQ^\top_{\{i\}} $, where $\Lambda_{\{i\}}$ are diagonal matrices containing the eigenvalues and the columns of $\matQ_{\{i\}}$ contain the corresponding eigenvectors. Set $\matQ_{\{0\}}=\Id$ and $\Lambda_{\{0\}}=\zeromtx$.
Notice that for every $k\in{[m]}$, by the eigenvalue decomposition of $\matT_{\{ i - 1\}}$, $\matT_{\{ i - 1\}} +  \widehat{\matA}_{(k)}\otimes \widehat{\matA}_{(k)} = \matQ_{\{i-1\}}\left(\Lambda_{\{i-1\}} + \matQ_{\{i-1\}}^\top \widehat{\matA}_{(k)} \otimes \matQ_{\{i-1\}}^\top \widehat{\matA}_{(k)}\right) \matQ_{\{i - 1\}}^\top.$ Observe that the above matrix (left hand side) and $\Lambda_{\{i-1\}} + \matQ_{\{i-1\}}^\top \widehat{\matA}_{(k)} \otimes \matQ_{\{i-1\}}^\top \widehat{\matA}_{(k)}$ have the same eigenvalues, since they are similar matrices. Let $\Lambda_{\{i-1\}} + \matQ_{\{i-1\}}^\top \widehat{\matA}_{(x_i^*)} \otimes \matQ_{\{i-1\}}^\top \widehat{\matA}_{(x_i^*)}  = \matU_{\{i\}} \Lambda_{\{i\}} \matU_{\{i\}}^\top$ be its eigenvalue decomposition\footnote{by its definition, $\matT_{\{i\}}$ has the same eigenvalues with $\Lambda_{\{i-1\}} + \matQ_{\{i-1\}}^\top \widehat{\matA}_{(x_i^*)} \otimes \matQ_{\{i-1\}}^\top \widehat{\matA}_{(x_i^*)}$.}. Then
\begin{align*}
\matT_{\{i - 1\}} + \widehat{\matA}_{(x_i^*)} \otimes \widehat{\matA}_{(x_i^*)} &  =  \matQ_{\{i-1\}}\left(\Lambda_{\{i-1\}} + \matQ_{\{i-1\}}^\top \widehat{\matA}_{(x_i^*)} \otimes \matQ_{\{i-1\}}^\top \widehat{\matA}_{(x_i^*)}\right) \matQ_{\{i - 1\}}^\top\\
																				&  =  \matQ_{\{i-1\}}\matU_{\{i\}} \Lambda_{\{i\}} \matU_{\{i\}}^\top  \matQ_{\{i - 1\}}^\top.
\end{align*}
It follows that $\matQ_{\{i\}} = \matQ_{\{i-1\}} \matU_{\{i\}}$ for every $i\geq 1$, so $\matQ_{\{i\}} = \matU_{\{1\}} \matU_{\{2\}} \dots  \matU_{\{i\}}$. The base case of the induction is immediate. Now assume that Algorithm~\ref{alg:fast:isotrop} has returned the same indices as Algorithm~\ref{alg:matrix:hyperbolic} up to the $(i-1)$-th iteration. It suffices to prove that at the $i$-th iteration Algorithm~\ref{alg:fast:isotrop} will return the index $x_i^*$.
%

%
We start with the expression in Step $4$ of Algorithm~\ref{alg:matrix:hyperbolic} and prove that it's equivalent (up to a fixed multiplicative constant factor) with the expression in Step $5$ of Algorithm~\ref{alg:fast:isotrop}. Indeed, for any $k\in{[m]}$, (let $\matC := \theta\sum_{j=1}^{i-1} f(x_j^*)$)
\begin{align*}
&2\trace{\coshm{ \matC + \theta f(k) }}  =  \trace{\expm{ \matC  + \theta f(k) } + \expm{-\matC - \theta f(k) }}  \\
																  & = \trace{\expm{ \matT_{\{i-1\}} + \widehat{\matA}_{(k)}\otimes \widehat{\matA}_{(k)} - \theta i \Id } + \expm{- \matT_{\{i-1\}} - \widehat{\matA}_{(k)}\otimes \widehat{\matA}_{(k)} + \theta i \Id}}  \\
																   &= \trace{\expm{ \matT_{\{i-1\}} + \widehat{\matA}_{(k)}\otimes \widehat{\matA}_{(k)} }\e^{- \theta i} + \expm{- \matT_{\{i-1\}} - \widehat{\matA}_{(k)}\otimes \widehat{\matA}_{(k)} }\e^{ \theta i}}
\end{align*}
where we used the definition of $\coshm{\cdot}$, $f(i)$ and $\matT_{\{i-1\}}$ and the fact that the matrices commute. In light of Algorithm~\ref{alg:fast:isotrop} and the induction hypothesis, observe that the $m\times n$ matrix $\matZ$ at the start of the $i$-th iteration of Algorithm~\ref{alg:fast:isotrop} is equal to $\widehat{\matA} \matU_{\{1\}} \matU_{\{2\}} \ldots \matU_{\{i -1 \}}=  \widehat{\matA} \matQ_{\{i - 1\}}$. Now, multiply the latter expression that appears inside the trace with $\matQ_{\{i-1\}}^\top $ from the left and $\matQ_{\{i-1\}}$ from the right, it follows that (let $\matC := \theta\sum_{j=1}^{i-1} f(x_j^*)$)
\begin{align*}
	2\trace{\coshm{ \matC + \theta f(k) }}  =  \trace{\expm{ \Lambda_{\{i-1\}} + \matZ_{(k)}\otimes \matZ_{(k)} }\e^{- \theta  i} + \expm{- \Lambda_{\{i-1\}} - \matZ_{(k)}\otimes \matZ_{(k)} }\e^{ \theta i}}
\end{align*}
using that $\matQ_{\{i-1\}}$ are the eigenvectors of $\matT_{\{i - 1\}}$ and the cyclic property of trace. This concludes part (\emph{a}).
%

%
Next we discuss how to deal with the technicality that arises from the approximate computation of the $\argmin$ expression in Step $5$ of Algorithm~\ref{alg:fast:isotrop}. First, let's assume that we have approximately (by invoking Lemma~\ref{lem:comp_eigs}) minimized the potential function in Step $5$ of Algorithm~\ref{alg:fast:isotrop}; denote this sequence of potential function values by  $\widetilde{\Phi}^{(1)},\ldots , \widetilde{\Phi}^{(t)}$. Next, we sufficiently bound the parameter $b$ of Lemma~\ref{lem:comp_eigs} so that the above approximation will not incur a significant multiplicative error.
%

%
Recall that at every iteration, by Ineq.~\eqref{ineq:barrier_incr} there exists an index over $[m]$ such that the current value of the potential function increases by at most a multiplicative factor $\exp\left( \eps^2 \rho^2 / \gamma^2\right)$. Lemma~\ref{lem:technical_cosh} tells us that at every iteration of Algorithm~\ref{alg:fast:isotrop} we increase the value of the potential function (by not selecting the optimal index over $[m]$) by at most an \emph{extra} multiplicative factor $\e^{2\delta}$, where $\delta$ is the additive error when computing the eigenvalues of the matrix in Step $5$ via Lemma~\ref{lem:comp_eigs}. Therefore, it follows that $\widetilde{\Phi}^{(t)} \leq \exp( 2\delta t) \Phi^{(t)}.$
%

%
Observe that, at the $i$-th iteration we apply Lemma\ref{lem:comp_eigs} on a matrix $\sum_{j=1}^{i} \widehat{\matA}_{(x_j)} \otimes \widehat{\matA}_{(x_j)}$ for some indices $x_j\in{[m]}$ and moreover $\norm{\sum_{j=1}^{i}{ \widehat{\matA}_{(x_j)} \otimes \widehat{\matA}_{(x_j)}} } =  \norm{ \theta \sum_{j=1}^{i}{ \matA_{(x_j)} \otimes \matA_{(x_j)} / p_{x_j} } } =   \norm{\theta \sum_{j=1}^{i} f(x_j ) - \theta i \Id}$. Triangle inequality tells us that $\norm{ \sum_{j=1}^{i}{ \widehat{\matA}_{(x_j)} \otimes \widehat{\matA}_{(x_j)}} }$ is at most $2\gamma\theta t$ for every $i \in{[t]}$. It follows that $\delta$ is at most $2^{-b+1} \theta t \gamma$ where $b$ is specified in Lemma~\ref{lem:comp_eigs}. The above discussion suggests that by setting $b= \OO(\log( \theta \gamma t))=\OO(\log (n\log n /\eps^3))$ we can guarantee that the potential function $\widetilde{\Phi}^{(t)} \leq  2n \exp\left( 3t \eps^2 \right)$. This concludes part~(\emph{b}).
%%%%%%%%%%%%%%%%%%%%%%%%%%%%%%%%%%%%%%%%%%%%%%%%%%%
%%%%%%%%%%%%%%%%%%%%%%%%%%%%%%%%%%%%%%%%%%%%%%%%%%%
%

%
Finally, we conclude the proof by analyzing the running time of Algorithm~\ref{alg:fast:isotrop}. Steps $2$ and $3$ can be done in $\OO(mn)$ time. Step $5$ requires $\widetilde{\OO}(mn\log^2 n )$ operations by invoking $m$ times Lemma~\ref{lem:comp_eigs}. Steps $6$ can be done in $\OO(n^2)$ time and Step $7$ requires $\widetilde{\OO}(mn\log^2 n )$ operations by invoking Lemma~\ref{lem:fast_mm:gerasoulis}. In total, since the number of iterations is $\OO(n\log n /\eps^2)$, the algorithm requires $\widetilde{\OO}( mn^2 \log^3 n /\eps^2)$ operations.
\end{proof}
%%%%%%%%%%%%%%%%%%%%%%%%%%%%%%%%%%%%%%%%%%%%%%%%%%%
%%%%%%%%%%%%%%%%%%%%%%%%%%%%%%%%%%%%%%%%%%%%%%%%%%%
%
%
%
%
%%%%%%%%%%%%%%%%%%%%%%%%%%%%%%%%%%%%%%%%%%%%%%%%%%%
%\clearpage
%%%%%%%%%%%%%%%%%%%%%%%%%%%%%%%%%%%%%%%%%%%%%%%%%%%
\subsection{Spectral Sparsification}\label{sec:sparsif:psd}
%%%%%%%%%%%%%%%%%%%%%%%%%%%%%%%%%%%%%%%%%%%%%%%%%%%
Here, we show that Algorithm~\ref{alg:fast:isotrop} can be used as a bootstrapping procedure to improve the time complexity of~\cite[Theorem~3.1]{phdthesis:Srivastava:2010}, see also~\cite[Theorem~$3.1$]{graph:sparsifiers:twice_ram}. Such an improvement implies faster algorithms for constructing graph spectral sparsifiers,,as we will see in \S~\ref{sec::graph_sparsifiers}, and element-wise sparsification of matrices,as we will see in \S~\ref{sec:sparsification:matrix}.
\begin{theorem}\label{thm:sparsification:here}
Suppose $0 < \eps < 1$ and $\matA = \sum_{i=1}^{m} \v_i \otimes \v_i $ are given, with column vectors $\v_i\in\reals^n $ and $m\geq n$. Then there are non-negative weights $\{s_i\}_{i\leq m}$, at most $ \lceil n/\eps^2 \rceil$ of which are non-zero, such that
\begin{equation}
	(1-\eps)^3 \matA \preceq \widetilde{\matA} \preceq (1+\eps)^3 \matA,
\end{equation}
where $\widetilde{\matA} = \sum_{i=1}^{m}s_i \v_i \otimes \v_i$. Moreover, there is an algorithm that computes the weights $\{s_i\}_{i\leq m}$ in deterministic $\widetilde{\OO}(mn^2 \log^3 n  /\eps^2 + n^4 \log n /\eps^4)$ time.
\end{theorem}
%%%%%%%%%%%%%%%%%%%%%%%%%%%%%%%%%%%%%%%%%%%%%%%%%%%
%%%%%%%%%%%%%%%%%%%%%%%%%%%%%%%%%%%%%%%%%%%%%%%%%%%
\begin{proof}
%(of Theorem~\ref{thm:sparsification:here})
%%%%%%%%%%%%%%%%%%%%%%%%%%%%%%%%%%%%%%%%%%%%%%%%%%%
Assume without loss of generality that $\matA$ has full rank. Define $\u_i = \matA^{-1/2}\v_i$ and notice that $\sum_{i=1}^{m} \u_i \otimes \u_i =\Id_n$. Run Algorithm~\ref{alg:fast:isotrop} with input $\{\u_i\}_{i\in{[m]}}$ and $\eps$ which returns $\{\tau_i\}_{i\leq m}$, at most $t=\OO(n\log n /\eps^2)$ of which are non-zero such that
\begin{equation}\label{eqn:spectral_sparse:inner}
\norm{\sum_{i=1}^{m} \tau_i \u_i \otimes \u_i - \Id_n } \leq \eps.
\end{equation}
Define $\widehat{\matA} = \matA^{1/2}\left(\sum_{i=1}^{m}\tau_i \u_i \otimes \u_i\right) \matA^{1/2} = \sum_{i=1}^{m}\tau_i \v_i\otimes \v_i$. Eqn.~\eqref{eqn:spectral_sparse:inner} is equivalent to $(1-\eps) \Id_n \preceq \sum_{i=1}^{m} \tau_i \u_i\otimes \u_i \preceq (1+\eps) \Id_n$. Conjugating the latter expression by $\matA^{1/2}$ (Lemma~\ref{lem:pert3}), we get that $ (1-\eps) \matA \preceq \widehat{\matA} \preceq (1+\eps) \matA.$ Apply~\cite[Theorem~3.1]{phdthesis:Srivastava:2010} on $\widehat{\matA}$ which outputs a matrix $\widetilde{\matA}=\sum_{i=1}^{m}s_i \v_i\otimes \v_i$ with non-negative weights $\{s_i\}_{i\in{[m]}}$ at most $\lceil n /\eps^2 \rceil$ of which are non-zero, such that $ (1-\eps)^2 \widehat{\matA} \preceq \widetilde{\matA} \preceq (1+\eps)^2 \widehat{\matA}.$ Using the positive semi-definite partial ordering, we conclude that $(1-\eps)^3 \matA \preceq \widetilde{\matA} \preceq (1+\eps)^3 \matA$.
\end{proof}
%%%%%%%%%%%%%%%%%%%%%%%%%%%%%%%%%%%%%%%%%%%%%%%%%%%
%%%%%%%%%%%%%%%%%%%%%%%%%%%%%%%%%%%%%%%%%%%%%%%%%%%
%
%


%
%
%
%%%%%%%%%%%%%%%%%%%%%%%%%%%%%%%%%%%%%%%%%%%%%%%%%%%%%%%
%%%%%%%%%%%%%%%%%%%%%%%%%%%%%%%%%%%%%%%%%%%%%%%%%%%%%%%
\section{Element-wise Matrix Sparsification}
%%%%%%%%%%%%%%%%%%%%%%%%%%%%%%%%%%%%%%%%%%%%%%%%%%%%%%%
%%%%%%%%%%%%%%%%%%%%%%%%%%%%%%%%%%%%%%%%%%%%%%%%%%%%%%%
Element-wise matrix sparsification was pioneered by Achlioptas and McSherry~\cite{matrix:sparsification:AM01,matrix:sparsification:optas}. The authors of~\cite{matrix:sparsification:optas} described sampling-based algorithms to select a small number of elements from an input matrix $\matA \in \RR^{n \times n}$ in order to construct a sparse sketch $\widetilde{\matA} \in \RR^{n \times n}$, which is close to $\matA$ in the operator norm. Such sketches were used in approximate eigenvector computations~\cite{matrix:sparsification:AM01,matrix:sparsification:arora,matrix:sparsification:optas}, semi-definite programming solvers~\cite{fast_SDP:AHK05,Asp09}, and matrix completion problems~\cite{CR09,CT09}. Motivated by their work, we present a simple matrix sparsification algorithm that achieves the best known upper bounds for element-wise matrix sparsification. Moreover, we present the first deterministic element-wise sparsification algorithm by derandomizing the result of Section~\ref{sec:msparse:conc} using the matrix hyperbolic cosine algorithm. Last but not least, we derive strong sparsification bounds for symmetric matrices that have an approximate diagonally dominant\footnote{A symmetric matrix $\matA$ of size $n$ is called \emph{diagonally dominant} if $|\Ae{ii}| \geq \sum_{j\neq i} |\Ae{ij}|$ for every $i\in{[n]}$.} property. Diagonally dominant matrices arise in many applications such as the solution of certain elliptic differential equations via the finite element method~\cite{SDD:Vavasis}, several optimization problems in computer vision~\cite{SDD:vision:Koutis} and computer graphics~\cite{SDD:graphics:Joshi}, to name a few.




%%%%%%%%%%%%%%%%%%%%%%%%%%%%%%%%%%%%%%%%%%%%%%%%%%%%%%%
%%%%%%%%%%%%%%%%%%%%%%%%%%%%%%%%%%%%%%%%%%%%%%%%%%%%%%%
\subsection{Sparsification via Matrix Concentration}\label{sec:msparse:conc}
%%%%%%%%%%%%%%%%%%%%%%%%%%%%%%%%%%%%%%%%%%%%%%%%%%%%%%%
%%%%%%%%%%%%%%%%%%%%%%%%%%%%%%%%%%%%%%%%%%%%%%%%%%%%%%%
%
The main algorithm (Algorithm~\ref{alg:msparse:IPL}) zeroes out ``small'' elements of $\matA$ and randomly samples the remaining elements of $\matA$ with respect to a probability distribution that favors ``larger'' entries.
%
\begin{algorithm}{}
\centerline{\caption{Matrix Sparsification Algorithm}\label{alg:msparse:IPL}}
\begin{algorithmic}[1]
%--------------------------------------------------
% Step 1
%--------------------------------------------------
\State \underline{\textbf{Input:}} $\matA \in \RR^{n \times n}$, accuracy parameter $\epsilon >0$.
%--------------------------------------------------
% Step 2
%--------------------------------------------------
\State \textbf{Let} $\widehat{\matA} = \matA$ and \textbf{zero-out} all entries of $\widehat{\matA}$ that are smaller (in absolute value) than $\epsilon/2n$.
%--------------------------------------------------
% Step 3
%--------------------------------------------------
\State \textbf{Set} $s$ as in Eqn.~\eqref{eqn:sfinal}.
%--------------------------------------------------
% Step 4
%--------------------------------------------------
\State \textbf{For} $t = 1\ldots s$ (i.i.d. trials with replacement) \textbf{randomly sample} indices $(i_t, j_t) $ (entries of $\widehat{\matA}$), with
%
\[\Prob{ (i_t, j_t) =  (i,j)}\ =\ p_{ij}, \quad \mbox{where }p_{ij}:=\widehat{\matA}_{ij}^2/\frobnorm{\widehat{\matA}}^2 \mbox{for all } (i,j) \in [n]\times[n].\]
%
%--------------------------------------------------
%--------------------------------------------------
% Output
%--------------------------------------------------
\State \underline{\textbf{Output:}} \[\widetilde{\matA} = \frac{1}{s}\sum_{t=1}^s \frac{\widehat{\matA}_{i_t j_t}}{p_{i_t j_t}} \e_{i_t} \e_{j_t}^\top \in \RR^{n \times n}.\]
\end{algorithmic}
\end{algorithm}
%
Our sampling procedure selects $s$ entries from $\matA$ (note that $\widehat{\matA}$ from the description of Algorithm~\ref{alg:msparse:IPL} is simply $\matA$, but with elements less than or equal to $\epsilon/(2n)$ zeroed out) in $s$ independent, identically distributed (i.i.d.) trials with replacement. In each trial, elements of $\matA$ are retained with probability proportional to their squared magnitude. Note that the same element of $\matA$ could be selected multiple times and that $\widetilde{\matA}$ contains at most $s$ non-zero entries. Theorem~\ref{thm::msparse:IPL} is our main quality-of-approximation result for Algorithm~\ref{alg:msparse:IPL} and achieves sparsity bounds proportional to $\frobnorm{\matA}^2$.
%
%
\begin{theorem} \label{thm::msparse:IPL}
%
Let $\matA \in \R ^{n \times n}$ be any matrix, let $\epsilon >0 $ be an accuracy parameter, and let $\widetilde{\matA}$ be the sparse sketch of $\matA$ constructed via Algorithm~\ref{alg:msparse:IPL}. If
%
\begin{equation}\label{eqn:sfinal}
%
s = \frac{28n \ln\left(\sqrt{2}n\right)}{\epsilon^{2}}\frobnorm{\matA}^2,
%
\end{equation}
%
then, with probability at least $1-n^{-1}$,
$$
\norm{\matA - \widetilde{\matA}} \leq \epsilon.
$$
$\widetilde{\matA}$ has at most $s$ non-zero entries and the construction of $\widetilde{\matA}$ can be implemented in one pass over the input matrix $\matA$ (see Section~\ref{sxn:onepass}).
%
\end{theorem}
%
We conclude this section with Corollary~\ref{cor:relativeerror}, which is a re-statement of Theorem~\ref{thm::msparse:IPL} involving the \emph{stable rank} of $\matA$, denoted by $\sr{\matA}$ (recall that the stable rank of any matrix $\matA$ is defined as the ratio $\sr{\matA} := \frobnorm{A}^2/\norm{\matA}^2$, which is upper bounded by the rank of $\matA$). The corollary guarantees relative error approximations for matrices of -- say -- constant stable rank, such as the ones that arise in~\cite{recht:simple_completion,CT09}.
%
\begin{corollary}\label{cor:relativeerror}
%
Let $\matA \in \R ^{n \times n}$ be any matrix and let $\varepsilon >0 $ be an accuracy parameter. Let $\widetilde{\matA}$ be the sparse sketch of $\matA$ constructed via Algorithm~\ref{alg:msparse:IPL} (with $\epsilon = \varepsilon\norm{\matA}$). If
%
$s = 28n\sr{\matA} \ln\left(\sqrt{2}n\right) / \varepsilon^{2},$
%
then, with probability at least $1-n^{-1}$,
\[
\norm{\matA - \widetilde{\matA}} \leq \varepsilon\norm{\matA}.
\]
%
\end{corollary}
%
It is worth noting that the sampling algorithm implied by Corollary~\ref{cor:relativeerror} can not be implemented in one pass, since we would need a priori knowledge of the spectral norm of $\matA$ in order to implement Step $2$ of Algorithm~\ref{alg:msparse:IPL}.
%
%\clearpage
%
%%%%%%%%%%%%%%%%%%%%%%%%%%%%%%%%%%%%%%%%%%%%%%%%%%%
%%%%%%%%%%%%%%%%%%%%%%%%%%%%%%%%%%%%%%%%%%%%%%%%%%%
%
\begin{center}
\begin{table}[ht]
	\small
\centering
    \begin{tabular}{ || c | c | c | c | c ||}
	\hline
    \multicolumn{5}{|c|}{} \\
	\multicolumn{5}{|c|}{\underline{\textbf{Randomized Element-wise Matrix Sparsification}}} \\
    \multicolumn{5}{|c|}{} \\
	\hline\hline
	\multirow{2}{*}{} & & & & \\
	\textbf{Sparsity of $\widetilde{\matA}$} &  & \textbf{Failure}  & \textbf{Citation}   & \textbf{Comments} \\
	 &  & \textbf{Probability}  & & \\
	\hline
	\hline
%--------------------------------------------------
%--------------------------------------------------
%	\multirow{2}{*}{} & & & & \\
	    & & & & $\eps > 4 \sqrt{n}\cdot b$  \\
	$16 n \frobnorm{\matA}^2 / \eps^2  + 8^4 n\log^4 n$    & {\footnotesize Expected} & $e^{-19\log^4 n}$ &   \cite{matrix:sparsification:optas}  & $n\geq 700\cdot 10^6$ \\

	\hline
%--------------------------------------------------
%--------------------------------------------------
% The above row has been completed using the following references.
%
% Using LEMMA 4.1 from [matrix:sparsification:optas] for comparison :
% SET m = n , s = 16 n \frobnorm{A}^2 / \eps^2
%
% Sparsity of \tilde{A} : Lemma 4.1 of [matrix:sparsification:optas]. Set s in Bullet (2) such that Bullet (1) gives error eps!
% Failure Probability   : Theorem 1.5 or Lemma 4.1 of [matrix:sparsification:optas].
% Comment 		: Theorem 1.5 Bullet (1), notice the p <= 1, hence we can only derive bound of order 4 b \sqrt{n}.
%
%  c_3 is THE constant of Latala's paper (see Theorem 5 of [Gmatrix:sparsification:Tropp])
%--------------------------------------------------
%--------------------------------------------------
% Also from THEOREM 1.5 of [matrix:sparsification:optas] notice that p should be smaller than 1!! Therefore, the result of [matrix:sparsification:optas] apply only when eps > 4 \sqrt{n} ||A||_{1->\infty}.
%
%--------------------------------------------------
%--------------------------------------------------
 	& & & & $n\geq 1$\\
	$R\cdot b \cdot n \frobnorm{\matA}^2 /\eps^2$   & {\footnotesize Expected} & $e^{-\Omega(R\cdot n ) }$ &   \cite{matrix:sparsification:Tropp}  & $\eps > c_1\sqrt{n\cdot R} \cdot b$ \\
	\hline
%--------------------------------------------------
%--------------------------------------------------
% The above row has been completed using the following references.
%
% Sparsity of \tilde{A} : Page 15 (Second Equation p nm Avg (a_{ij}^2 /b)) of [matrix:sparsification:Tropp].
% To see this result, we have to combine the first two equation of Page 15.
%
% First assume that 2C ( 2+ sqrt{R}) b sqrt{n/p} := \eps from the first equation. Solving for b we get b ~= eps sqrt{p/(n R)} .
% By plugging this value into the second equation of Page 15, we get that
%
% p nm Avg (a_{ij}^2 /b) = p ||A||_F^2 / b^2 = p ||A||_F^2 / (eps^2 p / (n R)) ~= n R ||A||_F^2 / eps^2
%
%
% Failure Probability   : Page 15 (First line) of [matrix:sparsification:Tropp].
% Comment         : Page 15 (Second line) of [matrix:sparsification:Tropp], notice the p <= 1 in the second line.
%
%  c_3 is THE constant of Latala's paper (see Theorem 5 of [matrix:sparsification:Tropp])
%--------------------------------------------------
	& & & & $c_2 \leq 45^2$ \\
	$c_2 n \log^2 \left(\frac{n}{\log^2 n}\right)\log n \frobnorm{\matA}^2/\eps^2$  & {\footnotesize Expected}  & $1/n$ &   \cite{drineas:sparsification_via_khintchine}  & $\eps>0,\ n\geq 300$, \\
	\hline
%--------------------------------------------------
%--------------------------------------------------
	& & & & $\eps>0,\ n\geq 300$ \\
	% See Theorem~$1$
	$c_3 n \log^3 n \frobnorm{\matA}^2/\eps^2$   & {\footnotesize Expected} & $1/n$ &   \cite{drineas:tensor_sparsification}  & {\footnotesize Extends to tensors}\\
 	\hline
%--------------------------------------------------
%--------------------------------------------------
    	\multirow{1}{*}{} & & & & \\
	$c_4 \sqrt{n}\sum_{ij}|\matA_{ij}| / \eps $   & {\footnotesize Exact} & $e^{-\Omega(n)}$ &   \cite{matrix:sparsification:arora}  & $\eps >0$, $n\geq 1$\\
	\hline
%--------------------------------------------------
%--------------------------------------------------
% Notes about AHK06 paper : The strongest result of [matrix:sparsification:AM01] is : With probability at least 1-1/n, we have that ||A - \widetilde{A} || < eps, Expected number \OO(n/eps^2 ||A||_F^2 + n) non-zero entries.
% To Be precise : The failure probability is equal to : \min\{e^{-C_2'\frac{\sqrt{n} \sum_{ij}|A_{ij}|}{\eps}},e^{-\Omega(n)}\}
%--------------------------------------------------
%--------------------------------------------------
	\multirow{2}{*}{} & & & & \\
	$14n \ln\left(2n/\delta\right)\frobnorm{\matA}^2/ \eps^{2}$ & {\footnotesize Exact} & $\delta$ &  Thm~\ref{thm::msparse:IPL}  & $\eps>0,\ n\geq 1$\\
	\hline
%--------------------------------------------------
%--------------------------------------------------
%--------------------------------------------------
%--------------------------------------------------
%	\multirow{2}{*}{} & & & & \\
%	$\OO\left(\sum_{ij}|\matA_{ij}| \log\left(\frac{\sum_{ij}|\matA_{ij}|}{\eps^{2}\delta}\right) \right)$ & {\footnotesize Exact} & $\delta$ &  Thm~\ref{thm:msparse:low}  & $0<\eps<1,\ n\geq 1$\\
%	\hline
%--------------------------------------------------
%--------------------------------------------------
%    \multirow{2}{*}{} & & & & \\
%	$17\sqrt{n} \sqrt{\ln\left(\sqrt{2}n\right)}\sum_{ij}|A_{ij}| / \eps$ & (D) & $1/n$ &   {\small Theorem~\ref{thm::msparse:IPL:aroralike}}  & $\eps>0,\ n\geq 1$\\
%& &       &                       & \\
%\hline
%--------------------------------------------------
%--------------------------------------------------
\end{tabular}

\caption{\small Summary of prior results in element-wise matrix sparsification. The first column indicates the number of non-zero entries in $\widetilde{\matA}$, whereas the second column indicates whether this number is exact or simply holds in expectation. In terms of notation, we let $b$ denote the $\max_{i,j}|\matA_{ij}|$ and $R$ denote $\max_{ij} {|\matA_{ij}|} / \min_{\matA_{ij}\neq 0} |\matA_{ij}|$. Finally, $c_1,c_2,c_3,c_4$ denote unspecified positive constants.}\label{table:summary}
\end{table}
\end{center}
%%%%%%%%%%%%%%%%%%%%%%%%%%%%%%%%%%%%%%%%%%%%%%%%%%%
%%%%%%%%%%%%%%%%%%%%%%%%%%%%%%%%%%%%%%%%%%%%%%%%%%%
%%%%%%%%%%%%%%%%%%%%%%%%%%%%%%%%%%%%%%%%%%%%%%%%%%%%%%%
%%%%%%%%%%%%%%%%%%%%%%%%%%%%%%%%%%%%%%%%%%%%%%%%%%%%%%%
\paragraph{Related Work.}
%%%%%%%%%%%%%%%%%%%%%%%%%%%%%%%%%%%%%%%%%%%%%%%%%%%%%%%
%%%%%%%%%%%%%%%%%%%%%%%%%%%%%%%%%%%%%%%%%%%%%%%%%%%%%%%
In this section (as well as in Table~\ref{table:summary}), we present a head-to-head comparison of our result (\cite{matrix:sparsification:IPL2011}) with all existing (to the best of our knowledge) bounds on randomized matrix sparsification. In~\cite{matrix:sparsification:AM01,matrix:sparsification:optas} the authors presented a sampling method that requires in \emph{expectation} $16 n \frobnorm{\matA}^2 / \eps^2  + 8^4 n\log^4 n$ non-zero entries in $\widetilde{\matA}$ in order to achieve an accuracy guarantee $\eps$ with a failure probability of at most $e^{-19\log^4 n}$. Compared with our result, their bound holds only when $\eps > 4\sqrt{n} \cdot \max_{i,j}\abs{\matA_{ij}}$ and, in this range, our bounds are superior when $\frobnorm{\matA}^2 / (\max_{i,j}\abs{\matA_{ij}})^2=o(n\log^3 n)$. It is worth mentioning that the constant involved in~\cite{matrix:sparsification:AM01,matrix:sparsification:optas} is two orders of magnitude larger than ours and more importantly their results hold only when $n\geq 700 \cdot 10^6$.
%

%
In~\cite{matrix:sparsification:Tropp}, the authors study the $\|\cdot\|_{\infty \rightarrow 2}$ and $\|\cdot\|_{\infty \rightarrow 1}$ norms in the matrix sparsification context and they also present a sampling scheme analogous to ours. They achieve (in expectation) a sparsity bound of $R n\frobnorm{\matA}^2 \max_{i,j}\abs{\matA_{ij}}  /\eps^2$  when $\eps \geq \sqrt{nR}\max_{i,j} \abs{\matA_{ij}}$; here $R:=\max_{ij} {\abs{\matA_{ij}}} / \min_{\matA_{ij}\neq 0} \abs{\matA_{ij}}$. Thus, our results are superior (in the above range of $\eps$) when $R\cdot \max_{i,j} \abs{\matA_{ij}} =\omega( \log n)$.
%

%
It is harder to compare our method to the work of~\cite{matrix:sparsification:arora}, which depends on the $\sum_{i,j=1}^n \abs{\matA_{ij}}$. The latter quantity is, in general, upper bounded only by $n \frobnorm{\matA}$, in which case the sampling complexity of~\cite{matrix:sparsification:arora} is much worse, namely $\OO(n^{3/2}\frobnorm{\matA}/\eps)$.
%
%However, our analysis is flexible enough to provide bounds that depend on $\sum_{i,j=1}^n \abs{A_{ij}}$ by a simple adjustment on Algorithm~\ref{alg:msparse:IPL}, see Theorem~\ref{thm::msparse:IPL:aroralike}. For comparison %purposes, we present a similar algorithm of our main algorithm (Algorithm~$3$) that achieves bounds weaker by a $\sqrt{\log n}$ factor with those of~\cite{matrix:sparsification:arora} (see Appendix).
%
Finally, the recent bounds on matrix sparsification via the non-commutative Khintchine's inequality in~\cite{drineas:sparsification_via_khintchine} are inferior compared to ours in terms of sparsity guarantees by at least $\OO(\ln^2 ( n /\ln^2 n))$. Nevertheless, we should mention that the bounds of~\cite{drineas:sparsification_via_khintchine} can be extended to multi-dimensional matrices (tensors), whereas our result does not generalize to this setting; see~\cite{drineas:tensor_sparsification} for details.
%
%
\subsubsection{Background}
%
\paragraph{Implementing the Sampling in one Pass over the Input Matrix.}\label{sxn:onepass}
%
We now discuss the implementation of Algorithm~\ref{alg:msparse:IPL} in one pass over the input matrix $\matA$. Towards that end, we will leverage (a slightly modified version of) Algorithm \textsc{Select} (p. 137 of~\cite{matrixmult:drineas}).
%
\begin{algorithm}
\centerline{\caption{One-pass \textsc{Select} algorithm}}
\begin{algorithmic}[1]
\State \underline{\textbf{Input:}} $\Ae{ij}$ for all $(i,j)\in [n]\times [n]$, arbitrarily ordered and $\epsilon >0$.
\State $N=0$.
\State \textbf{For all} $(i,j)\in [n]\times [n]$ \textbf{such that} $\Ae{ij}^2 > \frac{\epsilon^2}{4n^2}$
\begin{itemize}
\item $N = N + \Ae{ij}^2$.
\item \textbf{Set} $(I,J)=(i,j)$ and $S = \Ae{ij}$ \textbf{with probability} $\frac{\Ae{ij}^2}{N}$.
\end{itemize}
\State \underline{\textbf{Output:}} Return $(I,J)$, $S$ and $N$.
\end{algorithmic}
\end{algorithm}
%
\noindent We note that Step $3$ essentially operates on $\widehat{\matA}$. Clearly, in a single pass over the data we can run in parallel $s$ copies of the \textsc{Select} Algorithm (using a total of $\OO(s)$ memory) to effectively return $s$ independent samples from $\widehat{\matA}$. Lemma~$1$ (page $136$ of~\cite{matrixmult:drineas}, note that the sequence of the $\Ae{ij}^2$'s is all-positive) guarantees that each of the $s$ copies of \textsc{Select} returns a sample satisfying:
%
\[\Prob{ (i_t,j_t) = (i, j) }\ =\ \frac{\widehat{\matA}_{ij}^2}{\sum_{i,j=1}^n \widehat{\matA_{ij}}^2} = \frac{\widehat{\matA}_{ij}^2}{\frobnorm{\widehat{\matA}}^2},\quad \mbox{for all }t=1,\dots ,s.\]
%
Finally, in the parlance of Step $5$ of Algorithm~\ref{alg:msparse:IPL}, $(i_t, j_t)$ is set to $(I,J)$ and $p_{i_tj_t}$ is set to $S^2/N$ for all $t \in [s]$.
%
\paragraph{Proof of Theorem~\ref{thm::msparse:IPL}}
%
The proof of Theorem~\ref{thm::msparse:IPL} will combine Lemmas~\ref{lem:lem1} and~\ref{lem:lem4} in order to bound $\norm{\matA - \widetilde{\matA}}$ as follows:
%
\begin{align*}
	\norm{\matA-\widetilde{\matA}} = \norm{\matA-\widehat{\matA} + \widehat{\matA} - \widetilde{\matA}} \leq \norm{\matA-\widehat{\matA}} + \norm{\widehat{\matA} - \widetilde{\matA}} \leq \epsilon/2+\epsilon/2 = \epsilon.
\end{align*}
%
The failure probability of Theorem~\ref{thm::msparse:IPL} emerges from Lemma~\ref{lem:lem4}, which fails with probability at most $n^{-1}$ for the choice of $s$ in Eqn.~(\ref{eqn:sfinal}). The proof of Lemma~\ref{lem:lem4} will involve the matrix-valued Bernstein bound, see Chapter~\ref{chap:intro}.
%
\paragraph{Bounding $\norm{\matA - \widehat{\matA}}$}
%
\begin{lemma}\label{lem:lem1}
Using the notation of Algorithm~\ref{alg:msparse:IPL}, $\norm{\matA - \widehat{\matA}} \leq \epsilon/2$.
\end{lemma}
\begin{proof}
%
Recall that the entries of $\widehat{\matA}$ are either equal to the corresponding entries of $\matA$ or they are set to zero if the corresponding entry of $\matA$ is (in absolute value) smaller than $\epsilon/(2n)$. Thus,
%
\[\norm{\matA-\widehat{\matA}}^2 \leq \frobnorm{\matA-\widehat{\matA}}^2 = \sum_{i,j=1}^n \left(\matA - \widehat{\matA}\right)_{ij}^2 \leq \sum_{i,j=1}^n \frac{\epsilon^2}{4n^2}\leq \frac{\epsilon^2}{4}.\]
%
\end{proof}
\paragraph{Bounding $\norm{\widehat{\matA} - \widetilde{\matA} }$}
$\newline$
In order to prove our main result in this section (Lemma~\ref{lem:lem4}) we will leverage a powerful matrix-valued Bernstein bound originally proven in~\cite{recht:simple_completion} (Theorem 3.2). We restate this theorem, slightly rephrased to better suit our notation.
%
\begin{theorem}\label{thm::recht}\textsc{[Theorem 3.2~of~\cite{recht:simple_completion}]}
%
Let $\matM_1,\matM_2,\ldots,\matM_s$ be independent, zero-mean random matrices in $\RR^{n \times n}$. Suppose $\max_{t \in [s]} \left\{\norm{\EE\left(\matM_t \matM_t^\top\right)},\norm{\EE\left(\matM_t^\top \matM_t\right)}\right\}\leq \rho^2$ and $\norm{\matM_t} \leq \gamma$ for all $t \in [s]$. Then, for any $\tau > 0$,
%
$$\norm{\frac{1}{s} \sum_{t=1}^s \matM_t} \leq \tau$$
%
holds, subject to a failure probability of at most
%
$$2n \exp\left(-\frac{s\tau^2/2}{\rho^2 + \gamma \tau/3}\right).$$
%
\end{theorem}
%
In order to apply the above theorem, using the notation of Algorithm~\ref{alg:msparse:IPL}, we set $\matM_t = \frac{\widehat{\matA}_{i_t j_t}}{p_{i_t j_t}} \e_{i_t} \e_{j_t}^\top - \widehat{\matA}$ for all $t \in [s]$ to obtain
%
\begin{equation}\label{eqn:mainexp}
%
\frac{1}{s} \sum_{t=1}^s \matM_t = \frac{1}{s} \sum_{t=1}^s \left[\frac{\widehat{\matA}_{i_t j_t}}{p_{i_t j_t}} \e_{i_t} \e_{j_t}^\top - \widehat{\matA}\right] = \widetilde{\matA} - \widehat{\matA}.
%
\end{equation}
%
It is easy to argue that $\EE \left(\matM_t\right) = \zeromtx_{n}$ for all $t \in [s]$. Indeed, if we consider that $\sum_{i,j=1}^n p_{ij}=1$ and $\widehat{\matA} = \sum_{i,j=1}^n \widehat{\matA}_{i j} \e_{i}\e_{j}^\top$ we obtain
%
\[\EE\left(\matM_t\right) = \sum_{i,j=1}^n p_{ij} \left(\frac{\widehat{\matA}_{i j}}{p_{i j}} \e_{i}\e_{j}^\top - \widehat{\matA}\right) = \sum_{i,j=1}^n \widehat{\matA}_{i j} \e_{i}\e_{j}^\top - \sum_{i,j=1}^n p_{ij} \widehat{\matA} = \zeromtx_{n}.\]
%
Our next lemma bounds $\norm{\matM_t}$ for all $t \in [s]$.
%
\begin{lemma}\label{lem:lem2}
%
Using our notation, $\norm{\matM_t} \leq 4n\epsilon^{-1}\frobnorm{\widehat{\matA}}^2$ for all $t \in [s]$.
%
\end{lemma}

\begin{proof}
%
First, using the definition of $\matM_t$ and the fact that $p_{i_t j_t} = \widehat{\matA}_{i_tj_t}^2/\frobnorm{\widehat{\matA}}^2$,
%
$$\norm{\matM_t} = \norm{\frac{\widehat{\matA}_{i_t j_t}}{p_{i_t j_t}} \e_{i_t}\e_{j_t}^\top - \widehat{\matA}} \leq \frac{\frobnorm{\widehat{\matA}}^2}{\abs{\widehat{\matA}_{i_tj_t}}}+\norm{\widehat{\matA}} \leq \frac{2n\frobnorm{\widehat{\matA}}^2}{\epsilon}+\frobnorm{\widehat{\matA}}. $$
%
The last inequality follows since all entries of $\widehat{\matA}$ are at least $\epsilon/(2n)$ and the fact that $\norm{\widehat{\matA}} \leq \frobnorm{\widehat{\matA}}$. We can now assume that
%
\begin{equation}\label{eqn:assumption}
\frobnorm{\widehat{\matA}} \leq \frac{2n\frobnorm{\widehat{\matA}}^2}{\epsilon}
\end{equation}
%
to conclude the proof of the lemma. To justify our assumption in Eqn.~(\ref{eqn:assumption}), we note that if it is violated, then it must be the case that $\frobnorm{\widehat{\matA}} < \epsilon /(2n)$. If that were true, then all entries of $\widehat{\matA}$ would be equal to zero. (Recall that all entries of $\widehat{\matA}$ are either zero or, in absolute value, larger than $\epsilon/(2n)$.) Also, if $\widehat{\matA}$ were identically zero, then \textit{(i)} $\widetilde{\matA}$ would also be identically zero and, \textit{(ii)} all entries of $\matA$ would be at most $\epsilon/(2n)$. Thus, $$\norm{\matA-\widetilde{\matA}} = \norm{\matA} \leq \frobnorm{\matA} \leq \sqrt{n^2 \frac{\epsilon^2}{4n^2}}=\frac{\epsilon}{2}.$$
%
Thus, if the assumption of Eqn.~(\ref{eqn:assumption}) is not satisfied, the resulting all-zeros $\widetilde{\matA}$ still satisfies Theorem~\ref{thm::msparse:IPL}.
%
\end{proof}

\noindent Our next step towards applying Theorem~\ref{thm::recht} involves bounding the spectral norm of the expectation of $\matM_t\matM_t^\top$. The spectral norm of the expectation of $\matM_t^\top\matM_t$ admits a similar analysis and the same bound and is omitted.
%
\begin{lemma}\label{lem:lem3}
%
Using our notation, $\norm{\EE\left(\matM_t\matM_t^\top\right)} \leq n\frobnorm{\widehat{\matA}}^2$ for any $t \in [s]$.
%
\end{lemma}

\begin{proof}
%
We start by evaluating $\EE\left(\matM_t \matM_t^\top\right)$; recall that $p_{ij} = \widehat{\matA}_{ij}^2/\frobnorm{\widehat{\matA}}^2$:
%
\begin{align*}
%
\EE\left(\matM_t\matM_t^\top\right) &= \EE\left(\left(\frac{\widehat{\matA}_{i_tj_t}}{p_{i_tj_t}} \e_{i_t} \e_{j_t}^\top-\widehat{\matA}\right)\left(\frac{\widehat{\matA}_{i_tj_t}}{p_{i_tj_t}}\e_{j_t}\e_{i_t}^\top-\widehat{\matA}^\top\right)\right)\\
%
&= \sum_{i,j=1}^n p_{ij}\left(\frac{\widehat{\matA}_{ij}}{p_{ij}} \e_{i}\e_{j}^\top-\widehat{\matA}\right)\left(\frac{\widehat{\matA}_{ij}}{p_{ij}} \e_{j}\e_{i}^\top - \widehat{\matA}^\top\right)\\
%
&= \sum_{i,j=1}^n \left(\frac{\widehat{\matA}_{ij}^2}{p_{ij}}\e_{i}\e_{i}^\top-\widehat{\matA}_{ij} \widehat{\matA} \e_j \e_i^\top - \widehat{\matA}_{ij} \e_i \e_j^\top \widehat{\matA}^\top +p_{ij}\widehat{\matA}\widehat{\matA}^\top\right)\\
%
&= \frobnorm{\widehat{\matA}}^2 \sum_{i=1}^n m_i\cdot \e_i \e_i^\top  -\sum_{j=1}^n \widehat{\matA} \e_j \sum_{i=1}^n \widehat{\matA}_{ij} \e_i^\top - \sum_{j=1}^n\left(\sum_{i=1}^n\widehat{\matA}_{ij} \e_i\right) \left(\widehat{\matA} \e_j\right)^\top + \sum_{i,j=1}^n p_{ij} \widehat{\matA} \widehat{\matA}^\top,
%
\end{align*}
where $m_i$ is the number of non-zeroes of the $i$-th row of $\widehat{\matA}$.
%
We now simplify the above result using a few simple observations: $\sum_{i,j=1}^n p_{ij}=1$, $\widehat{\matA} \e_j = \widehat{\matA}^{(j)}$, $\sum_{i=1}^n \widehat{\matA}_{ij}e_i = \widehat{\matA}^{(j)}$, and $\sum_{j=1}^n \widehat{\matA}^{(j)} \left(\widehat{\matA}^{(j)}\right)^\top = \widehat{\matA} \widehat{\matA}^\top$. Thus, we get
%
\begin{align*}
%
\EE\left(\matM_t\matM_t^\top\right) &= \frobnorm{\widehat{\matA}}^2 \sum_{i=1}^n m_i\cdot \e_i \e_i^\top  -\sum_{j=1}^n \widehat{\matA}^{(j)} \left(\widehat{\matA}^{(j)}\right)^\top - \sum_{j=1}^n \widehat{\matA}^{(j)} \left(\widehat{\matA}^{(j)}\right)^\top + \widehat{\matA} \widehat{\matA}^\top\\
%
&= \frobnorm{\widehat{\matA}}^2 \sum_{i=1}^n m_i\cdot \e_i \e_i^\top  - \widehat{\matA} \widehat{\matA}^\top.
%
\end{align*}
%
Since $0\leq m_i \leq n$ and using Weyl's inequality (Theorem~$4.3.1$ of \cite{book:matrix_analysis:HornJohnson}), which states that by adding a positive semi-definite matrix to a symmetric matrix all its eigenvalues will increase, we get that
\[ -\widehat{\matA}\widehat{\matA}^\top \preceq \EE\left(\matM_t\matM_t^\top\right) \preceq n \frobnorm{\widehat{\matA}}^2 \Id_n.\]
%
Consequently $\norm{\EE{\left(\matM_t \matM_t^\top\right)}} = \max\left\{ \norm{\widehat{\matA}}^2, n\frobnorm{\widehat{\matA}}^2\right\} = n \frobnorm{\widehat{\matA}}^2$.
%
%
\end{proof}
%

\noindent We can now apply Theorem~\ref{thm::recht} on Eqn.~(\ref{eqn:mainexp}) with $\tau = \epsilon/2$, $\gamma = 4n\epsilon^{-1}\frobnorm{\widehat{\matA}}^2$ (Lemma~\ref{lem:lem2}), and $\rho^2 = n\frobnorm{\widehat{\matA}}^2$ (Lemma~\ref{lem:lem3}) . Thus, we get that $\norm{\widehat{\matA} - \widetilde{\matA}} \leq \epsilon/2$ holds, subject to a failure probability of at most $$2n \exp\left(-\frac{\epsilon^2 s/8}{\left(1+4/6\right)n\frobnorm{\widehat{\matA}}^2}\right).$$
%
Bounding the failure probability by  $\delta$ and solving for $s$, we get that $s \geq \frac{14}{\epsilon^2}n\frobnorm{\widehat{\matA}}^2 \ln\left(\frac{2n}{\delta}\right)$. Using $\frobnorm{\widehat{\matA}} \leq \frobnorm{\matA}$ (by construction) concludes the proof of the following lemma, which is the main result of this section.
%
\begin{lemma}\label{lem:lem4}
%
Using the notation of Algorithm~\ref{alg:msparse:IPL}, if
%
$s \geq 14n\epsilon^{-2} \frobnorm{\matA}^2\ln\left(2n/\delta\right),$
%
then, with probability at least $1-\delta$, $$\norm{\widehat{\matA} - \widetilde{\matA}} \leq \epsilon/2.$$
%
\end{lemma}
%
%
%%%%%%%%%%%%%%%%%%%%%%%%%%%%%%%%%%%%%%%%%%%%%%%%%%%
%%%%%%%%%%%%%%%%%%%%%%%%%%%%%%%%%%%%%%%%%%%%%%%%%%%
\subsection{Deterministic Matrix Sparsification}\label{sec:sparsification:matrix}
%%%%%%%%%%%%%%%%%%%%%%%%%%%%%%%%%%%%%%%%%%%%%%%%%%%
%
%
To the best of our knowledge, all known algorithms for this problem are randomized (see Table~\ref{table:summary}). In this section, we present the first deterministic algorithm. A deterministic algorithm for the element-wise matrix sparsification problem can be obtained by derandomizing Algorithm~\ref{alg:msparse:IPL}.
\begin{theorem}\label{thm:matrix_sparse:slow}
Let $\matA$ be an $n\times n$ matrix and $ 0 < \eps <1$.  There is a deterministic polynomial time algorithm that, given $\matA$ and $ \eps$, outputs a matrix $\widetilde{\matA}\in \reals^{n\times n}$ with at most $ 28 n \ln (\sqrt{2n} ) \sr{\matA} /\eps^2$ non-zero entries such that $\norm{\matA - \widetilde{\matA}} \leq \eps \norm{\matA}.$
\end{theorem}
%%%%%%%%%%%%%%%%%%%%%%%%%%%%%%%%%%%%%%%%%%%%%%%%%%%
\begin{proof}(of Theorem~\ref{thm:matrix_sparse:slow})
%%%%%%%%%%%%%%%%%%%%%%%%%%%%%%%%%%%%%%%%%%%%%%%%%%%
By homogeneity, assume that $\norm{\matA} = 1$. Following the proof of Theorem~\ref{thm::msparse:IPL}, we can assume that w.l.o.g. all non-zero entries of $\matA$ have magnitude at least $\eps/(2n)$ in absolute value, otherwise we can zero-out these entries and incur at most an error of $\eps/2$.
%

%
Consider the bijection $\pi$ between the sets $[n^2]$ and $[n]\times [n]$ defined by $\pi (l)  \mapsto ( \lceil l / n\rceil , (l - 1) \mod n + 1) $ for every $l\in[n^2]$. Let $\matE_{ij}\in\reals^{n\times n}$ be the all zeros matrix having one only in the $(i,j)$ entry. Set $h(l) = \dil{ \frac{\matA_{\pi (l)}}{p_{l}} \matE_{\pi (l)} - \matA }$ where $p_l = \matA_{\pi (l)}^2/\frobnorm{\matA}^2$ for every $l\in{[n^2]}$. Observe that $h(\cdot ) \in \Sym^{2n\times 2n}$. Let $X$ be a random variable over $[n^2]$ with distribution $p_l$, $l\in{[n^2]}$. The same analysis as in Lemmas~\ref{lem:lem2} and~\ref{lem:lem3} of~\cite{matrix:sparsification:IPL2011} together with properties of the dilation map imply that $\norm{h(l)} \leq 4n\sr{\matA} /\eps$ for every $l\in{[n^2]}$, $\EE{h(X)}=\zeromtx_{2n}$, and $\norm{\EE h(X)^2} \leq n\sr{\matA}$.

Run Algorithm~\ref{alg:matrix:hyperbolic} with $h(\cdot )$ as above. Algorithm~\ref{alg:matrix:hyperbolic} returns at most $t=28 n  \ln (\sqrt{2}n)\sr{\matA}/\eps^2$ indices $x_1^*,x_2^*,\ldots x_t^*$ over $[n^2]$ using $\OO(n^6 \sr{\matA} \log n /\eps^2 )$ operations such that
	\begin{equation}\label{ineq:esoteric}
		\norm{ \frac1{t}\sum_{l=1}^{t} h(x_l^*)} \leq \eps / 2.
	\end{equation}
Set $\widetilde{\matA} : = \frac1{t} \sum_{l=1}^{t} \matA_{\pi (x_l^*)} /p_{x_l^*} \matE_{\pi (x_l^*)}$. Observe that $\widetilde{\matA}$ has at most $t$ non-zero entries. Now, by the definition of $h(\cdot)$ and properties of the  dilation map, it follows that Ineq.~\eqref{ineq:esoteric} is equivalent to $\norm{ \dil{\widetilde{\matA} - \matA }} \ =\ \norm{ \widetilde{\matA} - \matA } \leq \eps /2.$
%%%%%%%%%%%%%%%%%%%%%%%%%%%%%%%%%%%%%%%%%%%%%%%%%%%
\end{proof}
%%%%%%%%%%%%%%%%%%%%%%%%%%%%%%%%%%%%%%%%%%%%%%%%%%%
%
%
%
%%%%%%%%%%%%%%%%%%%%%%%%%%%%%%%%%%%%%%%%%%%%%%%%%%%%%%%
%%%%%%%%%%%%%%%%%%%%%%%%%%%%%%%%%%%%%%%%%%%%%%%%%%%%%%%
\subsection{Sparsification for SDD Matrices}
%%%%%%%%%%%%%%%%%%%%%%%%%%%%%%%%%%%%%%%%%%%%%%%%%%%%%%%
%%%%%%%%%%%%%%%%%%%%%%%%%%%%%%%%%%%%%%%%%%%%%%%%%%%%%%%
In this section, we give an elementary connection between element-wise matrix sparsification and spectral sparsification of psd matrices. A direct application of this connection implies strong sparsification bounds for symmetric matrices that are close to being \emph{diagonally dominant}. More precisely, we give two element-wise sparsification algorithms for symmetric and diagonally dominant-like matrices; in its randomized and the other in its derandomized version (see Table~\ref{table:summary}). Both algorithms share a crucial difference with all previously known sampling-based algorithms for this problem; that is, during the sparsification process they arbitrarily densify the diagonal entries. As we will see later this twist turns out to allow strong sparsification bounds. The next theorem presents stronger sparsification algorithms for the special case of diagonally dominant matrices both randomized and deterministic.
%
\begin{theorem}
Let $\matA$ be any symmetric and diagonally dominant matrix of size $n$ and $ 0 < \eps <1$. Assume for normalization that $\norm{\matA}=1$.
\begin{enumerate}[(a)]
 \item
There is a randomized linear time algorithm that outputs a matrix $\widetilde{\matA}\in \reals^{n\times n}$ with at most $\OO( n \log n /\eps^2)$ non-zero entries such that, with probability at least $1-1/n$, $\norm{\matA - \widetilde{\matA}} \leq \eps.$
\item
There is a deterministic $\widetilde{\OO}( \eps^{-2} \nnz{\matA} n^2 \log n  \max\{ \log^2 n,1/\eps^2 \} )$ time algorithm that outputs a matrix $\widetilde{\matA}\in \reals^{n\times n}$ with at most $\OO(n/\eps^2)$ non-zero entries such that $\norm{\matA - \widetilde{\matA}} \leq \eps.$
\end{enumerate}
\end{theorem}
%

%
Recall that the results of~\cite{graph:sparsifiers:eff_resistance,graph:sparsifiers:twice_ram} imply an element-wise sparsification algorithm that works only for Laplacian matrices. It is easy to verify that Laplacian matrices are also diagonally dominant. Here we extend these results to a wider class of matrices (with a weaker notion of approximation). The diagonally dominant assumption is too restrictive and we will show that our sparsification algorithms work for a wider class of matrices. To accommodate this, we say that a matrix $\matA$ is $\theta$-symmetric diagonally dominant (abbreviate by $\theta$-SDD) if $\matA$ is symmetric and the inequality $\infnorm{\matA} \leq \sqrt{\theta} \norm{\matA}$ holds.
%

%
By definition, any diagonally dominant matrix is also a $4$-SDD matrix. On the other extreme, every symmetric matrix of size $n$ is $n$-SDD since the inequality $\infnorm{\matA}\leq \sqrt{n} \norm{\matA}$ is always valid. The following elementary lemma gives a connection between element-wise matrix sparsification and spectral sparsification as defined in~\cite{phdthesis:Srivastava:2010}.
\begin{lemma}\label{lem:sparsif:decomp}
Let $\matA$ be a symmetric matrix of size $n$ and $\matR=\diag{r_1,r_2,\ldots ,r_n}$ where $r_i = \sum_{j\neq i} |\matA_{ij}|$. Then there is a matrix $\matC$ of size $n\times m$ with $m \leq \binom{n}{2}$ such that
\begin{align}\label{eqn:sparsify_lemma}
 \matA = \matC\matC^\top +\diag{\matA} - \matR.
\end{align}
Moreover, each column of $\matC$ is indexed by the ordered pairs $(i,j)$, $i<j$ and equals to $\matC^{(i,j)} = \sqrt{|\matA_{ij}|} \e_i + \sign{\matA_{ij}}  \sqrt{|\matA_{ij}|} \e_j$ for every $i<j$, $i,j\in[n]$.
\end{lemma}
%%%%%%%%%%%%%%%%%%%%%%%%%%%%%%%%%%%%%%%%%%%%%%%%%%%
\begin{proof}
%(of Lemma~\ref{lem:sparsif:decomp})
%%%%%%%%%%%%%%%%%%%%%%%%%%%%%%%%%%%%%%%%%%%%%%%%%%%
The key identity is $\matC\matC^\top : = \sum_{l,k\in{[n]},\ l<k} \matC^{(l,k)} \otimes \matC^{(l,k)}$. Let $l,k\in{[n]}$ with $l<k$, it follows that
\begin{align*}
\matC^{(l,k)} \otimes \matC^{(l,k)}   & =  \left(\sqrt{|\matA_{lk}|} \e_l + \sign{\matA_{lk}}  \sqrt{|\matA_{lk}|} \e_k \right)\left(\sqrt{|\matA_{lk}|} \e_l + \sign{\matA_{lk}}  \sqrt{|\matA_{lk}|} \e_k\right)^\top\\
 & =  |\matA_{lk}| \e_l \otimes \e_l + \matA_{lk} \e_k \otimes \e_k + \matA_{lk} \e_k \otimes \e_l + |\matA_{lk}| \e_k \otimes \e_k.
\end{align*}
Therefore
\begin{equation}\label{eqn:sparse_decomp}
 \matC\matC^\top = \sum_{l,k\in{[n]}:\ l< k }\left[|\matA_{lk}| \e_l \otimes \e_l + \matA_{lk} \e_k \otimes \e_k + \matA_{lk} \e_k \otimes \e_l + |\matA_{lk}| \e_k \otimes \e_k\right].
\end{equation}
Let's first prove the equality for the off-diagonal entries of Eqn~\eqref{eqn:sparsify_lemma}. Let $l<k$ and $l,k\in{[n]}$. By construction, the only term of the sum that contributes to the $(i,j)$ and $(j,i)$ entry of the right hand side of Eqn.~\eqref{eqn:sparse_decomp} is the term $\matC^{(i,j)} \otimes \matC^{(i,j)} $. Moreover, this term equals $|\matA_{ij}| \e_i \otimes \e_i + \matA_{ij} \e_i \otimes \e_j + \matA_{ij} \e_j \otimes \e_i + |\matA_{ij}| \e_j \otimes \e_j$. Since $\matA_{ij} = \matA_{ji}$ this proves that the off-diagonal entries are equal.
%

%
For the diagonal entries of Eqn.~\eqref{eqn:sparsify_lemma}, it suffices to prove that $(\matC\matC^\top)_{ii} = r_i$. First observe that the last two terms of the sum in the right hand side of~\eqref{eqn:sparse_decomp} do not contribute to any diagonal entry. Second, the first two terms contribute only when $l=i$ or $k=i$. In the case where $l=i$, the contribution of the sum equals to $\sum_{i<k} |\matA_{ik}|$. On the other case ($k=i$), the contribution of the sum is equal to $\sum_{l<i} |\matA_{li}|$. However, $\matA$ is symmetric so $\matA_{li} = \matA_{il}$ for every $l<i$. It follows that the total contribution is $\sum_{i<k} |\matA_{ik}| + \sum_{l<i} |\matA_{il}| = \sum_{j\neq i} |\matA_{ij}| = r_i$.
%%%%%%%%%%%%%%%%%%%%%%%%%%%%%%%%%%%%%%%%%%%%%%%%%%%
\end{proof}
%%%%%%%%%%%%%%%%%%%%%%%%%%%%%%%%%%%%%%%%%%%%%%%%%%%
\begin{remark}
In the special case where $\matA$ is the Laplacian matrix of some graph, the above decomposition is precisely the vertex-edge decomposition of the Laplacian matrix, since in this case $\diag{\matA} =\matR$.
\end{remark}
Using the above lemma, we give a randomized and a deterministic algorithm for sparsifying $\theta$-SDD matrices. First we present the randomized algorithm.
\begin{theorem}\label{thm:matrix_sparsif:rand}
Let $\matA$ be a $\theta$-SDD matrix of size $n$ and $ 0 < \eps <1$.  There is a randomized linear time algorithm that, given $\matA$, $\norm{\matA}$ and $\eps$, outputs a matrix $\widetilde{\matA}\in \reals^{n\times n}$ with at most $\OO( n\theta \log n /\eps^2)$ non-zero entries such that w.p. at least $1-1/n$, $\norm{\matA - \widetilde{\matA}} \leq \eps \norm{\matA}.$
\end{theorem}
%%%%%%%%%%%%%%%%%%%%%%%%%%%%%%%%%%%%%%%%%%%%%%%%%%%
\begin{proof}
%( of Theorem~\ref{thm:matrix_sparsif:rand})
%%%%%%%%%%%%%%%%%%%%%%%%%%%%%%%%%%%%%%%%%%%%%%%%%%%
In one pass over the input matrix $\matA$ normalize the entries of $\matA$ by $\norm{\matA}$, so assume without loss of generality that $\norm{\matA}=1$. Let $\matC$ be the $n\times m$ matrix guaranteed by Lemma~\ref{lem:sparsif:decomp}, where $m= \binom{n}{2}$, each column of $\matC$ is indexed by the ordered pairs $(i,j)$, $i<j$ and $\matA = \matC\matC^\top +\diag{\matA} - \matR$. By definition of $\matC$ and the hypothesis, we have that $\norm{\matC\matC^\top} = \norm{\matA - \diag{\matA} +\matR} \leq \norm{\matA} +\infnorm{\matA}\leq 2\sqrt{\theta}$ and $\frobnorm{\matC}^2 =2 \sum_{i,j} |\matA_{ij}| \leq 2 n \infnorm{\matA}\leq 2n \sqrt{\theta} $.
%

%
Consider the bijection between the sets $[m]$ and $\{(i,j)\ |\ i<j,\ i,j\in{[n]}\}$ defined by $\pi (l)  \mapsto ( \lceil l / n\rceil , (l-1) \mod n + 1) $. For each $l\in{[m]}$, set $p_l=\norm{\matC^{\pi (l)}}^2 / \frobnorm{\matC}^2$ and define $f(l):= \matC^{\pi(l)} \otimes \matC^{\pi(l)}/p_l - \matC\matC^\top$. Let $X$ be a real-valued random variable over $[m]$ with distribution $p_l$. It is easy to verify that $\EE{f(X)} = \zeromtx_n$, $\norm{f(l)} \leq 2 \frobnorm{\matC}^2$ for every $l\in{[m]}$. A direct calculation gives that $\norm{\EE{ f(X)^2}} \leq 2\frobnorm{\matC}^2\norm{\matC\matC^\top}$. Matrix Bernstein inequality (see~\cite{chernoff:matrix_valued:Tropp}) with $f(\cdot)$ as above ($\gamma = 4n\sqrt{\theta}$ and $\rho^2 = 8 n \theta$) tells us that if we sample $t=38 n\theta \ln(\sqrt{2}n) /\eps^2 $ indices $x_1^*, x_2^*,\ldots , x_t^*$ over $[m]$ then with probability at least $1-1/n$, $\norm{ \frac1{t} \sum_{j=1}^{t} f(x_j^*)} \leq \eps$. Now, set $\widetilde{\matC}\in\reals^{n\times t}$ where the $j$-th column of $\widetilde{\matC}^{(j)}$ equals $\frac1{\sqrt{t}} \matC^{\pi(x_j^*)}$. It follows that $\norm{ \frac1{t} \sum_{j=1}^{t} f(x_j^*)} = \norm{ \frac1{t} \sum_{j=1}^{t} \matC^{\pi(x_j^*)} \otimes \matC^{\pi(x_j^*)} - \matC\matC^\top} = \norm{\widetilde{\matC}\widetilde{\matC}^\top - \matC\matC^\top}$. Define $\widetilde{\matA} = \widetilde{\matC}\widetilde{\matC}^\top +\diag{\matA} - \matR $. First notice that $\norm{\widetilde{\matA} - \matA} = \norm{\widetilde{\matC} \widetilde{\matC}^\top -\matC\matC^\top} \leq \eps$. It suffices to bound the number of non-zeros of $\widetilde{\matA}$. To do so, view the matrix-product $\widetilde{\matC}\widetilde{\matC}^\top$ as a sum of rank-one outer-products over all columns of $\widetilde{\matC}$. By the special structure of the entries of $\widetilde{\matC}$, every outer-product term of the sum contributes to at most four non-zero entries, two of which are off-diagonal. Since $\widetilde{\matC}$ has at most $t$ columns, $\widetilde{\matA}$ has at most $n + 2t$ non-zero entries; $n$ for the diagonal entries and $2t$ for the off-diagonal.
%%%%%%%%%%%%%%%%%%%%%%%%%%%%%%%%%%%%%%%%%%%%%%%%%%%
\end{proof}
%%%%%%%%%%%%%%%%%%%%%%%%%%%%%%%%%%%%%%%%%%%%%%%%%%%
Next we state the derandomized algorithm of the above result.
%
%
%%%%%%%%%%%%%%%%%%%%%%%%%%%%%%%%%%%%%%%%%%%%%%%%%%%
\begin{theorem}\label{thm:matrix_sparsif:det}
%%%%%%%%%%%%%%%%%%%%%%%%%%%%%%%%%%%%%%%%%%%%%%%%%%%
Let $\matA$ be a $\theta$-SDD matrix of size $n$ and $ 0 < \eps <1/2$.  There is an algorithm that, given $\matA$ and $ \eps$, outputs a matrix $\widetilde{\matA}\in \reals^{n\times n}$ with at most $\OO( n \theta /\eps^2)$ non-zero entries such that $\norm{\matA - \widetilde{\matA}} \leq \eps \norm{\matA}$. Moreover, the algorithm computes $\widetilde{\matA}$ in deterministic $\widetilde{\OO}(\nnz{\matA} n^2 \theta\log^3 n  /\eps^2 + n^4 \theta^2 \log n /\eps^4)$ time.
\end{theorem}
\begin{remark}
The results of~\cite{graph:sparsifiers:twice_ram,phdthesis:Srivastava:2010} imply a deterministic $\OO(\nnz{\matA} \theta n^3 /\eps^2 )$ time algorithm that outputs a matrix $\widetilde{\matA}$ with at most $ \lceil 19(1+\sqrt{\theta})^2 /\eps^2\rceil n $ non-zero entries such that $\norm{\widetilde{\matA}-\matA} \leq \eps\norm{\matA}$.
\end{remark}
%%%%%%%%%%%%%%%%%%%%%%%%%%%%%%%%%%%%%%%%%%%%%%%%%%%
\begin{proof}
%(of Theorem~\ref{thm:matrix_sparsif:det})
%%%%%%%%%%%%%%%%%%%%%%%%%%%%%%%%%%%%%%%%%%%%%%%%%%%
%
Let $\matC$ be the $n\times m$ matrix such that $\matA = \matC\matC^\top +\diag{\matA} - \matR$ and $m\leq \nnz{\matA}$ guaranteed by Lemma~\ref{lem:sparsif:decomp}. Apply Theorem~\ref{thm:sparsification:here} on the matrix $\matC\matC^\top$  and $\eps$ which outputs, in deterministic $\widetilde{\OO}(\nnz{\matA} n^2$
$ \theta \log^3 n  /\eps^2 + n^4 \theta^2 \log n /\eps^4)$ time, an $n\times \lceil n/\eps^2\rceil$ matrix $\widetilde{\matC}$ such that $(1-\eps)^3 \matC\matC^\top \preceq \widetilde{\matC}\widetilde{\matC}^\top \preceq (1+\eps)^3 \matC\matC^\top.$ By Weyl's inequality~\cite[Theorem~$4.3.1$]{book:matrix_analysis:HornJohnson} and the fact that $\eps<1/2$, it follows that $\norm{\matC\matC^\top - \widetilde{\matC}\widetilde{\matC}^\top} \leq 5 \eps \norm{\matC\matC^\top}$. Define $\widetilde{\matA}:= \widetilde{\matC}\widetilde{\matC}^\top + \diag{\matA} - \matR$. First we argue that the number of non-zero entries of $\widetilde{\matA}$ is at most $n+ \lceil 2n/\eps^2 \rceil $. Recall that every column of $\widetilde{\matC}$ is a rescaled column of $\matC$. Now, think the matrix-product $\widetilde{\matC}\widetilde{\matC}^\top$ as a sum of rank-one outer-products over all columns of $\widetilde{\matC}$. By the special structure of the entries of $\widetilde{\matC}$, every outer-product term of the sum contributes to at most four non-zero entries, two of which are off-diagonal. Since $\widetilde{\matC}$ has at most $\lceil n/\eps^2 \rceil$ columns, $\widetilde{\matA}$ has at most $n + \lceil 2 n/\eps^2\rceil$ non-zero entries; $n$ for the diagonal entries and $\lceil 2 n/\eps^2\rceil$ for the off-diagonal. Moreover, $\widetilde{\matA}$ is close to $\matA$ in the operator norm sense. Indeed,
\begin{align*}
	\norm{\matA - \widetilde{\matA} } &   =   \norm{\matC\matC^\top - \widetilde{\matC}\widetilde{\matC}^\top}  \leq\ 5\eps\norm{\matC\matC^\top } \   =  \  5\eps\norm{\matA - \diag{\matA} +\matR }\\
	  						  & \leq  5\eps(\norm{\matA} + \infnorm{\matA}) \ \leq\ 10 \eps \sqrt{\theta}\norm{\matA}
\end{align*}
where we used the definition of $\widetilde{\matA}$, Eqn.~\eqref{eqn:sparsify_lemma}, triangle inequality, the assumption that $\matA$ is $\theta$-SDD and the fact that $\theta \geq 1$. Repeating the proof with $\eps' =\frac{\eps}{10\sqrt{\theta}}$ and elementary manipulations conclude the proof.
\end{proof}
%
%
 % Linear Regression - Matrix Algorithms
\chapter{Graph Algorithms}\label{chap:graph}
In the present chapter\footnote{Both sections~\ref{sec:AR_graphs} and~\ref{sec::graph_sparsifiers} appeared in~\cite{ICALP12}. A preliminary version of Section~\ref{sec:gossip} appeared in~\cite{CDC12} (joint work with Nick Freris).}, we discuss applications of the tools analyzed in the previous chapters to graph theoretic problems. More precisely, we discuss three problems: (i) the construction of expanding Cayley graphs, (ii) an efficient deterministic algorithm for graph sparsification, and (iii) randomized gossip algorithms for solving Laplacian systems.
%A common theme underlying all the results here is that the corresponding graph theoretic problems are reformulated to an equivalent linear algebraic question that is amenable to be analyzed by the tools developed in the previous chapters.
%

%
%%%%%%%%%%%%%%%%%%%%%%%%%%%%%%%%%%%%%%%%%%%%%%%%%%%
%\clearpage
%%%%%%%%%%%%%%%%%%%%%%%%%%%%%%%%%%%%%%%%%%%%%%%%%%%
%\vspace*{-3.0ex}
\section{Alon-Roichman Expanding Cayley Graphs}\label{sec:AR_graphs}
%%%%%%%%%%%%%%%%%%%%%%%%%%%%%%%%%%%%%%%%%%%%%%%%%%%
%
The Alon-Roichman theorem asserts that Cayley graphs obtained by choosing a logarithmic number of group elements independently and uniformly at random are expanders~\cite{expander:AlonRoichman:orig}. The original proof of Alon and Roichman is based on Wigner's trace method, whereas recent proofs rely on matrix-valued deviation bounds~\cite{expander:AlonRoichman:RusLan}. Wigderson and Xiao's derandomization of the matrix Chernoff bound implies a deterministic $\OO(n^4 \log n )$ time algorithm for constructing Alon-Roichman graphs. Independently, Arora and Kale generalized the multiplicative weights update (MWU) method to the matrix-valued setting and, among other interesting implications, they improved the running time to $\OO(n^3\polylog{n})$~\cite{phdthesis:Kale:2008}. Here we further improve the running time to $\OO(n^2 \log^3 n)$ by exploiting the group structure of the problem. In addition, our algorithm is combinatorial in the sense that it only requires counting the number of all closed (even) paths of size at most $\OO(\log n)$ in Cayley graphs. All previous algorithms involve numerical matrix computations such as eigenvalue decompositions and matrix exponentiation.
%

%
We start by describing expander graphs. Given a connected undirected $d$-regular graph $H=(V,E)$ on $n$ vertices, let $\matA$ be its adjacency matrix, i.e., $\matA_{ij}=w_{ij}$ where $w_{ij}$ is the number of edges between vertices $i$ and $j$. Moreover, let $\widehat{\matA}:=\frac1{d}\matA$ be its normalized adjacency matrix. We allow self-loops and multiple edges. Let $\lambda_1(\widehat{\matA}),\ldots ,\lambda_n(\widehat{\matA})$ be its eigenvalues in decreasing order. We have that $\lambda_1(\widehat{\matA})=1$ with corresponding eigenvector $\mathbf{1}/\sqrt{n}$, where $\mathbf{1}$ is the all-one vector. The graph $H$ is called a \emph{spectral expander} if $\lambda(\widehat{\matA}):=\max_{2\leq j}\{ |\lambda_j(\widehat{\matA})|\}\leq \eps$ for some positive constant $\eps<1$.
%

%
Denote by $m_k=m_k(H):= \trace{\matA^k}$. By definition, $m_k$ is equal to the number of self-returning walks of length $k$ of the graph $H$. A graph-spectrum-based invariant, proposed by Estrada is defined as $EE(\matA) := \trace{\expm{\matA}}$~\cite{estrada}, which also equals to $\sum_{k=0}^{\infty} m_k/k!$. For $\theta>0$, we define the \emph{even $\theta$-Estrada index} by $EE_{\text{even}}(\matA,\theta) := \sum_{k=0}^{\infty}  m_{2k}(\theta \matA)/(2k)!$.
%

%
Now let $G$ be any finite group of order $n$ with identity element $\mathtt{id}$. Let $S$ be a multi-set of elements of $G$, we denote by $S\sqcup S^{-1}$ the symmetric closure of $S$, namely the number of occurrences of $s$ and $s^{-1}$ in $S\sqcup S^{-1}$ equals the number of occurrences of $s\in S$. Let $R$ be the right regular representation\footnote{In other words, represent each group algebra element with a permutation matrix of size $n$ that preserves the group structure. This is always possible due to Cayley's theorem.}, i.e., $(R(g_1)\phi)(g_2) = \phi(g_1 g_2)$ for every $\phi : G \to \reals$ and $g_1,g_2\in G$. The Cayley graph $\Cay{G}{S}$ on a group $G$ with respect to the mutli-set $S\subset G$ is the graph whose vertex set is $G$, and where $g_1$ and $g_2$ are connected by an edge if there exists $s\in S$ such that $g_2 = g_1 s$ (allowing multiple edges for multiple elements in $S$). In this section we prove the correctness of the following greedy algorithm for constructing expanding Cayley graphs.
%
%
\begin{theorem}\label{thm:AR_graphs}
Given the multiplication table of a finite group $G$ of size $n$ and $0<\eps<1$, Algorithm~\ref{alg:estradaAR} outputs a (symmetric) multi-set $S\subset G$ of size $\OO(\log n /\eps^2)$ such that $\lambda (\Cay{G}{S}) \leq \eps$ in $\OO(n^2\log^3 n /\eps^5)$ time. Moreover, the algorithm performs only group algebra operations that correspond to counting closed paths in Cayley graphs.
\end{theorem}
\begin{remark}
To the best of our knowledge, the above theorem improves the running time of all previously known deterministic constructions of Alon-Roichman Cayley graphs~\cite{arora:fast_SDP,chernoff:matrix_valued:derand:WX08,phdthesis:Kale:2008}, see also~\cite{cayley:latin12} for an alternative polynomial time construction. Moreover, notice that the running time of the above algorithm is optimal up-to poly-logarithmic factors since the size of the multiplication table of a finite group of size $n$ is $\OO(n^2)$.
\end{remark}
%%%%%%%%%%%%%%%%%%%%%%%%%%%%%%%%%%%%%%%%%%%%%%%%%%%
%%%%%%%%%%%%%%%%%%%%%%%%%%%%%%%%%%%%%%%%%%%%%%%%%%%
%\vspace*{-3.5ex}
\begin{algorithm}{}
	\caption{Expander Cayley Graph via even Estrada Index Minimization}\label{alg:estradaAR}
\begin{algorithmic}[1]
\Procedure{GreedyEstradaMin}{$G$, $\eps$}\Comment{Multiplication table of $G$, $0<\eps <1$}
\State Set $S^{(0)}=\emptyset$ and $t=\OO(\log n /\eps^2)$
\For {$i=1,\ldots t$ }
	\State Let $g_{*}\in G$ that (approximately) min. the even $\eps/ 2$-Estrada index of $\Cay{G}{S^{(i-1)}\cup g \cup g^{-1}}$ over all $g\in G $  \Comment{Use Lemma~\ref{lem:fastEstrada:Cayley}}
	\State Set $S^{(i)} = S^{(i-1)} \cup g_{*} \cup g_{*}^{-1}$
\EndFor
\State \textbf{Output:} A multi-set $S:=S^{(t)}$ of size $2t$ such that $\lambda(\Cay{G}{S}) \leq \eps$
\EndProcedure
\end{algorithmic}
\end{algorithm}
%\vspace*{-4.0ex}
%%%%%%%%%%%%%%%%%%%%%%%%%%%%%%%%%%%%%%%%%%%%%%%%%%%
%%%%%%%%%%%%%%%%%%%%%%%%%%%%%%%%%%%%%%%%%%%%%%%%%%%
%%%%%%%%%%%%%%%%%%%%%%%%%%%%%%%%%%%%%%%%%%%%%%%%%%%
%
%
Let $\widehat{\matA}$ be the normalized adjacency matrix of $\Cay{G}{S\sqcup S^{-1}}$ for some $S\subset G$. It is not hard to see that $ \widehat{\matA} = \frac1{2|S|} \sum_{s\in S}{ (R(s) + R(s^{-1}))}$. We want to bound $\lambda (\matA)$. Notice that $\lambda(\matA)=\norm{(\Id - \J/n) \matA}$. Since we want to analyze the second-largest eigenvalue (in absolute value), we consider $(\Id - \J/n)\matA = \frac1{|S|} \sum_{s\in S}{ (R(s) + R(s^{-1})) /2} - \J/n.$
Based on the above calculation, we define our matrix-valued function as
\begin{equation}\label{eq:AR:samplesnew}
f(g) := (R(g) + R(g^{-1})) / 2 - \J /n
\end{equation}
for every $g\in G$. The following lemma connects the potential function that is used in Theorem~\ref{thm:hypercosine:main} and the even Estrada index.
%
\begin{lemma}\label{lem:cosh_Estrada}
Let $S\subset G $ and $\matA$ be the adjacency matrix of $\Cay{G}{S\sqcup S^{-1}}$. For any $\theta>0$, $\trace{ \coshm{ \theta \sum_{s\in S} f(s) } } = EE_{even} (\matA,\theta/2)  + 1 - \cosh(\theta |S|).$
\end{lemma}
%%%%%%%%%%%%%%%%%%%%%%%%%%%%%%%%%%%%%%%%%%%%%%%%%%%
\begin{proof}
%(of Lemma~\ref{lem:cosh_Estrada})
%%%%%%%%%%%%%%%%%%%%%%%%%%%%%%%%%%%%%%%%%%%%%%%%%%%
For notational convenience, set $\matP:= \Id_n - \J_n/n$ and $\matB := \frac{\theta}{2} \sum_{s\in S} (R(s) + R(s)^{-1})  $. Since $\J R(g) = R(g) \J = \J$, we have that $\trace{ \coshm{ \theta \sum_{s\in S} f(s) } } = \trace{ \coshm{ \matP \matB }}$. Now using Lemma~\ref{lem:coshm_with_proj}, it follows $\trace{ \coshm{ \matP \matB }} = \trace{\matP \coshm{\matB} + \Id - \matP} = \trace{\coshm{\matB}} + \trace{-\frac{\J}{n} \coshm{\matB} + \Id - \matP}$. Notice that $\J/n$ is a projector matrix, hence applying Lemmata~\ref{lem:expm:outerprod} and \ref{lem:coshm_with_proj} we get that
%
\[\trace{-\frac{\J}{n} \coshm{\matB} + \Id - \matP} = \trace{-\coshm{\J/n \matB} + \matP +\Id - \matP} = 1 - \cosh(\theta |S|).\]
%
%%%%%%%%%%%%%%%%%%%%%%%%%%%%%%%%%%%%%%%%%%%%%%%%%%%
\end{proof}
%%%%%%%%%%%%%%%%%%%%%%%%%%%%%%%%%%%%%%%%%%%%%%%%%%%
%
The following lemma indicates that it is possible to efficiently compute the (even) Estrada index for Cayley graphs with small generating set.
\begin{lemma}\label{lem:fastEstrada:Cayley}
	Let $S\subset G $, $\theta,\delta >0$, and $\matA$ be the adjacency matrix of $\Cay{G}{S}$. There is an algorithm that, given $S$, computes an additive $\delta$ approximation to $EE(\theta \matA)$ or $EE_{\text{even}}(\matA,\theta)$ in $\OO(n|S| \max\{ \log (n/\delta) , 2\e^2 |S| \theta \})$ time.
\end{lemma}
%%%%%%%%%%%%%%%%%%%%%%%%%%%%%%%%%%%%%%%%%%%%%%%%%%%
%%%%%%%%%%%%%%%%%%%%%%%%%%%%%%%%%%%%%%%%%%%%%%%%%%%
\begin{proof}
%(of Lemma~\ref{lem:fastEstrada:Cayley})
%%%%%%%%%%%%%%%%%%%%%%%%%%%%%%%%%%%%%%%%%%%%%%%%%%%
We will prove the Lemma for $EE(\matA,\theta)$, the other case is similar. Let $h:=\theta \sum_{s\in S} s$ be a group algebra element of $G$, i.e, $h\in \reals [G]$. Define $\expm{h} := \mathtt{id} + \sum_{k=1}^{\infty} \frac{h^{\star k}}{k!}$ and $T_l(h):= \mathtt{id} + \sum_{k=1}^{l} \frac{h^{\star k}}{k!}$ (where $h^{\star k}$ is the $k$-folded convolution/multiplication over $\reals [G]$) the exponential operator and its $l$ truncated Taylor series, respectively. Notice that $\theta \matA=\theta \sum_{s\in S} R(s) = R(h) $, so $EE(\matA,\theta) = \trace{\expm{R(h)}}= \trace{R(\expm{h})}$. We will show that the quantity $\trace{R(T_l(h) )}$ is a $\delta$ approximation for $EE(\matA,\theta)$ when $l\geq  \max\{ \log (n/\delta) , 2\e^2 |S| \theta\}$.

Compute the sum of $T_l(h)$ by summing each term one by one and keeping track of all the coefficients of the group algebra elements. The main observation is that at each step there are at most $n$ such coefficients since we are working over $\reals [G]$. For $k > 1$, compute the $k$-th term of the sum by $(\sum_{s\in S} c_s s)^k /k! = (\sum_{s\in S} c_s s)^{k - 1 }/(k - 1 )! \cdot \sum_{s\in S} (c_s/k) s.$
	Assume that we have computed the first term of the above product, which is some group algebra element denote it by $\sum_{g\in G} \beta_g g$ for some $\beta_g\in\reals$. Hence, at the next iteration, we have to compute the product/convolution of $\sum_{g\in G} \beta_g g$ with $\theta /k \sum_{s\in S} s$, which can be done in  $\OO( n |S|)$ time. Since the sum has $l$ terms, in total we require $\OO(n|S| l)$ operations. Now, we show that it is a $\delta$ approximation. We need the following fact (see~\cite[Theorem~$10.1$,~p.~$234$]{book:Higham:Matrix_fcn})
\begin{fact}\label{fact:expm:taylor_exp}
For any $\matB\in \reals^{n\times n}$, let $T_{l}(B) := \sum_{k=0}^{l} \frac{\matB^k}{k!}$. Then, $ \norm{ \expm{\matB} - T_{l}(\matB) } \leq \frac{\norm{\matB}^{l+1}}{(l+1)!} \e^{\norm{\matB}}.$
\end{fact}
%
Notice that $ \norm{\theta \matA} = \norm{\sum_{s\in S}\theta R(s)} \leq \theta |S|$ by triangle inequality and the fact that $\norm{R(g)}=1$ for any $g\in G$. Applying Fact~\ref{fact:expm:taylor_exp} on $\theta \matA$ we get that
	\begin{align*}
		\norm{\expm{\theta \matA} - T_l(\theta \matA)}	& \leq  \frac{(\theta |S|)^{l+1}}{(l+1)!}\e^{\theta |S|}\ \leq\ \left(\frac{ \e\theta |S|}{l+1}\right)^{l+1} \e^{\theta |S|} \\
									&   =  \left(\frac{ \e^{1+ (\theta |S|)/(l+1)}\theta |S|}{l+1}\right)^{l+1}  \leq \frac1{2^{l+1}} \leq \frac{\delta}{n}.
	\end{align*}
	where we used the inequality $(l+1)! \geq  (\frac{l+1}{e})^{l+1}$ and the assumption that $l\geq \max\{ \log (n/\delta) , 2 \e^2 \theta |S|\}$.
\end{proof}
%

%
%
%%%%%%%%%%%%%%%%%%%%%%%%%%%%%%%%%%%%%%%%%%%%%%%%%%%
%
\begin{proof}(of Theorem~\ref{thm:AR_graphs})
By Lemma~\ref{lem:cosh_Estrada}, minimizing the even $\eps/2$-Estrada index in the $i$-th iteration is equivalent to minimizing $\trace{ \coshm{ \theta \sum_{s\in S^{(i-1)}} f(s) +\theta f(g) } }$ over all $g\in G$ with $\theta = \eps$. Notice that $f(g)\in \Sym^{n\times n}$ for $g\in G$, $\EE_{g\in_R{G}}{f(g)} = \zeromtx_n$ since $\sum_{g\in G}R(g) = \J$. It is easy to see that $\norm{f(g)} \leq 2$ and moreover a calculation implies that $\norm{\EE_{g\in_R{G}}{f(g)^2}} \leq 2$ as well. Theorem~\ref{thm:hypercosine:main} implies that we get a multi-set $S$ of size $t$ such that $\lambda (\Cay{G}{ S\sqcup S^{-1}})=\norm{\frac1{|S|} \sum_{s\in S} f(s) }  \leq \eps$. The moreover part follows from Lemma~\ref{lem:fastEstrada:Cayley} with $\delta = \frac{\e^{\eps^2}}{n^c}$ for a sufficient large constant $c>0$. Indeed, in total we incur (following the proof of Theorem~\ref{thm:hypercosine:main}) at most an additive $\ln( \delta n \e^{\eps^2 t}) / \eps$ error which is bounded by $\eps$.
\end{proof}
%
%
%
%%%%%%%%%%%%%%%%%%%%%%%%%%%%%%%%%%%%%%%%%%%%%%%%%%%
%%%%%%%%%%%%%%%%%%%%%%%%%%%%%%%%%%%%%%%%%%%%%%%%%%%
%\clearpage
%%%%%%%%%%%%%%%%%%%%%%%%%%%%%%%%%%%%%%%%%%%%%%%%%%%
%%%%%%%%%%%%%%%%%%%%%%%%%%%%%%%%%%%%%%%%%%%%%%%%%%%
\section{Deterministic Graph Sparsification}\label{sec::graph_sparsifiers}
%%%%%%%%%%%%%%%%%%%%%%%%%%%%%%%%%%%%%%%%%%%%%%%%%%%
%%%%%%%%%%%%%%%%%%%%%%%%%%%%%%%%%%%%%%%%%%%%%%%%%%%
%
The second problem that we study is the graph sparsification problem. This problem poses the question whether any dense graph can be approximated by a sparse graph under different notions of approximation. Given any undirected graph, the most well-studied notions of approximation by a sparse graph include approximating, (i) \emph{all} pairwise distances up to an additive error~\cite{graph:spanners:PelegS89}, (ii) every cut to an arbitrarily small multiplicative error~\cite{graph:sparsifier:BenczurK96} and (iii) every eigenvalue of the difference of their Laplacian matrices to an arbitrarily small relative error~\cite{graph:sparsifier:ICM2010}; the resulting graphs are usually called \emph{graph spanners}, \emph{cut sparsifiers} and \emph{spectral sparsifiers}, respectively. Given that the notion of spectral sparsification is stronger than cut sparsification, we focus on spectral sparsifiers. An efficient randomized algorithm to construct an $(1+\eps)$-spectral sparsifier with $\OO(n\log n /\eps^2)$ edges was given in~\cite{graph:sparsifiers:eff_resistance}. Furthermore, an $(1+\eps)$-spectral sparsifier with $\OO(n/\eps^2)$ edges can be computed in $\OO(mn^3/\eps^2)$ deterministic time~\cite{graph:sparsifiers:twice_ram}. The latter result is a direct corollary of the spectral sparsification of positive semi-definite (psd) matrices problem as defined in~\cite{phdthesis:Srivastava:2010}; see also~\cite{graph:sparsification:Naor} for more applications. For additional references, see~\cite{graph:sparse:Harvey}. Here we present an efficient deterministic spectral graph sparsification algorithm for the case of dense graphs.
%

%
Let us formalize the notion of cut and spectral sparsification. Let $G=(V,E,w_e)$ be a connected weighted undirected graph with $n$ vertices, $m$ edges and edge weights $w_e\geq 0$. Spectral sparsification was inspired by the notion of cut\footnote{Let $S\subseteq V$. A cut, denoted by $(S,\bar{S})$, is a partition of the vertices of a graph into two disjoint subsets $S$ and $\bar{S}$. The cut-set of the cut is the set of edges whose end points are in different subsets of the partition. The weight of a cut equals to the sum of the weights of all distinct edges contained in the cut-set.} sparsification introduced by Bencz\'{u}r and Karger~\cite{graph:sparsifier:BenczurK96} to accelerate cut algorithms whose running time depends on the number of edges. They designed algorithms that, given $G$ and a parameter $\eps>0$, output a weighted subgraph $\widetilde{G}=(V,\widetilde{E},\widetilde{w}_e)$ with $|\widetilde{E}|=\OO(n\log n /\eps^2)$ such that
\begin{equation}\label{cut_sparsifier:cond}
\forall S \subseteq V, \quad (1-\eps)\sum_{ (S,\bar{S}) \ni e\in{E}  } w_e \leq \sum_{(S,\bar{S}) \ni e\in{\widetilde{E}} } \widetilde{w}_e \leq (1+\eps)\sum_{(S,\bar{S}) \ni e\in{E} } w_e.
\end{equation}
%

%
We call such a graph $\widetilde{G}$, an $(1+\eps)$-\emph{cut sparsifier} of $G$. Let $\matL$ and $\widetilde{\matL}$ be the Laplacian matrices of $G$ and $\widetilde{G}$, respectively. Condition~\eqref{cut_sparsifier:cond} can be expressed using the language of Laplacians as follows
%
\begin{equation}\label{eqn:graph_sparse:comb}
(1-\eps) \x^\top \matL \x \ \leq\ \x^\top \widetilde{\matL} \x \leq (1+\eps) \x^\top \matL \x, \quad \text{for all }\x\in{\{0,1\}^n}.
\end{equation}
%
Spielman and Teng~\cite{graph:sparsifiers:SpielmanT08} devised stronger sparsifiers that extend~\eqref{eqn:graph_sparse:comb} to all $\vct{x}\in\reals^n$, but required $\OO(n\log^c n)$ edges for a large constant $c$. Quite recently, Spielman and Srivastava~\cite{graph:sparsifiers:eff_resistance} constructed sparsifiers with $\OO(n\log n /\eps^2)$ that satisfy
%
\begin{equation}\label{eqn:graph_sparse:forall}
(1-\eps) \x^\top \matL \x \ \leq\ \x^\top \widetilde{\matL} \x \leq (1+\eps) \x^\top \matL \x, \quad \text{for all }\x\in\reals^n.
\end{equation}
%
We say that $\widetilde{G}$ is an $(1+\eps)$-\emph{spectral sparsifier} of $G$, if it satisfies Ineq.~\eqref{eqn:graph_sparse:forall}.
%

%
The latter result of Spielman and Srivastava~\cite{graph:sparsifiers:eff_resistance} implicitly\footnote{To be precise, they used Vershynin and Rudelson's matrix Chernoff bound~\cite{lowrank:rankone:VR}, however the same bound follows via the matrix Bernstein bound as was noticed in~\cite{matrix:hypercosine_zouzias}.} used the matrix Bernstein inequality (Theorem~\ref{thm:matrix_valued:Bernstein}). In particular, they proved a stronger statement: they showed that there exists a probability distribution over the edges of any graph $G$, so that sampling $\OO(n\log n /\eps^2)$ edges with replacement will result to a sub-graph of $G$ that satisfies Ineq.~\eqref{eqn:graph_sparse:forall} with high probability. They also gave a nearly-linear time algorithm for constructing such spectral sparsifiers. Furthermore, an $(1+\eps)$-spectral sparsifier with $\OO(n/\eps^2)$ edges can be computed in $\OO(mn^3/\eps^2)$ deterministic time~\cite{graph:sparsifiers:twice_ram}. In a recent paper, the author obtained a faster deterministic algorithm than~\cite{graph:sparsifiers:twice_ram} for the case of dense graphs and constant $\eps$. This was achieved by combining the matrix hyperbolic cosine algorithm (Algorithm~\ref{alg:matrix:hyperbolic}) together with tools from numerical linear algebra such as the Fast Multipole Method and fast solvers for special type of eigensystems.
%
%
\begin{theorem}
Given a weighted graph $G=(V,E)$ on $n$ vertices, $\Omega(n^2)$ edges with positive weights and $0< \eps <1$, there is a deterministic algorithm that returns an $(1+\eps)$-spectral sparsifier with $\OO(n/ \eps^2)$ edges in $\widetilde{\OO}(n^4 \log n /\eps^2$ $ \max\{ \log^2 n, 1/\eps^2 \})$ time.
\end{theorem}
%
The proof is a direct corollary of the fast deterministic isotropic sparsification algorithm, Algorithm~\ref{alg:fast:isotrop}. The reduction from graph sparsification to sparsification of vectors in isotropic position was first observed in~\cite{graph:sparsifiers:twice_ram,phdthesis:Srivastava:2010}.
%
\begin{proof}

Given the weighted Laplacian matrix $\matL = \sum_{(i,j)\in E} w_{(i,j)} \b_{(i,j)} \otimes \b_{(i,j)}$ where the $i$-th coordinate of $\b_{(i,j)}$ equals to $1$, the $j$-th equals to $-1$, and zero otherwise.  Theorem~\ref{thm:sparsification:here} with input the vectors $\{\sqrt{w_{e}} \b_e\}_{e\in E}$, outputs a set of positive weights $\{s_e\}_{e\in{E}}$ (at most $\OO(n/\eps^2)$ of them are positive) in $\widetilde{\OO}(mn^2 \log^3 n  /\eps^2 + n^4 \log n /\eps^4)$ time so that
\[ (1-\eps)\matL \preceq \sum_{e\in E} s_e w_e \b_e \otimes \b_e \preceq (1+\eps) \matL.\]
It follows that the induced subgraph by the non-zero weights $\{s_e w_e\}_{e\in{E}}$ is an $(1+\eps)$-spectral sparsifier of $G$ having at most $\OO(n/ \eps^2)$ edges.
\end{proof}
%
%%%%%%%%%%%%%%%%%%%%%%%%%%%%%%%%%%%%%%%%%%%%%%%%%%%
\section{Randomized Gossip Algorithms for Solving Laplacian Systems}\label{sec:gossip}
%%%%%%%%%%%%%%%%%%%%%%%%%%%%%%%%%%%%%%%%%%%%%%%%%%%
%%%%%%%%%%%%%%%%%%%%%%%%%%%%%%%%%%%%%%%%%%%%%%%%%%%
We present distributed algorithms for solving Laplacian systems under the gossip model (a.k.a. asynchronous time model)~\cite{gossip:Boyd}, for earlier references see~\cite{gossip:TBA86,book:BT89}. The proposed algorithms are based on the randomized extended Kaczmarz algorithm that has been discussed in Section~\ref{sec:REK}. To the best of our knowledge, the connection between the gossip model and Kaczmarz-like algorithms was first observed in~\cite{CDC12} and will be discussed in Section~\ref{sec:LaplacianGossip}. In Section~\ref{sec:improvedGossip}, we present an improved gossip algorithm by exploiting the special structure of Laplacian matrices in the case of solvable Laplacian systems.
%
%
%
%%%%%%%%%%%%%%%%%%%%%%%%%%%%%%%%%%%%%%%%%%%%%%%%%%%%%%%%%%%%%%%%%%%%%%%%%%%%%%%%
\subsection{The Model of Computation: Gossip algorithms}\label{sec:model}
%%%%%%%%%%%%%%%%%%%%%%%%%%%%%%%%%%%%%%%%%%%%%%%%%%%%%%%%%%%%%%%%%%%%%%%%%%%%%%%%
The gossip model of computation is also known in the literature as \emph{asynchronous time model}~\cite{gossip:TBA86,book:BT89}. Gossip algorithms can be categorized as being randomized or deterministic. Here, we focus on the randomized gossip model (from now on we will drop the adjective randomized). The gossip model is, roughly speaking, defined as the classical asynchronous model of computation enriched with the additional assumption that each node can activate itself (randomly) in a fixed (pre-decided) rate. The model implicitly assumes that the computational power of all nodes is comparable with each other.
%

%
More formally, the gossip model is defined as follows: Each node $i$ has a clock which ticks at the times of a $\gamma_i$ Poisson process\footnote{See~\cite{book:Fellerv1} for more about Poisson processes.}. So, the inter-tick times of each node are rate $\gamma_i$ exponential random variables, independent over all the nodes and over time. Equivalently, using properties of the Poisson distribution\footnote{Let $W_1,\ldots ,W_n$ be $n$ independent exponential random variables with rates $\gamma_1,\ldots, \gamma_n$, respectively. Let $W_{\text{min}}$ be the minimum of $W_1,\ldots ,W_n$. Then $W_{\text{min}}$ is an exponential random variable of rate $\sum_{i=1}^{n} \gamma_i$.}, this corresponds to a single (global) clock ticking with a rate $\sum_{i\in V} \gamma_i$ Poisson time process at times $Z_0, Z_1, Z_2,\ldots$, where $\{Z_{k} - Z_{k-1}\}$ are i.i.d. exponentials of rate $\gamma_{I_k}$ assuming that node $I_k\in V$ is selected at the $k$-th tick. It is easy to see that $I_k$ are i.i.d. random variables distributed over $V$ with probability mass $\{\gamma_i\}_{i\in V}$.

A Chernoff bound type of argument can be used to relate the number of clock ticks to absolute time (time units) which allows us to discuss the results in terms of clock ticks instead of absolute time, see~\cite{gossip:Boyd} for the details. Therefore, the algorithmic design problem under the gossip model is to analyze distributed algorithms that require the minimum possible number of clock ticks in expectation given a particular problem.
%

%
In some sense the gossip model lies in the middle of the synchronous model and asynchronous model of computation. Recall that in the asynchronous model of computation, roughly speaking, the assumption is that each node performs a predefined computation infinitely often. Although such an assumption is very general, it is insufficient to hope for providing bounds on the rate of convergence of any asynchronous algorithm. As we will see shortly in the section, the asynchronous time model overcomes this limitation.
%
%%%%%%%%%%%%%%%%%%%%%%%%%%%%%%%%%%%%%%%%%%%%%%%%%%%%%%%%%%%%%%%%%%%%%%%%%%%%%%%%
\subsection{Related Work}
%%%%%%%%%%%%%%%%%%%%%%%%%%%%%%%%%%%%%%%%%%%%%%%%%%%%%%%%%%%%%%%%%%%%%%%%%%%%%%%%
%
%
The earliest reference in gossip or epidemic algorithms is the work of Demers et al.~\cite{gossip:Demers} in which they first coined the term ``gossip'', see also~\cite{gossip:Pittel}. The authors in~\cite{gossip:Demers} proposed gossip algorithms for maintaining up-to-date replicates over a network of databases.

Nevertheless, one of the most studied problems in distributed algorithms is the average consensus problem, i.e., initially each node receives a value and each node should compute the average over all the node values. The analysis of classical (synchronous and asynchronous) distributed algorithms for the averaging problem can be traced back to the work of Tsitsiklis~\cite{gossip:TBA86}. The work of Karp~et~al.~\cite{gossip:Karp} presented a general lower bound for the averaging problem for any graph and any gossip algorithm in the synchronous setting. Gossip-based algorithms for aggregating information where the underlying graph is the complete graph was studied in~\cite{gossip:Kempe}, see also~\cite{gossip:KempeImprov} for improvements. The analysis of randomized gossip-based averaging algorithms for an arbitrarily network topology was studied in~\cite{gossip:Boyd}. Although the results of~\cite{gossip:Boyd} are stated for computing the average function, their theoretical framework can be easily extended to the computation of other functions as well including maximum, minimum or product functions.

Solving Laplacian systems in a distributed manner is a fundamental computational primitive since several problems, such as clock synchronization, can be formulated as the solution of a Laplacian system~\cite{smoothing,karp,CDC12}. Laplacian solvers have been successfully analyzed in both the synchronous and asynchronous model of computation~\cite{gossip:TBA86}. However, to the best of our knowledge, solving Laplacian systems under the gossip model of computation has not been well-studied. That said, the techniques of~\cite{gossip:Boyd} have been applied in~\cite{gossip:Bolognani,gossip:IPSN05_Boyd,gossip:IPSN06_LS} to provide a naive solution to the problem, albeit with many drawbacks that we list in the following paragraph. For additional references on the least squares estimation problem, see the survey paper of Dimakis et al.~\cite[\S~IV]{gossip:Dimakis}.

The approach of~\cite{gossip:Bolognani,gossip:IPSN05_Boyd,gossip:IPSN06_LS} for solving Laplacian systems has several limitations. The underlying idea behind these results is to apply average consensus algorithms as the main building block towards solving the least-squares estimation problem. The main approach is as follows: in the first (distributed) step, each node applies a consensus-based algorithm to approximate all the entries of the normal equations matrix of the least squares problem together with the corresponding right hand side vector of the normal equation. In the next step, each node has all the required information to solve the least squares estimation problem individually. The main drawback of this approach is that each node has to compute (in parallel) a quadratic number of instances of the averaging problem and moreover each node is required to solve a linear system. Such solution clearly does not scale to large networks because it requires the transmission of quadratic number of messages in terms of the number of parameters to be estimated and the computational requirements per node are prohibitively intense (i.e., solving a linear system).  The following theorem summarizes\footnote{We should mention that the approaches of~\cite{gossip:Bolognani,gossip:IPSN05_Boyd,gossip:IPSN06_LS} operate under a more general setting of time-varying network topologies requiring very weak conditions of connectivity~\cite{gossip:IPSN06_LS}.} the outcome of the approaches in~\cite[\S~IV-A]{gossip:IPSN05_Boyd},~\cite{gossip:IPSN06_LS}. and~\cite[Theorem~5]{gossip:Bolognani} (see the following section for notation).
\begin{theorem}\cite[Theorem~5]{gossip:Bolognani}
	Let $G=(V,E)$ be a connected network of $n$ nodes and assume that each node $i\in V$ gets as input a value $b_i$. There is a consensus-based algorithm so that: every node $i\in V$ asymptotically computes the i-th entry of the vector $\xls$, where $\xls$ is the minimum $\ell_2$-norm solution vector of the Laplacian system $\matL (G) \x = \b$.
\end{theorem}
All the above approaches are able to only provide convergence in the limit as the number of iterations goes to infinity. In contrast, the results of the present section provide bounds on the rate of convergence, see Corollary~\ref{cor:gossip}.
%
%%%%%%%%%%%%%%%%%%%%%%%%%%%%%%%%%%%%%%%%%%%%%%%%%%%%%%%%%%%%%%%%%%%%%%%%%%%%%%%%
\subsection{Preliminaries and Problem Definition}
%%%%%%%%%%%%%%%%%%%%%%%%%%%%%%%%%%%%%%%%%%%%%%%%%%%%%%%%%%%%%%%%%%%%%%%%%%%%%%%%
%
The communication network is modeled by an undirected graph $G = (V,E)$. We let $n:=|V|$ be the number of agents and $m:=|E|$ be the number of communication links. For simplicity, communication is taken symmetric. Two neighboring nodes $i,j: (i,j)\in E$, can exchange packets to exchange information.
%

%
Let us label the nodes as $1,\ldots, n$ and write the undirected edge $e=(i,j)$ with $i<j$.  Consider any unknown vector $\x \in \R^n$ of node variables, where variable $x_i$ corresponds to node $i$. The \emph{edge-vertex incidence} matrix of the graph $\matB\in \R^{m \times n}$ has entries:
%
\begin{equation}
\matB_{ek} :=
\left\{
  \begin{array}{ll}
    -1, & \hbox{if $k=i$;} \\
    1, & \hbox{if $k=j$;} \\
    0, & \hbox{otherwise}.
  \end{array}
\right.
\end{equation}
%
Let $\matL$ be the unnormalized Laplacian matrix of $G$, i.e., $\matL=\matD - \matA$ where $\matA$ is the adjacency matrix of $G$ and $\matD$ is the diagonal matrix whose $(i,i)$-th entries is the degree of node $i$. It is a well-known fact that $\matL= \matB^\top \matB$. The goal of this section is to design distributed algorithms under the gossip model of computation that solve
\[\matL \x = \b.\]
We assume that every node $i\in V$ has access only to the values $b_i$ and $b_j $ for all $j$ that are adjacent to $i$. Since $\matL$ is singular the goal is to compute, in a distributed manner, the entries of the minimum $\ell_2$-norm least squares solution, $\xls = \pinv{\matL} \b$. More precisely, each node $i\in V$ has to compute (actually sufficiently approximate) the $i$-th coefficient of $\xls$.
%
%%%%%%%%%%%%%%%%%%%%%%%%%%%%%%%%%%%%%%%%%%%%%%%%%%%%%%%%%%%%%%%%%%%%%%%%%%%%%%%%
\paragraph{Basics from graph theory}\label{sec:graph}
%%%%%%%%%%%%%%%%%%%%%%%%%%%%%%%%%%%%%%%%%%%%%%%%%%%%%%%%%%%%%%%%%%%%%%%%%%%%%%%%
%
%
For each $i\in V$, we define its \emph{neighborhood}, $N_i:= \{j\in V: (i,j)\in E\}$. The \emph{degree} of the node is $d_i:=|N_i|$, and we define $d_{\max}:=\max_i d_i$.  Let $\lambda_1 \leq \lambda_2 \leq \ldots \leq \lambda_n$ be the eigenvalues of $\matL$.  For a connected graph we have that $\lambda_1 = 0$,  and $0<\lambda \le d_{\max}$, for $i=2,\ldots,n$.  The second smallest eigenvalue of $\matL$ denoted by $\lambda_2(G)$, also called the \emph{Fiedler value} or \emph{algebraic connectivity} of $G$, can be lower-bounded via Cheeger's inequality~\cite{book:spectralGraph} as follows: define for each non-empty $S\subseteq V$, the \emph{volume} to be the sum of the degrees of the vertices in $S$: $\vol{S} := \sum_{i\in S} d_i$; furthermore, let $E(S,\bar{S})$ be the set of edges with one vertex in $S$ and the other one in $V\setminus \bar{S}$; finally, let $h_G(S):= \frac{|E(S,\bar{S})|}{\min\{\vol{S}, \vol{\bar{S}}\} }$. The \emph{Cheeger constant} of $G$ is defined as $h_G:= \min_{S\neq \emptyset} h_G(S)$. Then:
\begin{equation}\label{eq:cheeger}
\lambda_2(G) \geq \frac{h_G^2}{2d_{\max}}.
\end{equation}
We will show that the rate of convergence of the presented random gossip algorithms depends on $\lambda_2(G)$. From now on we implicitly assume that that input graph is connected.

Let $\matB = \matU \matSig \matV^\top$ be the (truncated) singular value decomposition of $\matB$, i.e., $\matU$ and $\matV$ are $m\times (n-1)$ and $n\times (n-1)$ matrices with orthonormal columns respectively, and $\matSig$ is a diagonal matrix of size $(n-1)$ with positive elements. Since $\matL = \matB^\top \matB$, it holds that $\matL = \matV\matSig^2 \matV^\top$.
%
%
%%%%%%%%%%%%%%%%%%%%%%%%%%%%%%%%%%%%%%%%%%%%%%%%%%%%%%%%%%%%%%%%%%%%%%%%%%%%%%%%
\paragraph{Basics from Linear Algebra}\label{sec:linAlgebra}
%%%%%%%%%%%%%%%%%%%%%%%%%%%%%%%%%%%%%%%%%%%%%%%%%%%%%%%%%%%%%%%%%%%%%%%%%%%%%%%%
%
We summarize an observation regarding the structure of the minimum $\ell_2$-norm least-squares solution of Laplacian systems.
\begin{lemma}\label{lem:charLS}
	Let $\xls$ be the minimum $\ell_2$-norm least squares solution of $\matL \x = \b$. Let $\xls'$ be the returned vector after the following two-step procedure:
	\begin{enumerate}[(a)]
		\item
		Compute the minimum $\ell_2$-norm least squares solution of $\matB^\top \y = \b$, i.e., $\yls :=\pinv{(\matB^\top)} \b  $
		\item
		Compute and return the minimum $\ell_2$-norm least squares solution of $\matB \x = \yls$, i.e., $\xls' :=\pinv{\matB} \yls $
	\end{enumerate}
Then, $\xls'$ equals $\xls$.
\end{lemma}
\begin{proof}
Notice that $\xls' = \pinv{\matB} \yls = \pinv{\matB} (\matB^\top)^{\dagger} \b = \matV \matSig^{-1}\matU^{\top} \matU \matSig^{-1}\matV^{\top} \b  = \matV \matSig^{-2} \matV^{\top} \b = \pinv{\matL}\b$ where we used that $\matB = \matU\matSig\matV^\top $, $\matU^{\top} \matU = \matI_m$ and $\matV \matSig^{-2} \matV^{\top} = \pinv{\matL}$.
\end{proof}
%
%
%\clearpage
%%%%%%%%%%%%%%%%%%%%%%%%%%%%%%%%%%%%%%%%%%%%%%%%%%%%%%%
%%%%%%%%%%%%%%%%%%%%%%%%%%%%%%%%%%%%%%%%%%%%%%%%%%%%%%%
\subsection{Randomized Gossiping via Randomized Extended Kaczmarz}\label{sec:LaplacianGossip}
%%%%%%%%%%%%%%%%%%%%%%%%%%%%%%%%%%%%%%%%%%%%%%%%%%%%%%%
%%%%%%%%%%%%%%%%%%%%%%%%%%%%%%%%%%%%%%%%%%%%%%%%%%%%%%%
%
We propose a randomized gossip algorithm that exponentially converges to the least-squares solution of the Laplacian system corresponding to the underlying communication graph. The proposed gossip algorithm is based on the randomized extended Kaczmarz method (Algorithm~\ref{alg:REK}) as explained in detail next.
%

%
Consider a given column of $\matL$ with index $j\in [n]$. $\Lc{j}$ has $d_j + 1$ non-zero entries whose off-diagonal entries are equal to $-1$ and the diagonal equals to $d_j$, so $\norm{\Lc{j}}^2 = d_j^2 + d_j$. Moreover for any $\z\in\R^n$ we have $\ip{\z}{\Lc{j}} =  d_j z_j - \sum_{l\in N_j} z_{l}$. Step $6$ of Algorithm~\ref{alg:REK} is translated to
%
\[ \z^{(k+1)} \leftarrow \z^{(k)} + \frac{ z_j^{(k)} - \frac1{d_j}\sum_{l\in N_j} z^{(k)}_{l}}{d_j + 1} \Lc{j}.\]
%
In particular, if $j\in V$ is selected the only coordinates of $\z$ that are updated are $(z_{l})_{l\in N_i}$. This part of the algorithm is clearly distributed in the sense that a given iteration, only the one-hop neighbors of $i$ make updates of their local estimates based solely on exchanging their previous estimates.


Similarly, consider a given row of $\matL$, say $i\in{[n]}$, then for any $\x\in\R^n$ we have
$\ip{\x}{\Lr{i}} = d_i x_i - \sum_{l\in N_i} x_{l} $ as before. Step $7$ of Algorithm~\ref{alg:REK} is translated to
%
\[ \x^{(k+1)} \leftarrow \x^{(k)} + \frac{(b_{i}  - z_i^{(k)})/d_i -  x_i^{(k)} - \frac1{d_i}\sum_{l\in N_i} x_{i}^{(k)}}{d_i + 1} \Lr{i}.\]
%
%

%
Putting all the above observations together, we end up with Algorithm~\ref{alg:gossip}. This algorithm has exponential convergence in expectation as it follows from Theorem~\ref{thm:REK}, and its rate of convergence depends solely on the topology of the underlying communication network.
%
%
\begin{algorithm}{}
	\caption{Randomized Gossip Laplacian Solver}\label{alg:gossip}%RK applied to original noiseless equations
\begin{algorithmic}[1]
\Procedure{}{}
\ForAll{nodes $i\in V$}\Comment{Initialization step}
	\State Set $x^{(0)}_i = 0$ and detect neighbors $N_i$
	\State Node $i$ obtains $b_{j}$ for all $j\in N_i$ (hypothesis)
	\State Set $z^{(0)}_i = b_i$
\EndFor
\For {$k=0,1,2,\ldots $ (each clock tick)}
	\State\label{rek:sampl} Pick a node $i_k\in V$ with probability proportional $d_{i_k}^2 + d_{i_k}$
	\State Node $i_k$ collects $x_{j}$ and $z_{j}$ from all its neighbors $j\in N_i$
	\State Node $i_k$ broadcasts: $\theta : =  z_{i_k}^{(k)} - \frac1{d_{i_k}}\sum_{l\in N_{i_k}} z_l^{(k)}$ and $\xi : =   (b_{i_k} - z_{i_k}^{(k)})/d_{i_k} + x_{i_k}^{(k)} - \frac1{d_{i_k}}\sum_{l\in N_{i_k}} x_l^{(k)}$
	\State Node $i_k$ sets: $z^{(k+1)}_{i_k} \leftarrow  z^{(k)}_{i_k} + \frac{d_{i_k} \theta}{1  + d_{i_k} } $ and $x^{(k+1)}_{i_k} \leftarrow  x^{(k)}_{i_k} + \frac{d_{i_k} \xi }{1  + d_{i_k} } $
	\State Every node $j\in N_{i_k}$ sets: $ z^{(k+1)}_{j} \leftarrow z^{(k)}_{j} - \frac{\theta }{ 1 + d_{i_k} } $ and $ x^{(k+1)}_{j} \leftarrow x^{(k)}_{j} - \frac{\xi}{ 1 + d_{i_k} } $
\EndFor
\EndProcedure
\end{algorithmic}
\end{algorithm}
%
%
%The above algorithm is a translation of the randomized Kacmzarz algorithm in the clock synchronization setting. Therefore, we are able to analysis its speed of convergence using Theorem~\ref{thm:RK:consistent}.
\begin{corollary}[Convergence rate of Algorithm~\ref{alg:gossip}]\label{cor:gossip}
The updates of estimates produced by Algorithm~\ref{alg:gossip} satisfy:
%
\[ \EE {\norm{\x^{(k)} - \xls}^2} \leq \left(1 - \frac{\lambda_2^2(G)}{2\sum_{i}d_i^2 + d_i}\right)^{\lfloor k/4 \rfloor } \left(\norm{\xls}^2 + 2\norm{\b}^2 / \lambda_2^2(G) \right).\]
%
In particular, for any $\eps>0$, if $k\geq  \frac{8 \sum_{i}d_i^2 + d_i}{\lambda_2^2(G)} \ln\left( \frac{\norm{\xls}^2 + 2 \norm{\b}^2 / \lambda_2^2(G)}{\eps^2}\right)$, then $\EE {\norm{\x^{(k)} - \xls}} \leq \eps$.
\end{corollary}
%
%
\begin{proof}
The proof is based on the fact that the iterations of Algorithm~\ref{alg:gossip} are similar to the iterations of Algorithm~\ref{alg:REK} applied on $\matL$. First notice that $\matL$ is symmetric, so Step~\ref{rek:sampl} produces a sample that follows the correct distribution for both the rows and columns from $\matL$. The only difference between the iterations of Algorithm~\ref{alg:gossip} and Algorithm~\ref{alg:REK} is that only one sample is generated in Algorithm~\ref{alg:gossip} whereas Algorithm~\ref{alg:REK} two different samples; one for row sampling and the other for column sampling. However, $\matL$ is symmetric and the proof of Theorem~\ref{thm:REK} works through unchanged for Algorithm~\ref{alg:gossip} which uses a simple sample for both row and column sampling because $\matL$ is symmetric and by the linearity of expectation. Moreover, notice that $\frobnorm{\matL}^2 = \sum_{i\in V} (d_i^2 + d_i)$ and $\sigma^2_{\min}(\matL) = \lambda_2^2(G)$.
\end{proof}

%\clearpage
%%%%%%%%%%%%%%%%%%%%%%%%%%%%%%%%%%%%%%%%%%%%%%%%%%%%%%%
%%%%%%%%%%%%%%%%%%%%%%%%%%%%%%%%%%%%%%%%%%%%%%%%%%%%%%%
\subsection{Improved Randomized Gossiping for Laplacian Systems}\label{sec:improvedGossip}
%%%%%%%%%%%%%%%%%%%%%%%%%%%%%%%%%%%%%%%%%%%%%%%%%%%%%%%
%%%%%%%%%%%%%%%%%%%%%%%%%%%%%%%%%%%%%%%%%%%%%%%%%%%%%%%
Algorithm~\ref{alg:gossip} requires, roughly speaking, $\widetilde{\OO}( d_{\max} m / \lambda_2^2(G))$ number of rounds for convergence to a vector arbitrarily close to the least squares solution with high probability. Here we improve the above bound to $\OO( m / \lambda_2(G))$ iterations whenever $\matL \x = \b$ has a solution\footnote{The general setting can be achieved by utilizing the averaging algorithm~\cite{gossip:Boyd} on $\b$ and each node $i$ updates its right hand side coefficient to $b'_i:=b_i - \frac1{n}\sum_{i=1}^n{b_i}$. Notice that $\b'$ is in the range of $\matL$ and moreover, the minimum $\ell_2$-norm least-squares solution of $\matL \x = \b'$ equals to $\pinv{\matL} \b$.}. The main idea is based on the special decomposition of the normalized Laplacian matrix, i.e., $\matL= \matB^\top\matB $ and Lemma~\ref{lem:charLS}. That is, apply the procedure described in Lemma~\ref{lem:charLS} to solve the normalized Laplacian system. Assuming the notation of Lemma~\ref{lem:charLS}, observe that $\yls$ is in the column span of $\matB$, hence the linear system $\matB \x = \yls$ is consistent. The above lemma suggests that we can utilize the randomized Kaczmarz (RK) algorithm (Algorithm~\ref{alg:RK}) to compute an approximation to $\yls$, and then again invoke (in parallel) the randomized Kaczmarz algorithm to solve the linear system of Step $(b)$ of Lemma~\ref{lem:charLS}. The rationale behind this approach is based on the fact that we are solving two linear systems with coefficient matrix $\matB$ instead of $\matL=\matB^\top \matB$, so the condition number is now square-rooted compared to the original system $\matL$.

The sparsity structure of $\matB$ implies that the randomized Kaczmarz solver applied to both $\matB$ and $\matB^\top$ is implementable under the gossip model of computation as in the previous section.
%
%
\begin{theorem}\label{thm:improvedGossip}
Fix $\eps>0$. For every\footnote{$\widetilde{\Omega}(\cdot)$ hides logarithmic factors. See Lemma~\ref{lem:7} for the exact bound.} $k= \widetilde{\Omega}(m /\lambda_2(G))$, the updates of estimates produced by Algorithm~\ref{alg:RGimproved} satisfy:
%
\[ \EE {\norm{\x^{(k)} - \xls}^2} \leq \eps^2.\]
%
\end{theorem}
%
%
%
\begin{algorithm}{}
	\caption{Improved Randomized Gossip Laplacian Solver}\label{alg:RGimproved}
\begin{algorithmic}[1]
\Procedure{}{}
\ForAll{nodes $i\in V$}\Comment{Initialization step}
		\State Set $x^{(0)}_i = 0$ and detect neighbors $N_i$
		\State Node $i$ obtains $b_{j}$ and sets $y^{(0)}_{(i,j)} = 0$
		\State Each pair of adjacent nodes $(i,j)$ maintains a value of $y^{(k)}_{(i,j)}$
\EndFor
\For {$k=0,1,2,\ldots $ (each clock tick)}
		\State \label{rk1:1}Pick a node $s_k\in{[n]}$ w.p. proportional to $d_{s_k}$
		\State \label{rk2:1}Pick an edge $(i_k,j_k)$ uniformly from the edges adjacent to $s_k$ ($i_k$ or $j_k$ equals to $s_k$)
		\State\label{rk1:2} Node $s_k$ collects $y_{(s_k,j)}^{(k)}$ for all $j\in{N_{s_k}}$ \& computes $\theta = (b_{s_k} - \sum_{j\in{N_{s_k}}}y^{(k)}_{(s_k,j)})/d_{s_k} $.
		\Statex
		\State\label{rk1:3} Node $s_k$ broadcasts $\theta$ and  $y_{(s_k,j)}^{(k+1)} = y_{(s_k,j)}^{(k)} + \theta$ for every $j\in{N_{s_k}}$
		\State Comment: Steps~\ref{rk1:1},\ref{rk1:2} and~\ref{rk1:3} correspond to RK on $\matB^\top \y = \b$:
		\[\y^{(k+1)}  =  \y^{(k)}  + \frac{b_{s_k} - \sum_{j\in{N_{s_k}}}y^{(k)}_{(s_k,j)}  }{d_{s_k}} \Bc{s_k}\]
		\Statex
		\State\label{rk2:2} Node $i_k$ and $j_k$: $x_{i_k}^{(k+1)} = x_{i_k}^{(k)} + (y^{(k)}_{(i_k,j_k)} + x^{(k)}_{i_k} - x^{(k)}_{j_k}) /2 $ and $x_{j_k}^{(k+1)} = x_{j_k}^{(k)} - (y^{(k)}_{(i_k,j_k)} + x^{(k)}_{i_k} - x^{(k)}_{j_k}) /2 $, resp.
	\State Comment: Steps~\ref{rk2:1}and \ref{rk2:2} corresponds to applying RK on $\matB \x = \y^{(k)}$
	\[\x^{(k+1)} = \x^{(k)}  + \frac{y^{(k)}_{(i_k,j_k)} - (x_{i_k}^{(k)} - x_{j_k}^{(k)})}{2} \Br{(i_k,j_k)}\]
\EndFor
\EndProcedure
\end{algorithmic}
\end{algorithm}
%

%
We devote the rest of this section to prove Theorem~\ref{thm:improvedGossip}. The proof is based on the following three lemmas. Lemma~\ref{lem:5} indicates that the estimates of $\y^{(l)}$ converge exponentially to $\yls$ in expectation, since Steps~\ref{rk1:1},~\ref{rk1:2} and~\ref{rk1:3} perform updates of the randomized Kaczmarz algorithm on the system $\matB^\top \y = \b$. Lemma~\ref{lem:6} states that during the course of Algorithm~\ref{alg:RGimproved}, in expectation, the estimates $\x^{(k)}$ are within a ball of \emph{fixed} radius centered at $\xls$.

The main difficulty on proving Theorem~\ref{thm:improvedGossip} is the fact that Steps~\ref{rk2:1} and~\ref{rk2:2} of Algorithm~\ref{alg:RGimproved} applies a single iteration of RK on the linear systems $\{\matB \x = \y^{(k)}\}_{k\in \N}$ that are being updated after each iteration. We bypass this obstacle by observing the following: for sufficiently large $k$, $\y^{(k)}$ is arbitrarily close to $\yls$ (Lemma~\ref{lem:5}) and in addition $\norm{\x^{(k)} - \xls }$ is bounded in expectation (Lemma~\ref{lem:6}).
%

%
That is, after sufficiently many iterations, the current estimate $\x^{(k)}$ is within a bounded distance away from $\xls$. Moreover, Step $10$ of the algorithm is now applied on linear systems that, in expectation, are arbitrarily close to the linear system $\matB \x = \yls$, since $\matB \x = \y^{(k)} + \yls - \y^{(k)}$ and $\norm{\y^{(k)} - \yls}$ is arbitrarily small. Lemma~\ref{lem:7} formalized this discussion.
\begin{lemma}\label{lem:5}
Assuming the notation of Algorithm~\ref{alg:RGimproved}, for every $l>0$ it holds that
\begin{equation}\label{ineq:stopTime2}
	\EE \norm{\y^{(l)} - \yls}^2 \leq  \left(1- \frac1{\kappaFS(\matB)}\right)^{ l} \norm{\yls}^2.
\end{equation}
\end{lemma}
\begin{lemma}\label{lem:6}
Assuming the notation of Algorithm~\ref{alg:RGimproved}, for every $k>0$, it holds that
\[\EE \norm{\x^{(k + 1)} - \xls}^2 \leq \left(1- \frac1{\kappaFS(\matB)}\right)^{k+1}\norm{\xls}^2 + \norm{\yls}^2 / \sigma_{\min}^2(\matB).\]
\end{lemma}

%%%%%%%%%%%%%%%%%%%%%%%%%%%%%%%%%%%%%%%%%%%%%%%%%%%%%%%
%%%%%%%%%%%%%%%%%%%%%%%%%%%%%%%%%%%%%%%%%%%%%%%%%%%%%%%
\begin{proof}
%(of Lemma~\ref{lem:6})
%%%%%%%%%%%%%%%%%%%%%%%%%%%%%%%%%%%%%%%%%%%%%%%%%%%%%%%
Set $\alpha = 1 - 1/\kappaFS(\matB)$. We will show a uniform bound over all $k>0$ on $\EE \norm{\x^{(k)} - \pinv{\matB} \yls}$. The crucial point of the proof is to think of the evolution of the algorithm as an application of the randomized Kaczmarz algorithm applied on a noisy linear system. By definition of the algorithm, at the $k$-th iteration we update the estimate $\x^{(k)}$ to $\x^{(k+1)}$ by applying the randomized Kaczmarz update rule on $\matB \x = \y^{(k)}$. Now, we assume that the right hand side is ``noisy'', and the desired linear system is $\matB\x = \yls$, hence, applying the analysis of the noisy randomized Kaczmarz on the linear system $\matB \x = \y^{(k)}$ (set $\w$ in Theorem~3 to be $\w^{(k)}:=\y^{(k)} - \yls$) we get that
\begin{align}
		\EE_k \norm{\x^{(k + 1)} - \xls}^2 & \leq \alpha \norm{\x^{(k)} - \xls}^2 + \frac{\norm{\w^{(k)}}^2}{\frobnorm{\matB }^2}
\end{align}
Now averaging over the first $k$ steps of the algorithm and using linearity of expectation we get
\[		\EE \norm{\x^{(k + 1)} - \xls}^2  \leq  \alpha \EE\norm{\x^{(k)} - \xls}^2 + \frac{\EE\norm{\w^{(k)}}^2}{\frobnorm{\matB }^2} \]
Applying inductively the above reasoning on the right hand side, it follows that
\begin{equation}\label{ineq:generic}
\EE \norm{\x^{(k + 1)} - \xls}^2  \leq  \alpha^{k+1} \norm{\xls}^2 + \sum_{l=0}^{k}\alpha^l \frac{\EE\norm{\w^{(k-l)}}^2}{\frobnorm{\matB }^2}.
\end{equation}
Since $\alpha <1$, Lemma~\ref{lem:5} tells us that $\EE \norm{\w^{(l)}}^2 = \EE \norm{\y^{(l)} - \yls}^2 \leq  \norm{\yls}^2 $ for every $l>0$. We conclude that
\[
\EE \norm{\x^{(k + 1)} - \xls}^2  \leq  \alpha^{k+1}\norm{\xls}^2 + \frac{\norm{\yls}^2}{\frobnorm{\matB }^2}\sum_{l=0}^{\infty}\alpha^l.
\]
To conclude, observe that $\sum_{l=0}^{\infty}\alpha^l = \frobnorm{\matB}^2/\sigma_{\min}^2(\matB)$.
\end{proof}
%%%%%%%%%%%%%%%%%%%%%%%%%%%%%%%%%%%%%%%%%%%%%%%%%%%%%%%
%%%%%%%%%%%%%%%%%%%%%%%%%%%%%%%%%%%%%%%%%%%%%%%%%%%%%%%
%
%
\begin{lemma}\label{lem:7}
Fix any accuracy parameter $\eps>0$. If
\[k\geq \kappaFS(\matB)\left(\ln\left(\frac{2\norm{\xls}^2}{\eps^2} + \frac{2\norm{\yls}^2}{\eps^2\sigma_{\min}^2(\matB)} \right)
+
2\ln\left( \frac{2\norm{\yls}^2}{\eps^2\sigma_{\min}^2(\matB)} \right)\right),\]
then the updates of estimates produced by Algorithm~\ref{alg:RGimproved} satisfy $\EE \norm{\x^{(k)} - \xls }^2 \leq \eps^2$.
\end{lemma}
%
%
%%%%%%%%%%%%%%%%%%%%%%%%%%%%%%%%%%%%%%%%%%%%%%%%%%%%%%%
%%%%%%%%%%%%%%%%%%%%%%%%%%%%%%%%%%%%%%%%%%%%%%%%%%%%%%%
\begin{proof}
%(of Lemma~\ref{lem:7})
%%%%%%%%%%%%%%%%%%%%%%%%%%%%%%%%%%%%%%%%%%%%%%%%%%%%%%%
Set $\alpha = 1 - 1/\kappaFS(\matB)$. Let $l^*$ and $k^*$ be constant to be specified shortly. Repeat the proof of Lemma~\ref{lem:6} for $l^*$ iterations starting at the $(l^*+k^*)$-th iteration (use Ineq.~\eqref{ineq:generic}), it follows that
\begin{align*}
	\EE \norm{\x^{(l^* + k^*)} - \xls}^2  & \leq   \alpha^{l^*} \EE\norm{\x^{(k^*)} - \xls}^2 + \sum_{l=0}^{l^* - 1}\alpha^l \frac{\EE\norm{\w^{(k^* + l^* - l)}}^2}{\frobnorm{\matB}^2}.
\end{align*}
Now,
%
\begin{align*}
\EE \norm{\x^{(l^* + k^*)} - \xls}^2			  & \leq  \alpha^{l^*} \EE\norm{\x^{(k^*)} - \xls}^2 + \frac{\EE\norm{\w^{(k^*)}}^2}{\frobnorm{\matB}^2}\sum_{l=0}^{\infty}\alpha^l \ \leq \  \alpha^{l^*} \left(\norm{\xls}^2 + \frac{\norm{\yls}^2}{\sigma_{\min}^2(\matB)} \right) + \frac{\EE\norm{\w^{(k^*)}}^2}{\sigma_{\min}^2(\matB)} \\
\end{align*}
%
the first inequality follows by the fact that $\EE\norm{\w^{(k^* + j)}}^2 \leq \EE\norm{\w^{(k^*)}}^2$ for every $j>0$, the second by the convergence of the summation, the definition of $\alpha$ and Lemma~\ref{lem:6}. Now, set $k^* = 2\kappaFS(\matB) \ln( \frac{2\norm{\yls}^2}{\eps^2\sigma_{\min}^2(\matB)})$ in Lemma~\ref{lem:5} it holds that
\[
\EE\norm{\w^{(k^*)}}^2 = \EE\norm{\y^{(k)} - \yls}^2 \leq \eps^2 \sigma_{\min}^2(\matB) / 2.
\]
To conclude, set $l^* = \kappaFS(\matB)\ln\left(\frac{2\norm{\xls}^2}{\eps^2} + \frac{2\norm{\yls}^2}{\eps^2\sigma_{\min}^2(\matB)} \right)$, which implies that $\EE \norm{\x^{(k^*)} - \xls }^2 \leq \eps^2/2 \leq \eps^2$.
\end{proof}
%%%%%%%%%%%%%%%%%%%%%%%%%%%%%%%%%%%%%%%%%%%%%%%%%%%%%%%
%%%%%%%%%%%%%%%%%%%%%%%%%%%%%%%%%%%%%%%%%%%%%%%%%%%%%%%
The statement of Theorem~\ref{thm:improvedGossip} follows by Lemma~\ref{lem:7} since $\kappaFS(\matB) = \frobnorm{\matB}^2 / \sigma_{\min}^2(\matB) = 2m / \lambda_{\min}(\matB^\top \matB) = 2m / \lambda_2(G)$.
 % Graph Algorithms
\chapter{Conclusions}\label{chap:conclusion}
%
In the present thesis, we proposed randomized approximation algorithms for several computational problems that can be framed using the (highly descriptive) language of linear algebra. In almost all scenarios presented here, the proposed algorithms are asymptotically more efficient than the state-of-the-art exact deterministic procedures for constant approximation error. We believe that the proposed algorithms might be practical for applications in which an approximate solution within a few digits of precision away from the exact answer is sufficient.
%

%
However, many applications require highly accurate approximations to the exact solution, namely in the regime of ten to fifteen digits of precision. Most of the randomized algorithms discussed here seem not applicable in such high accuracy regimes mainly because their time complexity depends inverse polynomially on the approximation error, i.e., $\poly{1/\eps}$. Such an inverse polynomial dependency is a consequence of the sampling bounds that are required by probabilistic considerations. On the other hand there are cases, such as the paradigm of randomized proconditioning~\cite{RT08,AMT10}, where randomness turned out to be extremely useful on the high accuracy regime and even produced algorithms that outperformed well-developed software packages (LAPACK)~\cite{LAPACK}. We believe that there are many other such successful paradigms to be discovered.
%

%
Nevertheless, we hope that the research presented in this thesis triggers a lot of interesting questions to pursue. We briefly enumerate some of the them below.
%

%
As we mentioned in Chapter~\ref{chap:intro}, Wigderson and Xiao generalized the conditional expectation method to the matrix-valued setting~\cite{chernoff:matrix_valued:derand:WX08}. Along the same lines, we proposed the \emph{matrix hyperbolic cosine} algorithm in Section~\ref{sec:balancing}. Both these generalizations can be viewed as a derandomization on the space of matrices equipped with the operator norm. The usefulness of similar matrix concentration inequalities under a different class of norms, i.e., Schatten norms was presented in~\cite{chernoff:matrix_valued:Azuma_Naor}. The main motivation in~\cite{chernoff:matrix_valued:Azuma_Naor} was to construct small-set expanding graphs and we should highlight that this construction inspired researchers to design algorithms towards refuting the unique games conjecture~\cite{ABS10}. We believe that an interesting research direction is to ``derandomize'' such matrix concentration inequalities under the Schatten norm.
%

%
An additional open question that can be raised in Chapter~\ref{chap:intro} is about the balancing matrix game (see end of Section~\ref{sec:balancing}): Does Spencer's six standard deviation bound hold in the matrix-valued setting\footnote{The author would like to thank Toni Pitassi for bringing this question into his attention.}? Moreover, a better understanding of the connection between the matrix hyperbolic cosine algorithm and Arora-Kale's matrix multiplicative weights update method~\cite{arora:fast_SDP,phdthesis:Kale:2008} is an interesting direction of research.
%

%
In Chapter~\ref{chap:rnla}, we analyzed randomized approximation algorithms for fundamental linear algebraic computations. The main drawback of most of these algorithms is that they are effective only in the case of highly rectangular matrices, i.e., input matrices containing much more rows than columns or vice versa. Can randomness help us in the scenarios of  ``almost'' square matrices? For example, is there an efficient randomized algorithm for approximating the eigenvalues/singular values of squares matrices?
%

%
One of the problems that we studied in Chapter~\ref{chap:ma} was the design of approximation algorithms for solving linear regression problems. To the best of our knowledge, we are unaware of any lower bounds for this problem. Is there a near-linear time randomized approximate least-squares solver? There is some recent indication that this might be the case, however, under a weak notion of approximation~\cite{ls:nnzA}.
%

%
Last but not least, we would like to pose a question regarding the approximate matrix multiplication problem discussed in the beginning of Chapter~\ref{chap:rnla}. Unfortunately, the main theorem about approximate matrix multiplication (Theorem~\ref{thm:matrixmult}) is only effective in the case of highly rectangular matrices and the case of low stable rank matrices. So, is it possible that randomness is able to help us devise algorithms for the cases of high stable rank almost square matrices?
%
%
% The end
 % Conclusions
\chapter*{Appendix}
We briefly recall an efficient and numerically stable algorithm for computing all the eigenvalues of any rank-one updated diagonal matrix of size $n$ that was proposed in~\cite{Gu:update}. This algorithm is the main ingredient behind the fast isotropic sparsification algorithm~\ref{alg:fast:isotrop} presented in Chapter~\ref{chap:ma}.
%%%%%%%%%%%%%%%%%%%%%%%%%%%%%%%%%%%%%%%%%%%%%%%%%%%
\section*{Fast Multiplication with Cauchy Matrices and Special Eigensystems}
%%%%%%%%%%%%%%%%%%%%%%%%%%%%%%%%%%%%%%%%%%%%%%%%%%%
%%%%%%%%%%%%%%%%%%%%%%%%%%%%%%%%%%%%%%%%%%%%%%%%%%%
We start by defining the so-called Cauchy (generalized Hilbert) matrices. An $m\times n$ matrix $\matC$ defined by
\[ C_{i,j} := \frac1{t_i - s_j},\quad i\in{[m]},j\in{[n]},\]
where $\vct{t}=(t_1,\ldots ,t_m),\ \vct{t}\in\reals^m$ and $\vct{s}=(s_1,\ldots ,s_n),\ s\in\reals^n$  and $t_i\neq s_j$ for all $i\in{[m]}$ and $j\in{[n]}$ is called \emph{Cauchy}. Given a vector $\x\in\reals^n$, the naive algorithm for computing the matrix-vector product $\matC \x$ requires $\OO(mn)$ operations. It is not clear if it is possible to perform this computation in less than $\OO(mn)$ operations. Surprisingly enough, it is possible to compute this product with $\OO((m+n)\log^2 (m+n))$ operations. This computation can be done by two different approaches. The first one is based on fast polynomial multiplication, polynomial interpolation and polynomial evaluation at distinct points~\cite[Algorithm~$1$, p.~$130$]{book:fast_matrix:Bini_Pan}. The main drawback of this approach is its numerical instability. The second approach is based on the so-called Fast Multipole Method (FMM) introduced in~\cite{FMM:CGR}. This method returns an approximate solution to the matrix-vector product for any given error parameter\footnote{That is, given an $n\times n$ Cauchy matrix, a vector $\x\in\reals^n$ and $0<\eps< 1$, returns a vector $\y\in\reals^n$ so that $\infnorm{\y - \matC\x} \leq \eps \infnorm{\x}$ in time $\OO(n \log^2 (1/\eps))$. In an actual implementation, setting $\eps$ to be a small constant relative to the machine's (numerical) precision suffices; see~\cite[\S~$3$]{Gu:update} for a more careful implementation and discussion on numerical issues.}. Ignoring numerical issues that are beyond the scope of this work, we summarize our discussion to the following
%
%
\begin{lemma}\cite{book:fast_matrix:Bini_Pan,FMM:CGR}\label{lem:fast_mm:gerasoulis}
Let $\x\in\reals^n$ and $\matC$ be a Cauchy matrix defined as above with $\vct{t}\in\reals^m, \vct{s}\in\reals^n$. There is an algorithm that, given vectors $\vct{s},\vct{t},\x$, computes the product $\C  \x$ using $\OO((m+n)\log^2 (m+n))$ operations.
\end{lemma}
%
Given a self-adjoint matrix $\matB = \Sigma + \rho \u \otimes \u$, where $\Sigma = \diag{\sigma_1 ,\ldots ,\sigma_n}$, $\rho >0$ and $u\in\reals^n$ is a unit vector, our goal is to efficiently compute all the eigenvalues of $B$. It is well-known that the eigenvalues of $B$ are the roots of a special function, known as secular function~\cite{rank_one_update:Golub} and are interlaced with $\{\sigma_{i}\}_{i\leq n}$. In addition, evaluating the secular function requires $\OO(n)$ operations implying that a standard (Newton) root-finding procedure requires $\OO(n)$ operations per each eigenvalue. Hence, $\OO(n^2)$ operations are required for all eigenvalues. In their seminal paper~\cite{Gu:update}, Gu and Eisenstat showed that it is possible to encode the updates of the root-finding procedure for \emph{all} eigenvalues as matrix-vector multiplication with an $n\times n$ Cauchy matrix. Based on this observation, they showed how to use the Fast Multipole Method for approximately computing all the eigenvalues of this special type of eigenvalue problem.
%
\begin{lemma}\cite{Gu:update}\label{lem:comp_eigs}
Let $b\in\NN$, $\rho>0$, $\Sigma=\diag{\sigma_1,\sigma_2,\ldots , \sigma_n}$ and $\u\in\reals^n$ be a unit vector. There is an algorithm that given $\Sigma, \rho, \u$ computes all the eigenvalues of $\matB=\Sigma + \rho \u \otimes \u$ within an additive error $2^{-b}\norm{\matB}$ in $\OO(n \log^2 n \log b )$ operations.
\end{lemma}
%%%%%%%%%%%%%%%%%%%%%%%%%%%%%%%%%%%%%%%%%%%%%%%%%%%
%%%%%%%%%%%%%%%%%%%%%%%%%%%%%%%%%%%%%%%%%%%%%%%%%%%
 % Appendix (placeholder for proof)

%% This adds a line for the Bibliography in the Table of Contents.
\addcontentsline{toc}{chapter}{Bibliography}
%% ***   Set the bibliography style.   ***
%% (change according to your preference)
{\small
\bibliographystyle{alpha}
%% ***   Set the bibliography file.   ***
%% ("thesis.bib" by default; change if needed)
\bibliography{references}
}
%% ***   NOTE   ***
%% If you don't use bibliography files, comment out the previous line
%% and use \begin{thebibliography}...\end{thebibliography}.  (In that
%% case, you should probably put the bibliography in a separate file
%% and `\include' or `\input' it here).

\end{document}

%%%%%%%%%%%%%%%%%%%%%%%%%%%%%%%%%%%%%%%%%%%%%%%%%%%%%%%%%%%%%%%%%%%%%%
%%  End of UT-THESIS.TEX
%%%%%%%%%%%%%%%%%%%%%%%%%%%%%%%%%%%%%%%%%%%%%%%%%%%%%%%%%%%%%%%%%%%%%%
\endinput
